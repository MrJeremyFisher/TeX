\documentclass[10pt]{article}

\usepackage[margin=0.75in]{geometry}
\usepackage{amsmath,amsthm,amssymb}
\usepackage{xcolor}
\usepackage{cancel}
\usepackage{graphicx}
\usepackage{changepage}
\usepackage{quantikz}
\usepackage{tikz}
\usepackage{circuitikz}
\usepackage{pgfplots}
\usepackage{physics}
\usepackage{nicematrix}
\usepackage{hyperref}
\usepackage{siunitx}
\usepackage[breakable]{tcolorbox}
\usepackage[inline]{enumitem}

\theoremstyle{definition}
\newtheorem{problem}{Problem}
\newtheorem{soln}{Solution}

\pgfplotsset{compat=newest}
\usetikzlibrary{arrows, angles, calc, quotes}
\usetikzlibrary{quantikz2}

\definecolor{incolor}{HTML}{303F9F}
\definecolor{outcolor}{HTML}{D84315}
\definecolor{cellborder}{HTML}{CFCFCF}
\definecolor{cellbackground}{HTML}{F7F7F7}
\newcommand{\eq}{=}
\usetikzlibrary{positioning, fit, calc}
\pgfdeclarelayer{background}  
\pgfsetlayers{background,main}
\DeclareSIUnit[number-unit-product = {\,}]\calorie{cal}
\DeclareSIUnit[number-unit-product = {\,}]\atmosphere{atm}
\AtBeginDocument{\RenewCommandCopy\qty\SI}

\makeatletter
\newcommand{\boxspacing}{\kern\kvtcb@left@rule\kern\kvtcb@boxsep}
\makeatother
\newcommand{\prompt}[4]{
    \ttfamily\llap{{\color{#2}[#3]:\hspace{3pt}#4}}\vspace{-\baselineskip}
}

\newcommand{\thevenin}[2]{
  \begin{center}
    \begin{circuitikz} \draw
      (0,0) -- (2,0) to[battery1, l_=$V_{Th}\eq#1$] (2,2) 
      to[resistor, l_=$R_{Th}\eq#2$] (0,2)
      ;
      \draw [o-] (-.07,2.079);
      \draw [o-] (-.07,0.079);
    \end{circuitikz}
  \end{center}
}

\newcommand{\norton}[2]{
  \begin{center}
    \begin{circuitikz} \draw
      (0,0) -- (3,0) to[american current source, l_=$I_{N}\eq#1$] (3,2) -- (0,2) (2,0)
      to[resistor, l=$R_{N}\eq#2$] (2,2)
      ;
      \draw [o-] (-.07,2.079);
      \draw [o-] (-.07,0.079);
    \end{circuitikz}
  \end{center}
}

\newcommand{\highlight}[1]{\colorbox{yellow}{$\displaystyle #1$}}

\newcommand{\ti}[1]{\widetilde{#1}}

\NewDocumentCommand{\evalat}{sO{\big}mm}{%
  \IfBooleanTF{#1}
   {\mleft. #3 \mright|_{#4}}
   {#3#2|_{#4}}%
}

\title{Physics 2605H: Assignment IV}
\author{Jeremy Favro \\ Trent University, Peterborough, ON, Canada}
\date{\today}

\begin{document}
\maketitle
\begin{center}
  \begin{quantikz}
    \lstick{$\ket{c}$}&\gategroup[3,steps=4,style={dashed,rounded
          corners,fill=blue!20, inner
          xsep=2pt},background]{ALICE}&&\ctrl{1}&\gate{H}&\gategroup[3,steps=4,style={dashed,rounded
          corners,fill=blue!20, inner
          xsep=2pt},background]{BOB}&&\ctrl{2}&& \\
    \lstick{$\ket{b}=\ket{0}$}&\gate{H}&\ctrl{1}&\targ{}&&\ctrl{1}&&&&\\
    \lstick{$\ket{a}=\ket{0}$}&&\targ{}&&&\targ{}&\gate{H}&\targ{}&\gate{H}&
  \end{quantikz}
\end{center}
Here we will perform the analysis by creating a global transformation matrix and applying that to the input vector.
Our global matrix will look like
$$HG_{CNOT}HG_{CNOT}HG_{CNOT}G_{CNOT}H.$$
Here the matrix for a Hadamard gate,
$$
  \frac{1}{\sqrt{2}}\begin{pmatrix}
    1 & 1  \\
    1 & -1
  \end{pmatrix}
$$
and the matrix for a CNOT gate,
$$
  \begin{pmatrix}
    1 & 0 & 0 & 0 \\
    0 & 1 & 0 & 0 \\
    0 & 0 & 0 & 1 \\
    0 & 0 & 1 & 0
  \end{pmatrix}
$$
will need to be scaled using the Kronecker product to work with a 3-qubit system.
The scaled versions of these gates will be dependent on where in the circuit the qubit(s) are.
For example, the last operation which is the first matrix we are calculating will be $H\otimes I_2 \otimes I_2$ and then
the CNOT gate after that is given by $G_{CNOT}=\ket{0}\bra{0}\otimes I_2 + \ket{1}\bra{1}\otimes X\implies G_{CNOT}=\ket{0}\bra{0}\otimes I_2\otimes I_2 + \ket{1}\bra{1}\otimes I_2\otimes X$
eg applying the identity operation to the qubit(s) we ``don't care'' about in each case, going upwards from the bottom in order of Kronecker product. We can then follow this through for the whole circuit giving
$$(H\otimes I_2 \otimes I_2)
  (\ket{0}\bra{0}\otimes I_2\otimes I_2 + \ket{1}\bra{1}\otimes I_2\otimes X)
  (H\otimes I_2 \otimes I_2)
  (G_{CNOT}\otimes I_2)
  (I_2 \otimes I_2\otimes H)
  (I_2 \otimes G_{CNOT})
  (G_{CNOT}\otimes I_2)
  (I_2 \otimes H \otimes I_2).$$
\begin{center}
  \begin{quantikz}
    \lstick{$\ket{c}$}&\gategroup[3,steps=1,label style={label position=below, rotate=90, anchor=east, xshift=-0.1cm}]{$(I_2 \otimes H \otimes I_2)$}
    &\gategroup[3,steps=1,label style={rotate=90, anchor=west}]{$(G_{CNOT}\otimes I_2)$}
    &\ctrl{1}\gategroup[3,steps=1,label style={label position=below, rotate=90, anchor=east, xshift=-0.1cm}]{$(I_2 \otimes G_{CNOT})$}
    &\gate{H}\gategroup[3,steps=1,label style={rotate=90, anchor=west}]{$(I_2 \otimes I_2\otimes H)$}
    &\gategroup[3,steps=1,label style={label position=below, rotate=90, anchor=east, xshift=-0.1cm}]{$(G_{CNOT}\otimes I_2)$}
    &\gategroup[3,steps=1,label style={rotate=90, anchor=west}]{$(H\otimes I_2 \otimes I_2)$}
    &\ctrl{2}\gategroup[3,steps=1,label style={label position=below, rotate=90, anchor=east, xshift=-0.1cm}]{$(\ket{0}\bra{0}\otimes I_2\otimes I_2 + \ket{1}\bra{1}\otimes I_2\otimes X)$}
    &\gategroup[3,steps=1,label style={rotate=90, anchor=west}]{$(H\otimes I_2 \otimes I_2)$}& \\
    \lstick{$\ket{b}=\ket{0}$}&\gate{H}&\ctrl{1}&\targ{}&&\ctrl{1}&&&&\\
    \lstick{$\ket{a}=\ket{0}$}&&\targ{}&&&\targ{}&\gate{H}&\targ{}&\gate{H}&
  \end{quantikz}
\end{center}
Performing these tensor products gives us
\begin{align*}
  (H\otimes I_2 \otimes I_2)                                                  & =\frac{1}{\sqrt{2}}\begin{pmatrix}
                                                                                                     I_2 & I_2  \\
                                                                                                     I_2 & -I_2
                                                                                                   \end{pmatrix}\otimes I_2
  =\frac{1}{\sqrt{2}}\begin{pmatrix}
                       I_4 & I_4  \\
                       I_4 & -I_4
                     \end{pmatrix}                                                                                                  \\
  (\ket{0}\bra{0}\otimes I_2\otimes I_2 + \ket{1}\bra{1}\otimes I_2\otimes X) & =\begin{pmatrix}
                                                                                   1 & 0 & 0 & 0 & 0 & 0 & 0 & 0 \\
                                                                                   0 & 1 & 0 & 0 & 0 & 0 & 0 & 0 \\
                                                                                   0 & 0 & 1 & 0 & 0 & 0 & 0 & 0 \\
                                                                                   0 & 0 & 0 & 1 & 0 & 0 & 0 & 0 \\
                                                                                   0 & 0 & 0 & 0 & 0 & 1 & 0 & 0 \\
                                                                                   0 & 0 & 0 & 0 & 1 & 0 & 0 & 0 \\
                                                                                   0 & 0 & 0 & 0 & 0 & 0 & 0 & 1 \\
                                                                                   0 & 0 & 0 & 0 & 0 & 0 & 1 & 0
                                                                                 \end{pmatrix}                       \\
  (H\otimes I_2 \otimes I_2)                                                  & =\frac{1}{\sqrt{2}}\begin{pmatrix}
                                                                                                     I_4 & I_4  \\
                                                                                                     I_4 & -I_4
                                                                                                   \end{pmatrix}                    \\
  (G_{CNOT}\otimes I_2)                                                       & =\begin{pmatrix}
                                                                                   1 & 0 & 0 & 0 & 0 & 0 & 0 & 0 \\
                                                                                   0 & 1 & 0 & 0 & 0 & 0 & 0 & 0 \\
                                                                                   0 & 0 & 1 & 0 & 0 & 0 & 0 & 0 \\
                                                                                   0 & 0 & 0 & 1 & 0 & 0 & 0 & 0 \\
                                                                                   0 & 0 & 0 & 0 & 0 & 0 & 1 & 0 \\
                                                                                   0 & 0 & 0 & 0 & 0 & 0 & 0 & 1 \\
                                                                                   0 & 0 & 0 & 0 & 1 & 0 & 0 & 0 \\
                                                                                   0 & 0 & 0 & 0 & 0 & 1 & 0 & 0
                                                                                 \end{pmatrix}                       \\
  (I_2 \otimes I_2\otimes H)                                                  & =\frac{1}{\sqrt{2}}\begin{pmatrix}
                                                                                                     1 & 1  & 0 & 0  & 0 & 0  & 0 & 0  \\
                                                                                                     1 & -1 & 0 & 0  & 0 & 0  & 0 & 0  \\
                                                                                                     0 & 0  & 1 & 1  & 0 & 0  & 0 & 0  \\
                                                                                                     0 & 0  & 1 & -1 & 0 & 0  & 0 & 0  \\
                                                                                                     0 & 0  & 0 & 0  & 1 & 1  & 0 & 0  \\
                                                                                                     0 & 0  & 0 & 0  & 1 & -1 & 0 & 0  \\
                                                                                                     0 & 0  & 0 & 0  & 0 & 0  & 1 & 1  \\
                                                                                                     0 & 0  & 0 & 0  & 0 & 0  & 1 & -1
                                                                                                   \end{pmatrix} \\
  (I_2 \otimes G_{CNOT})                                                      & =\begin{pmatrix}
                                                                                   1 & 0 & 0 & 0 & 0 & 0 & 0 & 0 \\
                                                                                   0 & 1 & 0 & 0 & 0 & 0 & 0 & 0 \\
                                                                                   0 & 0 & 0 & 1 & 0 & 0 & 0 & 0 \\
                                                                                   0 & 0 & 1 & 0 & 0 & 0 & 0 & 0 \\
                                                                                   0 & 0 & 0 & 0 & 1 & 0 & 0 & 0 \\
                                                                                   0 & 0 & 0 & 0 & 0 & 1 & 0 & 0 \\
                                                                                   0 & 0 & 0 & 0 & 0 & 0 & 0 & 1 \\
                                                                                   0 & 0 & 0 & 0 & 0 & 0 & 1 & 0
                                                                                 \end{pmatrix}                       \\
  (G_{CNOT}\otimes I_2)                                                       & =\begin{pmatrix}
                                                                                   1 & 0 & 0 & 0 & 0 & 0 & 0 & 0 \\
                                                                                   0 & 1 & 0 & 0 & 0 & 0 & 0 & 0 \\
                                                                                   0 & 0 & 1 & 0 & 0 & 0 & 0 & 0 \\
                                                                                   0 & 0 & 0 & 1 & 0 & 0 & 0 & 0 \\
                                                                                   0 & 0 & 0 & 0 & 0 & 0 & 1 & 0 \\
                                                                                   0 & 0 & 0 & 0 & 0 & 0 & 0 & 1 \\
                                                                                   0 & 0 & 0 & 0 & 1 & 0 & 0 & 0 \\
                                                                                   0 & 0 & 0 & 0 & 0 & 1 & 0 & 0
                                                                                 \end{pmatrix}                       \\
  (I_2 \otimes H \otimes I_2)                                                 & =\frac{1}{\sqrt{2}}\begin{pmatrix}
                                                                                                     1 & 0 & 1  & 0  & 0 & 0 & 0  & 0  \\
                                                                                                     0 & 1 & 0  & 1  & 0 & 0 & 0  & 0  \\
                                                                                                     1 & 0 & -1 & 0  & 0 & 0 & 0  & 0  \\
                                                                                                     0 & 1 & 0  & -1 & 0 & 0 & 0  & 0  \\
                                                                                                     0 & 0 & 0  & 0  & 1 & 0 & 1  & 0  \\
                                                                                                     0 & 0 & 0  & 0  & 0 & 1 & 0  & 1  \\
                                                                                                     0 & 0 & 0  & 0  & 1 & 0 & -1 & 0  \\
                                                                                                     0 & 0 & 0  & 0  & 0 & 1 & 0  & -1
                                                                                                   \end{pmatrix}
\end{align*}
Then we multiply these together with $\cdot$ representing standard matrix multiplication,
\begin{align*}
  G_{TOTAL}=\frac{1}{4} & \begin{pmatrix}
                            1 & 0 & 0 & 0 & 1  & 0  & 0  & 0  \\
                            0 & 1 & 0 & 0 & 0  & 1  & 0  & 0  \\
                            0 & 0 & 1 & 0 & 0  & 0  & 1  & 0  \\
                            0 & 0 & 0 & 1 & 0  & 0  & 0  & 1  \\
                            1 & 0 & 0 & 0 & -1 & 0  & 0  & 0  \\
                            0 & 1 & 0 & 0 & 0  & -1 & 0  & 0  \\
                            0 & 0 & 1 & 0 & 0  & 0  & -1 & 0  \\
                            0 & 0 & 0 & 1 & 0  & 0  & 0  & -1
                          \end{pmatrix}\cdot
  \begin{pmatrix}
    1 & 0 & 0 & 0 & 0 & 0 & 0 & 0 \\
    0 & 1 & 0 & 0 & 0 & 0 & 0 & 0 \\
    0 & 0 & 1 & 0 & 0 & 0 & 0 & 0 \\
    0 & 0 & 0 & 1 & 0 & 0 & 0 & 0 \\
    0 & 0 & 0 & 0 & 0 & 1 & 0 & 0 \\
    0 & 0 & 0 & 0 & 1 & 0 & 0 & 0 \\
    0 & 0 & 0 & 0 & 0 & 0 & 0 & 1 \\
    0 & 0 & 0 & 0 & 0 & 0 & 1 & 0
  \end{pmatrix}\cdot
  \begin{pmatrix}
    1 & 0 & 0 & 0 & 1  & 0  & 0  & 0  \\
    0 & 1 & 0 & 0 & 0  & 1  & 0  & 0  \\
    0 & 0 & 1 & 0 & 0  & 0  & 1  & 0  \\
    0 & 0 & 0 & 1 & 0  & 0  & 0  & 1  \\
    1 & 0 & 0 & 0 & -1 & 0  & 0  & 0  \\
    0 & 1 & 0 & 0 & 0  & -1 & 0  & 0  \\
    0 & 0 & 1 & 0 & 0  & 0  & -1 & 0  \\
    0 & 0 & 0 & 1 & 0  & 0  & 0  & -1
  \end{pmatrix}\cdot                                          \\
                        & \begin{pmatrix}
                            1 & 0 & 0 & 0 & 0 & 0 & 0 & 0 \\
                            0 & 1 & 0 & 0 & 0 & 0 & 0 & 0 \\
                            0 & 0 & 1 & 0 & 0 & 0 & 0 & 0 \\
                            0 & 0 & 0 & 1 & 0 & 0 & 0 & 0 \\
                            0 & 0 & 0 & 0 & 0 & 0 & 1 & 0 \\
                            0 & 0 & 0 & 0 & 0 & 0 & 0 & 1 \\
                            0 & 0 & 0 & 0 & 1 & 0 & 0 & 0 \\
                            0 & 0 & 0 & 0 & 0 & 1 & 0 & 0
                          \end{pmatrix}\cdot
  \begin{pmatrix}
    1 & 1  & 0 & 0  & 0 & 0  & 0 & 0  \\
    1 & -1 & 0 & 0  & 0 & 0  & 0 & 0  \\
    0 & 0  & 1 & 1  & 0 & 0  & 0 & 0  \\
    0 & 0  & 1 & -1 & 0 & 0  & 0 & 0  \\
    0 & 0  & 0 & 0  & 1 & 1  & 0 & 0  \\
    0 & 0  & 0 & 0  & 1 & -1 & 0 & 0  \\
    0 & 0  & 0 & 0  & 0 & 0  & 1 & 1  \\
    0 & 0  & 0 & 0  & 0 & 0  & 1 & -1
  \end{pmatrix}\cdot
  \begin{pmatrix}
    1 & 0 & 0 & 0 & 0 & 0 & 0 & 0 \\
    0 & 1 & 0 & 0 & 0 & 0 & 0 & 0 \\
    0 & 0 & 0 & 1 & 0 & 0 & 0 & 0 \\
    0 & 0 & 1 & 0 & 0 & 0 & 0 & 0 \\
    0 & 0 & 0 & 0 & 1 & 0 & 0 & 0 \\
    0 & 0 & 0 & 0 & 0 & 1 & 0 & 0 \\
    0 & 0 & 0 & 0 & 0 & 0 & 0 & 1 \\
    0 & 0 & 0 & 0 & 0 & 0 & 1 & 0
  \end{pmatrix}\cdot                                              \\
                        & \begin{pmatrix}
                            1 & 0 & 0 & 0 & 0 & 0 & 0 & 0 \\
                            0 & 1 & 0 & 0 & 0 & 0 & 0 & 0 \\
                            0 & 0 & 1 & 0 & 0 & 0 & 0 & 0 \\
                            0 & 0 & 0 & 1 & 0 & 0 & 0 & 0 \\
                            0 & 0 & 0 & 0 & 0 & 0 & 1 & 0 \\
                            0 & 0 & 0 & 0 & 0 & 0 & 0 & 1 \\
                            0 & 0 & 0 & 0 & 1 & 0 & 0 & 0 \\
                            0 & 0 & 0 & 0 & 0 & 1 & 0 & 0
                          \end{pmatrix}\cdot
  \begin{pmatrix}
    1 & 0 & 1  & 0  & 0 & 0 & 0  & 0  \\
    0 & 1 & 0  & 1  & 0 & 0 & 0  & 0  \\
    1 & 0 & -1 & 0  & 0 & 0 & 0  & 0  \\
    0 & 1 & 0  & -1 & 0 & 0 & 0  & 0  \\
    0 & 0 & 0  & 0  & 1 & 0 & 1  & 0  \\
    0 & 0 & 0  & 0  & 0 & 1 & 0  & 1  \\
    0 & 0 & 0  & 0  & 1 & 0 & -1 & 0  \\
    0 & 0 & 0  & 0  & 0 & 1 & 0  & -1
  \end{pmatrix}                                          \\
                        & = \frac{1}{2}\begin{pmatrix}
                                         1  & 0  & 1  & 0  & 1  & 0  & 1  & 0  \\
                                         1  & 0  & 1  & 0  & -1 & 0  & -1 & 0  \\
                                         0  & 1  & 0  & -1 & 0  & 1  & 0  & -1 \\
                                         0  & 1  & 0  & -1 & 0  & -1 & 0  & 1  \\
                                         0  & 1  & 0  & 1  & 0  & 1  & 0  & 1  \\
                                         0  & -1 & 0  & -1 & 0  & 1  & 0  & 1  \\
                                         1  & 0  & -1 & 0  & 1  & 0  & -1 & 0  \\
                                         -1 & 0  & 1  & 0  & 1  & 0  & -1 & 0
                                       \end{pmatrix}
\end{align*}
Then we construct our input vector $\ket{a}\otimes \ket{b}\otimes \ket{c}=\ket{00}\otimes\begin{pmatrix}
    c_1 \\
    c_2
  \end{pmatrix}
  =\begin{pmatrix}
    c_1 \\
    c_2 \\
    0   \\
    0   \\
    0   \\
    0   \\
    0   \\
    0
  \end{pmatrix}$\\ and multiply,
$\frac{1}{2}\begin{pmatrix}
    1  & 0  & 1  & 0  & 1  & 0  & 1  & 0  \\
    1  & 0  & 1  & 0  & -1 & 0  & -1 & 0  \\
    0  & 1  & 0  & -1 & 0  & 1  & 0  & -1 \\
    0  & 1  & 0  & -1 & 0  & -1 & 0  & 1  \\
    0  & 1  & 0  & 1  & 0  & 1  & 0  & 1  \\
    0  & -1 & 0  & -1 & 0  & 1  & 0  & 1  \\
    1  & 0  & -1 & 0  & 1  & 0  & -1 & 0  \\
    -1 & 0  & 1  & 0  & 1  & 0  & -1 & 0
  \end{pmatrix}\begin{pmatrix}
    c_1 \\
    c_2 \\
    0   \\
    0   \\
    0   \\
    0   \\
    0   \\
    0
  \end{pmatrix}=
  \frac{1}{2}\begin{pmatrix}c_1  \\
    c_1  \\
    c_2  \\
    c_2  \\
    c_2  \\
    -c_2 \\
    c_1  \\
    -c_1
  \end{pmatrix}$



\end{document}