\documentclass[10pt]{article}

\usepackage[margin=0.75in]{geometry}
\usepackage{amsmath,amsthm,amssymb}
\usepackage{xcolor}
\usepackage{cancel}
\usepackage{graphicx}
\usepackage{changepage}
\usepackage{circuitikz}
\usepackage{pgfplots}
\usepackage{physics}
\usepackage{hyperref}
\usepackage{siunitx}
\usepackage[breakable]{tcolorbox}
\usepackage[inline]{enumitem}

\theoremstyle{definition}
\newtheorem{problem}{Problem}
\newtheorem{soln}{Solution}

\pgfplotsset{compat=newest}
\usetikzlibrary{arrows, angles, calc, quotes}

\definecolor{incolor}{HTML}{303F9F}
\definecolor{outcolor}{HTML}{D84315}
\definecolor{cellborder}{HTML}{CFCFCF}
\definecolor{cellbackground}{HTML}{F7F7F7}
\newcommand{\eq}{=}
\usetikzlibrary{positioning, fit, calc}
\pgfdeclarelayer{background}  
\pgfsetlayers{background,main}
\DeclareSIUnit[number-unit-product = {\,}]\calorie{cal}
\DeclareSIUnit[number-unit-product = {\,}]\atmosphere{atm}
\AtBeginDocument{\RenewCommandCopy\qty\SI}

\makeatletter
\newcommand{\boxspacing}{\kern\kvtcb@left@rule\kern\kvtcb@boxsep}
\makeatother
\newcommand{\prompt}[4]{
    \ttfamily\llap{{\color{#2}[#3]:\hspace{3pt}#4}}\vspace{-\baselineskip}
}

\newcommand{\thevenin}[2]{
  \begin{center}
    \begin{circuitikz} \draw
      (0,0) -- (2,0) to[battery1, l_=$V_{Th}\eq#1$] (2,2) 
      to[resistor, l_=$R_{Th}\eq#2$] (0,2)
      ;
      \draw [o-] (-.07,2.079);
      \draw [o-] (-.07,0.079);
    \end{circuitikz}
  \end{center}
}

\newcommand{\norton}[2]{
  \begin{center}
    \begin{circuitikz} \draw
      (0,0) -- (3,0) to[american current source, l_=$I_{N}\eq#1$] (3,2) -- (0,2) (2,0)
      to[resistor, l=$R_{N}\eq#2$] (2,2)
      ;
      \draw [o-] (-.07,2.079);
      \draw [o-] (-.07,0.079);
    \end{circuitikz}
  \end{center}
}

\newcommand{\highlight}[1]{\colorbox{yellow}{$\displaystyle #1$}}

\newcommand{\ti}[1]{\widetilde{#1}}

\NewDocumentCommand{\evalat}{sO{\big}mm}{%
  \IfBooleanTF{#1}
   {\mleft. #3 \mright|_{#4}}
   {#3#2|_{#4}}%
}

\title{Physics 2605H: Assignment II}
\author{Jeremy Favro \\ Trent University, Peterborough, ON, Canada}
\date{\today}

\begin{document}
\maketitle

% PROBLEM 1
\begin{problem} Two vectors are represented as
$$
  \ket{\psi}=\begin{pmatrix}
    3+4i \\
    4+3i
  \end{pmatrix},\qquad
  \ket{\phi}=\begin{pmatrix}
    2+i \\
    1+2i
  \end{pmatrix}
$$
\begin{enumerate}[label=(\alph*)]
  \item Find their scalar product, $\braket{\phi}{\psi}$
  \item Find the products \begin{enumerate}[label=(\roman*)]
          \item $\ket{\phi}\bra{\psi}$
          \item $\ket{\psi}\bra{\phi}$
        \end{enumerate}
  \item Do (i) and (ii) commute?
  \item Find \begin{enumerate} [label=(\roman*)]
          \item The inner product of $\ket{\psi}$
          \item The inner product of $\ket{\phi}$
          \item The outer product of $\ket{\phi}$
          \item The outer product of $\ket{\psi}$
        \end{enumerate}
\end{enumerate}
\end{problem}
\begin{soln} ~
  \begin{enumerate}[label=(\alph*)]
    \item \begin{align*}
            \braket{\phi}{\psi} & = \braket{\phi}{\psi} = \ket{\phi}^\dagger \ket{\psi} \\
                                & = \begin{pmatrix}
                                      2-i & 1-2i
                                    \end{pmatrix}
            \begin{pmatrix}
              3+4i \\
              4+3i
            \end{pmatrix}                                                              \\
                                & = 20
          \end{align*}
    \item Find the products \begin{enumerate}[label=(\roman*)]
            \item $\bra{\psi}=\ket{\psi}^\dagger=\begin{pmatrix}
                      3-4i & 4-3i
                    \end{pmatrix}$ So,
                  \begin{align*}
                    \ket{\phi}\bra{\psi} & =
                    \begin{pmatrix}
                      2+i \\
                      1+2i
                    \end{pmatrix}
                    \begin{pmatrix}
                      3-4i & 4-3i
                    \end{pmatrix}              \\
                                         & =
                    \begin{pmatrix}
                      (2+i)(3-4i)  & (2+i)(4-3i)  \\
                      (1+2i)(3-4i) & (1+2i)(4-3i) \\
                    \end{pmatrix} \\
                                         & =
                    \begin{pmatrix}
                      10-5i & 11-2i \\
                      11+2i & 10+5i \\
                    \end{pmatrix}
                  \end{align*}
            \item \begin{align*}
                    \ket{\psi}\bra{\phi} & =
                    \begin{pmatrix}
                      3+4i \\
                      4+3i
                    \end{pmatrix}
                    \begin{pmatrix}
                      2-i & 1-2i
                    \end{pmatrix}             \\
                                         & =
                    \begin{pmatrix}
                      (2-i)(3+4i) & (1-2i)(4+3i) \\
                      (2-i)(3+4i) & (1-2i)(4+3i) \\
                    \end{pmatrix} \\
                                         & =
                    \begin{pmatrix}
                      10+5i & 11-2i \\
                      11+2i & 10-5i \\
                    \end{pmatrix}
                  \end{align*}
          \end{enumerate}
    \item I'm not quite sure what this is asking as this isn't an area where commutativity can be defined. They aren't equal but they also aren't the commuted
          forms of each other.
    \item Find \begin{enumerate} [label=(\roman*)]
            \item \begin{align*}
                    \braket{\psi}{\psi} & =
                    \begin{pmatrix}
                      3-4i & 4-3i
                    \end{pmatrix}
                    \begin{pmatrix}
                      3+4i \\
                      4+3i
                    \end{pmatrix}             \\
                                        & = 50
                  \end{align*}
            \item \begin{align*}
                    \braket{\phi}{\phi} & =
                    \begin{pmatrix}
                      2-i & 1-2i
                    \end{pmatrix}
                    \begin{pmatrix}
                      2+i \\
                      1+2i
                    \end{pmatrix}             \\
                                        & = 10
                  \end{align*}
            \item \begin{align*}
                    \ket{\phi}\bra{\phi} & =
                    \begin{pmatrix}
                      2+i \\
                      1+2i
                    \end{pmatrix}
                    \begin{pmatrix}
                      2-i & 1-2i
                    \end{pmatrix}                           \\
                                         & =
                    \begin{pmatrix}
                      \left|2+i\right|^2 & (2+i)(1-2i)         \\
                      (2-i)(1+2i)        & \left|1+2i\right|^2
                    \end{pmatrix} \\
                                         & =
                    \begin{pmatrix}
                      5    & 4-3i \\
                      4+3i & 5
                    \end{pmatrix}
                  \end{align*}
            \item \begin{align*}
                    \ket{\psi}\bra{\psi} & =
                    \begin{pmatrix}
                      3+4i \\
                      4+3i
                    \end{pmatrix}
                    \begin{pmatrix}
                      3-4i & 4-3i
                    \end{pmatrix}                            \\
                                         & =
                    \begin{pmatrix}
                      \left|3+4i\right|^2 & (3+4i)(4-3i)        \\
                      (3-4i)(4+3i)        & \left|4+3i\right|^2
                    \end{pmatrix} \\
                                         & =
                    \begin{pmatrix}
                      25    & 24+7i \\
                      24-7i & 25
                    \end{pmatrix}
                  \end{align*}
          \end{enumerate}
  \end{enumerate}
\end{soln}
\newpage

% PROBLEM 2
\begin{problem}~
\begin{enumerate}[label=(\alph*)]
  \item Using the results from 1(d), normalize $\ket{\psi}$ and $\ket{\phi}$ so that they belong to Hilbert space. Here write about what is Hilbert space.
  \item Identify which of the following belong to Hilbert space and state why?
        \begin{enumerate}[label=(\roman*)]
          \item $\ket{\psi}=\frac{1}{\sqrt{3}}\ket{0}+\sqrt{\frac{2}{3}}\ket{1}$
          \item $\ket{\psi}=\frac{4}{5}\ket{+}+\frac{3}{5}\ket{-}$
        \end{enumerate}
\end{enumerate}
\end{problem}
\begin{soln}~
  \begin{enumerate}[label=(\alph*)]
    \item Because a Hilbert space is a collection of orthonormal vectors (and these vectors are already orthogonal) we need to multiply them by the inverse of their inner product.
          So $\ket{\psi}_H=\frac{1}{5\sqrt{2}}\begin{pmatrix}
              3+4i \\
              4+3i
            \end{pmatrix}$ and $\ket{\phi}_H=\frac{1}{\sqrt{10}}\begin{pmatrix}
              2+i \\
              1+2i
            \end{pmatrix}$
    \item Identify which of the following belong to Hilbert space and state why?
          \begin{enumerate}[label=(\roman*)]
            \item Because this is already in an orthonormal basis we only have to care about the magnitude of the vector which is (by inspection ;)) 1.
                  So this vector belongs to a vector space.
            \item Again this is an orthonormal basis so all we care about is if $\braket{\phi}{\phi}=1$,
                  \begin{align*}
                    \braket{\phi}{\phi} & =
                    \frac{1}{2}
                    \begin{pmatrix}
                      \frac{7}{5} & \frac{1}{5}
                    \end{pmatrix}
                    \begin{pmatrix}
                      \frac{7}{5} \\
                      \frac{1}{5}
                    \end{pmatrix}            \\
                                        & = 1
                  \end{align*}
          \end{enumerate}
          So this vector is also part of a Hilbert space.
  \end{enumerate}
\end{soln}

% PROBLEM 3
\begin{problem}~
\begin{enumerate}[label=(\alph*)]
  \item Would $\ket{0}$ and $\ket{+}$ together satisfy the criteria for a valid basis? Why?
  \item Review Stern and Gerlach Apparatus measurements from worksheet \#2. To measure
        the difference between an electron in a spin state $\ket{+}$ and $\ket{-}$, which of the following
        one could use? Explain.
        \begin{enumerate}[label=(\Roman*)]
          \item A horizontal SGA.
          \item A vertical SGA.
          \item A $\qty{45}{\degree}$ diagonal SGA.
        \end{enumerate}
\end{enumerate}
\end{problem}
\begin{soln} ~
  \begin{enumerate}[label=(\alph*)]
    \item Yes because they are both linearly independent (by inspection) and a spanning set as the matrix
          $$\begin{pmatrix}
              1 & \frac{1}{\sqrt{2}} \\
              0 & \frac{1}{\sqrt{2}}
            \end{pmatrix}$$
          has dimension 2 which implies that the vectors span $\mathbb{R}^2$.
    \item Because spin states are presented in basis $\ket{\uparrow}=\frac{1}{\sqrt{2}}\left(\ket{+}+\ket{-}\right)$ and
          $\ket{\downarrow}=\frac{1}{\sqrt{2}}\left(\ket{+}-\ket{-}\right)$ changing to basis $\ket{+}$ and $\ket{-}$ means we would need an angle such that
          $\sin(\theta)=\cos(\theta)=\frac{1}{\sqrt{2}}\implies\theta=\frac{\pi}{4}\unit{\radian}=\qty{45}{\degree}$.
  \end{enumerate}
\end{soln}
\newpage

% PROBLEM 4
\begin{problem}~
\begin{enumerate}[label=(\alph*)]
  \item Using the link given, obtain the mathematical form of a qubit with $\theta=\frac{2\pi}{3}$ and $\phi\frac{3\pi}{2}$
  \item Also find the corresponding orthonormal state for the state of qubit in (a).
  \item Show that the inner products of the two states lead to 1.
  \item Find their outer product.
\end{enumerate}
\end{problem}
\begin{soln} ~
  \begin{enumerate}[label=(\alph*)]
    \item $\ket{\psi}=\cos\left(\frac{\pi}{3}\right)\ket{\uparrow}+e^{i\frac{3\pi}{2}}\sin\left(\frac{\pi}{3}\right)\ket{\downarrow}=
            \frac{1}{2}\ket{\uparrow}-\frac{1}{2}\ket{\downarrow}$
    \item $\ket{\psi}=\begin{pmatrix}
              \frac{1}{\sqrt{2}} \\
              0
            \end{pmatrix}-\begin{pmatrix}
              0 \\
              \frac{1}{\sqrt{2}}
            \end{pmatrix}$ so the vector is already orthonormal (as expected as we used the Bloch sphere to create it).
    \item $\left[\begin{pmatrix}
                \frac{1}{\sqrt{2}} & 0
              \end{pmatrix}-\begin{pmatrix}
                0 & \frac{1}{\sqrt{2}}
              \end{pmatrix}\right]
            \left[\begin{pmatrix}
                \frac{1}{\sqrt{2}} \\
                0
              \end{pmatrix}-\begin{pmatrix}
                0 \\
                \frac{1}{\sqrt{2}}
              \end{pmatrix}\right]=\frac{1}{2}+\frac{1}{2}=1$
    \item $
            \ket{\psi}\bra{\psi}  = \frac{1}{2}
            \begin{pmatrix}
              1 \\
              -1
            \end{pmatrix}
            \begin{pmatrix}
              1 & -1
            \end{pmatrix}
            = \frac{1}{2}\begin{pmatrix}
              1  & -1 \\
              -1 & 1
            \end{pmatrix}
          $
  \end{enumerate}
\end{soln}

% PROBLEM 5
\begin{problem}~
\begin{center}
  \begin{circuitikz}
    \draw node[ieeestd xor port](T){}
    (T.in 1) node[left]{$x$}
    (T.in 2) node[left]{$y$}
    (T.out) node[right]{$o$};
  \end{circuitikz}
\end{center}
\begin{enumerate}[label=(\alph*)]
  \item The figure shows two input XOR gate. Tabulate the outputs when, the inputs or (0,1), (1,0), (1,1), and (0,1) \textbf{assuming typo}
  \item Compare the functionality of XOR gate with AND gate and OR gate.
  \item Which gate is called as the Universal gate and why?
\end{enumerate}
\end{problem}
\begin{soln} ~
  \begin{enumerate}[label=(\alph*)]
    \item\begin{displaymath}
      \begin{array}{|c c|c|}
        x & y & o \\
        \hline
        0 & 1 & 1 \\
        1 & 0 & 1 \\
        1 & 1 & 0 \\
        0 & 0 & 0
      \end{array}
    \end{displaymath}
    \item Exclusive or (XOR) is either but not both, AND is both, OR is either.
          \begin{displaymath}
            \begin{array}{|c c|c|c|c|}
              x & y & XOR & AND & OR \\
              \hline
              0 & 1 & 1   & 0   & 1  \\
              1 & 0 & 1   & 0   & 1  \\
              1 & 1 & 0   & 1   & 1  \\
              0 & 0 & 0   & 0   & 0
            \end{array}
          \end{displaymath}
    \item NAND and NOR gates are sometimes called ``universal'' gates as any other gate can be constructed from some number of either.
  \end{enumerate}
\end{soln}
\newpage

% PROBLEM 6
\begin{problem}~
\begin{enumerate}[label=(\alph*)]
  \item Use matrix multiplication to show how applying an $\hat{X}$ gate flips:
        \begin{enumerate}[label=(\roman*)]
          \item A qubit in the $\ket{0}$ state
          \item A qubit in the general state $\ket{\psi}=\alpha\ket{0}+\beta\ket{1}$
        \end{enumerate}
  \item Use matrix multiplication to show how applying the $\hat{Z}$ gate to $\ket{+}$ changes it to
        $\ket{-}$
  \item Show that Hadamard gate is unitary and hence reversible.
\end{enumerate}
\end{problem}
\begin{soln}~
  \begin{enumerate}[label=(\alph*)]
    \item Use matrix multiplication to show how applying an X gate flips:
          \begin{enumerate}[label=(\roman*)]
            \item \begin{align*}
                    \hat{X}\ket{0} & =
                    \begin{pmatrix}
                      0 & 1 \\
                      1 & 0
                    \end{pmatrix}
                    \begin{pmatrix}
                      1 \\
                      0
                    \end{pmatrix}             \\
                                   & =
                    \begin{pmatrix}
                      (1)(0)+(1)(0) \\
                      (1)(1)+(0)(0)
                    \end{pmatrix}
                                   & = \ket{1}
                  \end{align*}
            \item \begin{align*}
                    \hat{X}\ket{\psi} & = \hat{X}\alpha\ket{0}+\hat{X}\beta\ket{1}                                                                        \\
                                      & = \alpha\hat{X}\ket{0}+\beta\hat{X}\ket{1} \rightsquigarrow \text{ skipping as I'm in a rush and this is trivial} \\
                                      & = \alpha\ket{1}+\beta\ket{0}
                  \end{align*}
          \end{enumerate}
    \item \begin{align*}
            \hat{Z}\ket{+} & =
            \frac{1}{\sqrt{2}}
            \begin{pmatrix}
              1 & 0  \\
              0 & -1
            \end{pmatrix}
            \begin{pmatrix}
              1 \\
              1
            \end{pmatrix}             \\
                           & =
            \frac{1}{\sqrt{2}}
            \begin{pmatrix}
              (1)(1)+(1)(0) \\
              (1)(0)+(1)(-1)
            \end{pmatrix}             \\
                           & = \ket{-}
          \end{align*}
    \item \begin{align*}
            \hat{H}\hat{H} & =
            \frac{1}{2}
            \begin{pmatrix}
              1 & 1  \\
              1 & -1
            \end{pmatrix}
            \begin{pmatrix}
              1 & 1  \\
              1 & -1
            \end{pmatrix}                    \\
                           & =
            \frac{1}{2}
            \begin{pmatrix}
              (1)(1)+(1)(1)  & (1)(1) +(1)(-1)  \\
              (1)(1)+(-1)(1) & (1)(1) +(-1)(-1)
            \end{pmatrix} \\
                           & =
            \begin{pmatrix}
              1 & 0 \\
              0 & 1
            \end{pmatrix}                    \\
                           & = I
          \end{align*}
  \end{enumerate}
\end{soln}

% PROBLEM 7
\begin{problem}
$\ket{\psi}=\frac{4}{5}\ket{+}+\frac{3}{5}\ket{-}$ is measured in basis $\left\{\ket{0},\ket{1}\right\}$
\begin{enumerate}[label=(\alph*)]
  \item Perform the necessary basis change for the given state.
  \item What is the probability that the particle is in $\ket{1}$ state?
\end{enumerate}
\end{problem}
\begin{soln}
  $\ket{\psi}=\frac{4}{5}\ket{+}+\frac{3}{5}\ket{-}$ is measured in basis $\left\{\ket{0},\ket{1}\right\}$
  \begin{enumerate}[label=(\alph*)]
    \item \begin{align*}
             & = \frac{4}{5}\frac{1}{\sqrt{2}}\left(\ket{0}+\ket{1}\right)+\frac{3}{5}\frac{1}{\sqrt{2}}\left(\ket{0}-\ket{1}\right) \\
             & = \frac{7\sqrt{2}}{10}\ket{0}+\frac{\sqrt{2}}{10}\ket{1}
          \end{align*}
    \item $P(\ket{1})=\left(\frac{\sqrt{2}}{10}\right)^2=\frac{1}{50}=0.02$
  \end{enumerate}
\end{soln}
\end{document}