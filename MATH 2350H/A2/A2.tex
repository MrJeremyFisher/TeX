\documentclass[10pt]{article}

\usepackage[margin=0.75in]{geometry}
\usepackage{amsmath,amsthm,amssymb}
\usepackage{xcolor}
\usepackage{cancel}
\usepackage{graphicx}
\usepackage{changepage}
\usepackage{circuitikz}
\usepackage{pgfplots}
\usepackage{physics}
\usepackage{hyperref}
\usepackage{siunitx}
\usepackage[breakable]{tcolorbox}
\usepackage[inline]{enumitem}

\theoremstyle{definition}
\newtheorem{problem}{Problem}
\newtheorem{soln}{Solution}

\pgfplotsset{compat=newest}
\usetikzlibrary{lindenmayersystems}
\usetikzlibrary{arrows}
\usetikzlibrary{calc}

\definecolor{incolor}{HTML}{303F9F}
\definecolor{outcolor}{HTML}{D84315}
\definecolor{cellborder}{HTML}{CFCFCF}
\definecolor{cellbackground}{HTML}{F7F7F7}
\newcommand{\eq}{=}
\usetikzlibrary{positioning, fit, calc}
\pgfdeclarelayer{background}  
\pgfsetlayers{background,main}
\DeclareSIUnit[number-unit-product = {\,}]\calorie{cal}
\DeclareSIUnit[number-unit-product = {\,}]\atmosphere{atm}
\AtBeginDocument{\RenewCommandCopy\qty\SI}

\makeatletter
\newcommand{\boxspacing}{\kern\kvtcb@left@rule\kern\kvtcb@boxsep}
\makeatother
\newcommand{\prompt}[4]{
    \ttfamily\llap{{\color{#2}[#3]:\hspace{3pt}#4}}\vspace{-\baselineskip}
}

\newcommand{\thevenin}[2]{
  \begin{center}
    \begin{circuitikz} \draw
      (0,0) -- (2,0) to[battery1, l_=$V_{Th}\eq#1$] (2,2) 
      to[resistor, l_=$R_{Th}\eq#2$] (0,2)
      ;
      \draw [o-] (-.07,2.079);
      \draw [o-] (-.07,0.079);
    \end{circuitikz}
  \end{center}
}

\newcommand{\norton}[2]{
  \begin{center}
    \begin{circuitikz} \draw
      (0,0) -- (3,0) to[american current source, l_=$I_{N}\eq#1$] (3,2) -- (0,2) (2,0)
      to[resistor, l=$R_{N}\eq#2$] (2,2)
      ;
      \draw [o-] (-.07,2.079);
      \draw [o-] (-.07,0.079);
    \end{circuitikz}
  \end{center}
}

\newcommand{\highlight}[1]{\colorbox{yellow}{$\displaystyle #1$}}

\newcommand{\ti}[1]{\widetilde{#1}}

\NewDocumentCommand{\evalat}{sO{\big}mm}{%
  \IfBooleanTF{#1}
   {\mleft. #3 \mright|_{#4}}
   {#3#2|_{#4}}%
}

\title{Math 2350H: Assignment II}
\author{Jeremy Favro (0805980) \\ Trent University, Peterborough, ON, Canada}
\date{\today}

\begin{document}
\maketitle

% PROBLEM 1
\begin{problem}
Let $T\in\mathcal{L}(V)$. Prove that $T^2=T_0$ iff $\mathrm{range}\,T\subseteq\mathrm{null}\,T$.
\end{problem}
\begin{soln}
  Assuming
  \begin{align*}
             & \mathrm{range}\,T\subseteq\mathrm{null}\,T \\
    \implies & T=T_0                                      \\
    \implies & T(v\in V)=0_V                              \\
    \implies & T(T(v))=T^2=0_V\, (T:V\mapsto V)               \\
    \implies & T^2=T_0
  \end{align*}
  In the other direction assuming
  \begin{align*}
             & T^2=T_0          \\
    \implies & T(T(v\in V))=0_V \\
    \implies & T(v^\prime \in V)=0_V \, (T:V\mapsto V)\\
    \implies & \mathrm{range}\,T\subseteq\mathrm{null}\,T
  \end{align*}
  $\therefore T^2=T_0\iff \mathrm{range}\,T\subseteq\mathrm{null}\,T$
\end{soln}

% PROBLEM 2
\begin{problem}
Let $U$, $V$, $W$ be vector spaces over field $F$, and let $S\in\mathcal{L}(U,V)$ and $T\in\mathcal{L}(V,W)$.
\begin{enumerate}[label=(\alph*)]
  \item Show that if $T\circ S$ is injective, then $S$ is injective.
  \item Give an example showing that if $T\circ S$ is injective then $T$ need not be injective.
  \item Show that if $T\circ S$ is surjective, then $T$ is surjective.
  \item Give an example showing that if $T\circ S$ is surjective then $S$ need not be surjective.
\end{enumerate}
\end{problem}
\begin{soln}

\end{soln}

% PROBLEM 3
\begin{problem}
Let $x_0,x_1,\dots,x_n$ be $n+1$ distinct real numbers. For $i=0,\dots,n$, the set of polynomials $p_0,\dots,p_n$, defined by
$$p_i(x)=\frac{(x-x_0)(x-x_1)\dots(x-x_{i-1})(x-x_{i+1})\dots(x-x_n)}{(x_i-x_0)(x_i-x_1)\dots(x_i-x_{i-1})(x_i-x_{i+1})\dots(x_i-x_n)}$$
are called the \textit{Lagrange polynomials} associated to $x_0,x_1,\dots,x_n$.
\begin{enumerate}[label=(\alph*)]
  \item Find the Lagrange polynomials associated to $-5,-1,0,3$.
  \item Notice that $p_i(x_j)=0$ for $i\neq j$ and $p_i(x_i)=1$ (verify for yourself). Use this to prove that the set of Lagrange polynomials
        $p_0,\dots,p_n$ associated to $x_0,x_1,\dots,x_n$, is a basis for $\mathcal{P}_n(\mathbb{R})$.
  \item Let $(x_0, y_0),\dots,(x_n, y_n)$ be $n+1$ points in the plane $\mathbb{R}^2$, where $x_0,x_1,\dots,x_n$ are distinct.
        Verify that the polynomial $$f(x)=y_0p_0(x)+\dots+y_np_n(x)$$ is the unique polynomial in  $\mathcal{P}_n(\mathbb{R})$
        which passes through these $n+1$ points.
  \item Use the Lagrange polynomials found in part (a) to find a polynomial of degree at most 3 passing through
        $$(-5, 4), (-1, 1), (0, 2), (3, -3).$$
        (show how you obtain your answer).
  \item Suppose $g\in \mathcal{P}_n(\mathbb{R})$ and $g(r_i) = 0$ for $n + 1$ distinct scalars $r_0,r_1,\dots,r_n$.
        Use the results above to show that $g$ is the zero polynomial. This shows that a nonzero polynomial of degree $n$ cannot have more than $n$ distinct roots.
\end{enumerate}
\end{problem}
\begin{soln}~
  \begin{enumerate}[label=(\alph*)]
    \item \begin{align*}
            p_0(x) & =\frac{(x+1)(x)(x-3)}{(-5+1)(-5)(-5-3)}=\frac{x^3-2x^2-3x}{-160} \\
            p_1(x) & =\frac{(x+5)(x)(x-3)}{(-1+1)(-1)(-1-3)}=\frac{x^3+2x^2-15x}{16}  \\
            p_2(x) & =\frac{(x+1)(x+5)(x-3)}{(5)(1)(-3)}=\frac{x^3+3x^2-13x}{-15}+1   \\
            p_3(x) & =\frac{(x+1)(x+5)(x)}{(3+5)(3+1)(3)}=\frac{x^3+6x^2+5x}{96}
          \end{align*}
    \item For a set to form a basis it must be both linearly independent and a spanning set. Proving linear independence here is sufficient to
          imply the set is a spanning set as it means we have a linearly independent set of dimension $n+1$ for $\mathcal{P}_n$. For a set to be linearly independent
          we have
          $$c_0p_0(x)+\dots c_np_n(x)=0\implies c_0=\dots=c_n=0$$
          If we evaluate the above expression for every $x_i$ $i\leq n$ we obtain $n$ expressions where all polynomials $p_j(x_i)=0$ for $i\neq j$ and a single polynomial
          $p_i(x_i)=1$. This single polynomial has coefficient $c_i$ which, to satisfy the previous expression for linear independence must be 0. So,
          applying this to each $x_i$ we obtain that all scalars $c_0,\dots c_n$ must be zero. This implies linear independence which as mentioned above implies
          span due the nature of polynomial spaces which then implies that the set of Lagrange polynomials generated by distinct real numbers $x_0,x_1,\dots,x_n$ is
          a basis for $\mathcal{P}_n(\mathbb{R})$.
    \item Again using the fact that $p_i(x_i)=1$ it is evident that $y_ip_i(x_i)=y_i$ and so because we
          are also guaranteed that $p_i(x_j)=0$ for $i\neq j$ $\sum_{i=0}^{n}y_ip_i(x_i)$ will be $y_i$ at all corresponding $x_i$ because only a single polynomial
          in the sum will be nonzero at that point.
    \item From part (c) we simply sum all of the polynomials found in part (a) and multiply them by the given $y$-coordinate
          \begin{align*}
             & =4\cdot\frac{x^3-2x^2-3x}{-160}+1\cdot\frac{x^3+2x^2-15x}{16}+2\cdot\left(\frac{x^3+3x^2-13x}{-15}+1\right)-3\cdot\frac{x^3+6x^2+5x}{96} \\
             & =\frac{-61x^3-198x^2+343x}{480}+2                                                                                                        \\
          \end{align*}
    \item Suppose $g\in \mathcal{P}_n(\mathbb{R})$ and $g(r_i) = 0$ for $n + 1$ distinct scalars $r_0,r_1,\dots,r_n$.
          Use the results above to show that $g$ is the zero polynomial. This shows that a nonzero polynomial of degree $n$ cannot have more than $n$ distinct roots.
  \end{enumerate}
\end{soln}
\end{document}