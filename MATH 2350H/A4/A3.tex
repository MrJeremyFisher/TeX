\documentclass[10pt]{article}

\usepackage[margin=0.75in]{geometry}
\usepackage{amsmath,amsthm,amssymb}
\usepackage{xcolor}
\usepackage{cancel}
\usepackage{graphicx}
\usepackage{changepage}
\usepackage{circuitikz}
\usepackage{pgfplots}
\usepackage{physics}
\usepackage{hyperref}
\usepackage{siunitx}
\usepackage[breakable]{tcolorbox}
\usepackage[inline]{enumitem}

\theoremstyle{definition}
\newtheorem{problem}{Problem}
\newtheorem{soln}{Solution}

\pgfplotsset{compat=newest}
\usetikzlibrary{lindenmayersystems}
\usetikzlibrary{arrows}
\usetikzlibrary{calc}

\definecolor{incolor}{HTML}{303F9F}
\definecolor{outcolor}{HTML}{D84315}
\definecolor{cellborder}{HTML}{CFCFCF}
\definecolor{cellbackground}{HTML}{F7F7F7}
\newcommand{\eq}{=}
\usetikzlibrary{positioning, fit, calc}
\pgfdeclarelayer{background}  
\pgfsetlayers{background,main}
\DeclareSIUnit[number-unit-product = {\,}]\calorie{cal}
\DeclareSIUnit[number-unit-product = {\,}]\atmosphere{atm}
\AtBeginDocument{\RenewCommandCopy\qty\SI}

\makeatletter
\newcommand{\boxspacing}{\kern\kvtcb@left@rule\kern\kvtcb@boxsep}
\makeatother
\newcommand{\prompt}[4]{
    \ttfamily\llap{{\color{#2}[#3]:\hspace{3pt}#4}}\vspace{-\baselineskip}
}

\newcommand{\thevenin}[2]{
  \begin{center}
    \begin{circuitikz} \draw
      (0,0) -- (2,0) to[battery1, l_=$V_{Th}\eq#1$] (2,2) 
      to[resistor, l_=$R_{Th}\eq#2$] (0,2)
      ;
      \draw [o-] (-.07,2.079);
      \draw [o-] (-.07,0.079);
    \end{circuitikz}
  \end{center}
}

\newcommand{\norton}[2]{
  \begin{center}
    \begin{circuitikz} \draw
      (0,0) -- (3,0) to[american current source, l_=$I_{N}\eq#1$] (3,2) -- (0,2) (2,0)
      to[resistor, l=$R_{N}\eq#2$] (2,2)
      ;
      \draw [o-] (-.07,2.079);
      \draw [o-] (-.07,0.079);
    \end{circuitikz}
  \end{center}
}

\newcommand{\highlight}[1]{\colorbox{yellow}{$\displaystyle #1$}}

\newcommand{\ti}[1]{\widetilde{#1}}

\NewDocumentCommand{\evalat}{sO{\big}mm}{%
  \IfBooleanTF{#1}
   {\mleft. #3 \mright|_{#4}}
   {#3#2|_{#4}}%
}

\title{Math 2350H: Assignment IV}
\author{Jeremy Favro (0805980) \\ Trent University, Peterborough, ON, Canada}
\date{\today}

\begin{document}
\maketitle

% PROBLEM 1 
\begin{problem}
Consider the subspace
$$U=\mathrm{span}\left(
  \left(2,-1,-2,4\right),
  \left(-2,1,-5,5\right),
  \left(-1,3,7,11\right)
  \right)$$
\begin{enumerate}[label=(\alph*)]
  \item Apply the Gram-Schmidt process (with normalization) to find an orthonormal basis of $U$.
  \item Find a basis for $U^\perp$
  \item Express the vector $v = (-11, 8, -4, 18)$ as $v = x + y$ where
        $x\in $ and $y \in U^\perp$.
\end{enumerate}
\end{problem}
\begin{soln}
\end{soln}

% PROBLEM 2
\begin{problem}
The dot product is defined on $\mathcal{M}_{n\cross 1}(\mathbb{R})$ and $\mathcal{M}_{n\cross 1}(\mathbb{C})$ just as it is
for $\mathbb{R}^n$ and $\mathbb{C}^n$. For $u,v\in\mathcal{M}_{n\cross 1}(\mathbb{C})$, with
$$u=\begin{pmatrix}
    u_1    \\
    u_2    \\
    \vdots \\
    u_n
  \end{pmatrix},\quad v=\begin{pmatrix}
    v_1    \\
    v_2    \\
    \vdots \\
    v_n
  \end{pmatrix}$$
the dot product (or standard inner product) is
$$\braket{u}{v}=\sum_{k=1}^{n}u_kv_k^*$$
This can be written more compactly with matrix multiplication as
$$\braket{u}{v}=v^\dagger u$$
where $v^\dagger=\left(v^T\right)^*$. We use this inner product below.
\begin{enumerate}[label=(\alph*)]
  \item Let $P\in\mathcal{M}_{n\cross n}(\mathbb{R})$. Relative to the dot product on column vectors, show that the following are equivalent:
        \begin{enumerate}[label=(\roman*)]
          \item The columns $u_1,\dots,u_n$ of $P$ form an orthonormal basis $\mathcal{M}_{n\cross 1}(\mathbb{R})$.
          \item $P^T=P^{-1}$.
          \item The rows of $P$ form an orthonormal basis for $\mathbb{R}^n$.
        \end{enumerate}
  \item A matrix $P\in\mathcal{M}_{n\cross n}(\mathbb{R})$ is called an \emph{orthogonal matrix} if $P^T=P^{-1}$. Determine which
        of the following matrices are orthogonal.
        \begin{enumerate}[label=(\roman*)]
          \item $\begin{pmatrix}
                    3/5  & 4/5  \\
                    -4/5 & -3/5
                  \end{pmatrix}$
          \item $\begin{pmatrix}
                    5/13  & 12/13 \\
                    12/13 & -5/13
                  \end{pmatrix}$
          \item $\begin{pmatrix}
                    1 & 0 \\
                    0 & 1
                  \end{pmatrix}$
          \item $\begin{pmatrix}
                    1/\sqrt{2} & -1/\sqrt{2} \\
                    1/\sqrt{2} & 1/\sqrt{2}
                  \end{pmatrix}$
          \item $\begin{pmatrix}
                    1/3 & 2/3  & -2/3 \\
                    1/3 & -2/3 & 1/3  \\
                    2/3 & 1/3  & 2/3
                  \end{pmatrix}$
          \item $\begin{pmatrix}
                    1/3 & 2/3  & 2/3  \\
                    2/3 & -2/3 & 1/3  \\
                    2/3 & 1/3  & -2/3
                  \end{pmatrix}$
        \end{enumerate}
  \item Let $P,Q\in\mathcal{M}_{n\cross n}(\mathbb{R})$ be orthogonal matrices and $u,v\in\mathcal{M}_{n\cross 1}(\mathbb{R})$.
        Prove that
        \begin{enumerate}[label=(\roman*)]
          \item $\braket{Pu}{Pv}=\braket{u}{v}$
          \item $\|Pu \Vert =\|u \Vert $
          \item $PQ$ is an orthogonal matrix.
        \end{enumerate}
\end{enumerate}
\end{problem}
\begin{soln}
\end{soln}

% PROBLEM 3
\begin{problem}
  Let $V$ be the vector space of continuous, real-valued functions defined
  on the interval $[0, 1]$. Then $V$ is an inner product space with inner
  product
  $$\braket{f}{g}=\int_{0}^{1}f(x)g(x)dx,$$
  for $f,g\in V$. Consider the subspace $U$ of $V$ spanned by the functions $f(x)=\sqrt{x}$, $g(x)=x$, $h(x)=x^2$.
  \begin{enumerate}[label=(\alph*)]
    \item Show that $f, g, h$ is linearly independent.
    \item Find an orthonormal basis for $U$
    \item Let $p(x)=x^3$. Find the closest approximation of $p$ in $U$.
  \end{enumerate}
\end{problem}
\begin{soln}
\end{soln}
\end{document}