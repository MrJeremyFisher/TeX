\documentclass[10pt]{article}

\usepackage[margin=0.75in]{geometry}
\usepackage{amsmath,amsthm,amssymb}
\usepackage{xcolor}
\usepackage{cancel}
\usepackage{graphicx}
\usepackage{changepage}
\usepackage{circuitikz}
\usepackage{pgfplots}
\usepackage{physics}
\usepackage{hyperref}
\usepackage{siunitx}
\usepackage[breakable]{tcolorbox}
\usepackage[inline]{enumitem}

\theoremstyle{definition}
\newtheorem{problem}{Problem}
\newtheorem{soln}{Solution}

\pgfplotsset{compat=newest}
\usetikzlibrary{lindenmayersystems}
\usetikzlibrary{arrows}
\usetikzlibrary{calc}

\definecolor{incolor}{HTML}{303F9F}
\definecolor{outcolor}{HTML}{D84315}
\definecolor{cellborder}{HTML}{CFCFCF}
\definecolor{cellbackground}{HTML}{F7F7F7}
\newcommand{\eq}{=}
\usetikzlibrary{positioning, fit, calc}
\pgfdeclarelayer{background}  
\pgfsetlayers{background,main}
\DeclareSIUnit[number-unit-product = {\,}]\calorie{cal}
\DeclareSIUnit[number-unit-product = {\,}]\atmosphere{atm}
\AtBeginDocument{\RenewCommandCopy\qty\SI}

\makeatletter
\newcommand{\boxspacing}{\kern\kvtcb@left@rule\kern\kvtcb@boxsep}
\makeatother
\newcommand{\prompt}[4]{
    \ttfamily\llap{{\color{#2}[#3]:\hspace{3pt}#4}}\vspace{-\baselineskip}
}

\newcommand{\thevenin}[2]{
  \begin{center}
    \begin{circuitikz} \draw
      (0,0) -- (2,0) to[battery1, l_=$V_{Th}\eq#1$] (2,2) 
      to[resistor, l_=$R_{Th}\eq#2$] (0,2)
      ;
      \draw [o-] (-.07,2.079);
      \draw [o-] (-.07,0.079);
    \end{circuitikz}
  \end{center}
}

\newcommand{\norton}[2]{
  \begin{center}
    \begin{circuitikz} \draw
      (0,0) -- (3,0) to[american current source, l_=$I_{N}\eq#1$] (3,2) -- (0,2) (2,0)
      to[resistor, l=$R_{N}\eq#2$] (2,2)
      ;
      \draw [o-] (-.07,2.079);
      \draw [o-] (-.07,0.079);
    \end{circuitikz}
  \end{center}
}

\newcommand{\highlight}[1]{\colorbox{yellow}{$\displaystyle #1$}}

\newcommand{\ti}[1]{\widetilde{#1}}

\NewDocumentCommand{\evalat}{sO{\big}mm}{%
  \IfBooleanTF{#1}
   {\mleft. #3 \mright|_{#4}}
   {#3#2|_{#4}}%
}

\title{Math 2350H: Assignment III}
\author{Jeremy Favro (0805980) \\ Trent University, Peterborough, ON, Canada}
\date{\today}

\begin{document}
\maketitle

% PROBLEM 1
\begin{problem}
Let $T:\mathcal{P}_3(\mathbb{R})\to\mathcal{M}_{2\cross2}(\mathbb{R})$ be the linear transformation given by
$$T(a+bx+cx^2+dx^3)=\begin{pmatrix}
    3a + 7b - 2c - 5d  & 8a + 14b - 2c - 11d  \\
    -4a - 8b + 2c + 6d & 12a + 22b - 4c - 17d
  \end{pmatrix}$$
\begin{enumerate}[label=(\alph*)]
  \item Find the matrix representation $\left[T\right]_\beta^\gamma$ for bases
        $$\beta = \left\{1, x, x^2, x^3\right\},\qquad
          \gamma=\left\{
          \begin{pmatrix}
            1 & 0 \\
            0 & 0
          \end{pmatrix},
          \begin{pmatrix}
            0 & 1 \\
            0 & 0
          \end{pmatrix},
          \begin{pmatrix}
            0 & 0 \\
            1 & 0
          \end{pmatrix},
          \begin{pmatrix}
            0 & 0 \\
            0 & 1
          \end{pmatrix}
          \right\}.$$
  \item Find the matrix representation $\left[T\right]_\mathcal{B}^\mathcal{C}$ for bases
        $$\mathcal{B}=\left\{1 + x - x^2 + 2x^3, -1 + 2x + 2x^3, 2 + x - 2x^2 + 3x^3, 1 + x + 2x^3\right\},$$
        $$
          \mathcal{C}=\left\{
          \begin{pmatrix}
            1  & 1 \\
            -1 & 2
          \end{pmatrix},
          \begin{pmatrix}
            -1 & 2 \\
            0  & 2
          \end{pmatrix},
          \begin{pmatrix}
            2  & 1 \\
            -2 & 3
          \end{pmatrix},
          \begin{pmatrix}
            1 & 1 \\
            0 & 2
          \end{pmatrix}
          \right\}
          .$$
  \item Let $p(x)=3-x+2x^2-5x^3$. Find $\left[p(x)\right]_\beta$ and $\left[p(x)\right]_\mathcal{B}$.
  \item Find the image of $p(x)$ under $T$ in the following three ways:
        \begin{enumerate}[label=(\roman*)]
          \item By computing $T(p(x))$ directly
          \item By computing $\left[T\right]_\beta^\gamma\left[p(x)\right]_\beta=\left[T(p(x))\right]_\gamma$
          \item By computing $\left[T\right]_\mathcal{B}^\mathcal{C}\left[p(x)\right]_\mathcal{B}=\left[T(p(x))\right]_\mathcal{C}$
        \end{enumerate}
  \item Compute the matrix representations $\left[I\right]_\beta^\mathcal{B}$ and $\left[I\right]^\beta_\mathcal{B}$, where $I$ is the identity map
        on $\mathcal{P}_3\left(\mathbb{R}\right)$. Show that $\left(\left[I\right]_\beta^\mathcal{B}\right)^{-1}=\left[I\right]^\beta_\mathcal{B}$
  \item Compute the matrix product $\left[I\right]_\beta^\mathcal{B}\left[p(x)\right]_\beta$. What do we notice about the result?
\end{enumerate}
\end{problem}
\newpage

\begin{soln}~
  \begin{enumerate}[label=(\alph*)]
    \item Here we start by applying $T$ to each element of the input basis, $\beta$, and expressing the result as a linear combination of vectors in the output basis, $\gamma$,
          \begin{align*}
            T(1)   & =\begin{pmatrix}
                        3  & 8  \\
                        -4 & 12
                      \end{pmatrix} = 3\gamma_1+8\gamma_2-4\gamma_3+12\gamma_4 \rightsquigarrow (\text{treating $\gamma$ as ordered}) \\
            T(x)   & =\begin{pmatrix}
                        7  & 14 \\
                        -8 & 22
                      \end{pmatrix} = 7\gamma_1+14\gamma_2-8\gamma_3+22\gamma_4                                                       \\
            T(x^2) & =\begin{pmatrix}
                        -2 & -2 \\
                        2  & -4
                      \end{pmatrix} = -2\gamma_1-2\gamma_2+2\gamma_3-4\gamma_4                                                        \\
            T(x^3) & =\begin{pmatrix}
                        -5 & -11 \\
                        6  & -17
                      \end{pmatrix} = -5\gamma_1-11\gamma_2+6\gamma_3-17\gamma_4
          \end{align*}
          These linear combinations give us the columns of
          $$\left[T\right]_\beta^\gamma=\begin{pmatrix}
              3  & 7  & -2 & -5    \\
              8  & 14 & -2 & -11 & \\
              -4 & -8 & 2  & 6     \\
              12 & 22 & -4 & -17
            \end{pmatrix}$$
    \item Applying the same process as in part (a) (again treating $\mathcal{C}$ as ordered),
          \begin{align*}
            T(1+x-x^2+2x^3)  & =\begin{pmatrix}
                                  2  & 2 \\
                                  -2 & 4
                                \end{pmatrix} =2\mathcal{C}_1+0\mathcal{C}_2+0\mathcal{C}_3+0\mathcal{C}_4 \\
            T(-1+2x+2x^3)    & =\begin{pmatrix}
                                  1 & -2 \\
                                  0 & -2
                                \end{pmatrix} =0\mathcal{C}_1-1\mathcal{C}_2+0\mathcal{C}_3+0\mathcal{C}_4 \\
            T(2+x-2x^2+2x^3) & =\begin{pmatrix}
                                  2  & 1 \\
                                  -2 & 3
                                \end{pmatrix} =0\mathcal{C}_1+0\mathcal{C}_2+1\mathcal{C}_3+0\mathcal{C}_4 \\
            T(1+x+2x^3)      & =\begin{pmatrix}
                                  0 & 0 \\
                                  0 & 0
                                \end{pmatrix} =0\mathcal{C}_1+0\mathcal{C}_2+0\mathcal{C}_3+0\mathcal{C}_4
          \end{align*}
          $$\implies\left[T\right]_\mathcal{B}^\mathcal{C}=\begin{pmatrix}
              2 & 0  & 0 & 0 \\
              0 & -1 & 0 & 0 \\
              0 & 0  & 1 & 0 \\
              0 & 0  & 0 & 0
            \end{pmatrix}$$
    \item Again we represent $p(x)$ as a linear combination where the coefficients give us the columns of the matrix representation.
          For $\left[p(x)\right]_\beta$ we get $p(x)=3\cdot1-1\cdot x+2\cdot x^2-5\cdot x^3\implies \left[p(x)\right]_\beta=\begin{pmatrix}
              3  \\
              -1 \\
              2  \\
              -5
            \end{pmatrix}$. Then for $\left[p(x)\right]_\mathcal{B}$ we have a less obvious solution (obtained by row-reduction of the associated matrix) of
          $p(x)=32\mathcal{B}_1-7\mathcal{B}_2-17\mathcal{B}_3-2\mathcal{B}_4\implies \left[p(x)\right]_\mathcal{B}=\begin{pmatrix}
              32  \\
              -7  \\
              -17 \\
              -2
            \end{pmatrix}$

    \item Find the image of $p(x)$ under $T$ in the following three ways:
          \begin{enumerate}[label=(\roman*)]
            \item $T(p(x))=T(3-x+2x^2-5x^3)=\begin{pmatrix}
                      23  & 61 \\
                      -30 & 91
                    \end{pmatrix}$
            \item \begin{align*}
                    \left[T\right]_\beta^\gamma\left[p(x)\right]_\beta
                     & =\begin{pmatrix}
                          3  & 7  & -2 & -5    \\
                          8  & 14 & -2 & -11 & \\
                          -4 & -8 & 2  & 6     \\
                          12 & 22 & -4 & -17
                        \end{pmatrix}\begin{pmatrix}
                                       3  \\
                                       -1 \\
                                       2  \\
                                       -5
                                     \end{pmatrix}                       \\
                     & =\begin{pmatrix}
                          3\cdot 3+7\cdot (-1)+(-2)\cdot 2+(-5)^2          \\
                          8\cdot 3+14\cdot (-1)+(-2)\cdot 2+(-11)\cdot(-5) \\
                          -4\cdot 3+(-8)\cdot (-1)+2\cdot 2+6\cdot(-5)     \\
                          12\cdot 3+22\cdot (-1)+(-4)\cdot 2+(-17)\cdot (-5)
                        \end{pmatrix} \\
                     & =\begin{pmatrix}
                          23  \\
                          -49 \\
                          -30 \\
                          91
                        \end{pmatrix}                                    \\
                  \end{align*}
            \item \begin{align*}
                    \left[T\right]_\mathcal{B}^\mathcal{C}
                    \left[p(x)\right]_\mathcal{B} & =\begin{pmatrix}
                                                       2 & 0  & 0 & 0 \\
                                                       0 & -1 & 0 & 0 \\
                                                       0 & 0  & 1 & 0 \\
                                                       0 & 0  & 0 & 0
                                                     \end{pmatrix}\begin{pmatrix}
                                                                    32  \\
                                                                    -7  \\
                                                                    -17 \\
                                                                    -2
                                                                  \end{pmatrix} \\
                                                  & = \begin{pmatrix}
                                                        64  \\
                                                        7   \\
                                                        -17 \\
                                                        0
                                                      \end{pmatrix}
                  \end{align*}
          \end{enumerate}
    \item For $\left[I\right]_\beta^\mathcal{B}$,
          \begin{align*}
            I(1)   & =1\cdot 1 + 0\cdot x + 0\cdot x^2 + 0\cdot x^3 \\
            I(x)   & =0\cdot 1 + 1\cdot x + 0\cdot x^2 + 0\cdot x^3 \\
            I(x^2) & =0\cdot 1 + 0\cdot x + 1\cdot x^2 + 0\cdot x^3 \\
            I(x^3) & =0\cdot 1 + 0\cdot x + 0\cdot x^2 + 1\cdot x^3
          \end{align*}
          solving the associated matrices for coefficients of the elements of $\mathcal{B}$ yields
          $$\left[I\right]_\beta^\mathcal{B}=\begin{pmatrix}
              4  & 8  & -1 & -6 \\
              1  & 1  & 0  & -1 \\
              -2 & -4 & 0  & 3  \\
              0  & -1 & 1  & 1
            \end{pmatrix}.$$
          Then for $\left[I\right]^\beta_\mathcal{B}$,
          \begin{align*}
            I(1 + x - x^2 + 2x^3)  & =1\cdot 1 + 1\cdot x - 1\cdot x^2 + 2\cdot x^3  \\
            I(-1 + 2x + 2x^3)      & =-1\cdot 1 + 2\cdot x + 0\cdot x^2 + 2\cdot x^3 \\
            I(2 + x - 2x^2 + 3x^3) & =2\cdot 1 + 1\cdot x - 2\cdot x^2 + 3\cdot x^3  \\
            I(1 + x + 2x^3)        & =1\cdot 1 + 1\cdot x + 0\cdot x^2 + 2\cdot x^3
          \end{align*}
          which yields
          $$\left[I\right]^\beta_\mathcal{B}=
            \begin{pmatrix}
              1  & -1 & 2  & 1 \\
              1  & 2  & 1  & 1 \\
              -1 & 0  & -2 & 0 \\
              2  & 2  & 3  & 2
            \end{pmatrix}.$$
          To find $\left(\left[I\right]_\beta^\mathcal{B}\right)^{-1}$ we work the matrix $\left[\left[I\right]_\beta^\mathcal{B}\,|\,I \right]$ to
          $\left[I\,|\left(\left[I\right]_\beta^\mathcal{B}\right)^{-1}\right]$ which \textbf{TYPESET THIS :(}
    \item $\left[I\right]_\beta^\mathcal{B}\left[p(x)\right]_\beta=\begin{pmatrix}
              1  & -1 & 2  & 1 \\
              1  & 2  & 1  & 1 \\
              -1 & 0  & -2 & 0 \\
              2  & 2  & 3  & 2
            \end{pmatrix}\begin{pmatrix}
              3  \\
              -1 \\
              2  \\
              -5
            \end{pmatrix}=
            \begin{pmatrix}
              32  \\
              -7  \\
              -17 \\
              -2
            \end{pmatrix}=\left[p(x)\right]_\mathcal{B}$
  \end{enumerate}
\end{soln}

% PROBLEM 2
\begin{problem}
If $\beta$ and $\gamma$ are two bases for a finite dimensional vector space $V$,
and $I$ is the identity map on $V$, the matrix $\left[I\right]_\gamma^\beta$ is \emph{called a change of basis matrix}
(\emph{or change of coordinates matrix}). It is always invertible because $I$ is invertible.
\begin{enumerate}
  \item Let $S\in \mathcal{L}(V)$. Show that
        $$\left[S\right]_\gamma=\left(\left[I\right]_\gamma^\beta\right)^{-1}\left[S\right]_\beta\left[I\right]_\gamma^\beta$$
        by using the properties (given in class) which relate matrix multiplication and composition of linear maps
  \item Let $T\in\mathcal{L}(\mathcal{M}_{2\cross 2}(\mathbb{R}))$ be the map given by
        $$T\begin{pmatrix}
            a & b \\
            c & d
          \end{pmatrix}=
          \begin{pmatrix}
            -17a+11b+8c-11d & -57a+35b+24c-33d \\
            -14a+10b+6c-10d & -41a+25b+16c-23d
          \end{pmatrix}.$$
        Find the matrix representations $\left[T\right]_\beta$ and $\left[T\right]_\gamma$ for bases
        $$\beta=\left\{\begin{pmatrix}
            1 & 0 \\
            0 & 0
          \end{pmatrix},
          \begin{pmatrix}
            0 & 1 \\
            0 & 0
          \end{pmatrix},
          \begin{pmatrix}
            0 & 0 \\
            1 & 0
          \end{pmatrix},
          \begin{pmatrix}
            0 & 0 \\
            0 & 1
          \end{pmatrix}\right\},$$
        $$\gamma=
          \left\{\begin{pmatrix}
            0 & 1 \\
            0 & 1
          \end{pmatrix},
          \begin{pmatrix}
            1 & 1 \\
            1 & 0
          \end{pmatrix},
          \begin{pmatrix}
            1 & 3 \\
            2 & 3
          \end{pmatrix},
          \begin{pmatrix}
            2 & 6 \\
            1 & 4
          \end{pmatrix}\right\}.$$
  \item Find the change of basis matrix $\left[I\right]_\gamma^\beta$ and use it to verify what was shown in part (a) using the matrices computed in part (b).
\end{enumerate}
\end{problem}
\begin{soln}~
  \begin{enumerate}
    \item $$\left(\left[I\right]_\gamma^\beta\right)^{-1}\left[S\right]_\beta\left[I\right]_\gamma^\beta
            = \left[I\right]_\beta^\gamma\left[S\right]_\beta^\beta\left[I\right]_\gamma^\beta$$
          Because each of these are matrix representations of linear maps and composition of linear maps in matrix form is matrix multiplication the above expression can be
          re-written as
          $$I_{\beta\gamma}(S(I_{\gamma\beta}))$$
          which maps per $\gamma\to\beta\to\beta\to\gamma$, however the $\gamma\to\beta$ and $\beta\to\gamma$ steps are done using identity maps which do not change the
          mapped vector and can be removed from the expression yielding
          $$\left[I\right]_\beta^\gamma\left[S\right]_\beta^\beta\left[I\right]_\gamma^\beta\implies I_{\beta\gamma}(S(I_{\gamma\beta}))=S\implies\left[S\right]_\gamma.$$
    \item For $\left[T\right]_\beta$,
          \begin{align*}
            T(\beta_1) & =\begin{pmatrix}
                            -17 & -57 \\
                            -14 & -41
                          \end{pmatrix}=-17\beta_1-57\beta_2-14\beta_3-41\beta_4 \\
            T(\beta_2) & =\begin{pmatrix}
                            11 & 35 \\
                            10 & 25
                          \end{pmatrix}=11\beta_1+35\beta_2+10\beta_3+25\beta_4  \\
            T(\beta_3) & =\begin{pmatrix}
                            8 & 24 \\
                            6 & 16
                          \end{pmatrix}=8\beta_1+24\beta_2+6\beta_3+16\beta_4    \\
            T(\beta_4) & =\begin{pmatrix}
                            -11 & -33 \\
                            -10 & -23
                          \end{pmatrix}=-11\beta_1-33\beta_2-10\beta_3-23\beta_4
          \end{align*}
          so,
          $$\left[T\right]_\beta=\begin{pmatrix}
              -17 & 11 & 8  & -11 \\
              -57 & 35 & 24 & -33 \\
              -14 & 10 & 6  & -10 \\
              -41 & 25 & 16 & -23
            \end{pmatrix}.$$
          Then for $\left[T\right]_\gamma$ we get
          \begin{align*}
            T(\gamma_1) & =\begin{pmatrix}
                             0 & 2 \\
                             0 & 2
                           \end{pmatrix}=2\gamma_1+0\gamma_2+0\gamma_3+0\gamma_4 \\
            T(\gamma_2) & =\begin{pmatrix}
                             2 & 2 \\
                             2 & 0
                           \end{pmatrix}=0\gamma_1+2\gamma_2+0\gamma_3+0\gamma_4 \\
            T(\gamma_3) & =\begin{pmatrix}
                             -1 & -3 \\
                             -2 & -3
                           \end{pmatrix}=0\gamma_1+0\gamma_2-1\gamma_3+0\gamma_4 \\
            T(\gamma_4) & =\begin{pmatrix}
                             -4 & -12 \\
                             -2 & -8
                           \end{pmatrix}=0\gamma_1+0\gamma_2+0\gamma_3-2\gamma_4
          \end{align*}
          so,
          $$\left[T\right]_\gamma=\begin{pmatrix}
              2 & 0 & 0  & 0  \\
              0 & 2 & 0  & 0  \\
              0 & 0 & -1 & 0  \\
              0 & 0 & 0  & -2
            \end{pmatrix}.$$
    \item Applying $I$ to the input basis,
          \begin{align*}
            I(\gamma_1) & =\gamma_1=0\beta_1+1\beta_2+0\beta_3+1\beta_4 \\
            I(\gamma_2) & =\gamma_1=1\beta_1+1\beta_2+1\beta_3+0\beta_4 \\
            I(\gamma_3) & =\gamma_1=1\beta_1+3\beta_2+2\beta_3+3\beta_4 \\
            I(\gamma_4) & =\gamma_1=2\beta_1+6\beta_2+1\beta_3+4\beta_4
          \end{align*}
          so $\left[I\right]_\gamma^\beta=\begin{pmatrix}
              0 & 1 & 1 & 2 \\
              1 & 1 & 3 & 6 \\
              0 & 1 & 2 & 1 \\
              1 & 0 & 3 & 4
            \end{pmatrix}$. Now,
          \begin{align*}
             & = \left(\left[I\right]_\gamma^\beta\right)^{-1}\left[T\right]_\beta\left[I\right]_\gamma^\beta \\
             & = \begin{pmatrix}
                   0 & 1 & 1 & 2 \\
                   1 & 1 & 3 & 6 \\
                   0 & 1 & 2 & 1 \\
                   1 & 0 & 3 & 4
                 \end{pmatrix}^{-1}
            \begin{pmatrix}
              -17 & 11 & 8  & -11 \\
              -57 & 35 & 24 & -33 \\
              -14 & 10 & 6  & -10 \\
              -41 & 25 & 16 & -23
            \end{pmatrix}
            \begin{pmatrix}
              0 & 1 & 1 & 2 \\
              1 & 1 & 3 & 6 \\
              0 & 1 & 2 & 1 \\
              1 & 0 & 3 & 4
            \end{pmatrix}                                                                                    \\
             & =
            \begin{pmatrix}
              -11 & 7  & 4  & -6 \\
              -4  & 3  & 2  & -3 \\
              1   & -1 & 0  & 1  \\
              2   & -1 & -1 & 1
            \end{pmatrix}
            \begin{pmatrix}
              -17 & 11 & 8  & -11 \\
              -57 & 35 & 24 & -33 \\
              -14 & 10 & 6  & -10 \\
              -41 & 25 & 16 & -23
            \end{pmatrix}
            \begin{pmatrix}
              0 & 1 & 1 & 2 \\
              1 & 1 & 3 & 6 \\
              0 & 1 & 2 & 1 \\
              1 & 0 & 3 & 4
            \end{pmatrix}                                                                                    \\
             & =
            \begin{pmatrix}
              -22 & 14 & 8 & -12 \\
              -8  & 6  & 4 & -6  \\
              -1  & 1  & 0 & -1  \\
              -4  & 2  & 2 & -2
            \end{pmatrix}
            \begin{pmatrix}
              0 & 1 & 1 & 2 \\
              1 & 1 & 3 & 6 \\
              0 & 1 & 2 & 1 \\
              1 & 0 & 3 & 4
            \end{pmatrix}                                                                                    \\
             & =
            \begin{pmatrix}
              2 & 0 & 0  & 0 \\
              0 & 2 & 0  & 0 \\
              0 & 0 & -1 & 0 \\
              0 & 0 & 0  & 2
            \end{pmatrix}=\left[T\right]_\gamma                                                               \\
          \end{align*}
  \end{enumerate}
\end{soln}
\newpage

% PROBLEM 3 
\begin{problem}
Let $T:\mathcal{P}_3(\mathbb{R})\to\mathcal{M}_{2\cross 2}(\mathbb{R})$ be the linear transformation given by
$$T(a+bx+cx^2+dx^3)=\begin{pmatrix}
    a+b & a-2c \\
    d   & b-d
  \end{pmatrix}$$
\begin{enumerate}[label=(\alph*)]
  \item Find the null space of $T$.
  \item Show that $T$ is invertible without giving an explicit inverse.
  \item Find the matrix representation $\left[T\right]_\beta^\gamma$ for bases
        $$\beta=\left\{1,x,x^2,x^3\right\},\qquad \gamma=\left\{
          \begin{pmatrix}
            1 & 0 \\
            0 & 0
          \end{pmatrix},
          \begin{pmatrix}
            0 & 1 \\
            0 & 0
          \end{pmatrix},
          \begin{pmatrix}
            0 & 0 \\
            1 & 0
          \end{pmatrix},
          \begin{pmatrix}
            0 & 0 \\
            0 & 1
          \end{pmatrix}
          \right\}.$$
  \item Compute $\left(\left[T\right]_\beta^\gamma\right)^{-1}$ and use this to find an expression for $T^{-1}$; i.e. find $T^{-1}\begin{pmatrix}
            a & b \\
            c & d
          \end{pmatrix}$.
  \item Consider the $2\cross 2$ identity matrix
        $$I_2=\begin{pmatrix}
            1 & 0 \\
            0 & 1
          \end{pmatrix}$$
        Find its pre-image in $\mathcal{P}_3(\mathbb{R})$ under $T$ in two ways:
        \begin{enumerate}[label=(\roman*)]
          \item By computing $T^{-1}(I_2)$ is the result from (d)
          \item By first finding $\left[I_2\right]\gamma$, then computing $\left(\left[T\right]_\beta^\gamma\right)^{-1}\left[I_2\right]_\gamma$.
        \end{enumerate}
\end{enumerate}
\end{problem}
\begin{soln}~
  \begin{enumerate}[label=(\alph*)]
    \item $\text{null}\,T$ is all $p(x)=a+bx+cx^2+dx^3\in \mathcal{P}_3(\mathbb{R})$ such that $a+b=a-2c=d=b-d=0$ which has only the trivial solution,
          $a=b=c=d=0$ so $\text{null}\,T=\left\{0\right\}$.
    \item Because $\text{range}\, T=\mathcal{M}_{2\cross 2}(\mathbb{R})$ and $\text{null}\,T=\left\{0\right\}$ we are guaranteed that $T$ is both invertible and
          bijective.
    \item Applying $T$ to the elements of $\beta$,
          \begin{align*}
            T(1)   & =\begin{pmatrix}
                        1 & 1 \\
                        0 & 0
                      \end{pmatrix}=\gamma_1+\gamma_2+0\gamma_3+0\gamma_4   \\
            T(x)   & =\begin{pmatrix}
                        1 & 0 \\
                        0 & 1
                      \end{pmatrix}=1\gamma_1+0\gamma_2+0\gamma_3+1\gamma_4 \\
            T(x^2) & =\begin{pmatrix}
                        0 & -2 \\
                        0 & 0
                      \end{pmatrix}=0\gamma_1-2\gamma_2+0\gamma_3+0\gamma_4 \\
            T(x^3) & =\begin{pmatrix}
                        0 & 0  \\
                        1 & -1
                      \end{pmatrix}=0\gamma_1+0\gamma_2+1\gamma_3-1\gamma_4 \\
          \end{align*}
          so,
          $$\left[T\right]_\beta^\gamma=\begin{pmatrix}
              1 & 1 & 0  & 0  \\
              1 & 0 & -2 & 0  \\
              0 & 0 & 0  & 1  \\
              0 & 1 & 0  & -1
            \end{pmatrix}
          $$
    \item Compute $\left(\left[T\right]_\beta^\gamma\right)^{-1}$ and use this to find an expression for $T^{-1}$; i.e. find $T^{-1}\begin{pmatrix}
              a & b \\
              c & d
            \end{pmatrix}$.
    \item Consider the $2\cross 2$ identity matrix
          $$I_2=\begin{pmatrix}
              1 & 0 \\
              0 & 1
            \end{pmatrix}$$
          Find its pre-image in $\mathcal{P}_3(\mathbb{R})$ under $T$ in two ways:
          \begin{enumerate}[label=(\roman*)]
            \item By computing $T^{-1}(I_2)$ is the result from (d)
            \item By first finding $\left[I_2\right]\gamma$, then computing $\left(\left[T\right]_\beta^\gamma\right)^{-1}\left[I_2\right]_\gamma$.
          \end{enumerate}
  \end{enumerate}
\end{soln}
\end{document}