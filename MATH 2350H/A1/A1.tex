\documentclass[10pt]{article}

\usepackage[margin=0.75in]{geometry}
\usepackage{amsmath,amsthm,amssymb}
\usepackage{xcolor}
\usepackage{cancel}
\usepackage{graphicx}
\usepackage{changepage}
\usepackage{circuitikz}
\usepackage{pgfplots}
\usepackage{physics}
\usepackage{hyperref}
\usepackage{siunitx}
\usepackage[breakable]{tcolorbox}
\usepackage[inline]{enumitem}

\theoremstyle{definition}
\newtheorem{problem}{Problem}
\newtheorem{soln}{Solution}

\pgfplotsset{compat=newest}
\usetikzlibrary{lindenmayersystems}
\usetikzlibrary{arrows}
\usetikzlibrary{calc}

\definecolor{incolor}{HTML}{303F9F}
\definecolor{outcolor}{HTML}{D84315}
\definecolor{cellborder}{HTML}{CFCFCF}
\definecolor{cellbackground}{HTML}{F7F7F7}
\newcommand{\eq}{=}
\usetikzlibrary{positioning, fit, calc}
\pgfdeclarelayer{background}  
\pgfsetlayers{background,main}
\DeclareSIUnit[number-unit-product = {\,}]\calorie{cal}
\DeclareSIUnit[number-unit-product = {\,}]\atmosphere{atm}
\AtBeginDocument{\RenewCommandCopy\qty\SI}

\makeatletter
\newcommand{\boxspacing}{\kern\kvtcb@left@rule\kern\kvtcb@boxsep}
\makeatother
\newcommand{\prompt}[4]{
    \ttfamily\llap{{\color{#2}[#3]:\hspace{3pt}#4}}\vspace{-\baselineskip}
}

\newcommand{\thevenin}[2]{
  \begin{center}
    \begin{circuitikz} \draw
      (0,0) -- (2,0) to[battery1, l_=$V_{Th}\eq#1$] (2,2) 
      to[resistor, l_=$R_{Th}\eq#2$] (0,2)
      ;
      \draw [o-] (-.07,2.079);
      \draw [o-] (-.07,0.079);
    \end{circuitikz}
  \end{center}
}

\newcommand{\norton}[2]{
  \begin{center}
    \begin{circuitikz} \draw
      (0,0) -- (3,0) to[american current source, l_=$I_{N}\eq#1$] (3,2) -- (0,2) (2,0)
      to[resistor, l=$R_{N}\eq#2$] (2,2)
      ;
      \draw [o-] (-.07,2.079);
      \draw [o-] (-.07,0.079);
    \end{circuitikz}
  \end{center}
}

\newcommand{\highlight}[1]{\colorbox{yellow}{$\displaystyle #1$}}

\newcommand{\ti}[1]{\widetilde{#1}}

\NewDocumentCommand{\evalat}{sO{\big}mm}{%
  \IfBooleanTF{#1}
   {\mleft. #3 \mright|_{#4}}
   {#3#2|_{#4}}%
}

\title{Math 2350H: Assignment I}
\author{Jeremy Favro}
\date{\today}

\begin{document}
\maketitle

% PROBLEM 1
\begin{problem}
Demonstrate that there does not exist $\lambda \in \mathbb{C}$ such that
$$\lambda\left(2-3i, 5+4i, -6+i\right) = \left(12-5i, 7+22i, -32-9i\right)$$
\end{problem}
\begin{soln}
  Let $\lambda=a+bi$ then
  \begin{align*}
     & \lambda\left(2-3i, 5+4i, -6+i\right) = \left(12-5i, 7+22i, -32-9i\right)                    \\
     & \implies (2a+3b)+(-3a+2b)i=12-5i\wedge (5a-4b)+(4a+5b)i=7+22i\wedge(-6a-1b)+(1a-6b)i=-32-9i
  \end{align*}
  Here the determinant of the matrix of the coefficients of $i$ is
  $$  \begin{vmatrix}
      -3 & 2  & -5 \\
      4  & 5  & 2  \\
      1  & -6 & -9
    \end{vmatrix}=-3(5\cdot(-9)-22\cdot(-6))+(-1)2(4\cdot(-9)-1\cdot22)-5(4\cdot(-6)-1\cdot5)=0$$
  so there exist no $a,b$ which satisfy all three coefficients of $i$ and therefore there exists no $\lambda$ that satisfies the original equation.
\end{soln}



% PROBLEM 2
\begin{problem}
Let $V=\mathbb{R}^2$. If $(x_1, x_2)$ and $(y_1, y_2)$ are elements of $V$ , and $\alpha\in\mathbb{R}$, define
$$(x_1, x_2) + (y_1, y_2) = (x_1 + y_1, x_2y_2),$$ and $$\alpha\cdot(x_1, x_2) = (\alpha x_1, x_2).$$
Is $V$ a vector space over $\mathbb{R}$ with these operations? Justify your answer.
\end{problem}
\begin{soln} For $V$ to be a vector space over $\mathbb{R}$ with these operations the following must hold $\forall(x_1,y_1)\in V$, $\alpha\in\mathbb{R}$:
  \begin{enumerate}[label=(\roman*)]
    \item $+$ must be commutative and associative
    \item $\cdot$ must be associative
    \item $0\in V$
    \item There must exist a multiplicative identity for scalar multiplication
    \item There must exist an additive inverse
    \item The additive and multiplicative distributive law must hold
  \end{enumerate}
  \begin{enumerate}[label=(\roman*)]
    \item Commutativity: Let $(x_1, x_2), (y_1, y_2)\in V$. \\
          Given that $(x_1, x_2) + (y_1, y_2) = (x_1 + y_1, x_2y_2)$,
          \begin{align*}
             & (y_1, y_2) + (x_1, x_2) = (y_1 + x_1, y_2x_2) \, (\text{by definition})                   \\
             & (y_1, y_2) + (x_1, x_2) = (x_1 + y_1, x_2y_2) \, (\text{by commutativity in } \mathbb{R})
          \end{align*}
          Associativity: Let $(x_1, x_2), (y_1, y_2), (z_1, z_2)\in V$. \\
          Given that $(x_1, x_2) + ((y_1, y_2) + (z_1, z_2)) = (x_1, x_2) + (y_1 + z_1, y_2z_2) =(x_1 + y_1 + z_1, x_2y_2z_2)$
          \begin{align*}
             & =((x_1, x_2) + (y_1, y_2)) + (z_1, z_2)                                                \\
             & =(x_1 + y_1, x_2y_2) + (z_1, z_2) \, (\text{by definition})                            \\
             & =(z_1, z_2) + (x_1 + y_1, x_2y_2) \, (\text{by previously demonstrated commutativity}) \\
             & =(z_1 + x_1 + y_1, z_2x_2y_2)                                                          \\
             & =(x_1 + y_1 + z_1 , x_2y_2z_2) \, (\text{by commutativity in } \mathbb{R})             \\
          \end{align*}
    \item Associativity: Let $(x_1, x_2)\in V$ and $\alpha,\beta\in\mathbb{R}$\\
          Given that, by definition, $(\alpha\cdot\beta)\cdot(x_1, x_2)=(\alpha\cdot\beta\cdot x_1, x_2)$
          \begin{align*}
             & =\alpha\cdot(\beta\cdot(x_1, x_2)) \\
             & =\alpha\cdot(\beta\cdot x_1, x_2)  \\
             & =(\alpha\cdot\beta\cdot x_1, x_2)  \\
          \end{align*}


    \item Let $(x_1, x_2)\in V$
          \begin{align*}
             & = (x_1, x_2) + (0,1)  \\
             & = (x_1+0, x_2\cdot 1) \\
             & = (x_1, x_2)
          \end{align*}
    \item Let $(x_1, x_2)\in V$
    \begin{align*}
      & = 1\cdot(x_1, x_2)  \\
      & = (1\cdot x_1, x_2) \\
      & = (x_1, x_2)
   \end{align*}
    \item There must exist an additive inverse
    \item The additive and multiplicative distributive law must hold
  \end{enumerate}
\end{soln}

\end{document}