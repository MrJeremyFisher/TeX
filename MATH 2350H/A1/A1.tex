\documentclass[10pt]{article}

\usepackage[margin=0.75in]{geometry}
\usepackage{amsmath,amsthm,amssymb}
\usepackage{xcolor}
\usepackage{cancel}
\usepackage{graphicx}
\usepackage{changepage}
\usepackage{circuitikz}
\usepackage{pgfplots}
\usepackage{physics}
\usepackage{hyperref}
\usepackage{siunitx}
\usepackage[breakable]{tcolorbox}
\usepackage[inline]{enumitem}

\theoremstyle{definition}
\newtheorem{problem}{Problem}
\newtheorem{soln}{Solution}

\pgfplotsset{compat=newest}
\usetikzlibrary{lindenmayersystems}
\usetikzlibrary{arrows}
\usetikzlibrary{calc}

\definecolor{incolor}{HTML}{303F9F}
\definecolor{outcolor}{HTML}{D84315}
\definecolor{cellborder}{HTML}{CFCFCF}
\definecolor{cellbackground}{HTML}{F7F7F7}
\newcommand{\eq}{=}
\usetikzlibrary{positioning, fit, calc}
\pgfdeclarelayer{background}  
\pgfsetlayers{background,main}
\DeclareSIUnit[number-unit-product = {\,}]\calorie{cal}
\DeclareSIUnit[number-unit-product = {\,}]\atmosphere{atm}
\AtBeginDocument{\RenewCommandCopy\qty\SI}

\makeatletter
\newcommand{\boxspacing}{\kern\kvtcb@left@rule\kern\kvtcb@boxsep}
\makeatother
\newcommand{\prompt}[4]{
    \ttfamily\llap{{\color{#2}[#3]:\hspace{3pt}#4}}\vspace{-\baselineskip}
}

\newcommand{\thevenin}[2]{
  \begin{center}
    \begin{circuitikz} \draw
      (0,0) -- (2,0) to[battery1, l_=$V_{Th}\eq#1$] (2,2) 
      to[resistor, l_=$R_{Th}\eq#2$] (0,2)
      ;
      \draw [o-] (-.07,2.079);
      \draw [o-] (-.07,0.079);
    \end{circuitikz}
  \end{center}
}

\newcommand{\norton}[2]{
  \begin{center}
    \begin{circuitikz} \draw
      (0,0) -- (3,0) to[american current source, l_=$I_{N}\eq#1$] (3,2) -- (0,2) (2,0)
      to[resistor, l=$R_{N}\eq#2$] (2,2)
      ;
      \draw [o-] (-.07,2.079);
      \draw [o-] (-.07,0.079);
    \end{circuitikz}
  \end{center}
}

\newcommand{\highlight}[1]{\colorbox{yellow}{$\displaystyle #1$}}

\newcommand{\ti}[1]{\widetilde{#1}}

\NewDocumentCommand{\evalat}{sO{\big}mm}{%
  \IfBooleanTF{#1}
   {\mleft. #3 \mright|_{#4}}
   {#3#2|_{#4}}%
}

\title{Math 2350H: Assignment I}
\author{Jeremy Favro}
\date{\today}

\begin{document}
\maketitle

% PROBLEM 1
\begin{problem}
Demonstrate that there does not exist $\lambda \in \mathbb{C}$ such that
$$\lambda\left(2-3i, 5+4i, -6+i\right) = \left(12-5i, 7+22i, -32-9i\right)$$
\end{problem}
\begin{soln}
  Let $\lambda=a+bi$ then
  \begin{align*}
     & \lambda\left(2-3i, 5+4i, -6+i\right) = \left(12-5i, 7+22i, -32-9i\right)                    \\
     & \implies (2a+3b)+(-3a+2b)i=12-5i\wedge (5a-4b)+(4a+5b)i=7+22i\wedge(-6a-1b)+(1a-6b)i=-32-9i
  \end{align*}
  Here the determinant of the matrix of the coefficients of $i$ is
  $$  \begin{vmatrix}
      -3 & 2  & -5 \\
      4  & 5  & 2  \\
      1  & -6 & -9
    \end{vmatrix}=-3(5\cdot(-9)-22\cdot(-6))+(-1)2(4\cdot(-9)-1\cdot22)-5(4\cdot(-6)-1\cdot5)=0$$
  so there exist no $a,b$ which satisfy all three coefficients of $i$ and therefore there exists no $\lambda$ that satisfies the original equation.
\end{soln}

% PROBLEM 2
\begin{problem}
Let $V=\mathbb{R}^2$. If $(x_1, x_2)$ and $(y_1, y_2)$ are elements of $V$ , and $\alpha\in\mathbb{R}$, define
$$(x_1, x_2) + (y_1, y_2) = (x_1 + y_1, x_2y_2),$$ and $$\alpha\cdot(x_1, x_2) = (\alpha x_1, x_2).$$
Is $V$ a vector space over $\mathbb{R}$ with these operations? Justify your answer.
\end{problem}
\begin{soln} For $V$ to be a vector space over $\mathbb{R}$ with these operations the following must hold $\forall(x_1,y_1)\in V$, $\alpha\in\mathbb{R}$:
  \begin{enumerate}[label=(\roman*)]
    \item $+$ must be commutative and associative
    \item $\cdot$ must be associative
    \item $0\in V$
    \item There must exist a multiplicative identity for scalar multiplication
    \item There must exist an additive inverse
    \item The additive and multiplicative distributive law must hold
  \end{enumerate}
  \begin{enumerate}[label=(\roman*)]
    \item Commutativity: Let $(x_1, x_2), (y_1, y_2)\in V$. \\
          Given that $(x_1, x_2) + (y_1, y_2) = (x_1 + y_1, x_2y_2)$,
          \begin{align*}
             & (y_1, y_2) + (x_1, x_2) = (y_1 + x_1, y_2x_2) \, (\text{by definition})                                                                       \\
             & (y_1, y_2) + (x_1, x_2) = (x_1 + y_1, x_2y_2) \, (\text{by commutativity in } \mathbb{R}) \therefore + \text{ under } V \text{is commutative}
          \end{align*}
          Associativity: Let $(x_1, x_2), (y_1, y_2), (z_1, z_2)\in V$. \\
          Given that $(x_1, x_2) + ((y_1, y_2) + (z_1, z_2)) = (x_1, x_2) + (y_1 + z_1, y_2z_2) =(x_1 + y_1 + z_1, x_2y_2z_2)$
          \begin{align*}
             & =((x_1, x_2) + (y_1, y_2)) + (z_1, z_2)                                                                                        \\
             & =(x_1 + y_1, x_2y_2) + (z_1, z_2) \, (\text{by definition})                                                                    \\
             & =(z_1, z_2) + (x_1 + y_1, x_2y_2) \, (\text{by previously demonstrated commutativity})                                         \\
             & =(z_1 + x_1 + y_1, z_2x_2y_2)                                                                                                  \\
             & =(x_1 + y_1 + z_1 , x_2y_2z_2) \, (\text{by commutativity in } \mathbb{R}) \therefore + \text{ under } V \text{is associative}
          \end{align*}
    \item Associativity: Let $(x_1, x_2)\in V$ and $\alpha,\beta\in\mathbb{R}$\\
          Given that, by definition, $(\alpha\cdot\beta)\cdot(x_1, x_2)=(\alpha\cdot\beta\cdot x_1, x_2)$
          \begin{align*}
             & =\alpha\cdot(\beta\cdot(x_1, x_2))                                                         \\
             & =\alpha\cdot(\beta\cdot x_1, x_2)                                                          \\
             & =(\alpha\cdot\beta\cdot x_1, x_2)  \therefore \cdot \text{ under } V \text{is associative}
          \end{align*}
    \item Let $(x_1, x_2)\in V$
          \begin{align*}
             & = (x_1, x_2) + (0,1) \, (\text{by handwaving})  \\
             & = (x_1+0, x_2\cdot 1) \, (\text{by definition}) \\
             & = (x_1, x_2) \therefore (0,1)=0_V
          \end{align*}
    \item Let $(x_1, x_2)\in V$
          \begin{align*}
             & = 1\cdot(x_1, x_2)                                               \\
             & = (1\cdot x_1, x_2) \, (\text{by definition})                    \\
             & = (x_1, x_2) \therefore 1 \text{ is the multiplicative identity}
          \end{align*}
    \item Let $v,v^\prime\in V$. We are looking to show that $v+v^\prime=0_V$. By handwaving suppose that the additive inverse exists and is $v^\prime=\left(-x_1, \frac{1}{y_1}\right)$ then,
          \begin{align*}
             & v+v^\prime = (x_1,y_1)+\left(-x_1,\frac{1}{y_1}\right)                                                                                                  \\
             & = \left(x_1-x_1,y_1\frac{1}{y_1}\right) \text{ by definition}                                                                                           \\
             & = (0,1) \therefore\text{by rules for addition and multiplication under } \mathbb{R},\, \left(-x_1, \frac{1}{y_1}\right) \text{ is the additive inverse}
          \end{align*}
    \item Additive:
          Let $v_1,v_2\in V$, $\alpha\in\mathbb{R}$
          \begin{align*}
             & = \alpha(v_1+v_2)                                    \\
             & = \alpha((x_1,y_1)+(x_2,y_2))                        \\
             & = \alpha((x_1+x_2,y_1y_2)) \, (\text{by definition}) \\
             & = (\alpha(x_1+x_2),y_1y_2) \, (\text{by definition})
          \end{align*}
          Then,
          \begin{align*}
             & = \alpha v_1+\alpha v_2                                                   \\
             & = \alpha(x_1,y_1)+\alpha(x_2,y_2)                                         \\
             & = (\alpha x_1,y_1)+(\alpha x_2,y_2) \, (\text{by definition})             \\
             & = (\alpha x_1+\alpha x_2,y_1y_2) \, (\text{by definition} )               \\
             & = (\alpha(x_1+ x_2),y_1y_2) \therefore \text{ the distributive law holds}
          \end{align*}
          Multiplicative:
          Let $v\in V$, $\alpha,\beta\in\mathbb{R}$
          \begin{align*}
             & = \alpha v+\beta v                                             \\
             & = (\alpha x_1, y_1)+(\beta x_1, y_1) \, (\text{by definition}) \\
             & = (\alpha x_1+\beta x_1, y_1y_1) \, (\text{by definition})     \\
             & = ((\alpha +\beta)x_1, y_1y_1)
          \end{align*}
          Then,
          \begin{align*}
             & = (\alpha+\beta)v                                                                   \\
             & = ((\alpha+\beta)x_1, y_1)\, (\text{by definition} )                                \\
             & = (\alpha x_1+\beta x_1, y_1) \therefore \text{ the distributive law does not hold}
          \end{align*}
          $\therefore V$ is not a vector space over $\mathbb{R}$ with the given operations.
  \end{enumerate}
\end{soln}

% PROBLEM 3
\begin{problem}
Let $V=\mathbb{R}^2$. If $(x_1,y_1)$ and $(x_2,y_2)$ are elements of $V$, and $\alpha\in\mathbb{R}$, define
$$(x_1,y_1)+(x_2,y_2)=(x_1+2x_2, y_1+3y_2),$$
and
$$\alpha\cdot(x_1,y_1) = (\alpha x_1,\alpha y_1).$$
Is $V$ a vector space over $\mathbb{R}$ with these operations? Justify your answer.
\end{problem}
\begin{soln}
  Nicely, this one fails on the first one I checked, additive commutativity:
  Let $(x_1, x_2), (y_1, y_2)\in V$. \\
  Given that $(x_1, x_2) + (y_1, y_2) = (x_1+2x_2, y_1+3y_2)$,
  \begin{align*}
     & (y_1, y_2) + (x_1, x_2) = (x_2+2x_1, y_2+3y_1) \therefore + \text{ is not commutative} \\
  \end{align*}
  $\therefore V$ is not a vector space over $\mathbb{R}$ with the given operations.
\end{soln}

% PROBLEM 4
\begin{problem}
Which of the following sets are subspaces of $\mathbb{R}^3$ under the usual operations of addition and scalar multiplication in $\mathbb{R}^3$.
\begin{enumerate}[label=(\alph*)]
  \item $\displaystyle W_1 = \left\{(x_1, x_2, x_3) \in\mathbb{R}^3|x_1= x_2 \text{ and } x_3=-x_2\right\}.$
  \item $\displaystyle W_2 = \left\{(x_1, x_2, x_3) \in\mathbb{R}^3|x_1-4x_2-x_3=0\right\}.$
  \item $\displaystyle W_3 = \left\{(x_1, x_2, x_3) \in\mathbb{R}^3|5x_1^2-3x_2^2+6x_3^2=0\right\}.$
\end{enumerate}
\end{problem}
\begin{soln}
  The subspace criterion state that a non-empty subset $S$ is a subspace of $V$ (with scalars $F$) iff $\forall v_1,v_2\in S,\alpha\in F$
  \begin{enumerate}[label=(\roman*)]
    \item $0_V\in S$
    \item $v_1+v_2\in S$
    \item $\alpha v_1\in S$
  \end{enumerate}
  Now,
  \begin{enumerate}[label=(\alph*)]
    \item \begin{enumerate}[label=(\roman*)]
            \item $0_{\mathbb{R}^3}\in W_1$ because $x_1=0\implies x_2=0\implies x_3=0$
            \item Let $v_1,v_2\in W_1$,
                  \begin{align*}
                     & = v_1+v_2                                               \\
                     & = (x_1,x_2,x_3)+(y_1,y_2,y_3)                           \\
                     & = (x_2,x_2,-x_2)+(y_2,y_2,-y_2)\,(\text{by definition}) \\
                     & = (x_2+y_2,x_2+y_2,-x_2-y_2)
                  \end{align*}
                  Then, by the definition of $W_1$ addition here is closed as $x_2+y_2=x_2+y_2$ and $-(x_2+y_2)=-x_2-y_2$
            \item Let $v_1\in W_1$, $\alpha\in\mathbb{R}$,
                  \begin{align*}
                     & = \alpha v_1                          \\
                     & = \alpha(x_1,x_2,x_3)                 \\
                     & = (\alpha x_1,\alpha x_2,\alpha x_3)  \\
                     & = (\alpha x_2,\alpha x_2,-\alpha x_2)
                  \end{align*}
                  Which satisfies the definition of $W_1$ so $W_1$ is a subspace of $\mathbb{R}^3$ as it satisfies the subspace criterion.
          \end{enumerate}
    \item \begin{enumerate}[label=(\roman*)]
            \item $0_{\mathbb{R}^3}\in W_2$ because $x_1=x_2=x_3=0\implies 0-4\cdot0-0=0$
            \item Let $v_1,v_2\in W_2$,
                  \begin{align*}
                     & = v_1+v_2                     \\
                     & = (x_1,x_2,x_3)+(y_1,y_2,y_3) \\
                     & = (x_1+y_1,x_2+y_2,x_3+y_3)
                  \end{align*}
                  For this to be in $W_2$ it must satisfy the requirement $(x_1+y_1)-4(x_2+y_2)-(x_3+y_3)=0$,
                  \begin{align*}
                     & = (x_1+y_1)-4(x_2+y_2)-(x_3+y_3) \\
                     & = x_1+y_1-4x_2-4y_2-x_3-y_3      \\
                     & = (x_1-4x_2-x_3)+(y_1-4y_2-y_3)  \\
                     & = (0)+(0)=0 \, (v_1,v_2\in W_2)
                  \end{align*}
                  So this set is closed under addition.
            \item Let $v_1\in W_2$, $\alpha\in\mathbb{R}$,
                  \begin{align*}
                     & = \alpha v_1                         \\
                     & = (\alpha x_1,\alpha x_2,\alpha x_3) \\
                  \end{align*}
                  Then, again, we must satisfy $\alpha x_1-4\alpha x_2-\alpha x_3=0$,
                  \begin{align*}
                     & = \alpha x_1-4\alpha x_2-\alpha x_3 \\
                     & = \alpha(x_1-4 x_2- x_3)            \\
                     & = \alpha(0)=0
                  \end{align*}
                  So $W_2$ is a subspace of $\mathbb{R}^3$ as it satisfies the subspace criterion.
          \end{enumerate}
    \item \begin{enumerate}[label=(\roman*)]
            \item $0_{\mathbb{R}^3}\in W_3$ because $x_1=x_2=x_3=0\implies 5\cdot0^2-3\cdot0^2+6\cdot0^2=0$
            \item Let $v_1,v_2\in W_3$,
                  \begin{align*}
                     & = v_1+v_2                     \\
                     & = (x_1,x_2,x_3)+(y_1,y_2,y_3) \\
                     & = (x_1+y_1,x_2+y_2,x_3+y_3)
                  \end{align*}
                  For this to be in $W_3$ it must satisfy the requirement $5(x_1+y_1)^2-3(x_2+y_2)^2+6(x_3+y_3)^2=0$,
                  \begin{align*}
                     & = 5(x_1+y_1)^2-3(x_2+y_2)^2+6(x_3+y_3)^2                                  \\
                     & = 5(x_1^2+2x_1y_1+y_1^2)-3(x_2^2+2x_2y_2+y_2^2)+6(x_3^2+2x_3y_3+y_3^2)    \\
                     & = 5x_1^2+10x_1y_1+5y_1^2-3x_2^2-6x_2y_2-3y_2^2+6x_3^2+12x_3y_3+6y_3^2     \\
                     & = (5x_1^2-3x_2^2+6x_3^2)+(5y_1^2-3y_2^2+6y_3^2)+10x_1y_1-6x_2y_2+12x_3y_3 \\
                     & = 10x_1y_1-6x_2y_2+12x_3y_3 
                  \end{align*}
                  Which is not necessarily equal to 0 (e.g. $v_1=v_2=(1,1,1)$) so this set is not closed under addition and therefore not a subspace of $\mathbb{R}^3$.
          \end{enumerate}
  \end{enumerate}
\end{soln}
\end{document}