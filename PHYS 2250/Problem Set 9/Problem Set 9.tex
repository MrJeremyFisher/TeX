\documentclass[10pt]{article}

\usepackage[margin=0.75in]{geometry}
\usepackage{amsmath,amsthm,amssymb}
\usepackage{xcolor}
\usepackage{cancel}
\usepackage{graphicx}
\usepackage{changepage}
\usepackage{circuitikz}
\usepackage{pgfplots}
\usepackage{physics}
\usepackage{hyperref}
\usepackage{siunitx}
\usepackage[breakable]{tcolorbox}
\usepackage[inline]{enumitem}

\theoremstyle{definition}
\newtheorem{problem}{Problem}
\newtheorem{soln}{Solution}

\pgfplotsset{compat=newest}
\usetikzlibrary{lindenmayersystems}
\usetikzlibrary{arrows}
\usetikzlibrary{calc}

\definecolor{incolor}{HTML}{303F9F}
\definecolor{outcolor}{HTML}{D84315}
\definecolor{cellborder}{HTML}{CFCFCF}
\definecolor{cellbackground}{HTML}{F7F7F7}
\newcommand{\eq}{=}
\usetikzlibrary{positioning, fit, calc}
\pgfdeclarelayer{background}  
\pgfsetlayers{background,main}

\makeatletter
\newcommand{\boxspacing}{\kern\kvtcb@left@rule\kern\kvtcb@boxsep}
\makeatother
\newcommand{\prompt}[4]{
    \ttfamily\llap{{\color{#2}[#3]:\hspace{3pt}#4}}\vspace{-\baselineskip}
}

\newcommand{\thevenin}[2]{
  \begin{center}
    \begin{circuitikz} \draw
      (0,0) -- (2,0) to[battery1, l_=$V_{Th}\eq#1$] (2,2) 
      to[resistor, l_=$R_{Th}\eq#2$] (0,2)
      ;
      \draw [o-] (-.07,2.079);
      \draw [o-] (-.07,0.079);
    \end{circuitikz}
  \end{center}
}

\newcommand{\norton}[2]{
  \begin{center}
    \begin{circuitikz} \draw
      (0,0) -- (3,0) to[american current source, l_=$I_{N}\eq#1$] (3,2) -- (0,2) (2,0)
      to[resistor, l=$R_{N}\eq#2$] (2,2)
      ;
      \draw [o-] (-.07,2.079);
      \draw [o-] (-.07,0.079);
    \end{circuitikz}
  \end{center}
}

\newcommand{\highlight}[1]{\colorbox{yellow}{$\displaystyle #1$}}

\hypersetup{
    colorlinks=true,
    linkcolor=blue,
    filecolor=magenta,      
    urlcolor=cyan,
    pdftitle={Overleaf Example},
    pdfpagemode=FullScreen,
    }

\NewDocumentCommand{\evalat}{sO{\big}mm}{%
  \IfBooleanTF{#1}
   {\mleft. #3 \mright|_{#4}}
   {#3#2|_{#4}}%
}

\title{Physics 2250: Problem Set IX}
\author{Jeremy Favro}
\date{\today}

\begin{document}

\maketitle

% PROBLEM 1
\begin{problem} Determine both the DC impedance and the impedance at a frequency of $f=1\unit{\kilo\hertz}$ of the circuits shown below.\\
\begin{center}
  \begin{enumerate*}[label=(\alph*)]
    \item
    \begin{circuitikz}[scale=1.5]
            \draw (0,0) to[resistor, l_=$R\eq1\unit{\kilo\ohm}$, name=R] ++(1.5,0)
            to[inductor, l=$L\eq50\unit{\milli\henry}$] ++(1.5,0) ($(R.right) + (0.2,0)$) coordinate(D)
            (R.left) ++(-0.1,0) -- ++(0,0.5) coordinate(base) to[capacitor, l=$C\eq200\unit{\nano\farad}$, name=C] (base -| D) -- (D)
            ;
          \end{circuitikz}
    \item
    \begin{circuitikz}[scale=1.5]
            \draw (0,0) to[resistor, l_=$R\eq1\unit{\kilo\ohm}$, name=R] ++(1.5,0)
            to[capacitor, l=$C\eq200\unit{\nano\farad}$, name=C] ++(1.5,0) ($(R.right) + (0.2,0)$) coordinate(D)
            (R.left) ++(-0.1,0) -- ++(0,0.5) coordinate(base) to[inductor, l=$L\eq50\unit{\milli\henry}$] (base -| D) -- (D)
            ;
          \end{circuitikz}
  \end{enumerate*}
\end{center}
\end{problem}
\begin{soln} ~
  \begin{enumerate}[label=(\alph*)]
    \item \begin{align*}
      & Z = \left(\frac{1}{R}+j\omega C\right)^{-1}+j\omega L\\
      & = \frac{R}{1+j\omega CR}+j\omega L\\
      & = \frac{R+\left(j\omega L\right)\left(1+j\omega CR\right)}{1+j\omega CR} \\
      & = \frac{R\left(1-j\omega CR\right)+\left(j\omega L\right)\left(1+\left(\omega CR\right)^2\right)}{1+\left(\omega CR\right)^2} \\
      & = \frac{R-j\omega CR^2+j\omega L+j\omega L\left(\omega CR\right)^2}{1+\left(\omega CR\right)^2} \\
      & = \frac{R}{1+\left(\omega CR\right)^2}+\frac{-\omega CR^2+\omega L+\omega L\left(\omega CR\right)^2}{1+\left(\omega CR\right)^2}j \\
    \end{align*}
    Which means that the DC impedance ($\omega=0$) is just $R$, and impedance with a frequency of $1\unit{\kilo\hertz}$
    is $\frac{12500}{13}-\frac{1850}{13}j$
    \item \begin{align*}
      
    \end{align*}
  \end{enumerate}
\end{soln}
\end{document}