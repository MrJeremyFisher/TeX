\documentclass[10pt]{article}

\usepackage[margin=0.75in]{geometry}
\usepackage{amsmath,amsthm,amssymb}
\usepackage{xcolor}
\usepackage{cancel}
\usepackage{graphicx}
\usepackage{changepage}
\usepackage{circuitikz}
\usepackage{pgfplots}
\usepackage{physics}
\usepackage{hyperref}
\usepackage{siunitx}
\usepackage[breakable]{tcolorbox}
\usepackage[inline]{enumitem}

\theoremstyle{definition}
\newtheorem{problem}{Problem}
\newtheorem{soln}{Solution}

\pgfplotsset{compat=newest}
\usetikzlibrary{lindenmayersystems}
\usetikzlibrary{arrows}

\definecolor{incolor}{HTML}{303F9F}
\definecolor{outcolor}{HTML}{D84315}
\definecolor{cellborder}{HTML}{CFCFCF}
\definecolor{cellbackground}{HTML}{F7F7F7}
\newcommand{\eq}{=}
\usetikzlibrary{positioning, fit, calc}
\pgfdeclarelayer{background}  
\pgfsetlayers{background,main}

\makeatletter
\newcommand{\boxspacing}{\kern\kvtcb@left@rule\kern\kvtcb@boxsep}
\makeatother
\newcommand{\prompt}[4]{
    \ttfamily\llap{{\color{#2}[#3]:\hspace{3pt}#4}}\vspace{-\baselineskip}
}

\newcommand{\thevenin}[2]{
  \begin{center}
    \begin{circuitikz} \draw
      (0,0) -- (2,0) to[battery1, l_=$V_{Th}\eq#1$] (2,2) 
      to[resistor, l_=$R_{Th}\eq#2$] (0,2)
      ;
      \draw [o-] (-.07,2.079);
      \draw [o-] (-.07,0.079);
    \end{circuitikz}
  \end{center}
}

\newcommand{\norton}[2]{
  \begin{center}
    \begin{circuitikz} \draw
      (0,0) -- (3,0) to[american current source, l_=$I_{N}\eq#1$] (3,2) -- (0,2) (2,0)
      to[resistor, l=$R_{N}\eq#2$] (2,2)
      ;
      \draw [o-] (-.07,2.079);
      \draw [o-] (-.07,0.079);
    \end{circuitikz}
  \end{center}
}

\newcommand{\highlight}[1]{\colorbox{yellow}{$\displaystyle #1$}}

\hypersetup{
    colorlinks=true,
    linkcolor=blue,
    filecolor=magenta,      
    urlcolor=cyan,
    pdftitle={Overleaf Example},
    pdfpagemode=FullScreen,
    }

\NewDocumentCommand{\evalat}{sO{\big}mm}{%
  \IfBooleanTF{#1}
   {\mleft. #3 \mright|_{#4}}
   {#3#2|_{#4}}%
}

\title{Physics 2250: Problem Set VIII}
\author{Jeremy Favro}
\date{\today}

\begin{document}

\maketitle

% PROBLEM 1
\begin{problem}
Design a simple circuit comprising only a battery, a bulb (load), and simple switches, where the bulb lights up only when $q\cdot\overline{\left(r+s\right)}$ is true
\end{problem}
\begin{soln} ~\\
  a) Here if $q$ is closed and either $r$ or $s$ is also closed, the bulb is bypassed. If $q$ is open obviously no current reaches the bulb. 
  If both $r$ and $s$ are open (ergo $\overline{\left(r+s\right)}=1$ only true when $r,s=0$) the lamp sees current and turns on.
  \begin{center}
    \begin{circuitikz}
      \draw (0,0) to[battery1] ++(0,2) to[switch, l=$q$] ++(2,0) coordinate(J1) -- ++(0,-0.5) coordinate(J2) ++(-0.5, 0) -- ++(1, 0) to[switch, l=$r$] ++(0,-1)
      (J2) ++(-0.5,0) to[switch, l=$s$] ++(0,-1) -- ++(1,0) ++(-0.5,0) -- ++(0,-0.5)
      (J1) -- ++(1.5, 0) to[/tikz/circuitikz/bipoles/length=1cm, bulb] ++(0,-2) -- (0,0)
      ;
    \end{circuitikz}
  \end{center}
  b)
  
\end{soln}
\end{document}