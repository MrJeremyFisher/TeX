\documentclass[10pt]{article}

\usepackage[margin=0.75in]{geometry}
\usepackage{amsmath,amsthm,amssymb}
\usepackage{xcolor}
\usepackage{cancel}
\usepackage{graphicx}
\usepackage{changepage}
\usepackage{circuitikz}
\usepackage{pgfplots}
\usepackage{physics}
\usepackage{hyperref}
\usepackage{siunitx}
\usepackage[breakable]{tcolorbox}
\usepackage[inline]{enumitem}

\theoremstyle{definition}
\newtheorem{problem}{Problem}
\newtheorem{soln}{Solution}

\pgfplotsset{compat=newest}
\usetikzlibrary{lindenmayersystems}
\usetikzlibrary{arrows}

\definecolor{incolor}{HTML}{303F9F}
\definecolor{outcolor}{HTML}{D84315}
\definecolor{cellborder}{HTML}{CFCFCF}
\definecolor{cellbackground}{HTML}{F7F7F7}
\newcommand{\eq}{=}
\usetikzlibrary{positioning, fit, calc}
\pgfdeclarelayer{background}  
\pgfsetlayers{background,main}

\makeatletter
\newcommand{\boxspacing}{\kern\kvtcb@left@rule\kern\kvtcb@boxsep}
\makeatother
\newcommand{\prompt}[4]{
    \ttfamily\llap{{\color{#2}[#3]:\hspace{3pt}#4}}\vspace{-\baselineskip}
}

\newcommand{\thevenin}[2]{
  \begin{center}
    \begin{circuitikz} \draw
      (0,0) -- (2,0) to[battery1, l_=$V_{Th}\eq#1$] (2,2) 
      to[resistor, l_=$R_{Th}\eq#2$] (0,2)
      ;
      \draw [o-] (-.07,2.079);
      \draw [o-] (-.07,0.079);
    \end{circuitikz}
  \end{center}
}

\newcommand{\norton}[2]{
  \begin{center}
    \begin{circuitikz} \draw
      (0,0) -- (3,0) to[american current source, l_=$I_{N}\eq#1$] (3,2) -- (0,2) (2,0)
      to[resistor, l=$R_{N}\eq#2$] (2,2)
      ;
      \draw [o-] (-.07,2.079);
      \draw [o-] (-.07,0.079);
    \end{circuitikz}
  \end{center}
}

\newcommand{\highlight}[1]{\colorbox{yellow}{$\displaystyle #1$}}

\hypersetup{
    colorlinks=true,
    linkcolor=blue,
    filecolor=magenta,      
    urlcolor=cyan,
    pdftitle={Overleaf Example},
    pdfpagemode=FullScreen,
    }

\NewDocumentCommand{\evalat}{sO{\big}mm}{%
  \IfBooleanTF{#1}
   {\mleft. #3 \mright|_{#4}}
   {#3#2|_{#4}}%
}

\title{Physics 2250: Problem Set VIII}
\author{Jeremy Favro}
\date{\today}

\begin{document}

\maketitle

% PROBLEM 1
\begin{problem} ~
\begin{enumerate}[label=(\alph*)]
  \item Design a simple circuit comprising only a battery, a bulb (load), and simple switches, where the bulb lights up only when $q\cdot\overline{\left(r+s\right)}$ is true
  \item Fill out a complete truth table for the above expression.
\end{enumerate}
\end{problem}
\begin{soln} ~
  \begin{enumerate}[label=(\alph*)]
    \item Here if $q$ is closed and either $r$ or $s$ is also closed, the bulb is bypassed. If $q$ is open obviously no current reaches the bulb.
          If both $r$ and $s$ are open (ergo $\overline{\left(r+s\right)}=1$ only true when $r,s=0$) the lamp sees current and turns on.
          \begin{center}
            \begin{circuitikz}
              \draw (0,0) to[battery1] ++(0,2) to[switch, l=$q$] ++(2,0) coordinate(J1) -- ++(0,-0.5) coordinate(J2) ++(-0.5, 0) -- ++(1, 0) to[switch, l=$r$] ++(0,-1)
              (J2) ++(-0.5,0) to[switch, l=$s$] ++(0,-1) -- ++(1,0) ++(-0.5,0) -- ++(0,-0.5)
              (J1) -- ++(1.5, 0) to[/tikz/circuitikz/bipoles/length=1cm, bulb] ++(0,-2) -- (0,0)
              ;
            \end{circuitikz}
          \end{center}
    \item \begin{displaymath}
            \begin{array}{|c c c|c|c|}
              % |c c|c| means that there are three columns in the table and
              % a vertical bar ’|’ will be printed on the left and right borders,
              % and between the second and the third columns.
              % The letter ’c’ means the value will be centered within the column,
              % letter ’l’, left-aligned, and ’r’, right-aligned.
              q & r & s & \overline{\left(r+s\right)} & q\cdot\overline{\left(r+s\right)} \\ % Use & to separate the columns
              \hline % Put a horizontal line between the table header and the rest.
              T & T & T & F                           & F                                 \\
              T & T & F & F                           & F                                 \\
              T & F & F & T                           & T                                 \\
              F & F & F & T                           & F                                 \\
              F & F & T & F                           & F                                 \\
              F & T & T & F                           & F                                 \\
              T & F & T & F                           & F                                 \\
              F & T & F & F                           & F                                 \\
            \end{array}
          \end{displaymath}
  \end{enumerate}
\end{soln}
\newpage

% PROBLEM 2
\begin{problem}
For the following logic circuit,
\begin{center}
  \begin{circuitikz}
    \draw (0,0) node[ieeestd nor port, anchor=in 1](PQNOR){}
    (PQNOR.out) node[ieeestd nand port, anchor=in 1](PNRNAND){}
    (PNRNAND.out) node[ieeestd or port, anchor=in 1](OUTOUT){}
    (PNRNAND.in 2) node[ieeestd not port, rotate=90, scale=.5, anchor=out](NOT){}
    (NOT.in) node[ieeestd and port, anchor=in 1](RSAND){}
    (RSAND.out) -| (OUTOUT.in 2)
    (PQNOR.in 1) node[left]{$p$} (PQNOR.in 2) node[left]{$q$}
    (RSAND.in 1) -- ++(-1,0) node[left]{$r$} (RSAND.in 2) -- ++(-1,0) node[left]{$s$}
    ;
  \end{circuitikz}
\end{center}
\begin{enumerate}[label=(\alph*)]
  \item Write a simplified logic expression for the output. (Top marks will be given for an expression that cannot be simplified further.)
  \item Write a complete truth table for this circuit.
\end{enumerate}
\end{problem}
\begin{soln} ~
  \begin{enumerate}[label=(\alph*)]
    \item ~ \begin{center}
            \begin{circuitikz}
              \draw (0,0) node[ieeestd nor port, anchor=in 1](PQNOR){} (PQNOR.out) node[above right ]{$\overline{\left(p+q\right)}$}
              (PQNOR.out) -- ++(1,0) node[ieeestd nand port, anchor=in 1](PNRNAND){} (PNRNAND.out) node[above right]{$\overline{\overline{\left(p+q\right)}\overline{r}}$}
              (PNRNAND.out) -- ++(1.5,0) node[ieeestd or port, anchor=in 1](OUTOUT){} (OUTOUT.out) node[right]{$rs+\overline{\overline{\left(p+q\right)}\overline{r}}=rs+p+q+r=p+q+r+rs$}
              (PNRNAND.in 2) node[ieeestd not port, rotate=90, scale=.5, anchor=out](NOT){} (NOT.out) node[left]{$\overline{r}$}
              (NOT.in) node[ieeestd and port, anchor=in 1](RSAND){} (RSAND.out) node[above right]{$rs$}
              (RSAND.out) -| (OUTOUT.in 2)
              (PQNOR.in 1) node[left]{$p$} (PQNOR.in 2) node[left]{$q$}
              (RSAND.in 1) -- ++(-1,0) node[left]{$r$} (RSAND.in 2) -- ++(-1,0) node[left]{$s$}
              ;
            \end{circuitikz}
          \end{center}
    \item \begin{displaymath}
            \begin{array}{|c c c c|c|c|c|c}
              % |c c|c| means that there are three columns in the table and
              % a vertical bar ’|’ will be printed on the left and right borders,
              % and between the second and the third columns.
              % The letter ’c’ means the value will be centered within the column,
              % letter ’l’, left-aligned, and ’r’, right-aligned.
              p & q & r & s & rs & p+q & (p+q)r & p+q+r+rs \\ % Use & to separate the columns
              \hline % Put a horizontal line between the table header and the rest.
              F & F & F & F & F  & F   & F      & F        \\
              F & F & F & T & F  & F   & F      & F        \\
              F & F & T & F & F  & F   & F      & T        \\
              F & T & F & F & F  & T   & F      & T        \\
              T & F & F & F & F  & T   & F      & T        \\
              F & F & T & T & T  & F   & F      & T        \\
              F & T & T & T & T  & T   & T      & T        \\
              T & T & T & T & T  & T   & T      & T        \\
              T & F & F & T & F  & T   & F      & T        \\
              F & T & T & F & F  & T   & T      & T        \\
              T & F & T & F & F  & T   & T      & T        \\
              F & T & F & T & F  & T   & F      & T        \\
              T & F & T & T & T  & T   & T      & T        \\
              T & T & F & F & F  & T   & F      & T        \\
              T & T & T & F & F  & T   & T      & T        \\
              T & T & F & T & F  & T   & F      & T        \\
            \end{array}
          \end{displaymath}
  \end{enumerate}
\end{soln}
\newpage

% PROBLEM 3
\begin{problem}{\textbf{BONUS}}
Consider the circuit shown below with the two inputs $p$ and $q$ (of voltages $V_p$ and $V_q$). Note that in this logic circuit we are only interested in knowing when the inputs and outputs
voltages are high (HI) or low (LO), not theiractual value in volts. The convention is HI=1 and LO=0.
\begin{center}
  \begin{circuitikz}
    \draw \pgfextra{\ctikzset{bipoles/resistor/width=.4, bipoles/resistor/height=.2}}
    (0,0) node[left]{$V_q$} -- ++(0.25,0) to[resistor, l=$R_q$] ++(1,0) -- ++(0.85,0) node[npn, anchor=E](T){1}
    (T.base) to[resistor, l_=$R_p$] ++(-1,0) -- ++(-0.25,0) node[left]{$V_p$} 
    (T.collector) |- ++(1,1) node[npn, anchor=B](T2){2} (T2.collector) -- ++(1.5,0) node[right]{$V_{out}$}
    (T.emitter) -- ++(1,0) node[npn, anchor=B](T3){3} (T3.collector) -- ++(1,0) to[crossing] ++(0,2) -- ++(0,1)
    to[resistor, l=$R_{C2}$] ++(0,1) coordinate(VCC) -- ++(1,0) node[right]{$V_{CC}$} 
    (VCC) -- (VCC -| T.collector) to[resistor, l=$R_{C1}$] ++(0,-1) -- (T.collector)
    (T2.emitter) -- ++(1.25,0) coordinate(R) -- (T3.emitter -| R) coordinate(G) node[ground]{}
    (T3.emitter) -- (G)
    ;
  \end{circuitikz}
\end{center}
\begin{enumerate}[label=(\alph*)]
  \item Construct and complete a full truth table for $q$, $p$, and $V_{out}$ (and any intermediate conditions you wish to include). Highlight the $p$, $q$ conditions for which the output $V_{out}=$HI (or 1).
  \item Write a \underline{simplified} logic expression in terms of $p$ and $q$ for the network.
  \item Reproduce an equivalent logic network using only NOR gates.
\end{enumerate}
\end{problem}
\begin{soln} 
\end{soln}
\newpage

% PROBLEM 4
\begin{problem}
Consider the seven-segment (a-g) LED numeric display. To display a given ``decimal'' number (i.e. 0-9), various segments need to be powered. 
A 4-bit input (DCBA) is more than sufficient to create each of the 10 needed numerals. Let's presume that a 4-bit device is only being used to display 
the numerals 0-9 (and no other useful ``patterns'').
\begin{enumerate}[label=(\alph*)]
  \item Complete the (DCBA and a-g) portion of the truth table for the necessary inputs and outputs of the circuit that controls the LED display.
        An example for the numeral 6 has already been filled out. (for now, ignore the Decoder-4 column)
  \item How many inputs and how many outputs are needed to control the display?
  \item Consider the Decoder \#4 circuit, used for controlling one of the seven segments. Note how the circuit only uses NAND/OR logic.
        Fill in the Decoder \#4 column in the truth table and determine which segment (a-g) this circuit controls.
  \item Write an algebraic expression for the logic gate portion of Decoder \#4, then simplify the expression using logic algebra.
        (Hint: You should be able to simplify the expression to include only OR and NOT operations.)
  \item ~[\textbf{BONUS}] Construct the analogous NAND/OR logic circuit (using only NAND/OR gates) to control segment \textbf{f}. (My solution uses four gates. You may be able to do it with fewer)
\end{enumerate}
\end{problem}
\begin{soln} ~
  \begin{enumerate}[label=(\alph*)]
    \item \begin{displaymath}
            \begin{array}{|c|c|c|c|c|c|c|c|c}
              % |c c|c| means that there are three columns in the table and
              % a vertical bar ’|’ will be printed on the left and right borders,
              % and between the second and the third columns.
              % The letter ’c’ means the value will be centered within the column,
              % letter ’l’, left-aligned, and ’r’, right-aligned.
              Dec & DCBA & a & b & c & d & e & f & g \\ % Use & to separate the columns
              \hline % Put a horizontal line between the table header and the rest.
              1   & 0001 & 0 & 0 & 1 & 1 & 0 & 0 & 0 \\
              2   & 0010 & 0 & 0 & 1 & 0 & 1 & 1 & 1 \\
              3   & 0011 & 0 & 0 & 1 & 1 & 1 & 1 & 1 \\
              4   & 0100 & 1 & 0 & 1 & 1 & 0 & 1 & 0 \\
              5   & 0101 & 1 & 0 & 0 & 1 & 1 & 1 & 1 \\
              6   & 0110 & 1 & 1 & 0 & 1 & 1 & 1 & 1 \\
              7   & 0111 & 0 & 0 & 1 & 1 & 1 & 0 & 0 \\
              8   & 1000 & 1 & 1 & 1 & 1 & 1 & 1 & 1 \\
              9   & 1001 & 1 & 0 & 1 & 1 & 1 & 1 & 1 \\
              0   & 0000 & 1 & 1 & 1 & 1 & 1 & 0 & 1 \\
            \end{array}
          \end{displaymath}
    \item We are trying to represent 10 distinct input states in binary, so we need
    \item When the decoder's ``result'' (signal at the base of the transistor) is HI, the transistor will be on and the diode will in turn be ON. The opposite
          occurs when the output is LO as the transistor is cut off and therefore no current flows to activate the diode. 
          \begin{displaymath}
            \begin{array}{|c|c|c|c|c|c|c|c|c|c}
              % |c c|c| means that there are three columns in the table and
              % a vertical bar ’|’ will be printed on the left and right borders,
              % and between the second and the third columns.
              % The letter ’c’ means the value will be centered within the column,
              % letter ’l’, left-aligned, and ’r’, right-aligned.
              Dec & DCBA & a & b & c & d & e & f & g & Dec \,\#4 \\ % Use & to separate the columns
              \hline % Put a horizontal line between the table header and the rest.
              1   & 0001 & 0 & 0 & 1 & 1 & 0 & 0 & 0 & 1         \\
              2   & 0010 & 0 & 0 & 1 & 0 & 1 & 1 & 1 & 0         \\
              3   & 0011 & 0 & 0 & 1 & 1 & 1 & 1 & 1 & 1         \\
              4   & 0100 & 1 & 0 & 1 & 1 & 0 & 1 & 0 & 1         \\
              5   & 0101 & 1 & 0 & 0 & 1 & 1 & 1 & 1 & 1         \\
              6   & 0110 & 1 & 1 & 0 & 1 & 1 & 1 & 1 & 1         \\
              7   & 0111 & 0 & 0 & 1 & 1 & 1 & 0 & 0 & 1         \\
              8   & 1000 & 1 & 1 & 1 & 1 & 1 & 1 & 1 & 1         \\
              9   & 1001 & 1 & 0 & 1 & 1 & 1 & 1 & 1 & 1         \\
              0   & 0000 & 1 & 1 & 1 & 1 & 1 & 0 & 1 & 1         \\
            \end{array}
          \end{displaymath}
          As can be seen in the table, the $d$ segment is the segment that is off when the ``result'' is LO, meanining the transistor is cut off and the lamp is OFF.
    \item Traversing the circuit we get $\overline{B\overline{(A+C+D)}}$ then, by DeMorgan's Theorem, $\overline{BX}=\overline{B}+\overline{X}=\overline{B}+(A+C+D)$.
          This result is consistent with the above truth table.
    \item ~[\textbf{BONUS}] \begin{center}
            \begin{circuitikz}
              \draw (0,0) node[ieeestd nand port, anchor=in 1](NAND){} (NAND.in 1) -- (NAND.in 2) -- ++(-0.25, 0) node[left]{$B$}
              (NAND.out) node[ieeestd or port, anchor=in 2](OR1){} (OR1.in 1) node[left]{$A$}
              (OR1.out) node[ieeestd or port, anchor=in 1](OR2){}
              (OR2.in 2) -- ++(0,-1) node[ieeestd or port, anchor=out](OR3){} (OR3.in 1) node[left]{$C$} (OR3.in 2) node[left]{$D$}
              (OR2.out) node[right]{$\overline{B}+A+C+D$}
              ;
            \end{circuitikz}
          \end{center}
  \end{enumerate}
\end{soln}
\newpage

% PROBLEM 5
\begin{problem} ~
\begin{enumerate}[label=(\alph*)]
  \item $1100011_2+77_{10}=?_2$
  \item $\left(101_2\right)^3=?_2$
  \item $111.11_2=?_{10}$
  \item $17.34_{10}=?_2$
  \item $101.01_2+110.10_2=?_{10}$
\end{enumerate}
\end{problem}
\begin{soln} ~
  \begin{enumerate}[label=(\alph*)]
    \item First convert 77 to binary \begin{align*}
             & \frac{77}{2}=38R1                           \\
             & \frac{38}{2}=19R0                           \\
             & \frac{19}{2}=9R1                            \\
             & \frac{9}{2}=4R1                             \\
             & \frac{4}{2}=2R0                             \\
             & \frac{2}{2}=1R0                             \\
             & \frac{1}{2}=0R1\implies 77_{10}=1001101_2 
          \end{align*}
          Then add,
          \begin{tabular}{ccccccccc}
            ~ & (1) & ~ & ~ & (1) & (1) & (1) & (1) & ~           \\
            ~ & ~   & 1 & 0 & 0   & 1   & 1   & 0   & 1           \\
            ~ & +   & 1 & 1 & 0   & 0   & 0   & 1   & 1           \\
            \hline
            ~ & 1   & 0 & 1 & 1   & 0   & 0   & 0   & 0         
          \end{tabular}
          
    \item ~\\
          \begin{tabular}{cccccc}
            ~ & ~ & ~        & 1 & 0 & 1   \\
            ~ & ~ & $\cross$ & 1 & 0 & 1   \\
            \hline
            ~ & ~ & ~        & 1 & 0 & 1   \\
            ~ & ~ & 0        & 0 & 0 & ~   \\ 
            + & 1 & 0        & 1 & ~ & ~   \\ 
            \hline
            ~ & 1 & 1        & 0 & 0 & 1 
          \end{tabular} ~\\
          Then, \\
          \begin{tabular}{cccccccc}
            ~ & ~        & ~ & ~ & 1 & 0 & 1         \\
            ~ & $\cross$ & 1 & 1 & 0 & 0 & 1         \\
            \hline
            ~ & ~        & 1 & 1 & 0 & 0 & 1         \\
            ~ & 0        & 0 & 0 & 0 & 0             \\
            1 & 1        & 0 & 0 & 1                 \\ 
            \hline
            1 & 1        & 1 & 1 & 1 & 0 & 1       
          \end{tabular}
    \item $111.11_2=1\left(2^2+2^1+2^0+2^{-1}+2^{-2}\right)_{10}=7.75_{10}$
    \item First convert 17 to base 2
          \begin{align*}
             & \frac{17}{2}=8R1                           \\
             & \frac{8}{2}=4R0                            \\
             & \frac{4}{2}=2R0                            \\
             & \frac{2}{2}=1R0                            \\
             & \frac{1}{2}=0R1 \implies 17_{10}=10001_2 
          \end{align*}
          Then convert 0.34 to base 2. This process would repeat forever as $0.34=\frac{17}{50}$ and 50 is not a power of 2 so I'm stopping one a reasonable approximation is reached.
          \begin{align*}
             & 0.34\cdot 2 = 0+0.68                                       \\
             & 0.68\cdot 2 = 1+0.36                                       \\
             & 0.36\cdot 2 = 0+0.72                                       \\
             & 0.72\cdot 2 = 1+0.44                                       \\
             & 0.44\cdot 2 = 0+0.88                                       \\
             & 0.88\cdot 2 = 1+0.76                                       \\
             & 0.76\cdot 2 = 1+0.52                                       \\
             & 0.52\cdot 2 = 1+0.04\implies 0.34_{10}\approx 0.01010111 
          \end{align*}
          So $17.34_{10}\approx 10001.01010111_2$
    \item
          \begin{tabular}{ccccccc}
            ~ & (1) & ~ & ~ & ~ & ~ & ~     \\
            ~ & 1   & 0 & 1 & . & 0 & 1     \\
            + & 1   & 1 & 0 & . & 1 & 0     \\
            \hline  
            1 & 0   & 1 & 1 & . & 1 & 1   
          \end{tabular} \\
          Then $1011.11_2=(2^3+2^1+2^0)_{10}+(2^{-1}+2^{-2})_{10}=11.75_{10}$
  \end{enumerate}
\end{soln}
\end{document}