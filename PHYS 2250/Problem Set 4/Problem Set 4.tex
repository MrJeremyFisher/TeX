\documentclass[10pt]{article}

\usepackage[margin=0.75in]{geometry}
\usepackage{amsmath,amsthm,amssymb}
\usepackage{xcolor}
\usepackage{cancel}
\usepackage{graphicx}
\usepackage{changepage}
\usepackage{circuitikz}
\usepackage{pgfplots}
\usepackage{physics}
\usepackage{hyperref}
\usepackage{siunitx}
\usepackage[breakable]{tcolorbox}
\usepackage[inline]{enumitem}

\theoremstyle{definition}
\newtheorem{problem}{Problem}
\newtheorem{soln}{Solution}

\pgfplotsset{compat=newest}
\usetikzlibrary{lindenmayersystems}
\usetikzlibrary{arrows}

\definecolor{incolor}{HTML}{303F9F}
\definecolor{outcolor}{HTML}{D84315}
\definecolor{cellborder}{HTML}{CFCFCF}
\definecolor{cellbackground}{HTML}{F7F7F7}
\newcommand{\eq}{=}
\usetikzlibrary{positioning, fit, calc}
\pgfdeclarelayer{background}  
\pgfsetlayers{background,main}

\makeatletter
\newcommand{\boxspacing}{\kern\kvtcb@left@rule\kern\kvtcb@boxsep}
\makeatother
\newcommand{\prompt}[4]{
    \ttfamily\llap{{\color{#2}[#3]:\hspace{3pt}#4}}\vspace{-\baselineskip}
}

\newcommand{\thevenin}[2]{
  \begin{center}
    \begin{circuitikz} \draw
      (0,0) -- (2,0) to[battery1, l_=$V_{Th}\eq#1$] (2,2) 
      to[resistor, l_=$R_{Th}\eq#2$] (0,2)
      ;
      \draw [o-] (-.07,2.079);
      \draw [o-] (-.07,0.079);
    \end{circuitikz}
  \end{center}
}

\newcommand{\norton}[2]{
  \begin{center}
    \begin{circuitikz} \draw
      (0,0) -- (3,0) to[american current source, l_=$I_{N}\eq#1$] (3,2) -- (0,2) (2,0)
      to[resistor, l=$R_{N}\eq#2$] (2,2)
      ;
      \draw [o-] (-.07,2.079);
      \draw [o-] (-.07,0.079);
    \end{circuitikz}
  \end{center}
}

\newcommand\amplifier[2]{%1: name of this amplifier, %2 start coordinate, %3 R1, %4 R2
\draw #2 coordinate(#1-in) to[short] ++(0,0)
 node[op amp, noinv input up, anchor=+](#1-OA){#1};
}

\newcommand\aamplifier[2]{%1: name of this amplifier, %2 start coordinate, %3 R1, %4 R2
\draw #2 coordinate(#1-in) to[short] ++(0,0)
 node[op amp, noinv input up, anchor=+](#1-OA){#1};
}

\newcommand{\highlight}[1]{\colorbox{yellow}{$\displaystyle #1$}}

\hypersetup{
    colorlinks=true,
    linkcolor=blue,
    filecolor=magenta,      
    urlcolor=cyan,
    pdftitle={Overleaf Example},
    pdfpagemode=FullScreen,
    }

\NewDocumentCommand{\evalat}{sO{\big}mm}{%
  \IfBooleanTF{#1}
   {\mleft. #3 \mright|_{#4}}
   {#3#2|_{#4}}%
}

\title{Physics 2250: Problem Set IV}
\author{Jeremy Favro}
\date{\today \space- thank you for losing the key.}

\begin{document}

\maketitle

% PROBLEM 1
\begin{problem}
Consider a current-to-voltage converter that uses one op-amp and one resistor.
\begin{center}
  \begin{circuitikz} 
    \draw node[op amp, scale=0.75](A){A} % (a -| b) takes the y of a and the x of b!
    (A.-) -- ++(0, 1) coordinate(R) to[resistor, l=$R_f\eq1\unit{\kilo\ohm}$] (R -| A.out) -- (A.out) -- ++(.5, 0) node[above right]{$V_o$}
    (A.-) -- ++(-1, 0) node[above left]{$I_i$}
    (A.+) -- ++(-1, 0) -- ++(0,-0.5) node[ground]{}
    ;
    
  \end{circuitikz}
\end{center}
\end{problem}
\begin{soln} ~\\
  a) Using simplified op-amp approximations here means that no current will flow inside the op amp between the terminals. Therefore all of $I_i$ flows through $R_f$.
  Then, by the virtual ground approximation, $V^-\approxeq V^+=0 \implies I_i=\displaystyle\frac{V^--V_o}{R_f}=-\frac{V_o}{R_f}$ which can be rearranged to obtain $V_o=-I_iR_f$.
  \\
  b) 
  \begin{center}
    \begin{circuitikz} 
      % (a -| b) takes the y of a and the x of b!
      \draw (0,0) node[left]{$V^-$} -- (1,0) to[resistor, l_=$R_i\eq1\unit{\mega\ohm}$] ++(0,-2) -- ++(-1,0) node[left]{$V^+$} node[ground]{}
      (0.75,0) -- ++(0,1) to[resistor, l=$R_f\eq1\unit{\kilo\ohm}$] ++ (3,0) -- ++(0,-1) coordinate(J) -- ++(.25,0) node[right]{$V_o$}
      (J) to[resistor, l=$R_o\eq10\unit{\ohm}$] ++(-1.5, 0) -- ++(0,-.5) to[american controlled voltage source, l=$A(V^+-V^-)$] ++ (0,-1.5) node[ground]{}
      ;
    \end{circuitikz}
  \end{center}
  For some reason this question really messed with me, likely because I initially put it off until later and then got sick and am now attempting it while sick. Anyway:\\
  The current coming in at the $V^-$ node, $I_i$ (yarrr), splits between the $R_f$ and $R_i$ branches. Knowing that $V^+$ will be zero because of the ground point $V_o$
  becomes $-V^-$ and $V^-=I_iR_i\displaystyle\frac{R_f+R_o}{R_f+R_o+R_i}$ which is the voltage drop across $R_i$ due to the current through that branch. This means that
  $V_o=-I_iR_i\displaystyle\frac{R_f+R_o}{R_f+R_o+R_i}$. To compare this to part a) it's easiest to sub in the given numbers and partially evaluate
  \begin{align*}
    & V_o=-I_iR_i\frac{R_f+R_o}{R_f+R_o+R_i} \\
    & V_o=-I_i\frac{10^6\left(10^3+10\right)}{10^3+10+10^6} \\
    & V_o\approxeq-1009\cdot I_i \\
  \end{align*}
  which is only slightly greater than the answer to part a) ($V_o=1000\cdot I_i$).
\end{soln}

% PROBLEM 2
\begin{problem}
Determine the value of the leftmost resistor in the circuit below.
\begin{center}
  \begin{circuitikz} 
    \draw node[op amp, scale=0.75](A){A} % (a -| b) takes the y of a and the x of b!
    (A.-) -- ++(-3, 0) -- ++(0,-4) coordinate(G)
    (A.+) to[resistor, l=$R_?\eq?$] ++(-2.25,0) coordinate(R?) (G -| R?) to[battery1, l_=$12\unit{\volt}$] (R?) 
    (A.+) to[resistor, l=$R_2\eq4\unit{\kilo\ohm}$] ++(0,-1.5) coordinate(R2) to[resistor, l=$R_1\eq10\unit{\kilo\ohm}$] (G -| R2) coordinate(R1) node[ground]{}
    (R2) -- (R2 -| A.out) -- ++(0.5,0) coordinate(F) (A.out) -* ++(0.5,0) node[above right]{$8\unit{\volt}$} -- (F)
    (G) -- (G -| A.out) -- ++(0.5,0)
    ;
  \end{circuitikz}
\end{center}
\end{problem}
\begin{soln} Defining the potential at the ground junction as $0\unit{\volt}$ we get that the potential across $R_1$ is $8V$ and by virtual ground the potential
  at the positive terminal is $0\unit{\volt}$ which means that the voltage drop across $R_2$ must be $8\unit{\volt}$ which implies that $I_{R2}=\displaystyle\frac{-8\unit{\volt}}{4\unit{\kilo\ohm}}=-2\unit{\milli\ampere}$ with the negative
  sign indicating that the current is flowing upwards because the potential drop across $R_2$ goes from the bottom up. In order to satisfy the loop law
  the potential after $R_?$ must be $-12\unit{\volt}$ so $V_?=-12\unit{\volt}=I_{R2}R_? \implies R_?=\frac{-12\unit{\volt}}{-12\unit{\milli\ampere}}=6\unit{\kilo\ohm}$
  
\end{soln}
% PROBLEM 3
\begin{problem}[\textbf{BONUS}]
Determine the output voltage, $V_o$ in the circuit below.
\begin{center}
  \begin{circuitikz}[scale=1.1] 
    % (a -| b) takes the y of a and the x of b!
    \draw (0,0) to[battery1, l_=$V_A\eq12\unit{\volt}$] (0,-2) -- (2,-2)
    to[american current source, l_=$I_B\eq6\unit{\ampere}$] (2,0)
    to[resistor, l_=$R_1\eq1\unit{\ohm}$] (0,0) (2,0)
    to[resistor, l=$R_2\eq2\unit{\ohm}$] (4,0)
    to[resistor, l=$R_3\eq3\unit{\ohm}$] (4,-2) -- (2,-2)
    (4,0) -- (5.75,0) node[op amp,  noinv input up, scale=0.75, anchor=+](A){A}
    (A.-) -- ++(0,-1) coordinate(J) to[resistor, l=$R_4\eq2\unit{\ohm}$] ++(0,-1.5) node[ground]{}
    (A.out) -- (J -| A.out) to[resistor, l_=$R_5\eq10\unit{\ohm}$] (J)
    ;
  \end{circuitikz}
\end{center}
\end{problem}
\begin{soln} 
  The right side of the circuit can be solved a number of ways to determine that the voltage where it connects to the positive terminal
  of the op-amp is $V^+=3\unit{\volt}$. Assuming the amplifier is standard (high gain), it will output the maximum voltage that it can as $V_o=A(V^+-V^-)$
  and though $V^-$ will only be slightly less than $V^+$ (whatever the voltage drop across $R_5$ is), the gain factor will increase that slight difference significantly.
  I believe the standard maximum output for an op-amp is $15\unit{\volt}$, so $V_o=15\unit{\volt}$ provided this is a standard op-amp.
\end{soln}
% PROBLEM 4
\newpage
\begin{problem}
What is the output $V_o$ of the following circuit
\begin{center}
  \begin{circuitikz} 
    \draw node[op amp, scale=0.75](A){A} % (a -| b) takes the y of a and the x of b!
    (A.+) -- ++(0,-1.25) coordinate(G) node[ground]{}
    (A.-) ++(-0.5,0) -- ++(0,1) coordinate(R3S) to[resistor, l=$R_3\eq5\unit{\kilo\ohm}$] (R3S -| A.out) -- (A.out)
    (A.-) -- ++(-1,0) coordinate(J1) to[resistor, l=$R_2\eq5\unit{\kilo\ohm}$] ++(-1.5,0) -- ++(-0.5,0) coordinate(VX) node[left]{$V_y\eq2\unit{\volt}$}
    (J1) -- ++(0,1) to[resistor, l_=$R_1\eq1\unit{\kilo\ohm}$] ++(-1.5,0) -- ++(-0.5,0) node[left]{$V_x\eq3\unit{\volt}$}
    (A.out) -- ++(2.5,0) to[resistor, l=$R_4\eq3\unit{\kilo\ohm}$] ++(1.5,0) -- ++(0, -1) coordinate(J2) 
    to[resistor, l=$R_5\eq5\unit{\kilo\ohm}$] ++(-1.5,0) -- ++(-0.5,0) node[left]{$V_z\eq1\unit{\volt}$}
    (J2) -- ++(1,0) node[op amp, scale=0.75, anchor=-](A1){A1} (A1.+) -- (A1.+ -| G)
    (J2) ++(0.5,0) -- ++(0,1) coordinate(R6S) to[resistor, l=$R_6\eq5\unit{\kilo\ohm}$] (R6S -| A1.out) -- (A1.out) -- ++(1,0) node[right]{$V_o$}
    ;
    \draw     
    node[fit={($(VX.north)+(-1.25, 1.4)$) ($(A.out)+(0,-.5)$)}, draw, dashed, label={Circuit A}, inner sep=10pt]{}
    node[fit={($(J2.west)+(-3.2, 1.4)$) ($(A1.out)+(1.4,-.5)$)}, draw, dashed, label={Circuit B}, inner sep=10pt]{}
    ;
  \end{circuitikz}
\end{center}
\end{problem}
\begin{soln} 
  We can break this circuit down into two subcircuits (illustrated in the problem statement to avoid repetition here), circuit A and circuit B. To determine
  the output voltage of circuit A, $V_{oA}$, we start by analyzing current flow. The currents through $R_1$ and $R_2$ created by their respective voltages
  combine at the negative terminal junction into a single current, all of which travels through $R_3$ by the ideal amplifier approximations ($R_i\to\infty$).
  Mathematically this means that $I=I_1+I_2$ and that $I_1=\displaystyle\frac{V_x}{R_1}$, $I_2=\displaystyle\frac{V_y}{R_2}$, and $I=-\displaystyle\frac{V_{oA}}{R_3}$
  which means that $\displaystyle\frac{V_x}{R_1}+\frac{V_y}{R_2}=-\frac{V_{oA}}{R_3}\implies V_{oA}=-R_3\left(\frac{V_x}{R_1}+\frac{V_y}{R_2}\right)$.
  
  \noindent Next for circuit B we basically perform the same steps except $I_4=-\displaystyle\frac{R_3}{R_4}\left(\frac{V_x}{R_1}+\frac{V_y}{R_2}\right)$, $I_5=\displaystyle\frac{V_z}{R_5}$, and $I=-\displaystyle\frac{V_{oB}}{R_6}$.
  Again we also have that $I=I_4+I_5$ which means that $\displaystyle-\frac{R_3}{R_4}\left(\frac{V_x}{R_1}+\frac{V_y}{R_2}\right)+\frac{V_z}{R_5}=-\displaystyle\frac{V_{oB}}{R_6}
    \implies V_{oB}=-R_6\left[-\frac{R_3}{R_4}\left(\frac{V_x}{R_1}+\frac{V_y}{R_2}\right)+\frac{V_z}{R_5}\right]$. Finally, $V_{oB}=
    \displaystyle V_o=-5\unit{\kilo\ohm}\left[-\frac{5\unit{\kilo\ohm}}{3\unit{\kilo\ohm}}\left(\frac{3\unit{\volt}}{1\unit{\kilo\ohm}}+\frac{2\unit{\volt}}{5\unit{\kilo\ohm}}\right)+\frac{1\unit{\volt}}{5\unit{\kilo\ohm}}\right]
    =\frac{82}{3}\unit{\volt}$
\end{soln}
% PROBLEM 5
\begin{problem}[\textbf{BONUS}]
You are to be a writer for the most boring episode of MacGyver. Our hero needs to
briefly power a device with -5V, but only has a defective battery. In testing this battery, MacGyver
finds that the voltage steadily (linearly) ramps down from 10V to 0V over precisely one minute.
Conveniently, there is a box of op-amps, wires, resistors, and capacitors. Devise a circuit that
accomplishes the desired task. BRIEFLY explain in words the basic idea behind your circuit
design.
\end{problem}
\begin{soln} ~\\
  \begin{center}
    \begin{circuitikz} 
      \draw node[op amp, scale=0.75](A){A} % (a -| b) takes the y of a and the x of b!
      (A.+) -- ++(0,-1.25) coordinate(G) node[ground]{}
      (A.-) ++(-0.5,0) -- ++(0,1) coordinate(R3S) to[resistor, l=$R_d\eq1\unit{\mega\ohm}$] (R3S -| A.out) -- (A.out)
      (A.-) -- ++(-1,0) coordinate(J1) to[capacitor, l=$C\eq500\unit{\nano\farad}$] ++(-1.5,0)
      (A.out) -- ++(1,0) to[resistor, l=$10\unit{\ohm}$] ++(1.5,0) coordinate(J2) 
      (J2) -- ++(1,0) node[op amp, scale=0.75, anchor=-](A1){A1} (A1.+) -- ++(0, -0.75) coordinate(G2) -- (G2 -| G)
      (J2) ++(0.5,0) -- ++(0,1) coordinate(R6S) to[resistor, l=$10\unit{\ohm}$] (R6S -| A1.out) -- (A1.out) -- ++(1,0) node[right]{$V_o\eq-5\unit{\volt}$}
      ;
      \draw     
      node[fit={($(J1.north)+(-1.4, 1.4)$) ($(A.out)+(0,-.5)$)}, draw, dashed, label={Differentiator}, inner sep=10pt]{}
      node[fit={($(J2.west)+(-1.5, 1.4)$) ($(A1.out)+(0,-.5)$)}, draw, dashed, label={Inverting Amplifier}, inner sep=10pt]{}
      ;
    \end{circuitikz}
  \end{center}
  
  The idea here is that the differentiator converts the linear ramp-down voltage of the battery to a constant signal with voltage given by $10R_dC$. The goal
  for this part of the circuit was to have an output of $5\unit{\volt}$ that I could then invert with an inverting amplifier with a gain of 1. I fixed $C=500\unit{\nano\farad}$
  and then solved for $R_d$ after trying a few other capacitors which didn't have a nice equivalent resistor in the given set. The second part of the circuit was just pulled from my notes and the resistors chosen so that the amplifier had an effective gain of 1 meaning that 
  $V_o=\displaystyle -V_i=-5\unit{\volt}$.
\end{soln}
\end{document}