\documentclass[10pt]{article}

\usepackage[margin=0.75in]{geometry}
\usepackage{amsmath,amsthm,amssymb}
\usepackage{xcolor}
\usepackage{cancel}
\usepackage{graphicx}
\usepackage{changepage}
\usepackage{circuitikz}
\usepackage{pgfplots}
\usepackage{physics}
\usepackage{hyperref}
\usepackage[breakable]{tcolorbox}
\usepackage[inline]{enumitem}

\theoremstyle{definition}
\newtheorem{problem}{Problem}
\newtheorem{soln}{Solution}

\pgfplotsset{compat=newest}
\usetikzlibrary{lindenmayersystems}
\usetikzlibrary{arrows}

\definecolor{incolor}{HTML}{303F9F}
\definecolor{outcolor}{HTML}{D84315}
\definecolor{cellborder}{HTML}{CFCFCF}
\definecolor{cellbackground}{HTML}{F7F7F7}
\newcommand{\eq}{=}
\tikzset
{%
  axes/.style={thick,-latex},
  cylinder/.style={right color=blue!80,left color=white,fill opacity=0.7},
  paraboloid back/.style={left color=magenta!80,fill opacity=0.4},
  paraboloid front/.style={left color=white, right color=magenta!80,fill opacity=0.4},
}

\makeatletter
\newcommand{\boxspacing}{\kern\kvtcb@left@rule\kern\kvtcb@boxsep}
\makeatother
\newcommand{\prompt}[4]{
    \ttfamily\llap{{\color{#2}[#3]:\hspace{3pt}#4}}\vspace{-\baselineskip}
}


\hypersetup{
    colorlinks=true,
    linkcolor=blue,
    filecolor=magenta,      
    urlcolor=cyan,
    pdftitle={Overleaf Example},
    pdfpagemode=FullScreen,
    }

\NewDocumentCommand{\evalat}{sO{\big}mm}{%
  \IfBooleanTF{#1}
   {\mleft. #3 \mright|_{#4}}
   {#3#2|_{#4}}%
}

\title{Physics 2250: Problem Set I}
\author{Jeremy Favro}
\date{\today}

\begin{document}

\maketitle

% PROBLEM 1
\begin{problem}
Determine the resistance of resistors A and B in the figure (not included here) below. (A: Red, Red, Red, Red; B: Blue, Violet, Green, Brown).
\end{problem}
\begin{soln} 
  A: $2.2\cdot100\Omega\pm2\%=2.2\mathrm{k}\Omega\pm44\Omega$;
  B: $6.7\cdot100000\Omega\pm1\%=6.7\mathrm{M}\Omega\pm6.7\mathrm{k}\Omega$
\end{soln}

% PROBLEM 2
\begin{problem}
Consider the 5-resistor circuit shown below. Redrawn using circuitikz for ease of labeling in \LaTeX
\end{problem}
\begin{soln} ~\\
  \begin{center}
    \begin{circuitikz} \draw
      (0,4) to[battery1, l_=6V] (0,0) (0,4)
      -- (2,4) to[resistor, l_=$R_1\eq1\Omega$] (2,0) to[resistor, l_=$R_3\eq3\Omega$] (4,0) 
      -- (4,1) -- (3,1) -- (5,1) to [resistor, l_=$R_5\eq2\Omega$] (5,3) -- (3,3) to[resistor, l=$R_4\eq2\Omega$] (3,1) 
      (4,3) -- (4,4) to[resistor, l_=$R_2\eq2\Omega$] (2,4) (2,0) -- (0,0)
      ; 
      
      \draw [->] (6.5,3) -- (7.5,3) node[midway,above] {simplifies to};
      
      \draw 
      (9,4) to[battery1, l_=6V] (9,0) (9,4)
      -- (11,4) to[resistor, l_=$\displaystyle{\frac{7}{6}}\Omega$] (11,0) -- (9,0)
      ;
    \end{circuitikz}
  \end{center}
  So, $V=6\mathrm{V}; R_{tot}=\displaystyle{\frac{7}{6}}\Omega \therefore I_{tot}=7\mathrm{A}$. \\
  a) $P=IV=7\mathrm{A}\cdot6\mathrm{V}=42\mathrm{W}$\\
  b) By the current divider rule the current entering 
  $R_1$ is $\displaystyle{I_{tot}\cdot\frac{R_2+(\frac{1}{R_4}+\frac{1}{R_5})^{-1}+R_3}{R_{tot}}}=7\mathrm{A}\cdot\displaystyle{\frac{6\Omega}{7\Omega}}=6\mathrm{A}$
  which means that 1A enters $R_2$ and is evenly split between $R_4$ and $R_5$ which both cause a voltage drop of $0.5\mathrm{A}\cdot2\Omega=1\mathrm{V}$ because they are of equivalent resistance and by Kirchoff's junction law $R_3$ therefore 
  experiences the same 1A of current as $R_2$.
\end{soln}

% PROBLEM 3
\begin{problem}
Consider the network shown below. Again redrawn in circuitikz (the voltage drop arrow almost clipping is killing me).
\end{problem}
\begin{soln} ~\\
  \begin{center}
    \begin{circuitikz} \draw
      (4,2) to[battery1, l=$V_B\eq5\mathrm{V}$, o-] (4,0) (4,1) -- (4,2) to[battery1, l_=$V_A\eq10\mathrm{V}$] (4,4) 
      to[resistor, l=$R_4\eq1\Omega$] (0,4) to[resistor, l=$R_3\eq1\Omega$] (0,2)
      to[resistor, l_=$R_2\eq1\Omega$, -o] (2,2) -- (4,2) (0,2) -- (0,0) 
      to[resistor, v_<=$V_1$, l=$R_1\eq1\Omega$] (2,0) -- (4,0)
      ;
      \draw (4,2) node[above left] {$\epsilon_1$} (2,2) node[above left] {$\epsilon_2$};
      \draw [Latex-, very thick] (2.25,2) -- (3.5,2) node[midway,above] {$I^*$};
      \draw [->, very thick] (0,2) -- (0,1) node[midway,left] {$I_2$};
      \draw [->, very thick] (0,2) -- (0,2.4) node[midway,left] {$I_1$};
    \end{circuitikz}
  \end{center}
  a) Going around the upper loop clockwise and the lower loop counter-clockwise starting at $\epsilon_1$ we get $-I^*R_2-I_1R_3-I_1R_4+V_A=0$ for the upper loop and $-I^*-I_2R_1+V_B=0$ for the lower loop. By KCL we know that
  $I^*=I_1+I_2$. Solving the upper loop equation for $I_1$ we get that $I_1=\frac{-I^*+10V}{2\Omega}$. Doing the same for the lower loop equation we get that $I_2=-I^*+5$.
  Therefore
  \begin{align*}
     & I^*=\frac{-I^*+10\mathrm{V}}{2\Omega}-I^*+5\mathrm{V}  \\
     & 2\Omega I^*=-I^*+10\mathrm{V}-2\Omega I^*+10\mathrm{V} \\
     & 2\Omega I^*=-3\Omega I^*+20\mathrm{V}                  \\
     & 5\Omega I^*=20\mathrm{V}                               \\
     & I^*=4\mathrm{A}
  \end{align*} ~\\
  So $I^*$ is 4 Amps in the direction of the loops at $\epsilon_1$ (towards $R_2$)\\
  b) 0V, I think? There's no resistor or anything there, and a wire is to be treated as having zero resistance so it would have zero voltage drop by Ohm's law.\\
  c) Now that we know $I^*$ we can solve the loop equations to get that $I_1=3\mathrm{A}$ \& $I_2=1\mathrm{A}$ so the voltage drop across $R_1$ is $V=I_2R_1=1\mathrm{V}$
\end{soln}

% PROBLEM 4
\begin{problem}
\textbf{[BONUS]} If the source voltage in the figure below is $V_S=12\mathrm{V}$, what is the potential at node $\epsilon$?
\end{problem}
\begin{soln} ~\\
  \begin{center}
    \begin{circuitikz} \draw
      (0,2) to[battery1, l_=$V_S\eq12\mathrm{V}$,] (0,0) to[resistor, l=$R$] (2,0) 
      to[resistor, l_=$R$] (2,2) -- (4,2)
      to[resistor, l=$R$, -o] (4,0) -- (6,0)
      to[resistor, l_=$R$] (6,2) -- (0,2) (2,0) -- (4,0)
      ;
      \draw (4,0) node[below] {$\epsilon$};
    \end{circuitikz}
  \end{center}
  \begin{align*}
     & \epsilon = V_S-I_{tot}\mathrm{\colorbox{yellow}{$R_{before\,\epsilon}$}} \\
  \end{align*}
  \begin{align*}
     & R_{eq} = \mathrm{\colorbox{yellow}{$(\frac{1}{R}+\frac{1}{R}+\frac{1}{R})^{-1}$}}+R \\
     & R_{eq} = \frac{4R}{3}\Omega                                                         \\
  \end{align*}
  \begin{align*}
     & I_{tot} = \frac{V_S}{R_{eq}}                      \\
     & I_{tot} = \frac{12\mathrm{V}}{\frac{4R}{3}\Omega} \\
     & I_{tot} = \frac{9}{R}\mathrm{A}                   \\
  \end{align*}
  \begin{align*}
     & \epsilon = 12\mathrm{V}-\frac{9}{R}\mathrm{A}\frac{R}{3}\Omega \\
     & \epsilon = 9\mathrm{V}                                         \\
  \end{align*}
\end{soln}
\end{document}