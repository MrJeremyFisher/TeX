\documentclass[10pt]{article}

\usepackage[margin=0.75in]{geometry}
\usepackage{amsmath,amsthm,amssymb}
\usepackage{xcolor}
\usepackage{cancel}
\usepackage{graphicx}
\usepackage{changepage}
\usepackage{circuitikz}
\usepackage{pgfplots}
\usepackage{physics}
\usepackage{hyperref}
\usepackage{siunitx}
\usepackage[breakable]{tcolorbox}
\usepackage[inline]{enumitem}

\theoremstyle{definition}
\newtheorem{problem}{Problem}
\newtheorem{soln}{Solution}

\pgfplotsset{compat=newest}
\usetikzlibrary{lindenmayersystems}
\usetikzlibrary{arrows}
\usetikzlibrary{calc}

\definecolor{incolor}{HTML}{303F9F}
\definecolor{outcolor}{HTML}{D84315}
\definecolor{cellborder}{HTML}{CFCFCF}
\definecolor{cellbackground}{HTML}{F7F7F7}
\newcommand{\eq}{=}
\newcommand{\ui}{\mathbf{i}}
\newcommand{\uj}{\mathbf{j}}
\newcommand{\uk}{\mathbf{k}}
\usetikzlibrary{positioning, fit, calc}
\pgfdeclarelayer{background}  
\pgfsetlayers{background,main}

\makeatletter
\newcommand{\boxspacing}{\kern\kvtcb@left@rule\kern\kvtcb@boxsep}
\makeatother
\newcommand{\prompt}[4]{
    \ttfamily\llap{{\color{#2}[#3]:\hspace{3pt}#4}}\vspace{-\baselineskip}
}

\newcommand{\thevenin}[2]{
  \begin{center}
    \begin{circuitikz} \draw
      (0,0) -- (2,0) to[battery1, l_=$V_{Th}\eq#1$] (2,2) 
      to[resistor, l_=$R_{Th}\eq#2$] (0,2)
      ;
      \draw [o-] (-.07,2.079);
      \draw [o-] (-.07,0.079);
    \end{circuitikz}
  \end{center}
}

\newcommand{\norton}[2]{
  \begin{center}
    \begin{circuitikz} \draw
      (0,0) -- (3,0) to[american current source, l_=$I_{N}\eq#1$] (3,2) -- (0,2) (2,0)
      to[resistor, l=$R_{N}\eq#2$] (2,2)
      ;
      \draw [o-] (-.07,2.079);
      \draw [o-] (-.07,0.079);
    \end{circuitikz}
  \end{center}
}

\newcommand{\highlight}[1]{\colorbox{yellow}{$\displaystyle #1$}}

\newcommand{\ti}[1]{\widetilde{#1}}

\title{Math 2120H: Assignment V}
\author{Jeremy Favro (0805980) \\ Trent University, Peterborough, ON, Canada}
\date{\today}

\begin{document}
\maketitle

% PROBLEM 1
\begin{problem}
Verify that the below vector field is conservative and find a potential function for $\mathbf{F}$
$$\mathbf{F}=e^{y+2z}\left(\ui+x\uj+2x\uk\right)$$
\end{problem}
\begin{soln}
  For conservative fields
  $$\frac{\partial M}{\partial y}=\frac{\partial N}{\partial x},
    \frac{\partial N}{\partial z}=\frac{\partial P}{\partial y},
    \frac{\partial P}{\partial x}=\frac{\partial M}{\partial z}.
  $$
  Here $M=e^{y+2z}$, $N=xe^{y+2z}$, $P=2xe^{y+2z}$ so,
  \begin{align*}
    \frac{\partial M}{\partial y} & =e^{y+2z},
                                  & \frac{\partial M}{\partial z} & =2e^{y+2z}  \\
    \frac{\partial N}{\partial x} & =e^{y+2z},
                                  & \frac{\partial N}{\partial z} & =2xe^{y+2z} \\
    \frac{\partial P}{\partial y} & =2xe^{y+2z},
                                  & \frac{\partial P}{\partial x} & =2e^{y+2z}
  \end{align*}
  which satisfies the previous set of equalities so the field is conservative. For the potential function we find
  $f(x,y,z)\int M\,dx=\int e^{y+2z}\,dx=xe^{y+2z}+G(y,z)$, then differentiate with respect to $y$ and $z$ to determine
  $G$. First with respect to $y$,
  $$\frac{\partial f}{\partial y}=N=xe^{y+2z}=xe^{y+2z}+\frac{\partial G}{\partial y}\implies \frac{\partial G}{\partial y}=0\therefore G(y,z)=G(z)$$
  Then with respect to $z$,
  $$\frac{\partial f}{\partial z}=P=2xe^{y+2z}=2xe^{y+2z}+\frac{\partial G}{\partial z}\implies \frac{\partial G}{\partial z}=0\therefore G=C$$
  so $f(x,y,z)=xe^{y+2z}+C$ where $C$ is a constant of integration.
\end{soln}

% PROBLEM 2
\begin{problem}
Find a potential function for the field below and evaluate the integral
$$\int_{(1,1,1)}^{(1,2,3)}3x^2\,dx+\frac{z^2}{y}\,dy+2z\ln y\,dz$$
\end{problem}
\begin{soln}
  Here we first ensure that this field is conservative,
  \begin{align*}
    \frac{\partial M}{\partial y} & =0,
                                  & \frac{\partial M}{\partial z} & =0            \\
    \frac{\partial N}{\partial x} & =0,
                                  & \frac{\partial N}{\partial z} & =\frac{2z}{y} \\
    \frac{\partial P}{\partial y} & =\frac{2z}{y},
                                  & \frac{\partial P}{\partial x} & =0
  \end{align*}
  which it is. Then we follow the standard procedure to determine $f(x,y,z)$,
  $$f(x,y,z)=\int M\,dx=x^3+G(y,z)$$
  then we differentiate with respect to $y$,
  $$\frac{\partial f}{\partial y}=\frac{z^2}{y}=0+\frac{\partial G}{\partial y}
    \implies G(y,z)=z^2\ln y+C$$
  then with respect to $z$,
  $$\frac{\partial f}{\partial z}=2z\ln y=0+\frac{\partial G}{\partial z}
    \implies G(z,y)=z^2\ln y + C$$
  So $f(x,y,z)=x^3+z^2\ln y + C$
\end{soln}

% PROBLEM 3
\begin{problem}
Find the outward flux of the field $\mathbf{F} = xy\ui + y^2\uj$ over the boundary of the region enclosed by
the curve $y = x^2$ and the line $y = x$.
\end{problem}
\begin{soln}
  Here the divergence of the field is $3y$. By Green's theorem we then evaluate the integral of the divergence over the given region,
  \begin{align*}
     & =\int_{0}^{1}\int_{x^2}^{x}3y\,dydx \\
     & =\frac{3}{2}\int_{0}^{1}x^2-x^4\,dx \\
     & =\frac{1}{5}
  \end{align*}
\end{soln}

% PROBLEM 4
\begin{problem}
Apply Green's Theorem to evaluate the integral below
$$\oint_C 3y\,dx+2x\,dy,\qquad\text{Where $C$ is the boundary of } 0\leq x\leq\pi,\,0\leq y\leq \sin x$$
\end{problem}
\begin{soln}
  Green's Theorem states that $$\oint M\,dx+N\,dy=\iint_R\frac{\partial N}{\partial x}-\frac{\partial M}{\partial y}\,dx\,dy.$$
  Applying this here we get
  $$\oint_C 3y\,dx+2x\,dy=\int_0^\pi\int_{0}^{\sin x}-1\,dydx=\eval{\cos x}_{0}^{\pi}=-2$$
\end{soln}
\end{document}