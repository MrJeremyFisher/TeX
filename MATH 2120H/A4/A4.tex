\documentclass[10pt]{article}

\usepackage[margin=0.75in]{geometry}
\usepackage{amsmath,amsthm,amssymb}
\usepackage{xcolor}
\usepackage{cancel}
\usepackage{graphicx}
\usepackage{changepage}
\usepackage{circuitikz}
\usepackage{pgfplots}
\usepackage{physics}
\usepackage{hyperref}
\usepackage{siunitx}
\usepackage[breakable]{tcolorbox}
\usepackage[inline]{enumitem}

\theoremstyle{definition}
\newtheorem{problem}{Problem}
\newtheorem{soln}{Solution}

\pgfplotsset{compat=newest}
\usetikzlibrary{lindenmayersystems}
\usetikzlibrary{arrows}
\usetikzlibrary{calc}

\definecolor{incolor}{HTML}{303F9F}
\definecolor{outcolor}{HTML}{D84315}
\definecolor{cellborder}{HTML}{CFCFCF}
\definecolor{cellbackground}{HTML}{F7F7F7}
\newcommand{\eq}{=}
\newcommand{\ui}{\mathbf{i}}
\newcommand{\uj}{\mathbf{j}}
\newcommand{\uk}{\mathbf{k}}
\usetikzlibrary{positioning, fit, calc}
\pgfdeclarelayer{background}  
\pgfsetlayers{background,main}

\makeatletter
\newcommand{\boxspacing}{\kern\kvtcb@left@rule\kern\kvtcb@boxsep}
\makeatother
\newcommand{\prompt}[4]{
    \ttfamily\llap{{\color{#2}[#3]:\hspace{3pt}#4}}\vspace{-\baselineskip}
}

\newcommand{\thevenin}[2]{
  \begin{center}
    \begin{circuitikz} \draw
      (0,0) -- (2,0) to[battery1, l_=$V_{Th}\eq#1$] (2,2) 
      to[resistor, l_=$R_{Th}\eq#2$] (0,2)
      ;
      \draw [o-] (-.07,2.079);
      \draw [o-] (-.07,0.079);
    \end{circuitikz}
  \end{center}
}

\newcommand{\norton}[2]{
  \begin{center}
    \begin{circuitikz} \draw
      (0,0) -- (3,0) to[american current source, l_=$I_{N}\eq#1$] (3,2) -- (0,2) (2,0)
      to[resistor, l=$R_{N}\eq#2$] (2,2)
      ;
      \draw [o-] (-.07,2.079);
      \draw [o-] (-.07,0.079);
    \end{circuitikz}
  \end{center}
}

\newcommand{\highlight}[1]{\colorbox{yellow}{$\displaystyle #1$}}

\newcommand{\ti}[1]{\widetilde{#1}}

\title{Math 2120H: Assignment IV}
\author{Jeremy Favro (0805980) \\ Trent University, Peterborough, ON, Canada}
\date{\today}

\begin{document}
\maketitle

% PROBLEM 1
\begin{problem}
Evaluate $\int_C(xy+y+z)\,ds$ along the curve $\mathbf{r}(t)=2t\ui+t\uj+(2-2t)\uk,\,0\leq t\leq 1$.
\end{problem}
\begin{soln}
  \begin{align*}
    \int_C(xy+y+z)\,ds & =\int_0^1\left[(2t)(t)+t+2-2t\right]\left|2\ui+1\uj+(2-2t)\uk\right|\,dt \\
                       & =\int_0^1\left[(2t)(t)+t+2-2t\right]\left|2\ui+1\uj-2\uk\right|\,dt      \\
                       & =\int_0^1 \left[2t^2-t+2\right]\sqrt{9}\,dt=\frac{13}{2}
  \end{align*}
\end{soln}

% PROBLEM 2
\begin{problem}
Find the mass of a thin wire lying along the curve $\mathbf{r}(t)=(\sqrt{2})t\ui+(\sqrt{2})t\uj+(4-t^2)\uk,\,0\leq t\leq 1$, if the density is $\delta=3t$.
\end{problem}
\begin{soln}
  The mass of an object with a continuous density function is given by $\int_C\delta(x,y,z)\,ds$ so,
  \begin{align*}
    \int_C\delta(x,y,z)\,ds & =\int_0^1 3t\left|(\sqrt{2})\ui+(\sqrt{2})\uj-2t\uk\right|\,dt \\
                            & =\int_0^1 3t\sqrt{2+2+4t^2}\,dt                                \\
                            & =\int_1^2 3t\sqrt{4+4t^2}\,dt                                  \\
                            & =3\int_0^1 \sqrt{u}\,du=2^{5/2}-2
  \end{align*}
\end{soln}

% PROBLEM 3
\begin{problem}
Find the line integral of $F = 3y\ui + 2x\uj + 4z\uk$ over the path $\mathbf{r}(t) = t\ui + t^2\uj + t^4\uk,\,0 \leq t \leq 1$.
\end{problem}
\begin{soln}
  \begin{align*}
     & =\int_0^1F(\mathbf{r}(t))\mathbf{r}^\prime(t)\,dt                                      \\
     & =\int_0^1 \left[3t^2\ui + 2t\uj + 4t^4\uk\right]\left[\ui + 2t\uj + 4t^3\uk\right]\,dt \\
     & =\int_0^1 3t^2 + 4t^2 + 16t^7\uk\,dt=\frac{13}{3}
  \end{align*}
\end{soln}

% PROBLEM 4
\begin{problem}
Find the flux of the fields $F = 2x\ui + (x - y)\uj$ across the circle $\mathbf{r}(t) =(a\cos t)\ui + (a\sin t)\uj,\,0\leq t \leq 2\pi$.
\end{problem}
\begin{soln}
  The outward normal vector for $\mathbf{r}(t)$ is $\nabla\mathbf{r}(t)=-a\sin t\ui+a\cos t\uj$, for $a\neq1$ this must be divided by
  $a$ to unitize(? is that a word) it so,
  \begin{align*}
     & =\int_0^{2\pi}\left[-2\sin t\ui + (-\sin t - \cos t)\uj\right]\left[-\sin t\ui+\cos t\uj\right]\,dt \\
     & =\int_0^{2\pi}-2\sin^2t-\sin (t)\cos (t) - \cos^2 t\,dt                                             \\
     & =\int_0^{2\pi}\sin (t)\cos (t) - 1\,dt=-2\pi
  \end{align*}
\end{soln}
\end{document}