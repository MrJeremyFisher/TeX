\documentclass[10pt]{article}

\usepackage[margin=0.75in]{geometry}
\usepackage{amsmath,amsthm,amssymb}
\usepackage{xcolor}
\usepackage{cancel}
\usepackage{graphicx}
\usepackage{changepage}
\usepackage{circuitikz}
\usepackage{minted}
\usepackage{pgfplots}
\usepackage{physics}
\usepackage{siunitx}
\usepackage[breakable]{tcolorbox}
\usepackage[inline]{enumitem}

\theoremstyle{definition}
\newtheorem{problem}{Problem}
\newtheorem{soln}{Solution}

\pgfplotsset{compat=newest}
\usetikzlibrary{lindenmayersystems}
\usetikzlibrary{arrows}

\definecolor{incolor}{HTML}{303F9F}
\definecolor{outcolor}{HTML}{D84315}
\definecolor{cellborder}{HTML}{CFCFCF}
\definecolor{cellbackground}{HTML}{F7F7F7}
\newcommand{\eq}{=}
\usetikzlibrary{positioning, fit, calc}
\pgfdeclarelayer{background}  
\pgfsetlayers{background,main}

\makeatletter
\newcommand{\boxspacing}{\kern\kvtcb@left@rule\kern\kvtcb@boxsep}
\makeatother
\newcommand{\prompt}[4]{
    \ttfamily\llap{{\color{#2}[#3]:\hspace{3pt}#4}}\vspace{-\baselineskip}
}

\newcommand{\thevenin}[2]{
  \begin{center}
    \begin{circuitikz} \draw
      (0,0) -- (2,0) to[battery1, l_=$V_{Th}\eq#1$] (2,2) 
      to[resistor, l_=$R_{Th}\eq#2$] (0,2)
      ;
      \draw [o-] (-.07,2.079);
      \draw [o-] (-.07,0.079);
    \end{circuitikz}
  \end{center}
}

\newcommand{\norton}[2]{
  \begin{center}
    \begin{circuitikz} \draw
      (0,0) -- (3,0) to[american current source, l_=$I_{N}\eq#1$] (3,2) -- (0,2) (2,0)
      to[resistor, l=$R_{N}\eq#2$] (2,2)
      ;
      \draw [o-] (-.07,2.079);
      \draw [o-] (-.07,0.079);
    \end{circuitikz}
  \end{center}
}

\newcommand{\highlight}[1]{\colorbox{yellow}{$\displaystyle #1$}}

\NewDocumentCommand{\evalat}{sO{\big}mm}{%
  \IfBooleanTF{#1}
   {\mleft. #3 \mright|_{#4}}
   {#3#2|_{#4}}%
}

\title{Math 2150: Assignment I}
\author{Jeremy Favro}
\date{\today}

\begin{document}

\maketitle
\noindent Note: I've kept $p$, $q$, and $r$ throughout my solutions and only substituted the actual numbers in at the end. This is because I find it easier,
especially when dealing with things that might cancel nicely, to deal with variables rather than the numbers they represent. In my case my
student number is 0805980 so $p=9$, $q=5$, and $r=22$.
\\
% PROBLEM 1
\begin{problem}
\end{problem}
\begin{soln}
\end{soln}
\newpage

% PROBLEM 2
\begin{problem}
Solve the following differential equation using the method of undetermined coefficients.
$$\frac{d^2y}{dx^2}-2q\frac{dy}{dx}+q^2y=2q^2\left(q^2+4\right)^2\cos^2\left(x\right)+6pxe^{qx}+q^3x+(q^2+1)^2p\cos\left(x\right)$$
\end{problem}
\begin{soln} First we solve the homogeneous portion
  \begin{align*}
     & \frac{d^2y}{dx^2}-2q\frac{dy}{dx}+q^2y=0\rightsquigarrow y=e^{mx}                                                    \\
     & m^2-2qm+q^2=0 \implies m = \frac{2q\pm\sqrt{4q^2-4q^2}}{2} = q                                                       \\
     & \therefore y_h(x)=C_1e^{qx}+C_2xe^{qx} \text{ where } xe^{qx} \text{ is the result of reduction of order on } e^{qx} \\
  \end{align*}
  Then, using undetermined coefficients
  \begin{align*}
    y_p=                & A\cos\left(2x\right)+B\sin\left(2x\right)+Cx+D+x^2\left(Ex+F\right)e^{qx}+G\cos(x)+H\sin(x) \\
    y_p^\prime=         & -2A\sin(2x)+2B\cos(2x)+C+3Ex^2e^{qx}+qEx^3e^{qx}+2Fxe^{qx}+Fqx^2e^{qx}-G\sin(x)+H\cos(x)    \\
    y_p^{\prime\prime}= & -4A\cos(2x)-4B\sin(2x)+6Exe^{qx}+3Eqx^2e^{qx}+3Eqx^2e^{qx}+Eq^2x^3e^{qx}+2Fe^{qx}+          \\
                        & 2Fqxe^qx+2Fxe^{qx}+Fq^2x^2e^{qx}-G\cos(x)-H\sin(x)
  \end{align*}
  Which can then be plugged into the original differential equation in place of $y$, $y^\prime$, and $y^{\prime\prime}$
  \begin{align*}
     & -4A\cos(2x)-4B\sin(2x)+6Exe^{qx}+3Eqx^2e^{qx}+3Eqx^2e^{qx}+Eq^2x^3e^{qx}+2Fe^{qx}+2Fqxe^{qx}+2Fxe^{qx}+Fq^2x^2e^{qx} \\ &-G\cos(x)-H\sin(x)    \\
     & -2q\left[-2A\sin(2x)+2B\cos(2x)+C+3Ex^2e^{qx}+qEx^3e^{qx}+2Fxe^{qx}+Fqx^2e^{qx}-G\sin(x)+H\cos(x)\right]             \\
     & +q^2\left[A\cos\left(2x\right)+B\sin\left(2x\right)+Cx+D+x^2\left(Ex+F\right)e^{qx}+G\cos(x)+H\sin(x)\right]         \\
     & =q^2\left(q^2+4\right)^2+q^2\left(q^2+4\right)^2\cos\left(2x\right)+6pxe^{qx}+q^3x+(q^2+1)^2p\cos\left(x\right)
  \end{align*}
  Which when multiplied out yields
  \begin{align*}
     & -4A\cos(2x)-4B\sin(2x)+6Exe^{qx}+3Eqx^2e^{qx}+3Eqx^2e^{qx}+Eq^2x^3e^{qx}+2Fe^{qx}+2Fqxe^{qx}+2Fxe^{qx}+Fq^2x^2e^{qx}               \\
     & -G\cos(x)-H\sin(x) +4qA\sin(2x)-4qB\cos(2x)-2qC-6qEx^2e^{qx}-2q^2Ex^3e^{qx}-4qFxe^{qx}-2Fq^2x^2e^{qx}                              \\
     & +2qG\sin(x)-2qH\cos(x)+Aq^2\cos\left(2x\right)+Bq^2\sin\left(2x\right)+Cq^2x+D+Eq^2x^3e^{qx}+Fq^2x^2e^{qx}+q^2G\cos(x)+q^2H\sin(x) \\
     & =q^2\left(q^2+4\right)^2+q^2\left(q^2+4\right)^2\cos\left(2x\right)+6pxe^{qx}+q^3x+(q^2+1)^2p\cos\left(x\right)
  \end{align*}
  Which can be solved to determine the unknown coefficients
  \begin{align*}
     & -4B+4qA+Bq^2=0 \implies\frac{-4qA}{q^2-4}=B                                                                                \\
     & -4A-4qB+Aq^2=q^2\left(q^2+4\right)^2\implies\frac{q^2\left(q^2+4\right)^2}{q^2-4+\frac{16q^2}{q^2-4}}=A=525\implies B=-500 \\
     & Cq^2=q^3\implies C=q=5                                                                                                     \\
     & -2qC+D=q^2\left(q^2+4\right)^2\implies D=q^2\left(q^2+4\right)^2+2qC=21075                                                 \\
     & 2F=0\implies F=0                                                                                                           \\
     & 6E+2Fq+2F-4qF=6E=6p\implies E=p=9                                                                                          \\
     & -H+2qG+q^2H=0\implies \frac{-2qG}{q^2-1}=H                                                                                 \\
     & -G-2qH+q^2G=\left(q^2+1\right)^2p\implies G=\frac{\left(q^2+1\right)^2p}{\frac{4q^2}{q^2-1}-1+q^2}=216\implies H=-90       \\
  \end{align*}
  \begin{align*}
     & A=525   \\
     & B=-500  \\
     & C=5     \\
     & D=21075 \\
     & E=9     \\
     & F=0     \\
     & G=216   \\
     & H=-90   \\
  \end{align*}
  So,
  $$y_p=525\cos\left(2x\right)-500\sin\left(2x\right)+5x+21075+9x^3e^{5x}+216\cos(x)-90\sin(x)$$
  $$\therefore y=C_1e^{qx}+C_2xe^{qx}+525\cos\left(2x\right)-500\sin\left(2x\right)+5x+21075+9x^3e^{5x}+216\cos(x)-90\sin(x)$$
\end{soln}
\newpage

% PROBLEM 3
\begin{problem}
Solve the following differential equation using variation of parameters
$$qx^2\frac{d^2y}{dx^2}-9qx\frac{dy}{dx}+26qy=px^5\ln^2\left(x\right),\qquad x>0$$
\end{problem}
\begin{soln} First we solve the homogeneous portion
  \begin{align*}
     & qx^2\frac{d^2y}{dx^2}-9qx\frac{dy}{dx}+26qy=0 \rightsquigarrow y=x^m                                          \\
     & qx^2m^2x^{m-2}-9qxmx^{m-1}+26qx^m=0                                                                           \\
     & qm^2 -10qm+26q=0 \implies m=\frac{10q\pm\sqrt{100q^2-104q^2}}{2q}=5\pm i                                      \\
     & \therefore y_h(x)=C_1x^{5+i}+C_2x^{5-i}=x^5\left(C_3\cos\left(\ln(x)\right)+C_4\sin\left(\ln(x)\right)\right) \\
  \end{align*}
  Then, using variation of parameters
  $$y_p(x)=-y_1(x)\int\frac{g(x)y_2(x)}{W\left[y_1,y_2\right]}\,dx+y_2(x)\int\frac{g(x)y_1(x)}{W\left[y_1,y_2\right]}\,dx$$
  Where $g(x)=\frac{p}{q}x^3\ln^2\left(x\right)$ (the left hand side divided by $qx^2$).
  \begin{align*}
     & W\left[y_1,y_2\right]=
    \begin{vmatrix}
      y_1        & y_2        \\
      y_1^\prime & y_2^\prime
    \end{vmatrix}
    =
    \begin{vmatrix}
      x^5\cos\left(\ln(x)\right)                             & x^5\sin\left(\ln(x)\right)                             \\
      5x^4\cos\left(\ln(x)\right)-x^4\sin\left(\ln(x)\right) & 5x^4\sin\left(\ln(x)\right)+x^4\cos\left(\ln(x)\right)
    \end{vmatrix} \\
     & = \left[5x^4\sin\left(\ln(x)\right)+x^4\cos\left(\ln(x)\right)\right]x^5\cos\left(\ln(x)\right)
    -\left[5x^4\cos\left(\ln(x)\right)-x^4\sin\left(\ln(x)\right)\right]x^5\sin\left(\ln(x)\right)                  \\
     & = \cancel{5x^9\sin\left(\ln(x)\right)\cos\left(\ln(x)\right)}+x^9\cos^2\left(\ln(x)\right)
    \cancel{-5x^9\cos\left(\ln(x)\right)\sin\left(\ln(x)\right)} +x^9\sin^2\left(\ln(x)\right)                      \\
     & = x^9\cos^2\left(\ln(x)\right)+x^9\sin^2\left(\ln(x)\right) = x^9
  \end{align*}
  So,
  \begin{align*}
     & y_p(x)=-x^5\cos\left(\ln(x)\right)\int\frac{px^{8}\ln^2\left(x\right)\sin\left(\ln(x)\right)}{qx^9}\,dx \\
     & +x^5\sin\left(\ln(x)\right)\int\frac{px^{8}\ln^2\left(x\right)\cos\left(\ln(x)\right)}{qx^9}\,dx  \\
     & =-\frac{p}{q}x^5\cos\left(\ln(x)\right)\left[2\ln(x)\sin\left(\ln(x)\right)+\left(2-\ln^2(x)\right)\cos\left(\ln(x)\right)\right] \\
     & +\frac{p}{q}x^5\sin\left(\ln(x)\right)\left[\left(\ln^2(x)-2\right)\sin\left(\ln(x)\right)+2\ln(x)\cos\left(\ln(x)\right)\right]  \\
     & =-\frac{p}{q}x^5\cos\left(\ln(x)\right)2\ln(x)\sin\left(\ln(x)\right)-\frac{p}{q}x^5\cos\left(\ln(x)\right)\left(2-\ln^2(x)\right)\cos\left(\ln(x)\right) \\
     & +\frac{p}{q}x^5\sin\left(\ln(x)\right)\left(\ln^2(x)-2\right)\sin\left(\ln(x)\right)+\frac{p}{q}x^5\sin\left(\ln(x)\right)2\ln(x)\cos\left(\ln(x)\right)  \\
     & = -\frac{p}{q}x^5\cos\left(\ln(x)\right)\left(2-\ln^2(x)\right)\cos\left(\ln(x)\right)+\frac{p}{q}\sin\left(\ln(x)\right)\left(\ln^2(x)-2\right)\sin\left(\ln(x)\right) \\
     & -\frac{p}{q}x^5\left(2-\ln^2(x)\right)\cos^2\left(\ln(x)\right)+\frac{p}{q}\left(\ln^2(x)-2\right)\sin^2\left(\ln(x)\right)                                             \\
     & = \frac{p}{q}x^5\left[-2\cos^2\left(\ln(x)\right)+\ln^2(x)\cos^2\left(\ln(x)\right)+\ln^2(x)\sin^2\left(\ln(x)\right)-2\sin^2\left(\ln(x)\right)\right]                 \\
     & = \frac{p}{q}x^5\left[\ln^2(x)-2\right]                                                                                                                                  \\
  \end{align*}
  $$\therefore y(x)=x^5\left(C_3\cos\left(\ln(x)\right)+C_4\sin\left(\ln(x)\right)+\frac{9}{5}\left[\ln^2(x)-2\right]\right) $$
\end{soln}
\end{document}