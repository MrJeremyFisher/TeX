\documentclass[10pt]{article}

\usepackage[margin=0.75in]{geometry}
\usepackage{amsmath,amsthm,amssymb}
\usepackage{xcolor}
\usepackage{cancel}
\usepackage{graphicx}
\usepackage{changepage}
\usepackage{circuitikz}
\usepackage{pgfplots}
\usepackage{physics}
\usepackage{hyperref}
\usepackage{siunitx}
\usepackage[breakable]{tcolorbox}
\usepackage[inline]{enumitem}

\theoremstyle{definition}
\newtheorem{problem}{Problem}
\newtheorem{soln}{Solution}

\pgfplotsset{compat=newest}
\usetikzlibrary{lindenmayersystems}
\usetikzlibrary{arrows}

\definecolor{incolor}{HTML}{303F9F}
\definecolor{outcolor}{HTML}{D84315}
\definecolor{cellborder}{HTML}{CFCFCF}
\definecolor{cellbackground}{HTML}{F7F7F7}
\newcommand{\eq}{=}
\usetikzlibrary{positioning, fit, calc}
\pgfdeclarelayer{background}  
\pgfsetlayers{background,main}

\makeatletter
\newcommand{\boxspacing}{\kern\kvtcb@left@rule\kern\kvtcb@boxsep}
\makeatother
\newcommand{\prompt}[4]{
    \ttfamily\llap{{\color{#2}[#3]:\hspace{3pt}#4}}\vspace{-\baselineskip}
}

\newcommand{\thevenin}[2]{
  \begin{center}
    \begin{circuitikz} \draw
      (0,0) -- (2,0) to[battery1, l_=$V_{Th}\eq#1$] (2,2) 
      to[resistor, l_=$R_{Th}\eq#2$] (0,2)
      ;
      \draw [o-] (-.07,2.079);
      \draw [o-] (-.07,0.079);
    \end{circuitikz}
  \end{center}
}

\newcommand{\norton}[2]{
  \begin{center}
    \begin{circuitikz} \draw
      (0,0) -- (3,0) to[american current source, l_=$I_{N}\eq#1$] (3,2) -- (0,2) (2,0)
      to[resistor, l=$R_{N}\eq#2$] (2,2)
      ;
      \draw [o-] (-.07,2.079);
      \draw [o-] (-.07,0.079);
    \end{circuitikz}
  \end{center}
}

\newcommand{\highlight}[1]{\colorbox{yellow}{$\displaystyle #1$}}

\hypersetup{
    colorlinks=true,
    linkcolor=blue,
    filecolor=magenta,      
    urlcolor=cyan,
    pdftitle={Overleaf Example},
    pdfpagemode=FullScreen,
    }

\NewDocumentCommand{\evalat}{sO{\big}mm}{%
  \IfBooleanTF{#1}
   {\mleft. #3 \mright|_{#4}}
   {#3#2|_{#4}}%
}

\title{Physics 2250: Problem Set III}
\author{Jeremy Favro}
\date{\today}

\begin{document}

\maketitle

% PROBLEM 1
\begin{problem}
[\textbf{BONUS}]Consider the circuit below.
\begin{center}
  \begin{circuitikz} \draw
    (0,2) to[battery1, l_=$9\unit{\volt}$] (0,0) (0,2) 
    to[capacitor, l=$C\eq3\unit{\micro\farad}$] (2,2)
    to[resistor, l=$R_1\eq6\unit{\kilo\ohm}$] (4,2) -- (4,0) -- (2,0)
    to[resistor, l_=$R_2\eq3\unit{\kilo\ohm}$] (2,2)  (0,0)
    to[switch, l_=$S$] (2,0)
    ;
  \end{circuitikz}
\end{center}
\end{problem}
\begin{soln} ~\\
  a) Because resistors in parallel share a voltage drop we can simplify $R_1$ and $R_2$ here
  $R_{eq}=\displaystyle{\left(\frac{1}{6\unit{\kilo\ohm}}+\frac{1}{3\unit{\kilo\ohm}}\right)^{-1}}=2\unit{\kilo\ohm}$
  Using the formula for voltage across a resistor as a function of time in an RC circuit given in the course text we get that
  \begin{align*}
     & 4.5\unit{\volt}=9\unit{\volt}e^{\frac{-t}{R_{eq}C}}                                                            \\
     & 4.5\unit{\volt}=9\unit{\volt}e^{\frac{-t}{\num{2d3}\unit{\ohm}\cdot\num{3d-6}\unit{\farad}}}                   \\
     & 4.5\unit{\volt}=9\unit{\volt}e^{\frac{-t}{\num{6d-3}\unit{\ohm\farad}}}                                        \\
     & \ln(\frac{1}{2})={\frac{-t}{\num{6d-3}\unit{\ohm\farad}}}                                                      \\
     & -\num{6d-3}\unit{\ohm\farad}\ln(\frac{1}{2})=t\approxeq \num{4.159d-3}\unit{\second}=4.159\unit{\milli\second}
  \end{align*}
  b) Again using the course text to obtain a formula for the voltage across a capacitor versus time, which, when equal to the source voltage will indicate a fully charged capacitor
  $$v_c(t)=V_s\left[1-e^{\frac{-t}{R_{eq}C}} \right]$$
  without doing any rearranging for $t$ it can be seen that for $v_c=9\unit{\volt}$ $\displaystyle{1-e^{\frac{-t}{R_{eq}C}}}$ must be 1, thus the exponential term must be zero which only occurs
  at an infinitely distant time. So the capacitor can never be ``fully'' charged in a mathematical sense. However (I think this makes sense) because the universe
  isn't infinitely granular (right?) when $9\unit{\volt}C-v_cC < e \text{ (the electron charge)}$ the capacitor can be considered effectively fully charged as there exists
  no smaller charge carrier to further equalize the charge at that instant. The aformentioned instant occurs at ~0.0132s. See ``I think this makes sense'' for rationalization.
\end{soln}

% PROBLEM 2
\begin{problem}
Find the value of current $I$ flowing through the $1\unit{\ohm}$ resistor.
\begin{center}
  \begin{circuitikz} \draw
    (0,0) to[resistor, l_=$R_3\eq1\unit{\ohm}$] (0,2) (2,2)
    to[battery1, l_=$3\unit{\volt}$] (0,2) (2,2)
    to[resistor, l=$R_1\eq3\unit{\ohm}$] (4.5,2)
    to[battery1, l=$15\unit{\volt}$] (4.5,0) -- (2,0)
    to[resistor, l_=$R_2\eq6\unit{\ohm}$, o-] (2,2) (2,0) -- (0,0)
    ;
    
    \draw [-stealth, line width=0.5mm] (1,0) node[below right] {$I_3$} -- (2,0) ;
    \draw [-stealth, line width=0.5mm] (2,0.44) node[right] {$I_2$} -- (2,0) ;
    \draw [-stealth, line width=0.5mm] (2,0) node[below right] {$I_1$} -- (2.5,0) ;
    \draw (2,0) node[below left] {$\epsilon_1$};
    \draw [*-] (.25,2.079) node[above left] {B};
    \draw [*-] (.25,0.079) node[below left] {A};
  \end{circuitikz}
\end{center}
\end{problem}
\begin{soln} ~\\
  a) Going around both loops clockwise starting at $\epsilon_1$ (and looking at the junction at $\epsilon_1$) we get
  \begin{align*}
     & 0 = -I_3R_3+3\unit{\volt}-I_2R_2 \\
     & 0 = 15\unit{\volt}-I_1R_1-I_2R_2 \\
     & 0 = I_3 + I_2 - I_1
  \end{align*}
  which when solved yields $I_3=\displaystyle{\frac{7}{3}}\unit{\ampere}$ \\
  b) Removing the $3\unit{\volt}$ battery and replacing it with a wire we get a simple three resistor circuit where $R_{eq}=\displaystyle\left(\frac{1}{1\unit{\ohm}}+\frac{1}{6\unit{\ohm}}\right)^{-1}+3\unit{\ohm}=\frac{27}{7}\unit{\ohm}$
  which therefore has an $I_{tot}=\displaystyle\frac{15\unit{\volt}}{\frac{27}{7}\unit{\ohm}}=\frac{35}{9}\unit{\ampere}$ which is divided into the $R_3$ branch
  per the current divider rule $I_3=\displaystyle\frac{35}{9}\unit{\ampere}\cdot\frac{6}{7}=\frac{10}{3}\unit{\ampere}$ flowing ``downwards'' (negative). Then removing the $15\unit{\volt}$ resistor
  we get a circuit with $R_{eq}=\displaystyle\left(\frac{1}{3\unit{\ohm}}+\frac{1}{6\unit{\ohm}}\right)^{-1}+1\unit{\ohm}=3\unit{\ohm}$ which has an $I_{tot}=\displaystyle\frac{3\unit{\volt}}{3\unit{\ohm}}=1\unit{\ampere}$
  all of which flows ``upwards'' through $R_3$. Therefore the total current when these two are superimposed is $I_3=1\unit{\ampere}-\displaystyle\frac{10}{3}\unit{\ampere}=-\frac{7}{3}\unit{\ampere}$
  upwards, which is actually downwards because of the sign. \\
  c) We can solve the circuit created by removing $R_3$ using superposition to determine that the 3V-only circuit will have a potential drop across port AB of $3\unit{\volt}$ and the 15V-only
  circuit will have one of $10\unit{\volt}$. This means that $V_{Th}=V_{15AB}-V_{3AB}=10\unit{\volt}-3\unit{\volt}=7\unit{\volt}$. $R_{Th}$ is obtained by shorting both voltage sources and looking
  at the current path from terminal A to terminal B so $R_{Th}=\displaystyle\left(\frac{1}{3\unit{\ohm}}+\frac{1}{6\unit{\ohm}}\right)^{-1}=2\unit{\ohm}$ so the circuit's Th\'evenin equivalent looks like
  \begin{center}
    \begin{circuitikz} \draw
      (0,0) -- (2,0) to[battery1, l_=$V_{Th}\eq7\unit{\volt}$] (2,2) 
      to[resistor, l_=$R_{Th}\eq2\unit{\ohm}$] (0,2)
      ;
      \draw [*-] (0,2.079) node[above left] {B};
      \draw [*-] (0,0.079) node[below left] {A};
    \end{circuitikz}
  \end{center}
  So when a $1\unit{\ohm}$ load is placed across port AB it will experience a current of $I_{load}=\displaystyle\frac{V_{Th}}{R_{tot}}=\frac{7\unit{\volt}}{2\unit{\ohm}+1\unit{\ohm}}=\frac{7}{3}\unit{\ampere}$\\
  d) Using superposition to solve for $I_N$ we get two circuits
  \begin{center}
    \begin{circuitikz} \draw
      (2,2)
      to[battery1, l_=$3\unit{\volt}$] (0,2) (2,2)
      to[resistor, l=$R_1\eq3\unit{\ohm}$] (4.5,2)
      -- (4.5,0) -- (2,0)
      to[resistor, l_=$R_2\eq6\unit{\ohm}$, o-] (2,2) (2,0) -- (0,0)
      (.0,2.079) -- (.0,0.079)
      ;
      \node at (2.5,3.2) {Circuit A};
      \draw [*-] (.0,2.079) node[above left] {B};
      \draw [*-] (.0,0.079) node[below left] {A};
    \end{circuitikz}
    \begin{circuitikz} \draw
      (2,2)
      -- (0,2) (2,2)
      to[resistor, l=$R_1\eq3\unit{\ohm}$] (4.5,2)
      to[battery1, l=$15\unit{\volt}$] (4.5,0) -- (2,0)
      to[resistor, l_=$R_2\eq6\unit{\ohm}$, o-] (2,2) (2,0) -- (0,0)
      (0,2.079) -- (0,0.079)
      ;
      \node at (2.5,3.2) {Circuit B};
      \draw [*-] (0,2.079) node[above left] {B};
      \draw [*-] (0,0.079) node[below left] {A};
    \end{circuitikz}
  \end{center}
  In circuit A we get a current through port AB of $I_{AB}=\displaystyle3\unit{\volt}\left(3^{-1}+6^{-1}\right)=\frac{3}{2}\unit{\ampere}$. In circuit B we've shorted $R_2$ so the current through
  port AB is $I_{AB}=\displaystyle\frac{15\unit{\volt}}{3\unit{\ohm}}=5\unit{\ampere}$. So when superimposing these currents we get that $I_N=5\unit{\ampere}-\displaystyle\frac{3}{2}\unit{\ampere}=\frac{7}{2}\unit{\ampere}$.
  $R_N$ can be found by shorting both voltage sources and looking in from port AB at how current will travel so $R_N=\left(6^{-1}\unit{\ohm}^{-1}+3^{-1}\unit{\ohm}^{-1}\right)^{-1}=2\unit{\ohm}$. Therefore the Norton equivalent looks like
  \begin{center}
    \begin{circuitikz} \draw
      (0,0) -- (3,0) to[american current source, l_=$I_{N}\eq\frac{7}{2}\unit{\ampere}$] (3,2) -- (0,2) (2,0)
      to[resistor, l=$R_{N}\eq2\unit{\ohm}$] (2,2)
      ;
      \draw [*-] (0,2.079) node[above left] {B};
      \draw [*-] (0,0.079) node[below left] {A};
    \end{circuitikz}
  \end{center}
  So when a $1\unit{\ohm}$ load is placed between terminals A and B it will experience a current given by the associated current divider $I_{load}=I_N\cdot\displaystyle\frac{R_N}{R_N+1\unit{\ohm}}=\frac{7}{2}\unit{\ampere}\cdot\displaystyle\frac{2\unit{\ohm}}{2\unit{\ohm}+1\unit{\ohm}}=\frac{7}{3}\unit{\ampere}$
\end{soln}

% PROBLEM 3
\begin{problem}
Determine the Norton \textbf{and} Th\'evenin equivalents of the circuit below.
\begin{center}
  \begin{circuitikz}[scale=1.25]
    \draw
    (0,0) to[american current source, l=$I_S\eq10\unit{\ampere}$] (0,2)
    to[resistor, l=$R_1\eq3\unit{\ohm}$] (2,2)
    to[battery1, l_=$V_A\eq20\unit{\volt}$] (2,1)
    to[resistor, l=$R_3\eq6\unit{\ohm}$] (2,0)
    to[resistor, l=$R_2\eq3\unit{\ohm}$] (0,0) (2,0)
    to[resistor, l_=$R_4\eq4\unit{\ohm}$] (4,0)
    to[resistor, l=$R_5\eq2\unit{\ohm}$] (4,2) -- (2,2)
    ;
    \draw [-o] (4,2) -- (4.5,2);
    \draw [-o] (4,0) -- (4.5,0);
  \end{circuitikz}
\end{center}
\end{problem}
\begin{soln} ~\\
  Th\'evenin:
  Solving the circuit using superposition yields two sub circuits
  \begin{center}
    \begin{circuitikz}[scale=1.25] \draw
      (0,0) to[open] (0,2)
      to[resistor, l=$R_1\eq3\unit{\ohm}$] (2,2)
      to[battery1, l_=$V_A\eq20\unit{\volt}$] (2,1)
      to[resistor, l=$R_3\eq6\unit{\ohm}$] (2,0)
      to[resistor, l=$R_2\eq3\unit{\ohm}$] (0,0) (2,0)
      to[resistor, l_=$R_4\eq4\unit{\ohm}$] (4,0)
      to[resistor, l=$R_5\eq2\unit{\ohm}$] (4,2) -- (2,2)
      ;
      \node at (2.5,3.2) {Circuit A};
      \draw [-o] (4,2) -- (4.5,2);
      \draw [-o] (4,0) -- (4.5,0);
    \end{circuitikz}
    \begin{circuitikz}[scale=1.25] \draw
      (0,0) to[american current source, l=$I_S\eq10\unit{\ampere}$] (0,2)
      to[resistor, l=$R_1\eq3\unit{\ohm}$] (2,2)
      to[resistor, l_=$R_3\eq6\unit{\ohm}$] (2,0)
      to[resistor, l=$R_2\eq3\unit{\ohm}$] (0,0) (2,0)
      to[resistor, l_=$R_4\eq4\unit{\ohm}$] (4,0)
      to[resistor, l=$R_5\eq2\unit{\ohm}$] (4,2) -- (2,2)
      ;
      \node at (2.5,3.2) {Circuit B};
      \draw [-o] (4,2) -- (4.5,2);
      \draw [-o] (4,0) -- (4.5,0);
    \end{circuitikz}
  \end{center}
  In this case we are trying to solve for the voltage across $R_5$ in both circuits. In circuit A $R_1$ and $R_2$ don't have any effect on the circuit as they have no current passing through them so they can be ignored.
  To determine the voltage across $R_{5A}$ we can find the equivalent resistance $R_{eq}=R_3+R_4+R_5=12\unit{\ohm}$ and thus $I_{tot}=\displaystyle\frac{20\unit{\volt}}{12\unit{\ohm}}=\frac{5}{3}\unit{\ampere}$
  therefore $V_{5A}=\frac{5}{3}\unit{\ampere}\cdot2\unit{\ohm}=\frac{10}{3}\unit{\volt}$. In circuit B the voltage across $R_5$ is determined by the current divider created by the $R_3$-$R_5$
  branch, $\displaystyle I_5=I_S\frac{R_3}{R_5+R_4+R_3}=5\unit{\ampere}\implies V_{5B}=5\unit{\ampere}\cdot2\unit{\ohm}=10\unit{\volt}$. Superimposing these two voltages gives us the voltage across the port 
  $V_{port}=V_{5A}+V_{5B}=\displaystyle\frac{40}{3}\unit{\volt}=V_{Th}$. $R_{Th}$ can be found by eliminating the active components which removes $R_1$ and $R_2$ as the current source is replaced with an open. From there
  $R_{Th}=\displaystyle\left((R_3+R_4)^{-1}+R_5^{-1}\right)^{-1}=\frac{5}{3}\unit{\ohm}$. Therefore the Th\'evenin equivalent becomes \\
  \thevenin{\frac{40}{3}\unit{\volt}}{\frac{5}{3}\unit{\ohm}} ~\\
  Norton: Solving the circuit using superposition and shorting across the port 
  \begin{center}
    \begin{circuitikz}[scale=1.25] \draw
      (0,0) to[open] (0,2)
      to[resistor, l=$R_1\eq3\unit{\ohm}$] (2,2)
      to[battery1, l_=$V_A\eq20\unit{\volt}$] (2,1)
      to[resistor, l=$R_3\eq6\unit{\ohm}$] (2,0)
      to[resistor, l=$R_2\eq3\unit{\ohm}$] (0,0) (2,0)
      to[resistor, l_=$R_4\eq4\unit{\ohm}$] (4,0)
      to[resistor, l=$R_5\eq2\unit{\ohm}$] (4,2) -- (2,2)
      ;
      \node at (2.5,3.2) {Circuit A};
      \draw (4,2) -- (4.5,2) -- (4.5,0) -- (4,0);
    \end{circuitikz}
    \begin{circuitikz}[scale=1.25] \draw
      (0,0) to[american current source, l=$I_S\eq10\unit{\ampere}$] (0,2)
      to[resistor, l=$R_1\eq3\unit{\ohm}$] (2,2)
      to[resistor, l_=$R_3\eq6\unit{\ohm}$] (2,0)
      to[resistor, l=$R_2\eq3\unit{\ohm}$] (0,0) (2,0)
      to[resistor, l_=$R_4\eq4\unit{\ohm}$] (4,0)
      to[resistor, l=$R_5\eq2\unit{\ohm}$] (4,2) -- (2,2)
      ;
      \node at (2.5,3.2) {Circuit B};
      \draw (4,2) -- (4.5,2) -- (4.5,0) -- (4,0);
    \end{circuitikz}
  \end{center}
  In circuit A $R_1$ and $R_2$ experience no current and therefore have no effect on the circuit. $R_5$ is shorted and therefore also has no effect on the circuit, so the current through the shorted port is $I_{tot}=I_{B}=\displaystyle\frac{20\unit{\volt}}{6\unit{\ohm}+4\unit{\ohm}}=2\unit{\ampere}$.
  In circuit B $R_5$ is again shorted and therefore has no effect on the circuit so the current through the shorted port is the current through $R_4$ which is determined by the $R_3$-$R_4$ current divider
  $I_{port}=I_B=I_S\cdot\displaystyle\frac{R_3}{R_3+R_4}=10\unit{\ampere}\cdot\frac{6\unit{\ohm}}{6\unit{\ohm}+4\unit{\ohm}}=6\unit{\ampere}$. So, $I_N=I_A+I_B=8\unit{\ampere}$. $R_N$ is determined through the same 
  method as Th\'evenin so I will spare the details I went through previously and just state that $R_N=\displaystyle\frac{5}{3}\unit{\ohm}$. Therefore the Norton equivalent looks like
  \norton{8\unit{\ampere}}{\frac{5}{3}\unit{\ohm}}
\end{soln}

% PROBLEM 4
\begin{problem}
[\textbf{BONUS}] A simple model of a photovoltaic solar cell is a current source with the current proportional to the power of sunlight falling on it. 
(A more accurate model includes a diode, which is a nonlinear element we will learn about later in the course). There is some leakage current in the
source, which can be modelled with a parallel resistor, Rp. There is also a voltage drop in the transmission of the power that can be modelled by resistors in series, Rs, connecting to the load. 
The load resistance is given by RL. This crude model is represented in the figure below
\begin{center}
  \begin{circuitikz} \draw
    (0,0) coordinate(bottom) to[american current source, l=$I_S\eq0.2\unit{\ampere}$, name=s] (0,2) coordinate(top) -- (1,2)
    to[resistor, l=$R_P\eq3\unit{\ohm}$] (1,0) -- (0,0) (1,0) 
    to[resistor, l=$R_S\eq1.8\unit{\ohm}$, name=rt] (7,0) 
    to[resistor, l=$R_L$, name=rl] (7,2)
    to[resistor, l=$R_S\eq1.8\unit{\ohm}$] (1,2) (7,0) coordinate(rl2);
    \begin{pgfonlayer}{background}
      \draw     
      node[fit={($(top.north)+(0,.5)$) ($(bottom.south)+(0,-.5)$) ($(s.west)+(-2,0)$) ($(rt.east)+(0.4,0)$)}, draw, fill=blue, opacity=0.2, dashed, label={Power Source}, inner sep=10pt]{}
      node[fit={($(top.north)+(3.25,0)$) ($(bottom.south)+(3.25,0)$) ($(rt.east)+(0.25,0)$)}, draw, fill=green, opacity=0.2, dashed, label={Transmission}, inner sep=10pt]{}
      node[fit={($(rl.north)+(0,1)$) ($(rl2.west)+(-0.75,0)$) (7.25,0)}, draw, fill=red, opacity=0.2, dashed, label={Load}, inner sep=10pt]{};
    \end{pgfonlayer}
  \end{circuitikz}
\end{center}
\end{problem}
\begin{soln} ~\\
  a) Knowing that $P=IV=I^2R$ we can set up an equation for $P_{L}$ in terms of $R_{L}$ which can be differentiated to determine maxima and minima. 
  $I_L=I_S\cdot\displaystyle\frac{R_P}{R_P+R_L+2R_S}\implies P_L=I_L^2R_L=\left(I_S\cdot\frac{R_P}{R_P+R_L+2R_S}\right)^2\cdot R_L$
  \begin{align*}
   & \frac{d}{dR_L}\left[\left(I_S\cdot\frac{R_P}{R_P+R_L+2R_S}\right)^2\cdot R_L\right] \\
   & \frac{d}{dR_L}\left[\frac{I_S^2R_P^2R_L}{\left(R_P+R_L+2R_S\right)^2} \right] \\
   & \frac{\frac{d}{dR_L}\left[I_S^2R_P^2R_L\right]\left(R_P+R_L+2R_S\right)^2-\frac{d}{dR_L}\left[\left(R_P+R_L+2R_S\right)^2\right]I_S^2R_P^2R_L}{\left(R_P+R_L+2R_S\right)^4} \\
   & I_S^2R_P^2\cdot\frac{\frac{d}{dR_L}\left[R_L\right]\left(R_P+R_L+2R_S\right)^2-\frac{d}{dR_L}\left[\left(R_P+R_L+2R_S\right)^2\right]R_L}{\left(R_P+R_L+2R_S\right)^4} \\
   & I_S^2R_P^2\cdot\frac{\left(R_P+R_L+2R_S\right)^2-2\left(R_P+R_L+2R_S\right)\frac{d}{dR_L}\left[\left(R_P+R_L+2R_S\right)\right]R_L}{\left(R_P+R_L+2R_S\right)^4} \\
   & I_S^2R_P^2\cdot\frac{\left(R_P+R_L+2R_S\right)^2-2\left(R_P+R_L+2R_S\right)R_L}{\left(R_P+R_L+2R_S\right)^4} \\
  \end{align*}
  Wow, great, I think I did that correctly. Now we solve $P_L^\prime=0$
  \begin{align*}
    & 0 = \frac{I_S^2R_P^2\left(R_P+R_L+2R_S\right)^2}{\left(R_P+R_L+2R_S\right)^4} - \frac{2I_S^2R_P^2\left(R_P+R_L+2R_S\right)R_L}{\left(R_P+R_L+2R_S\right)^4}\\
    & 0 = \frac{I_S^2R_P^2}{\left(R_P+R_L+2R_S\right)^2} - \frac{2I_S^2R_P^2R_L}{\left(R_P+R_L+2R_S\right)^3} \\
    & \frac{2I_S^2R_P^2R_L}{\left(R_P+R_L+2R_S\right)} = I_S^2R_P^2\\
    & \frac{1}{2R_L} = \frac{1}{\left(R_P+R_L+2R_S\right)} \\
    & 2R_L = R_P+R_L+2R_S \\
    & R_L = R_P+2R_S \\
   \end{align*}
   So $P_L^\prime$ has only one zero at $R_L = 3+2(1.8)=6.6\unit{\ohm}$. Looking at the function on a graph which I will omit here to save myself the paper and ink reveals this point to be a maximum. The same conclusion could
   also be arrived at through testing the second derivative.\\
   b) Removing $R_L$ also takes both transmission resistors out of the equation whichs means that $V_{Th}=I_SR_P=0.2\unit{\ampere}\cdot3\unit{\ohm}=\displaystyle\frac{3}{5}\unit{\volt}$. $R_{Th}$ is 
   the combined resistance of both transmission resistors and the parallel resistor as when $I_S$ is opened that is the only path through which current flows from the port so $R_{Th}=2R_S+R_P=6.6\unit{\ohm}$.
   Therefore the Th\'evenin equivalent looks like
   \thevenin{\frac{3}{5}\unit{\volt}}{6.6\unit{\ohm}}
   $R_{Th}$ is the same as the optimal $R_L$ because the optimal $R_L$ will shift to match $R_{Th}$ per $P_L=I_SV_L=I_S^2R_L$. Because the current is constant if $R_L<R_{Th}$ then $R_L$ is no longer optimal and must increase
   up to $R_{Th}$ to create the greatest possible voltage drop across itself. In reverse, increasing $R_{L}$ past this happy point will mean more current travels through $R_{P}$ which defeats the purpose of increasing $R_{L}$.
   c) The maximum power delivery occurs when $R_L$ is optimal so is $P_L=I_L^2R_L=\displaystyle\left(0.2\unit{\ampere}\frac{3\unit{\ohm}}{2\cdot1.8\unit{\ohm}+6.6\unit{\ohm}+3\unit{\ohm}}\right)^26.6\unit{\ohm}\approxeq13.64\unit{\milli\watt}$. $P_{tot}=V_{tot}I_S=(V_P+V_S+V_L)I_S\approxeq92.73\unit{\milli\watt}$,
   therefore the resistor uses $\displaystyle\frac{13.64\unit{\milli\watt}}{92.73\unit{\milli\watt}}\cdot100\approxeq14.71\%$ of the total expendable power.
\end{soln}
\end{document}