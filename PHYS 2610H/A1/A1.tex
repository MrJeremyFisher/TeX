\documentclass[10pt]{article}

\usepackage[margin=0.75in]{geometry}
\usepackage{amsmath,amsthm,amssymb}
\usepackage{xcolor}
\usepackage{cancel}
\usepackage{graphicx}
\usepackage{changepage}
\usepackage{circuitikz}
\usepackage{pgfplots}
%\usepackage{physics}
\usepackage{hyperref}
\usepackage{siunitx}
\usepackage[breakable]{tcolorbox}
\usepackage[inline]{enumitem}

\theoremstyle{definition}
\newtheorem{problem}{Problem}
\newtheorem{soln}{Solution}

\pgfplotsset{compat=newest}
\usetikzlibrary{lindenmayersystems}
\usetikzlibrary{arrows}
\usetikzlibrary{calc}

\definecolor{incolor}{HTML}{303F9F}
\definecolor{outcolor}{HTML}{D84315}
\definecolor{cellborder}{HTML}{CFCFCF}
\definecolor{cellbackground}{HTML}{F7F7F7}
\newcommand{\eq}{=}
\usetikzlibrary{positioning, fit, calc}
\pgfdeclarelayer{background}  
\pgfsetlayers{background,main}

\makeatletter
\newcommand{\boxspacing}{\kern\kvtcb@left@rule\kern\kvtcb@boxsep}
\makeatother
\newcommand{\prompt}[4]{
    \ttfamily\llap{{\color{#2}[#3]:\hspace{3pt}#4}}\vspace{-\baselineskip}
}

\newcommand{\thevenin}[2]{
  \begin{center}
    \begin{circuitikz} \draw
      (0,0) -- (2,0) to[battery1, l_=$V_{Th}\eq#1$] (2,2) 
      to[resistor, l_=$R_{Th}\eq#2$] (0,2)
      ;
      \draw [o-] (-.07,2.079);
      \draw [o-] (-.07,0.079);
    \end{circuitikz}
  \end{center}
}

\newcommand{\norton}[2]{
  \begin{center}
    \begin{circuitikz} \draw
      (0,0) -- (3,0) to[american current source, l_=$I_{N}\eq#1$] (3,2) -- (0,2) (2,0)
      to[resistor, l=$R_{N}\eq#2$] (2,2)
      ;
      \draw [o-] (-.07,2.079);
      \draw [o-] (-.07,0.079);
    \end{circuitikz}
  \end{center}
}

\newcommand{\highlight}[1]{\colorbox{yellow}{$\displaystyle #1$}}

\newcommand{\ti}[1]{\widetilde{#1}}

\hypersetup{
    colorlinks=true,
    linkcolor=blue,
    filecolor=magenta,      
    urlcolor=cyan,
    pdftitle={Overleaf Example},
    pdfpagemode=FullScreen,
    }

\NewDocumentCommand{\evalat}{sO{\big}mm}{%
  \IfBooleanTF{#1}
   {\mleft. #3 \mright|_{#4}}
   {#3#2|_{#4}}%
}

\title{Physics 2610H: Assignment I}
\author{Jeremy Favro}
\date{\today}

\begin{document}
\maketitle

% PROBLEM 1
\begin{problem}
At what wavelength does the human body emit the maximum electromagnetic radiation? Use Wien's law from Exercise 14 and assume a skin
temperature of 70$\unit{\degree}\mathrm{F}$.
\end{problem}
\begin{soln}
  By Wien's law, $\lambda_{max}T=\alpha$ where $\alpha=\qty{2.898d-3}{\meter\kelvin}$. Here $T=\qty{294.261}{\kelvin}$ so $\lambda_{max}=\frac{\alpha}{T}\approx10\unit{\micro\meter}$
\end{soln}

% PROBLEM 2
\begin{problem}
With light of wavelength $520\unit{\nano\meter}$, photoelectrons are ejected from a metal surface with a maximum speed of $\qty{1.78d5}{\meter\per\second}$.
\begin{enumerate}[label=(\alph*)]
  \item What wavelength would be needed to give a maximum speed of $\qty{4.81d5}{\meter\per\second}$?
  \item Can you guess what metal it is?
\end{enumerate}
\end{problem}
\begin{soln}
  ~\begin{enumerate}[label=(\alph*)]
    \item Here we first need to determine the work function, $\phi$, of the surface. Using equation 1.2 from the formula sheet,
          \begin{align*}
             & E_{kmax}=\frac{1}{2}m_ev^2=hf-\phi                 \\
             & E_{kmax}=\frac{hc}{\lambda}-\frac{1}{2}m_ev^2=\phi
          \end{align*}
          Then the required wavelength for $v^\prime=\qty{4.81d5}{\meter\per\second}$ will be proportional to the base kinetic energy of those electrons and the work function energy
          \begin{align*}
             & \frac{1}{2}m_e\left(v^\prime\right)^2+\phi=\frac{hc}{\lambda^\prime}                                             \\
             & \frac{hc}{\frac{1}{2}m_e\left(v^\prime\right)^2+\phi}=\lambda^\prime                                             \\
             & \frac{2hc}{m_e\left[\left(v^\prime\right)^2-v^2\right]+\frac{2hc}{\lambda}}=\lambda^\prime=420\unit{\nano\meter}
          \end{align*}
    \item Going by table 1 from the textbook with $\phi=\frac{hc}{\lambda}-\frac{1}{2}m_ev^2\approx2.3\unit{\electronvolt}$ the metal is probably sodium.
  \end{enumerate}
\end{soln}

% PROBLEM 3
\begin{problem}
When a beam of monoenergetic electrons is directed at a tungsten target, X-rays are produced with wavelengths no shorter than
$0.062\unit{\nano\meter}$. How fast are the electrons in the beam moving?
\end{problem}
\begin{soln} This is probably off by a bit because it doesn't account for relativistic effects.
  $\lambda_{min}=0.062\unit{\nano\meter}\implies E=\frac{hc}{\lambda}=\frac{1}{2}m_ev^2\implies\sqrt{\frac{2hc}{m_e\lambda}}=v\approx\qty{8.4d7}{\meter\per\second}$
\end{soln}

% PROBLEM 4
\begin{problem}
A $0.057\unit{\nano\meter}$ X-ray photon ``bounces off'' an initially stationary electron and scatters with a wavelength of $0.061\unit{\nano\meter}$. Find the directions of scatter of
\begin{enumerate}[label=(\alph*)]
  \item The photon.
  \item The electron.
\end{enumerate}
\end{problem}
\begin{soln}~
  \begin{enumerate}[label=(\alph*)]
    \item Using equation 1.3 from the formula sheet,
          \begin{align*}
             & \lambda^\prime-\lambda=\frac{h}{m_ec}\left(1-\cos(\theta)\right)                   \\
             & \theta=\arccos\left(1-\frac{\Delta\lambda m_e c}{h}\right)\approx2.3\unit{\radian} \\
          \end{align*}
    \item Using equations 4 and 5 from the textbook,
          \begin{align*}
             & \frac{h\left(\lambda^\prime-\lambda\cos\theta\right)}{\lambda\lambda^\prime}=\gamma_um_eu\cos\phi
          \end{align*}
          and
          \begin{align*}
             & \frac{h\sin\theta}{\lambda^\prime}=\gamma_um_eu\sin\phi
          \end{align*}
          then,
          \begin{align*}
             & \frac{\frac{h\sin\theta}{\lambda^\prime}}{\frac{h\left(\lambda^\prime-\lambda\cos\theta\right)}{\lambda\lambda^\prime}}=\frac{\gamma_um_eu\sin\phi}{\gamma_um_eu\cos\phi}                                                                          \\
             & \phi=\arctan\left(\frac{\frac{h\sin\theta}{\lambda^\prime}}{\frac{h\left(\lambda^\prime-\lambda\cos\theta\right)}{\lambda\lambda^\prime}}\right)=\arctan\left(\frac{\lambda\sin\theta}{\lambda^\prime-\lambda\cos\theta} \right)=0.4\unit{\radian}
          \end{align*}
  \end{enumerate}
\end{soln}

% PROBLEM 5
\begin{problem}
The setup depicted in Figure 6 is used in a diffraction experiment using X-rays of $0.26\unit{\nano\meter}$ wavelength. Constructive interference is noticed
at angles of $23\unit{\degree}$ and $51.4\unit{\degree}$, but none between. What is the spacing $d$ of atomic planes?
\end{problem}
\begin{soln}
  Using equation 1.4 from the formula sheet, $\frac{m\lambda}{2\sin\theta}=d$. Here, $m=1$, $\lambda=0.26\unit{\nano\meter}$ and $\theta=\frac{23\pi}{180}\unit{\radian}$ so $d\approx330\unit{\pico\meter}$
\end{soln}

% PROBLEM 6
\begin{problem}
The average kinetic energy of a particle at temperature $T$ is $\frac{3}{2}k_BT$.
\begin{enumerate}[label=(\alph*)]
  \item What is the wavelength of a room-temperature ($22\unit{\degreeCelsius}$) electron?
  \item Of a room-temperature proton?
  \item In what circumstances should each behave as a wave?
\end{enumerate}
\end{problem}
\begin{soln}
  $E=\frac{3}{2}k_BT\approx\qty{6d-4}{\joule}$
  \begin{enumerate}[label=(\alph*)]
    \item Knowing that $\lambda=\frac{h}{p}$,
          \begin{align*}
             & E=\frac{1}{2}m_ev^2                                                                                    \\
             & \implies \sqrt{\frac{2E}{m_e}}=v                                                                       \\
             & \implies \frac{h}{m_e\sqrt{\frac{2E}{m_e}}}=\frac{h}{\sqrt{2m_eE}}=\lambda\approx6.3\unit{\nano\meter}
          \end{align*}
    \item We can use the same derived equation as in part (a), $\displaystyle\frac{h}{\sqrt{2m_eE}}=\lambda\approx0.15\unit{\nano\meter}$
    \item When interacting with surroundings of similar scale. E.g. the electron will behave like a wave when interacting with a slit of width on the order of $10\unit{\nano\meter}$ whereas
    the proton would probably only act as a wave when diffracted with something on the order of an Angstrom.
  \end{enumerate}
\end{soln}
\end{document}