\documentclass[10pt]{article}

\usepackage[margin=0.75in]{geometry}
\usepackage{amsmath,amsthm,amssymb}
\usepackage{xcolor}
\usepackage{cancel}
\usepackage{graphicx}
\usepackage{changepage}
\usepackage{circuitikz}
\usepackage{pgfplots}
\usepackage{physics}
\usepackage{hyperref}
\usepackage{siunitx}
\usepackage[breakable]{tcolorbox}
\usepackage[inline]{enumitem}

\theoremstyle{definition}
\newtheorem{problem}{Problem}
\newtheorem{soln}{Solution}

\pgfplotsset{compat=newest}
\usetikzlibrary{lindenmayersystems}
\usetikzlibrary{arrows}
\usetikzlibrary{calc}

\definecolor{incolor}{HTML}{303F9F}
\definecolor{outcolor}{HTML}{D84315}
\definecolor{cellborder}{HTML}{CFCFCF}
\definecolor{cellbackground}{HTML}{F7F7F7}
\newcommand{\eq}{=}
\usetikzlibrary{positioning, fit, calc}
\pgfdeclarelayer{background}  
\pgfsetlayers{background,main}
\AtBeginDocument{\RenewCommandCopy\qty\SI}

\makeatletter
\newcommand{\boxspacing}{\kern\kvtcb@left@rule\kern\kvtcb@boxsep}
\makeatother
\newcommand{\prompt}[4]{
    \ttfamily\llap{{\color{#2}[#3]:\hspace{3pt}#4}}\vspace{-\baselineskip}
}

\newcommand{\thevenin}[2]{
  \begin{center}
    \begin{circuitikz} \draw
      (0,0) -- (2,0) to[battery1, l_=$V_{Th}\eq#1$] (2,2) 
      to[resistor, l_=$R_{Th}\eq#2$] (0,2)
      ;
      \draw [o-] (-.07,2.079);
      \draw [o-] (-.07,0.079);
    \end{circuitikz}
  \end{center}
}

\newcommand{\norton}[2]{
  \begin{center}
    \begin{circuitikz} \draw
      (0,0) -- (3,0) to[american current source, l_=$I_{N}\eq#1$] (3,2) -- (0,2) (2,0)
      to[resistor, l=$R_{N}\eq#2$] (2,2)
      ;
      \draw [o-] (-.07,2.079);
      \draw [o-] (-.07,0.079);
    \end{circuitikz}
  \end{center}
}

\newcommand{\highlight}[1]{\colorbox{yellow}{$\displaystyle #1$}}

\newcommand{\ti}[1]{\widetilde{#1}}

\title{Physics 2610H: Assignment III}
\author{Jeremy Favro}
\date{\today}

\begin{document}
\maketitle

% PROBLEM 1
\begin{problem}
A $\qty{50}{\electronvolt}$ electron is trapped between electrostatic walls $\qty{200}{\electronvolt}$ high.
How far does its wave function extend beyond the walls?
\end{problem}
\begin{soln}
  The penetration depth, given in the text as equation 24, is
  $$\delta=\frac{\hbar}{\sqrt{2m_e\left(U_0-E\right)}}=\frac{\hbar}{\sqrt{2m_e\left(\qty{200}{\electronvolt}-\qty{50}{\electronvolt}\right)\cdot\qty{1.602d-19}{\joule\per\electronvolt}}}\approx\qty{1.59d-11}{\meter}$$
\end{soln}

% PROBLEM 2
\begin{problem}
To a good approximation, the hydrogen chloride molecule, HCl, behaves vibrationally as a quantum harmonic
oscillator of spring constant $\qty{480}{\newton\per\meter}$ and with effective oscillating mass just that of the lighter atom, hydrogen.
If it were in its ground vibrational state, what wavelength photon would be just right to bump this molecule up to its next-higher vibrational energy state?
\end{problem}
\begin{soln}
  The energy of a quantum harmonic oscillator at state $n$ is given by $E=\left(\frac{1}{2}+n\right)\hbar\sqrt{\frac{\kappa}{m}}$ here $\kappa=\qty{480}{\newton\per\meter}$.
  $\Delta E$ for this situation is then $\left(\frac{1}{2}+1-\frac{1}{2}-0\right)\hbar\sqrt{\frac{\kappa}{m}}=\hbar\sqrt{\frac{\kappa}{m_e+m_p}}$ then the wavelength is given by
  $\frac{hc}{E}=\lambda\approx\qty{3.52}{\micro\meter}$
\end{soln}

% PROBLEM 3
\begin{problem}
Calculate the reflection probability for a $\qty{5}{\electronvolt}$ electron encountering a step in which the potential drops by $\qty{2}{\electronvolt}$.
\end{problem}
\begin{soln}
  Reflection probability is given by
  $$\left(\frac{\sqrt{E}-\sqrt{E-U_0}}{\sqrt{E}+\sqrt{E-U_0}}\right)^2$$
  where $E$ is the energy of the particle and $U_0$ the energy of the step. For a potential drop
  $U_0$ will be negative provided we set zero so that (in this case) $U_0$ is $\qty{2}{\electronvolt}$ below zero potential. So the probability is
  $$\left(\frac{\sqrt{\qty{5}{\electronvolt}}-\sqrt{\qty{5}{\electronvolt}+\qty{2}{\electronvolt}}}{\sqrt{\qty{5}{\electronvolt}}+\sqrt{\qty{5}{\electronvolt}+\qty{2}{\electronvolt}}}\right)^2\approx0.704\%$$
\end{soln}

% PROBLEM 4
\begin{problem}
A beam of particles of energy $E$ and incident upon a potential step of $U_0 = \frac{5}{4}E$ is described by the wave function
$$\psi_{inc}(x)=e^{ikx}$$
\begin{enumerate}[label=(\alph*)]
  \item Determine completely the reflected wave and the
        wave inside the step by enforcing the required
        continuity conditions to obtain their (possibly
        complex) amplitudes.
  \item Verify by explicit calculation that the ratio of
        reflected probability density to the incident probability density is 1.
\end{enumerate}
\end{problem}
\begin{soln}~
  \begin{enumerate}[label=(\alph*)]
    \item Here I'll skip a few steps because they are direct from the text and say that because of the condition requiring that the incident, reflected, and transmitted wavefunctions
          must be continuous and so must there first derivatives we obtain that before the step $\psi(x)=Ae^{ikx}+Be^{-ikx}$ and ``inside'' the step $\psi(x)=Ce^{-\alpha x}$ which gives us that
          $A+B=C$ and $ik(A-B)=-\alpha C$. Here $A$ is the amplitude of the incident wave, $B$ that of the reflected wave, and $C$ that of the transmitted wave. Now, by the requirement that
          total probability never exceeds 1 we assume that the incident wave is normalized so $A=1$. So,
          $$1+B=C\implies ik(1-B)=-\alpha \left(1+B\right)\implies B=\frac{-\alpha-ik}{\alpha-ik}$$
          which we then substitute the given values for $\alpha$ and $k$ into,
          \begin{align*}
            B & =\frac{-\frac{\sqrt{\cancel{2m}\left(U_0-E\right)}}{\cancel{\hbar}}-i\frac{\sqrt{\cancel{2m}E}}{\cancel{\hbar}}}{\frac{\sqrt{\cancel{2m}\left(U_0-E\right)}}{\cancel{\hbar}}-i\frac{\sqrt{\cancel{2m}E}}{\cancel{\hbar}}} \\
              & =\frac{-\sqrt{\frac{5}{4}E-E}-i\sqrt{E}}{\sqrt{\frac{5}{4}E-E}-i\sqrt{E}}                                                                                                                                                 \\
              & =\frac{-\frac{1}{2}-i}{\frac{1}{2}-i}                                                                                                                                                                                     \\
              & =\frac{3}{5}-\frac{4}{5}i
          \end{align*}
          And $C=1+B=\frac{8}{5}-\frac{4}{5}i$
    \item \begin{align*}
            R & =\frac{AA^*}{BB^*}                                                                    \\
              & =\frac{1}{\left(\frac{3}{5}-\frac{4}{5}i\right)\left(\frac{3}{5}+\frac{4}{5}i\right)} \\
              & =\frac{1}{\frac{9}{25}+\frac{16}{25}}=1
          \end{align*}
  \end{enumerate}
\end{soln}

% PROBLEM 5
\begin{problem}
What fraction of a beam of $\qty{50}{\electronvolt}$ electrons would get through a $\qty{200}{\volt}$, $\qty{1}{\nano\meter}$ wide electrostatic barrier?
\end{problem}
\begin{soln}
  The transmission probability $T$ for electrons of energy $E$ for potential step of height $U_0$ and ``width'' $L$ is given by
  $$T+\frac{4\left(E/U_0\right)\left(1-E/U_0\right)}{\sinh^2\left(\sqrt{2m\left(U_0-E\right)}\cdot L/\hbar\right)+4\left(E/U_0\right)\left(1-E/U_0\right)}$$
  here we have $U_0=\qty{200}{\electronvolt}$ as an $\unit{\electronvolt}$ is defined as the change in potential energy experienced by an electron across $\qty{1}{\volt}$.
  This yields $T\approx\qty{9.55d-55}{}$ which means that 1 in $\qty{9.55d55}{}$ electrons would make it through the barrier. Note that this answer is a bit far from what is given
  in the back of the book as I used the constants pre-programmed into my calculator to perform the calculation which I assume are more accurate than those used in the book's answers section.
\end{soln}
\end{document}