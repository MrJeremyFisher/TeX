\documentclass[10pt]{article}

\usepackage[margin=0.75in]{geometry}
\usepackage{amsmath,amsthm,amssymb}
\usepackage{xcolor}
\usepackage{cancel}
\usepackage{graphicx}
\usepackage{changepage}
\usepackage{circuitikz}
\usepackage{pgfplots}
\usepackage{physics}
\usepackage{hyperref}
\usepackage{siunitx}
\usepackage[breakable]{tcolorbox}
\usepackage[inline]{enumitem}

\theoremstyle{definition}
\newtheorem{problem}{Problem}
\newtheorem{soln}{Solution}

\pgfplotsset{compat=newest}
\usetikzlibrary{lindenmayersystems}
\usetikzlibrary{arrows}
\usetikzlibrary{calc}

\definecolor{incolor}{HTML}{303F9F}
\definecolor{outcolor}{HTML}{D84315}
\definecolor{cellborder}{HTML}{CFCFCF}
\definecolor{cellbackground}{HTML}{F7F7F7}
\newcommand{\eq}{=}
\usetikzlibrary{positioning, fit, calc}
\pgfdeclarelayer{background}  
\pgfsetlayers{background,main}
\AtBeginDocument{\RenewCommandCopy\qty\SI}

\makeatletter
\newcommand{\boxspacing}{\kern\kvtcb@left@rule\kern\kvtcb@boxsep}
\makeatother
\newcommand{\prompt}[4]{
    \ttfamily\llap{{\color{#2}[#3]:\hspace{3pt}#4}}\vspace{-\baselineskip}
}

\newcommand{\thevenin}[2]{
  \begin{center}
    \begin{circuitikz} \draw
      (0,0) -- (2,0) to[battery1, l_=$V_{Th}\eq#1$] (2,2) 
      to[resistor, l_=$R_{Th}\eq#2$] (0,2)
      ;
      \draw [o-] (-.07,2.079);
      \draw [o-] (-.07,0.079);
    \end{circuitikz}
  \end{center}
}

\newcommand{\norton}[2]{
  \begin{center}
    \begin{circuitikz} \draw
      (0,0) -- (3,0) to[american current source, l_=$I_{N}\eq#1$] (3,2) -- (0,2) (2,0)
      to[resistor, l=$R_{N}\eq#2$] (2,2)
      ;
      \draw [o-] (-.07,2.079);
      \draw [o-] (-.07,0.079);
    \end{circuitikz}
  \end{center}
}

\newcommand{\highlight}[1]{\colorbox{yellow}{$\displaystyle #1$}}

\newcommand{\ti}[1]{\widetilde{#1}}

\title{Physics 2610H: Assignment III}
\author{Jeremy Favro}
\date{\today}

\begin{document}
\maketitle

% PROBLEM 1
\begin{problem}
A $\qty{50}{\electronvolt}$ electron is trapped between electrostatic walls $\qty{200}{\electronvolt}$ high.
How far does its wave function extend beyond the walls?
\end{problem}
\begin{soln}

\end{soln}

% PROBLEM 2
\begin{problem}
To a good approximation, the hydrogen chloride molecule, HCl, behaves vibrationally as a quantum harmonic
oscillator of spring constant $\qty{480}{\newton\per\meter}$ and with effective oscillating mass just that of the lighter atom, hydrogen.
If it were in its ground vibrational state, what wavelength photon would be just right to bump this molecule up to its next-higher vibrational energy state?
\end{problem}
\begin{soln}

\end{soln}

% PROBLEM 3
\begin{problem}
A particle moving in a region of zero force encounters a precipice---a sudden drop in the potential energy to an arbitrarily large negative value.
What is the probability that it will “go over the edge”?
\end{problem}
\begin{soln}

\end{soln}

% PROBLEM 4
\begin{problem}
A beam of particles of energy $E$ and incident upon a potential step of $U_0 = 54E$ is described by the wave function
$$\psi_{inc}(x)=e^{ikx}$$
\begin{enumerate}[label=(\alph*)]
  \item Determine completely the reflected wave and the
        wave inside the step by enforcing the required
        continuity conditions to obtain their (possibly
        complex) amplitudes.
  \item Verify by explicit calculation that the ratio of
        reflected probability density to the incident probability density is 1.
\end{enumerate}
\end{problem}
\begin{soln}

\end{soln}

% PROBLEM 5
\begin{problem}
What fraction of a beam of $\qty{50}{\electronvolt}$ electrons would get through a $\qty{50}{volt}$, $\qty{1}{\nano\meter}$ wide electrostatic barrier?
\end{problem}
\begin{soln}

\end{soln}
\end{document}