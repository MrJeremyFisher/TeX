\documentclass[10pt]{article}

\usepackage[margin=0.75in]{geometry}
\usepackage{amsmath,amsthm,amssymb}
\usepackage{xcolor}
\usepackage{cancel}
\usepackage{graphicx}
\usepackage{changepage}
\usepackage{circuitikz}
\usepackage{pgfplots}
\usepackage{physics}
\usepackage{hyperref}
\usepackage{siunitx}
\usepackage[breakable]{tcolorbox}
\usepackage[inline]{enumitem}

\theoremstyle{definition}
\newtheorem{problem}{Problem}
\newtheorem{soln}{Solution}

\pgfplotsset{compat=newest}
\usetikzlibrary{lindenmayersystems}
\usetikzlibrary{arrows}
\usetikzlibrary{calc}

\definecolor{incolor}{HTML}{303F9F}
\definecolor{outcolor}{HTML}{D84315}
\definecolor{cellborder}{HTML}{CFCFCF}
\definecolor{cellbackground}{HTML}{F7F7F7}
\newcommand{\eq}{=}
\usetikzlibrary{positioning, fit, calc}
\pgfdeclarelayer{background}  
\pgfsetlayers{background,main}
\AtBeginDocument{\RenewCommandCopy\qty\SI}

\makeatletter
\newcommand{\boxspacing}{\kern\kvtcb@left@rule\kern\kvtcb@boxsep}
\makeatother
\newcommand{\prompt}[4]{
    \ttfamily\llap{{\color{#2}[#3]:\hspace{3pt}#4}}\vspace{-\baselineskip}
}

\newcommand{\thevenin}[2]{
  \begin{center}
    \begin{circuitikz} \draw
      (0,0) -- (2,0) to[battery1, l_=$V_{Th}\eq#1$] (2,2) 
      to[resistor, l_=$R_{Th}\eq#2$] (0,2)
      ;
      \draw [o-] (-.07,2.079);
      \draw [o-] (-.07,0.079);
    \end{circuitikz}
  \end{center}
}

\newcommand{\norton}[2]{
  \begin{center}
    \begin{circuitikz} \draw
      (0,0) -- (3,0) to[american current source, l_=$I_{N}\eq#1$] (3,2) -- (0,2) (2,0)
      to[resistor, l=$R_{N}\eq#2$] (2,2)
      ;
      \draw [o-] (-.07,2.079);
      \draw [o-] (-.07,0.079);
    \end{circuitikz}
  \end{center}
}

\newcommand{\highlight}[1]{\colorbox{yellow}{$\displaystyle #1$}}

\newcommand{\ti}[1]{\widetilde{#1}}

\title{Physics 2610H: Assignment IV}
\author{Jeremy Favro}
\date{\today}

\begin{document}
\maketitle

% PROBLEM 1
\begin{problem}
Use the tables of $R(r)$, $\Theta(\theta)$, and $\Phi(\phi)$ to describe the mathematical dependence on the $r$, $\theta$, and $\phi$
variables of the $n = 3$, $\ell=2$, $m_\ell=2$ energy eigenstate of the hydrogen atom. Use computer software of
your choice (Excel, Matlab, \emph{etc.}) to plot the shape of the radial probability density, $P(r)$, of this state
against $r$. Include units! Denote on your plot the most probable radius for an electron in this state.
\end{problem}
\begin{soln}
  $$R(r)=\frac{1}{\left(3a_0\right)^{3/2}}\frac{2\sqrt{2}r^2}{27\sqrt{5}a_0^2}e^{-r/3a_0}$$ 
  $$\Theta(\theta)=\sqrt{\frac{15}{32\pi}}\sin^2\theta$$ 
  $$\Phi(\phi)=e^{2i\phi}$$
\end{soln}

% PROBLEM 2
\begin{problem}
Show that the degeneracy of the $n$-th state of the hydrogen atom \emph{if} the electron had no spin (so course
material only up to Chapter 7) would be $n^2$. That is, prove the following equality and in a sentence or two
state why this equation answers this question:
$$\sum_{\ell=0}^{n-1}\left(2\ell+1\right)=n^2$$
\end{problem}
\begin{soln}
\end{soln}

% PROBLEM 3
\begin{problem}
For a hydrogen atom in the $1s$ state
\begin{enumerate}[label=(\alph*)]
  \item What is the most probable value of $r$ from a measurement of $r$?
  \item What is the average value of $r$?
  \item where in 3-dimensional space is the electron most likely to be within a small volume of \underline{fixed} size $dV$?
\end{enumerate}
\end{problem}
\begin{soln}
\end{soln}

% PROBLEM 4
\begin{problem}
Evaluate, in electron volts, the energies of the three lowest energy levels of the hydrogen atom. Then
calculate the frequencies in $\unit{\hertz}$, and the wavelengths in $\unit{\angstrom}$ of all the photons that can be emitted by the
atom in transitions between these levels. In what range of the electromagnetic spectrum are these photons?
\end{problem}
\begin{soln}
\end{soln}

% PROBLEM 5
\begin{problem}
Consider the probability of finding the electron in the hydrogen atom somewhere inside a cone of $\qty{23.5}{\degree}$
(``arctic polar region'').
\begin{enumerate}[label=(\alph*)]
  \item If the electron were equally likely to be found anywhere in space, what would be the probability of finding the
        electron in the arctic polar region?
  \item Suppose the electron is in the state $n = 2$, $\ell = 1$, $m_\ell = 0$ (i.e. a $2p_z$ orbital); recalculate the probability of
        finding the electron in the arctic polar region.
\end{enumerate}
\end{problem}
\begin{soln}
\end{soln}

% PROBLEM 6
\begin{problem}
An electron (with spin) in a hydrogen atom is in the $4f_{5/2}$ state. Find the values of the quantum numbers
$n$, $\ell$ and $j$. What is the magnitude of the electron's total angular momentum? What are the possible values
for the $z$-component of the electron's total angular momentum?
\end{problem}
\begin{soln}
\end{soln}

% PROBLEM 7
\begin{problem}
Neglecting any effects of spin (i.e. if both particles are spinless), how many quantum states are possible
for a two-particle system with orbital angular momentum quantum numbers $\ell_1=2$ and $\ell_2=1$? Write down
all these states in what is called the \emph{uncoupled} form: ($\ell_1m_{\ell_1}\ell_2m_{\ell_2}$). Then find all possible values of the
total orbital angular momentum quantum number $\ell_T$ and the total orbital magnetic quantum number $m_{\ell_T}$. Write down all these states
in the \emph{coupled} form: ($\ell_1\ell_2\ell_Tm_{\ell_T}$).
\end{problem}
\begin{soln}
\end{soln}

% PROBLEM 8
\begin{problem}
Two spinless particles, assumed initially to be distinguishable from each other, occupy two different
states ($n=2$ and $n^\prime=3$) in a 1-D infinite well of length $L$. The \emph{product wave function} for the case with particle 1 in
state $n=2$ and particle 2 in state $n^\prime=3$ is
$$\psi(x_1,x_2)=A\sin\left(\frac{2\pi x_1}{L}\right)\sin\left(\frac{3\pi x_2}{L}\right),\qquad A=\frac{2}{L}$$
\begin{enumerate}[label=(\alph*)]
  \item Show that with this normalization constant this product wave function is correctly normalized.
  \item If the two particles are instead \emph{identical} how should this product wave function be adapted to form two
        realistic wave functions? Given that the normalization constant for each of these wave functions is $L=\sqrt{2}/L$
        in each case what is the probability of both particles being found in the left half of the well (i.e. between $0$ and $L/2$).
\end{enumerate}
\end{problem}
\begin{soln}
\end{soln}
\end{document}