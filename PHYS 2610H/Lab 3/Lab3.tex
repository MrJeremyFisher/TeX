\documentclass[10pt, twocolumn]{article}

\usepackage[margin=0.75in]{geometry}
\usepackage{amsmath,amsthm,amssymb}
\usepackage{xcolor}
\usepackage{lipsum}
\usepackage{cancel}
\usepackage{pdflscape}
\usepackage{graphicx}
\usepackage{changepage}
\usepackage{circuitikz}
\usepackage{pgfplots}
\usepackage[maxbibnames=10]{biblatex}
\usepackage[separate-uncertainty=true]{siunitx}
\usepackage{hyperref}
\usepackage{multicol, multirow, booktabs}
\usepackage[breakable]{tcolorbox}
\usepackage[inline]{enumitem}

\theoremstyle{definition}
\newtheorem{problem}{Problem}
\newtheorem{soln}{Solution}

\pgfplotsset{compat=newest}
\usetikzlibrary{lindenmayersystems}
\usetikzlibrary{arrows}
\usetikzlibrary{calc}
\usetikzlibrary{plotmarks}

\definecolor{incolor}{HTML}{303F9F}
\definecolor{outcolor}{HTML}{D84315}
\definecolor{cellborder}{HTML}{CFCFCF}
\definecolor{cellbackground}{HTML}{F7F7F7}
\newcommand{\eq}{=}
\usetikzlibrary{positioning, fit, calc}
\pgfdeclarelayer{background}  
\pgfsetlayers{background,main}
\DeclareSIUnit[number-unit-product = {\,}]\calorie{cal}
\DeclareSIUnit[number-unit-product = {\,}]\atmosphere{atm}

\makeatletter
\newcommand{\boxspacing}{\kern\kvtcb@left@rule\kern\kvtcb@boxsep}
\makeatother
\newcommand{\prompt}[4]{
    \ttfamily\llap{{\color{#2}[#3]:\hspace{3pt}#4}}\vspace{-\baselineskip}
}

\addbibresource{citations.bib}

\newcommand{\thevenin}[2]{
  \begin{center}
    \begin{circuitikz} \draw
      (0,0) -- (2,0) to[battery1, l_=$V_{Th}\eq#1$] (2,2) 
      to[resistor, l_=$R_{Th}\eq#2$] (0,2)
      ;
      \draw [o-] (-.07,2.079);
      \draw [o-] (-.07,0.079);
    \end{circuitikz}
  \end{center}
}

\newcommand{\norton}[2]{
  \begin{center}
    \begin{circuitikz} \draw
      (0,0) -- (3,0) to[american current source, l_=$I_{N}\eq#1$] (3,2) -- (0,2) (2,0)
      to[resistor, l=$R_{N}\eq#2$] (2,2)
      ;
      \draw [o-] (-.07,2.079);
      \draw [o-] (-.07,0.079);
    \end{circuitikz}
  \end{center}
}

\newcommand{\highlight}[1]{\colorbox{yellow}{$\displaystyle #1$}}

\newcommand{\ti}[1]{\widetilde{#1}}

\title{Physics 2610H: Lab III}
\author{Jeremy Favro (0805980),
Manan Ravat (0791811),
Layla Scrimgeour-Brown (0766619)
 \\\emph{Department of Physics \& Astronomy}\\ Trent University, Peterborough, ON, Canada}
\date{\today}
\begin{document}
\maketitle
\begin{abstract}
\end{abstract}
\section{Introduction}
Spectroscopy is a useful tool in many fields. A visible light spectrometer---such as the one employed in this experiment---can be used to determine the composition of an object by measuring emission or absorption of
specific wavelengths of photons by that object.

This experiment seeks to employ a digital visible light spectrometer to verify calculated values for electron energy level transitions in a helium atom,
determine the components of unknown visible light emitters, and determine a method to resolve merged spectral lines for energy
level transitions with similar emitted photons.
\section{Theory}
Electrons in orbit around atomic nuclei exist in discrete, quantized, energy levels and may only move between these levels when a ``hole''
is available for them to do so. In order to ``jump'' up to a higher energy level an electron must be given energy, usually by a photon, who's energy is given by
$$E=\frac{hc}{\lambda}$$ where $h$ is Planck's constant\autocite{codata}, $c$ the speed of light in the medium through which the photon moves, and $\lambda$ the
wavelength of the photon. Electrons may also decay to a lower energy level, emitting the energy lost in that process as a photon of energy given by the difference in energy
between the final and initial energy levels.

Visible light spectrometers record these energy level transitions by recording the intensity of light due to the emitted photons
and mapping it to its corresponding wavelength.
\vfill\null
\section{Methods}
This experiment utilized a software controlled device known as a monochromator to diffract incoming light into its component wavelengths
and measure the intensity of each individually. The monochromator utilizes a reflective diffraction grating alongside entrance and exit slits
and several internal mirrors to diffract, collimate, and deliver incoming light to a visible light sensitive photodiode connected to an Arduino.
A MicroPython script was utilized to control the device and record data produced by the photodiode.

Three light sources were measured with the monochromator. The light from each was directed to a lens which focused it onto
the input slit. The output slit width remained constant throughout the experiment and only the wavelength step, range, and input
slit width were varied.

Light sources were placed such that the photodiode registered the maximum emission strength when the diffraction grating was in its rest position.
\newpage
\section{Discussion}
\subsection{Helium Spectrum}
\begin{table}[h]
  \centering%
  \begin{tabular}{p{0.2\columnwidth}p{0.1\columnwidth}p{0.2\columnwidth}p{0.06\columnwidth}p{0.1\columnwidth}}
    \toprule
    Assigned Transition         & Pred. wavelength, $\lambda_p (\unit{\nano\meter})$ & Obs. wavelength, $\lambda_o (\unit{\nano\meter})$ & \% Diff & Comments        \\
    \midrule
    $1s0s{}^1\!S$ $1s3p{}^1\!P$ & $392$                                              &                                                   &         &                 \\
    \hline
    $1s0s{}^3\!S$ $1s3p{}^3\!P$ & $412$                                              &                                                   &         &                 \\
    \hline
    $1s1s{}^1\!S$ $1s3p{}^1\!P$ & $505$                                              & $505\pm1.7$                                       & 0.337   & Extremely faint \\
    \hline
    $1s1s{}^3\!S$ $1s3p{}^3\!P$ & $471$                                              & $471\pm0.6$                                       & 0.127   &                 \\
    \hline
    $1s2s{}^1\!S$ $1s1p{}^1\!P$ & $397$                                              &                                                   &         &                 \\
    \hline
    $1s2s{}^1\!S$ $1s2p{}^1\!P$ & $502$                                              & $502\pm1.6$                                       & 0.319   &                 \\
    \hline
    $1s2s{}^3\!S$ $1s3p{}^3\!P$ & $707$                                              & $707\pm1.5$                                       & 0.212   &                 \\
    \hline
    $1s3s{}^1\!P$ $1s0p{}^1\!D$ & $439$                                              &                                                   &         &                 \\
    \hline
    $1s3s{}^1\!P$ $1s1p{}^1\!D$ & $492$                                              & $492\pm1.6$                                       & 0.325   &                 \\
    \hline
    $1s3s{}^1\!P$ $1s2p{}^1\!D$ & $668$                                              & $668\pm1.8$                                       & 0.269   &                 \\
    \hline
    $1s3s{}^1\!S$ $1s3p{}^1\!P$ & $728$                                              & $728\pm1.5$                                       & 0.206   &                 \\
    \hline
    $1s3s{}^3\!P$ $1s0p{}^3\!D$ & $403$                                              &                                                   &         &                 \\
    \hline
    $1s3s{}^3\!P$ $1s1p{}^3\!D$ & $447$                                              & $447\pm1.6$                                       & 0.358   &                 \\
    \hline
    $1s3s{}^3\!P$ $1s2p{}^3\!D$ & $588$                                              & $588\pm1.5$                                       & 0.255   &                 \\
    \hline
    $1s3s{}^3\!S$ $1s2p{}^3\!P$ & $389$                                              & $389\pm1.4$                                       & 0.360   &                 \\
    \bottomrule
  \end{tabular}
  \label{he-res-table}
\end{table}
\begin{figure}
  \begin{tikzpicture}[define rgb/.code={\definecolor{mycolor}{RGB}{#1}},
      rgb color/.style={define rgb={#1},mycolor}]
    \begin{axis}[
        % ymode=log,
        width=\columnwidth,
        axis background/.style={fill=black!60},
        xmin=350,
        xmax=750,
        legend pos=north west
      ]
      \addplot table [col sep=comma,
          header=false,
          x index=0,
          y index=1,
          mark=none
        ] {He/He_VIS_100_0.1nm_250um_350-750-PROCESSED.csv};
      \addplot table [col sep=comma,
          header=false,
          x index=0,
          y index=1,
          mark=none
        ] {Psi/Psi_VIS_100_1nm_250um_350-750-PROCESSED.csv};
      \addplot[green] table [col sep=comma,
          header=false,
          x index=0,
          y index=1,
          mark=none
        ] {Phi/Phi_VIS_100_0.1nm_250um_350-750-PROCESSED.csv};
      \draw[rgb color={124, 0, 149}] (392.233771,0.01) -- (392.233771,0.7);
      \draw[rgb color={0, 255, 97}] (504.922999,0.01) -- (504.922999,0.7);
      \draw[rgb color={129, 0, 165}] (396.573604,0.01) -- (396.573604,0.7);
      \draw[rgb color={0, 255, 135}] (501.705800,0.01) -- (501.705800,0.7);
      \draw[rgb color={179, 0, 0}] (728.119994,0.01) -- (728.119994,0.7);
      \draw[rgb color={17, 0, 255}] (438.788943,0.01) -- (438.788943,0.7);
      \draw[rgb color={0, 255, 234}] (492.319811,0.01) -- (492.319811,0.7);
      \draw[rgb color={234, 0, 0}] (668.002672,0.01) -- (668.002672,0.7);
      \draw[rgb color={123, 0, 226}] (412.201154,0.01) --(412.201154,0.7);
      \draw[rgb color={0, 174, 255}] (471.453491,0.01) -- (471.453491,0.7);
      \draw[rgb color={131, 0, 188}] (706.514060,0.01) -- (706.514060,0.7);
      \draw[rgb color={118, 0, 132}] (388.983974,0.01) -- (388.983974,0.7);
      \draw[rgb color={131, 0, 188}] (402.608906,0.01) -- (402.608906,0.7);
      \draw[rgb color={0, 53, 255}] (447.287203,0.01) -- (447.287203,0.7);
      \draw[rgb color={255, 233, 0}] (587.716721,0.01) -- (587.716721,0.7); % Yuck
      \addlegendentry{He Spectrum}
      \addlegendentry{$\phi$ Spectrum}
      \addlegendentry{$\psi$ Specrum}
    \end{axis}
    \label{helium-spectrum-1}
  \end{tikzpicture}
  \caption{Helium, $\psi$, and $\phi$ lamp spectrum. Coloured lines denote predicted values for helium.
    See appendix for full size with log scale.
    Generated with TikZ}
\end{figure}
\subsubsection{``Missing'' Wavelengths}
Five expected wavelengths (see \ref{he-res-table}) in the low wavelength region are missing from the recorded data.
This is likely because the diode is less sensitive at those lower wavelengths\autocite{diode-manufacturer} however there is
still a strong peak at $\qty{389\pm1.4}{\nano\meter}$ which is the lowest expected approximately visible wavelength.
\subsection{Doublet Resolution}
\begin{figure}
  \begin{tikzpicture}
    \begin{axis}[
        ymode=log,
        width=\columnwidth,
        xmin=587,
        xmax=591,
        legend pos=south west
      ]
      \addplot table [col sep=comma,
          header=false,
          x index=0,
          y index=1,
          mark=none
        ] {Psi/Doublets/Psi_VIS_100_0.02nm_0um_587-591-PROCESSED.csv};
      \addplot table [col sep=comma,
          header=false,
          x index=0,
          y index=1,
          mark=none
        ] {Psi/Doublets/Psi_VIS_100_0.02nm_5um_588-590.5-PROCESSED.csv};
      \addplot table [col sep=comma,
          header=false,
          x index=0,
          y index=1,
          mark=none
        ] {Psi/Doublets/Psi_VIS_100_0.02nm_10um_588-590.5-PROCESSED.csv};
      \addplot table [col sep=comma,
          header=false,
          x index=0,
          y index=1,
          mark=none
        ] {Psi/Doublets/Psi_VIS_100_0.02nm_50um_588-590.5-PROCESSED.csv};
      \addplot[green] table [col sep=comma,
          header=false,
          x index=0,
          y index=1,
          mark=none
        ] {Psi/Doublets/Psi_VIS_100_0.02nm_250um_588-590.5-PROCESSED.csv};
      \addlegendentry{$0\unit{\micro\meter}$}
      \addlegendentry{$5\unit{\micro\meter}$}
      \addlegendentry{$10\unit{\micro\meter}$}
      \addlegendentry{$50\unit{\micro\meter}$}
      \addlegendentry{$250\unit{\micro\meter}$}
    \end{axis}
    \label{psi-spectrum-1}
  \end{tikzpicture}
  \caption{$\psi$ lamp spectrum focused on yellow doublet. Note significantly decreased intensity for a $0\unit{\micro\meter}$ slit width requiring semilog scale. See appendix for full size.}
\end{figure}
The $\psi$ lamp exhibited two doublets between $\qty{588}{\nano\meter}$ and $\qty{590.5}{\nano\meter}$. As can be seen in
\ref{psi-spectrum-1} the doublets were successfully resolved by decreasing the input slit width. And optimal value for this slit
width is somewhere between $\qty{5}{\micro\meter}$ and $\qty{10}{\micro\meter}$ as a $\qty{0}{\micro\meter}$ slit significantly impacts
the amount of light entering the apparatus whereas a slit widths above $\qty{10}{\micro\meter}$ do not resolve the doublet well.
\subsection{Guessing the $\phi$ Lamp}
As can be see in \ref{helium-spectrum-1} the $\phi$ lamp has strong peaks at approximately
$\qty{436}{\nano\meter}$,$\qty{546}{\nano\meter}$, $\qty{577}{\nano\meter}$, $\qty{580}{\nano\meter}$. Using the
NIST Atomic Spectra Database\autocite{spectrum-db} to determine materials with similar peaks suggests Mercury as a likely
match. It exhibits many of the same emission peaks as the recorded data,
with the recorded data only missing the peaks on the far ends of the visible spectrum where the photodiode employed in
this experiment may not be as sensitive as the instruments employed by NIST.
\subsection{Sources of error}
\subsubsection{Systematic Error}
The most significant source of systematic error in this experiment stems from the fact that the experiment
was not conducted in ``total'' darkness. A nearby window introduced not insignificant---though not quantified---
levels of stray light alongside the screen of the computer used to control the monochromator and the lamps not
being studied. These sources were largely constant throughout the experiment and the calibration of the raw data by
subtracting the minimum nonzero reading from the photodiode eliminated the majority of their
influence.
\subsubsection{Random Error}
The most significant source of random error in this experiment was likely the non-rigidity of the apparatus. Only the helium
lamp was placed directly infront of the monochromator's entrance slit, the $\psi$ and $\phi$ lamps both had their light directed
towards the entrance slit using a mirror that was not securely fixed to any rigid mounting point. The simplest way of
reducing the effects of this non-rigidity would likely be performing the experiment on an optical table with securely fixed
mirrors, lenses, and lightsources.
\section{Conclusion}
This experiment successfully verified the hydrogen emission spectrum save for some lower wavelength emission peaks,
likely due to the lower sensitivity of the employed photodiode in this range. The experiment also matched the $\phi$
lamp to the emission spectrum of mercury and successfully resolved the yellow doublet of the $\psi$ lamp using a 
decreased entrance slit width on the monochromator.
\section{Bibliography}
\printbibliography

\section{Appendix}
\subsection{Figures}

\begin{landscape}
  \begin{figure}
    \begin{tikzpicture}[define rgb/.code={\definecolor{mycolor}{RGB}{#1}},
        rgb color/.style={define rgb={#1},mycolor}]
      \begin{axis}[
          ymode=log,
          width=1.33*\textwidth,
          height=\textheight,
          axis background/.style={fill=black!60},
          xmin=350,
          xmax=750,
          legend pos=south east
        ]
        \addplot table [col sep=comma,
            header=false,
            x index=0,
            y index=1,
            mark=none,
            blue] {He/He_VIS_100_0.1nm_250um_350-750-PROCESSED.csv};
        \addplot table [col sep=comma,
            header=false,
            x index=0,
            y index=1,
            mark=none
          ] {Phi/Phi_VIS_100_0.1nm_250um_350-750-PROCESSED.csv};
        \addplot[green] table [col sep=comma,
            header=false,
            x index=0,
            y index=1,
            mark=none
          ] {Psi/Psi_VIS_100_1nm_250um_350-750-PROCESSED.csv};
        \draw[rgb color={124, 0, 149}] (392.233771,0.01) -- (392.233771,0.7);
        \draw[rgb color={0, 255, 97}] (504.922999,0.01) -- (504.922999,0.7);
        \draw[rgb color={129, 0, 165}] (396.573604,0.01) -- (396.573604,0.7);
        \draw[rgb color={0, 255, 135}] (501.705800,0.01) -- (501.705800,0.7);
        \draw[rgb color={179, 0, 0}] (728.119994,0.01) -- (728.119994,0.7);
        \draw[rgb color={17, 0, 255}] (438.788943,0.01) -- (438.788943,0.7);
        \draw[rgb color={0, 255, 234}] (492.319811,0.01) -- (492.319811,0.7);
        \draw[rgb color={234, 0, 0}] (668.002672,0.01) -- (668.002672,0.7);
        \draw[rgb color={123, 0, 226}] (412.201154,0.01) --(412.201154,0.7);
        \draw[rgb color={0, 174, 255}] (471.453491,0.01) -- (471.453491,0.7);
        \draw[rgb color={131, 0, 188}] (706.514060,0.01) -- (706.514060,0.7);
        \draw[rgb color={118, 0, 132}] (388.983974,0.01) -- (388.983974,0.7);
        \draw[rgb color={131, 0, 188}] (402.608906,0.01) -- (402.608906,0.7);
        \draw[rgb color={0, 53, 255}] (447.287203,0.01) -- (447.287203,0.7);
        \draw[rgb color={255, 233, 0}] (587.716721,0.01) -- (587.716721,0.7); % Yuck
        \addlegendentry{He Spectrum}
        \addlegendentry{$\phi$ Spectrum}
        \addlegendentry{$\psi$ Specrum}
      \end{axis}
      \label{figure-1-large}
    \end{tikzpicture}
    \caption{Helium spectrum. Coloured lines denote predicted values.}
  \end{figure}
\end{landscape}
\newpage
\begin{landscape}
  \begin{figure}
    \begin{tikzpicture}
      \begin{axis}[
          ymode=log,
          width=\columnwidth,
          height=\textheight,
          xmin=588,
          xmax=590.5,
          legend pos=south west
        ]
        \addplot table [col sep=comma,
            header=false,
            x index=0,
            y index=1,
            mark=none
          ] {Psi/Doublets/Psi_VIS_100_0.02nm_0um_587-591-PROCESSED.csv};
        \addplot table [col sep=comma,
            header=false,
            x index=0,
            y index=1,
            mark=none
          ] {Psi/Doublets/Psi_VIS_100_0.02nm_5um_588-590.5-PROCESSED.csv};
        \addplot table [col sep=comma,
            header=false,
            x index=0,
            y index=1,
            mark=none
          ] {Psi/Doublets/Psi_VIS_100_0.02nm_10um_588-590.5-PROCESSED.csv};
        \addplot table [col sep=comma,
            header=false,
            x index=0,
            y index=1,
            mark=none
          ] {Psi/Doublets/Psi_VIS_100_0.02nm_50um_588-590.5-PROCESSED.csv};
        \addplot[green] table [col sep=comma,
            header=false,
            x index=0,
            y index=1,
            mark=none
          ] {Psi/Doublets/Psi_VIS_100_0.02nm_250um_588-590.5-PROCESSED.csv};
        \addlegendentry{$0\unit{\micro\meter}$}
        \addlegendentry{$5\unit{\micro\meter}$}
        \addlegendentry{$10\unit{\micro\meter}$}
        \addlegendentry{$50\unit{\micro\meter}$}
        \addlegendentry{$250\unit{\micro\meter}$}
      \end{axis}
      \label{psi-spectrum-1}
    \end{tikzpicture}
    \caption{$\psi$ lamp spectrum focused on yellow doublet. Note significantly decreased intensity for a $0\unit{\micro\meter}$ slit width requiring semilog scale.}
  \end{figure}
\end{landscape}
\end{document}