\documentclass[10pt, twocolumn]{article}

\usepackage[margin=0.75in]{geometry}
\usepackage{amsmath,amsthm,amssymb}
\usepackage{xcolor}
\usepackage{lipsum}
\usepackage{cancel}
\usepackage{pdflscape}
\usepackage{graphicx}
\usepackage{changepage}
\usepackage{circuitikz}
\usepackage{pgfplots}
\usepackage{siunitx}
\usepackage{hyperref}
\usepackage{cite}
\usepackage{multicol, multirow, booktabs}
\usepackage[breakable]{tcolorbox}
\usepackage[inline]{enumitem}

\theoremstyle{definition}
\newtheorem{problem}{Problem}
\newtheorem{soln}{Solution}

\pgfplotsset{compat=newest}
\usetikzlibrary{lindenmayersystems}
\usetikzlibrary{arrows}
\usetikzlibrary{calc}

\definecolor{incolor}{HTML}{303F9F}
\definecolor{outcolor}{HTML}{D84315}
\definecolor{cellborder}{HTML}{CFCFCF}
\definecolor{cellbackground}{HTML}{F7F7F7}
\newcommand{\eq}{=}
\usetikzlibrary{positioning, fit, calc}
\pgfdeclarelayer{background}  
\pgfsetlayers{background,main}
\DeclareSIUnit[number-unit-product = {\,}]\calorie{cal}
\DeclareSIUnit[number-unit-product = {\,}]\atmosphere{atm}

\makeatletter
\newcommand{\boxspacing}{\kern\kvtcb@left@rule\kern\kvtcb@boxsep}
\makeatother
\newcommand{\prompt}[4]{
    \ttfamily\llap{{\color{#2}[#3]:\hspace{3pt}#4}}\vspace{-\baselineskip}
}

\newcommand{\thevenin}[2]{
  \begin{center}
    \begin{circuitikz} \draw
      (0,0) -- (2,0) to[battery1, l_=$V_{Th}\eq#1$] (2,2) 
      to[resistor, l_=$R_{Th}\eq#2$] (0,2)
      ;
      \draw [o-] (-.07,2.079);
      \draw [o-] (-.07,0.079);
    \end{circuitikz}
  \end{center}
}

\newcommand{\norton}[2]{
  \begin{center}
    \begin{circuitikz} \draw
      (0,0) -- (3,0) to[american current source, l_=$I_{N}\eq#1$] (3,2) -- (0,2) (2,0)
      to[resistor, l=$R_{N}\eq#2$] (2,2)
      ;
      \draw [o-] (-.07,2.079);
      \draw [o-] (-.07,0.079);
    \end{circuitikz}
  \end{center}
}

\newcommand{\highlight}[1]{\colorbox{yellow}{$\displaystyle #1$}}

\newcommand{\ti}[1]{\widetilde{#1}}

\title{Physics 2700H: Lab III}
\author{Jeremy Favro (0805980),
Manan Ravat (0791811),
Layla Scrimgeour-Brown (0766619)
 \\\emph{Department of Physics \& Astronomy}\\ Trent University, Peterborough, ON, Canada}
\date{\today}
\begin{document}
\maketitle
\begin{abstract}
\end{abstract}
\section{Introduction}
Spectroscopy is a useful tool in many fields. A visible light spectrometer---such as the one employed in this experiment---can be used to determine the composition of an object by measuring emission or absorption of
specific wavelengths of photons by that object.

This experiment seeks to employ a digital visible light spectrometer to verify calculated values for electron energy level transitions in a helium atom,
determine the active ``ingredient'' \textbf{REPHRASE} in unknown visible light emitters, and determine a method to resolve merged spectral lines for energy
level transitions with similar emitted photons.
\section{Theory}
Electrons in orbit around atomic nuclei exist in specific, quantized, energy levels and may only move between these levels when a ``hole'' \textbf{REWORK}
is available for them to do so. In order to ``jump'' up to a higher energy level an electron must be given energy, usually by a photon, who's energy is given by
$$E=\frac{hc}{\lambda}$$ where $h$ is Planck's constant\cite{codata}, $c$ the speed of light in the medium through which the photon moves, and $\lambda$ the
wavelength of the photon. Electrons may also decay to a lower energy level, emitting the energy lost in that process as a photon of energy given by the difference in energy
between the final and initial energy levels.

The spectrometer employed in this experiment records the number of these emitted photons for a given wavelength by separating\dots
\section{Methods}
\lipsum[3-5]
\section{Discussion}
\subsection{Helium Spectrum}
\begin{figure}
  \begin{tikzpicture}
    \begin{axis}[
        ymode=log,
        width=\columnwidth,
      ]
      \addplot table [col sep=comma,
          header=false,
          x index=0,
          y index=1,
          mark=none] {He/He_VIS_100_0.1nm_250um_350-750-PROCESSED.csv};
    \end{axis}
    \label{figure-1}
  \end{tikzpicture}
  \caption{Helium spectrum}
\end{figure}
\subsection{Sources of error}
\lipsum[10]
\section{Conclusion}
\lipsum[11]
\section{Bibliography}
\bibliography{bib}{}
\bibliographystyle{plain}

\section{Appendix}
\lipsum[12]
\subsection{Figures}
\begin{landscape}
  \begin{tikzpicture}
    \begin{axis}[
        ymode=log,
        width=1.33*\textwidth,
        height=\textheight
      ]
      \addplot table [col sep=comma,
          header=false,
          x index=0,
          y index=1,
          mark=none] {He/He_VIS_100_0.1nm_250um_350-750-PROCESSED.csv};
    \end{axis}
    \label{figure-1-large}
  \end{tikzpicture}
\end{landscape}
\end{document}