\documentclass[10pt, twocolumn]{article}

\usepackage[margin=0.75in]{geometry}
\usepackage{amsmath,amsthm,amssymb}
\usepackage{xcolor}
\usepackage{lipsum}
\usepackage{cancel}
\usepackage{pdflscape}
\usepackage{graphicx}
\usepackage{changepage}
\usepackage{circuitikz}
\usepackage{pgfplots}
\usepackage{siunitx}
\usepackage{hyperref}
\usepackage{cite}
\usepackage{multicol, multirow, booktabs}
\usepackage[breakable]{tcolorbox}
\usepackage[inline]{enumitem}

\theoremstyle{definition}
\newtheorem{problem}{Problem}
\newtheorem{soln}{Solution}

\pgfplotsset{compat=newest}
\usetikzlibrary{lindenmayersystems}
\usetikzlibrary{arrows}
\usetikzlibrary{calc}
\usetikzlibrary{plotmarks}

\definecolor{incolor}{HTML}{303F9F}
\definecolor{outcolor}{HTML}{D84315}
\definecolor{cellborder}{HTML}{CFCFCF}
\definecolor{cellbackground}{HTML}{F7F7F7}
\newcommand{\eq}{=}
\usetikzlibrary{positioning, fit, calc}
\pgfdeclarelayer{background}  
\pgfsetlayers{background,main}
\DeclareSIUnit[number-unit-product = {\,}]\calorie{cal}
\DeclareSIUnit[number-unit-product = {\,}]\atmosphere{atm}

\makeatletter
\newcommand{\boxspacing}{\kern\kvtcb@left@rule\kern\kvtcb@boxsep}
\makeatother
\newcommand{\prompt}[4]{
    \ttfamily\llap{{\color{#2}[#3]:\hspace{3pt}#4}}\vspace{-\baselineskip}
}

\newcommand{\thevenin}[2]{
  \begin{center}
    \begin{circuitikz} \draw
      (0,0) -- (2,0) to[battery1, l_=$V_{Th}\eq#1$] (2,2) 
      to[resistor, l_=$R_{Th}\eq#2$] (0,2)
      ;
      \draw [o-] (-.07,2.079);
      \draw [o-] (-.07,0.079);
    \end{circuitikz}
  \end{center}
}

\newcommand{\norton}[2]{
  \begin{center}
    \begin{circuitikz} \draw
      (0,0) -- (3,0) to[american current source, l_=$I_{N}\eq#1$] (3,2) -- (0,2) (2,0)
      to[resistor, l=$R_{N}\eq#2$] (2,2)
      ;
      \draw [o-] (-.07,2.079);
      \draw [o-] (-.07,0.079);
    \end{circuitikz}
  \end{center}
}

\newcommand{\highlight}[1]{\colorbox{yellow}{$\displaystyle #1$}}

\newcommand{\ti}[1]{\widetilde{#1}}

\title{Physics 2700H: Lab III}
\author{Jeremy Favro (0805980),
Manan Ravat (0791811),
Layla Scrimgeour-Brown (0766619)
 \\\emph{Department of Physics \& Astronomy}\\ Trent University, Peterborough, ON, Canada}
\date{\today}
\begin{document}
\maketitle
\begin{abstract}
\end{abstract}
\section{Introduction}
Spectroscopy is a useful tool in many fields. A visible light spectrometer---such as the one employed in this experiment---can be used to determine the composition of an object by measuring emission or absorption of
specific wavelengths of photons by that object.

This experiment seeks to employ a digital visible light spectrometer to verify calculated values for electron energy level transitions in a helium atom,
determine the active ``ingredient'' \textbf{REPHRASE} in unknown visible light emitters, and determine a method to resolve merged spectral lines for energy
level transitions with similar emitted photons.
\section{Theory}
Electrons in orbit around atomic nuclei exist in specific, quantized, energy levels and may only move between these levels when a ``hole'' \textbf{REWORK}
is available for them to do so. In order to ``jump'' up to a higher energy level an electron must be given energy, usually by a photon, who's energy is given by
$$E=\frac{hc}{\lambda}$$ where $h$ is Planck's constant\cite{codata}, $c$ the speed of light in the medium through which the photon moves, and $\lambda$ the
wavelength of the photon. Electrons may also decay to a lower energy level, emitting the energy lost in that process as a photon of energy given by the difference in energy
between the final and initial energy levels.

The spectrometer employed in this experiment records the number of these emitted photons for a given wavelength by separating\dots
\section{Methods}
\lipsum[3-5]
\section{Discussion}
\subsection{Helium Spectrum}
% \begin{figure}
%   \begin{tikzpicture}[define rgb/.code={\definecolor{mycolor}{RGB}{#1}},
%       rgb color/.style={define rgb={#1},mycolor}]
%     \begin{axis}[
%         ymode=log,
%         width=\columnwidth,
%         axis background/.style={fill=black!60},
%         xmin=350,
%         xmax=750
%       ]
%       \addplot table [col sep=comma,
%           header=false,
%           x index=0,
%           y index=1,
%           mark=none
%         ] {He/He_VIS_100_0.1nm_250um_350-750-PROCESSED.csv};
%       \draw[rgb color={124, 0, 149}] (392.233771,0.01) -- (392.233771,0.7);
%       \draw[rgb color={0, 255, 97}] (504.922999,0.01) -- (504.922999,0.7);
%       \draw[rgb color={129, 0, 165}] (396.573604,0.01) -- (396.573604,0.7);
%       \draw[rgb color={0, 255, 135}] (501.705800,0.01) -- (501.705800,0.7);
%       \draw[rgb color={179, 0, 0}] (728.119994,0.01) -- (728.119994,0.7);
%       \draw[rgb color={17, 0, 255}] (438.788943,0.01) -- (438.788943,0.7);
%       \draw[rgb color={0, 255, 234}] (492.319811,0.01) -- (492.319811,0.7);
%       \draw[rgb color={234, 0, 0}] (668.002672,0.01) -- (668.002672,0.7);
%       \draw[rgb color={123, 0, 226}] (412.201154,0.01) --(412.201154,0.7);
%       \draw[rgb color={0, 174, 255}] (471.453491,0.01) -- (471.453491,0.7);
%       \draw[rgb color={131, 0, 188}] (706.514060,0.01) -- (706.514060,0.7);
%       \draw[rgb color={118, 0, 132}] (388.983974,0.01) -- (388.983974,0.7);
%       \draw[rgb color={131, 0, 188}] (402.608906,0.01) -- (402.608906,0.7);
%       \draw[rgb color={0, 53, 255}] (447.287203,0.01) -- (447.287203,0.7);
%       \draw[rgb color={255, 233, 0}] (587.716721,0.01) -- (587.716721,0.7); % Yuck
%     \end{axis}
%     \label{helium-spectrum-1}
%   \end{tikzpicture}
%   \caption{Helium lamp spectrum. Coloured lines denote predicted values. See appendix for full size and overlay. Generated with TikZ}
% \end{figure}
\begin{table}[ht!]
  \centering%
  \begin{tabular}{p{0.2\columnwidth}p{0.1\columnwidth}p{0.2\columnwidth}p{0.06\columnwidth}p{0.1\columnwidth}}
    \toprule
    Assigned Transition         & Pred. wavelength, $\lambda_p (\unit{\nano\meter})$ & Obs. wavelength, $\lambda_o (\unit{\nano\meter})$ & \% Diff & Comments        \\
    \midrule
    $1s0s{}^1\!S$ $1s3p{}^1\!P$ & $392$                                              & 0                                                 & 0       & `'              \\
    \hline
    $1s0s{}^3\!S$ $1s3p{}^3\!P$ & $412$                                              & 0                                                 & 0       & `'              \\
    \hline
    $1s1s{}^1\!S$ $1s3p{}^1\!P$ & $505$                                              & $505\pm1.7$                                       & 0.337   & Extremely faint \\
    \hline
    $1s1s{}^3\!S$ $1s3p{}^3\!P$ & $471$                                              & $471\pm0.6$                                       & 0.127   & `'              \\
    \hline
    $1s2s{}^1\!S$ $1s1p{}^1\!P$ & $397$                                              & 0                                                 & 0       & `'              \\
    \hline
    $1s2s{}^1\!S$ $1s2p{}^1\!P$ & $502$                                              & $502\pm1.6$                                       & 0.319   & `'              \\
    \hline
    $1s2s{}^3\!S$ $1s3p{}^3\!P$ & $707$                                              & $707\pm1.5$                                       & 0.212   & `'              \\
    \hline
    $1s3s{}^1\!P$ $1s0p{}^1\!D$ & $439$                                              & 0                                                 & 0       & `'              \\
    \hline
    $1s3s{}^1\!P$ $1s1p{}^1\!D$ & $492$                                              & $492\pm1.6$                                       & 0.325   & `'              \\
    \hline
    $1s3s{}^1\!P$ $1s2p{}^1\!D$ & $668$                                              & $668\pm1.8$                                       & 0.269   & `'              \\
    \hline
    $1s3s{}^1\!S$ $1s3p{}^1\!P$ & $728$                                              & $728\pm1.5$                                       & 0.206   & `'              \\
    \hline
    $1s3s{}^3\!P$ $1s0p{}^3\!D$ & $403$                                              & 0                                                 & 0       & `'              \\
    \hline
    $1s3s{}^3\!P$ $1s1p{}^3\!D$ & $447$                                              & $447\pm1.6$                                       & 0.358   & `'              \\
    \hline
    $1s3s{}^3\!P$ $1s2p{}^3\!D$ & $588$                                              & $588\pm1.5$                                       & 0.255   & `'              \\
    \hline
    $1s3s{}^3\!S$ $1s2p{}^3\!P$ & $389$                                              & $389\pm1.4 $                                      & 0.360   & `'              \\
    \bottomrule
  \end{tabular}
\end{table}
% \begin{figure}
%   \begin{tikzpicture}
%     \begin{axis}[
%         ymode=log,
%         width=\columnwidth,
%         xmin=350,
%         xmax=750
%       ]
%       \addplot table [col sep=comma,
%           header=false,
%           x index=0,
%           y index=1,
%           mark=none
%         ] {Phi/Phi_VIS_100_0.1nm_250um_350-750-PROCESSED.csv};
%     \end{axis}
%     \label{phi-spectrum-1}
%   \end{tikzpicture}
%   \caption{$\phi$ lamp spectrum. See appendix for full size and overlay.}
% \end{figure}
% \begin{figure}
%   \begin{tikzpicture}
%     \begin{axis}[
%         ymode=log,
%         width=\columnwidth,
%         xmin=350,
%         xmax=750
%       ]
%       \addplot table [col sep=comma,
%           header=false,
%           x index=0,
%           y index=1,
%           mark=none
%         ] {Psi/Psi_VIS_100_1nm_250um_350-750-PROCESSED.csv};
%     \end{axis}
%     \label{psi-spectrum-1}
%   \end{tikzpicture}
%   \caption{$\psi$ lamp spectrum. See appendix for full size and overlay.}
% \end{figure}
\begin{figure}
  \begin{tikzpicture}
    \begin{axis}[
        ymode=log,
        width=\columnwidth,
        xmin=587,
        xmax=591,
        legend pos=south west
      ]
      \addplot table [col sep=comma,
          header=false,
          x index=0,
          y index=1,
          mark=none
        ] {Psi/Doublets/Psi_VIS_100_0.02nm_0um_587-591-PROCESSED.csv};
      \addplot table [col sep=comma,
          header=false,
          x index=0,
          y index=1,
          mark=none
        ] {Psi/Doublets/Psi_VIS_100_0.02nm_5um_588-590.5-PROCESSED.csv};
      \addplot table [col sep=comma,
          header=false,
          x index=0,
          y index=1,
          mark=none
        ] {Psi/Doublets/Psi_VIS_100_0.02nm_10um_588-590.5-PROCESSED.csv};
      \addplot table [col sep=comma,
          header=false,
          x index=0,
          y index=1,
          mark=none
        ] {Psi/Doublets/Psi_VIS_100_0.02nm_50um_588-590.5-PROCESSED.csv};
      \addplot[green] table [col sep=comma,
          header=false,
          x index=0,
          y index=1,
          mark=none
        ] {Psi/Doublets/Psi_VIS_100_0.02nm_250um_588-590.5-PROCESSED.csv};
      \addlegendentry{$0\unit{\micro\meter}$}
      \addlegendentry{$5\unit{\micro\meter}$}
      \addlegendentry{$10\unit{\micro\meter}$}
      \addlegendentry{$50\unit{\micro\meter}$}
      \addlegendentry{$250\unit{\micro\meter}$}
    \end{axis}
    \label{psi-spectrum-1}
  \end{tikzpicture}
  \caption{$\psi$ lamp spectrum focused on yellow doublet. Note significantly decreased intensity for a $0\unit{\micro\meter}$ slit width requiring semilog scale. See appendix for full size.}
\end{figure}
\subsection{Sources of error}
\lipsum[10]
\section{Conclusion}
\lipsum[11]
\section{Bibliography}
\bibliography{bib}{}
\bibliographystyle{plain}

\section{Appendix}
\lipsum[12]
\subsection{Figures}
\begin{landscape}
  \begin{figure}
    \begin{tikzpicture}[define rgb/.code={\definecolor{mycolor}{RGB}{#1}},
        rgb color/.style={define rgb={#1},mycolor}]
      \begin{axis}[
          ymode=log,
          width=1.33*\textwidth,
          height=\textheight,
          axis background/.style={fill=black!60},
          xmin=350,
          xmax=750
        ]
        \addplot table [col sep=comma,
            header=false,
            x index=0,
            y index=1,
            mark=none,
            blue] {He/He_VIS_100_0.1nm_250um_350-750-PROCESSED.csv};
        \addplot table [col sep=comma,
            header=false,
            x index=0,
            y index=1,
            mark=none
          ] {Phi/Phi_VIS_100_0.1nm_250um_350-750-PROCESSED.csv};
        \addplot table [col sep=comma,
            header=false,
            x index=0,
            y index=1,
            mark=none
          ] {Psi/Psi_VIS_100_1nm_250um_350-750-PROCESSED.csv};
        \draw[rgb color={124, 0, 149}] (392.233771,0.01) -- (392.233771,0.7);
        \draw[rgb color={0, 255, 97}] (504.922999,0.01) -- (504.922999,0.7);
        \draw[rgb color={129, 0, 165}] (396.573604,0.01) -- (396.573604,0.7);
        \draw[rgb color={0, 255, 135}] (501.705800,0.01) -- (501.705800,0.7);
        \draw[rgb color={179, 0, 0}] (728.119994,0.01) -- (728.119994,0.7);
        \draw[rgb color={17, 0, 255}] (438.788943,0.01) -- (438.788943,0.7);
        \draw[rgb color={0, 255, 234}] (492.319811,0.01) -- (492.319811,0.7);
        \draw[rgb color={234, 0, 0}] (668.002672,0.01) -- (668.002672,0.7);
        \draw[rgb color={123, 0, 226}] (412.201154,0.01) --(412.201154,0.7);
        \draw[rgb color={0, 174, 255}] (471.453491,0.01) -- (471.453491,0.7);
        \draw[rgb color={131, 0, 188}] (706.514060,0.01) -- (706.514060,0.7);
        \draw[rgb color={118, 0, 132}] (388.983974,0.01) -- (388.983974,0.7);
        \draw[rgb color={131, 0, 188}] (402.608906,0.01) -- (402.608906,0.7);
        \draw[rgb color={0, 53, 255}] (447.287203,0.01) -- (447.287203,0.7);
        \draw[rgb color={255, 233, 0}] (587.716721,0.01) -- (587.716721,0.7); % Yuck
      \end{axis}
      \label{figure-1-large}
    \end{tikzpicture}
    \caption{Helium spectrum. Coloured lines denote predicted values. See appendix for full size. Generated with TikZ}
  \end{figure}
\end{landscape}
\newpage
\begin{landscape}
  \begin{figure}
    \begin{tikzpicture}
      \begin{axis}[
          ymode=log,
          width=\columnwidth,
          height=\textheight,
          xmin=588,
          xmax=590.5,
          legend pos=south west
        ]
        \addplot table [col sep=comma,
            header=false,
            x index=0,
            y index=1,
            mark=none
          ] {Psi/Doublets/Psi_VIS_100_0.02nm_0um_587-591-PROCESSED.csv};
        \addplot table [col sep=comma,
            header=false,
            x index=0,
            y index=1,
            mark=none
          ] {Psi/Doublets/Psi_VIS_100_0.02nm_5um_588-590.5-PROCESSED.csv};
        \addplot table [col sep=comma,
            header=false,
            x index=0,
            y index=1,
            mark=none
          ] {Psi/Doublets/Psi_VIS_100_0.02nm_10um_588-590.5-PROCESSED.csv};
        \addplot table [col sep=comma,
            header=false,
            x index=0,
            y index=1,
            mark=none
          ] {Psi/Doublets/Psi_VIS_100_0.02nm_50um_588-590.5-PROCESSED.csv};
        \addplot[green] table [col sep=comma,
            header=false,
            x index=0,
            y index=1,
            mark=none
          ] {Psi/Doublets/Psi_VIS_100_0.02nm_250um_588-590.5-PROCESSED.csv};
        \addlegendentry{$0\unit{\micro\meter}$}
        \addlegendentry{$5\unit{\micro\meter}$}
        \addlegendentry{$10\unit{\micro\meter}$}
        \addlegendentry{$50\unit{\micro\meter}$}
        \addlegendentry{$250\unit{\micro\meter}$}
      \end{axis}
      \label{psi-spectrum-1}
    \end{tikzpicture}
    \caption{$\psi$ lamp spectrum focused on yellow doublet. Note significantly decreased intensity for a $0\unit{\micro\meter}$ slit width requiring semilog scale. See appendix for full size.}
  \end{figure}
\end{landscape}
\end{document}