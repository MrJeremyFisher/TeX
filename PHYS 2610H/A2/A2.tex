\documentclass[10pt]{article}

\usepackage[margin=0.75in]{geometry}
\usepackage{amsmath,amsthm,amssymb}
\usepackage{xcolor}
\usepackage{cancel}
\usepackage{graphicx}
\usepackage{changepage}
\usepackage{circuitikz}
\usepackage{pgfplots}
\usepackage{physics}
\usepackage{hyperref}
\usepackage{siunitx}
\usepackage[breakable]{tcolorbox}
\usepackage[inline]{enumitem}

\theoremstyle{definition}
\newtheorem{problem}{Problem}
\newtheorem{soln}{Solution}

\pgfplotsset{compat=newest}
\usetikzlibrary{lindenmayersystems}
\usetikzlibrary{arrows}
\usetikzlibrary{calc}

\definecolor{incolor}{HTML}{303F9F}
\definecolor{outcolor}{HTML}{D84315}
\definecolor{cellborder}{HTML}{CFCFCF}
\definecolor{cellbackground}{HTML}{F7F7F7}
\newcommand{\eq}{=}
\usetikzlibrary{positioning, fit, calc}
\pgfdeclarelayer{background}  
\pgfsetlayers{background,main}
\AtBeginDocument{\RenewCommandCopy\qty\SI}

\makeatletter
\newcommand{\boxspacing}{\kern\kvtcb@left@rule\kern\kvtcb@boxsep}
\makeatother
\newcommand{\prompt}[4]{
    \ttfamily\llap{{\color{#2}[#3]:\hspace{3pt}#4}}\vspace{-\baselineskip}
}

\newcommand{\thevenin}[2]{
  \begin{center}
    \begin{circuitikz} \draw
      (0,0) -- (2,0) to[battery1, l_=$V_{Th}\eq#1$] (2,2) 
      to[resistor, l_=$R_{Th}\eq#2$] (0,2)
      ;
      \draw [o-] (-.07,2.079);
      \draw [o-] (-.07,0.079);
    \end{circuitikz}
  \end{center}
}

\newcommand{\norton}[2]{
  \begin{center}
    \begin{circuitikz} \draw
      (0,0) -- (3,0) to[american current source, l_=$I_{N}\eq#1$] (3,2) -- (0,2) (2,0)
      to[resistor, l=$R_{N}\eq#2$] (2,2)
      ;
      \draw [o-] (-.07,2.079);
      \draw [o-] (-.07,0.079);
    \end{circuitikz}
  \end{center}
}

\newcommand{\highlight}[1]{\colorbox{yellow}{$\displaystyle #1$}}

\newcommand{\ti}[1]{\widetilde{#1}}

\title{Physics 2610H: Assignment II}
\author{Jeremy Favro}
\date{\today}

\begin{document}
\maketitle

% PROBLEM 1
\begin{problem}
Write out the total wave function $\Psi(x, t)$ for an electron in the $n=3$ state of a $\qty{10}{\nano\meter}$ wide infinite well.
Other than the symbols $x$ and $t$, the function should include only numerical values.
\end{problem}
\begin{soln}
  We can split $\Psi(x, t)=\psi(x)\phi(t)=\psi(x)\exp\left(\frac{-iEt}{\hbar}\right)$ then for a particle in an infinite potential well
  $\psi(x)=\sqrt{\frac{2}{L}}\sin\left(\sqrt{\frac{2mE}{\hbar^2}}x\right)$ and $E=\frac{n^2\pi^2\hbar^2}{2mL^2}$ So,
  \begin{align*}
    \Psi(x, t) = &\,\sqrt{\frac{2}{L}}\sin\left(\sqrt{\frac{2m\frac{n^2\pi^2\hbar^2}{2mL^2}}{\hbar^2}}x\right)\exp\left(\frac{-i\frac{n^2\pi^2\hbar^2}{2mL^2}t}{\hbar}\right)\\
     = &\,\sqrt{\frac{2}{L}}\sin\left(\frac{n\pi x}{L}\right)\exp\left(-i\frac{n^2\pi^2\hbar t}{2mL^2}\right)\\
     = &\,\sqrt{\frac{2}{\qty{10d-9}{}}}\sin\left(\frac{3\pi x}{\qty{10d-9}{}}\right)\exp\left(-i\frac{3^2\pi^2\hbar t}{2\cdot\qty{9.109d-31}{}\cdot(\qty{10d-9})^2}\right)\\
     \approx &\,\qty{1.414d3}{\per\meter\tothe{1/2}}\sin\left(\qty{9.425d17}{\per\meter}x\right)\exp\left(-i\cdot \qty{5.142d13}{\per\second}\right)\\
  \end{align*}
\end{soln}

% PROBLEM 2
\begin{problem}
An electron in the $n=4$ state of a $\qty{5}{\nano\meter}$ wide infinite well makes a transition to the ground state,
giving off energy in the form of a photon. What is the photon's wavelength?
\end{problem}
\begin{soln} By equation 1.11 from the formula sheet the energy of a particle of mass $m$ in state $n$ in an infinite well of width $L$ is
  $$E=\frac{n^2h^2}{8mL^2}$$
  Here we're looking at $\Delta E$ of an electron between $n=4$ and $n=1$ so,
  $$\Delta E = \eval{\frac{n^2h^2}{8mL^2}}_{4}^{1}=\frac{15h^2}{8m_eL^2}$$
  Then, $\displaystyle E=\frac{hc}{\lambda}\implies \lambda = hc\frac{8m_e(\qty{5}{\nano\meter})^2}{15h^2}=\frac{8cm_e(\qty{5}{\nano\meter})^2}{15h}\approx \qty{5500}{\nano\meter}$
\end{soln}
\newpage

% PROBLEM 3
\begin{problem}
What is the probability that a particle in the first excited $(n=2)$ state of an infinite well would be
found in the middle third of the well? How does this compare with the classical expectation? Why?
\end{problem}
\begin{soln}
  \begin{align*}
     & = \int_{\frac{L}{3}}^{\frac{2L}{3}}\left[\sqrt{\frac{2}{L}}\sin\left(\frac{n\pi x}{L}\right)\right]^2\,dx= \frac{2}{L}\int_{\frac{L}{3}}^{\frac{2L}{3}}\sin^2\left(\frac{n\pi x}{L}\right)\,dx  \\
     & = \frac{2}{L}\left[\frac{x}{2}-\frac{\sin\left(\frac{2n\pi x}{L}\right)}{\frac{4n\pi}{L}}\right]_{\frac{L}{3}}^{\frac{2L}{3}}\,(\text{From formula sheet})                                      \\
     & = \frac{2}{L}\left[\frac{x}{2}-\frac{L}{4n\pi}\sin\left(\frac{2n\pi x}{L}\right)\right]_{\frac{L}{3}}^{\frac{2L}{3}}                                                                            \\
     & = \frac{2}{L}\left[\frac{\frac{2L}{3}}{2}-\frac{L}{4n\pi}\sin\left(\frac{2n\pi \frac{2L}{3}}{L}\right)-\frac{\frac{L}{3}}{2}+\frac{L}{4n\pi}\sin\left(\frac{2n\pi \frac{L}{3}}{L}\right)\right] \\
     & = \frac{2}{3}-\frac{2}{6}+\frac{2}{4n\pi}\sin\left(\frac{2n\pi}{3}\right)-\frac{2}{4n\pi}\sin\left(\frac{4n\pi}{3}\right)                                                                       \\
     & = \frac{1}{3}+\frac{1}{2n\pi}\left[\sin\left(\frac{2n\pi}{3}\right)-\sin\left(\frac{4n\pi}{3}\right)\right]\approx0.196                                                                         \\
  \end{align*}
  Classically I think we'd expect that the particle has an even probability density over the entire well and therefore the probability that it is inside one third of the well is $\frac{1}{3}$.
  The difference between these two arises from the fact that in the $n=2$ state we have an antinode in the middle third which means that the probability is lesser than the outer two thirds.
\end{soln}
\end{document}