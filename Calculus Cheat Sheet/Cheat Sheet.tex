\documentclass[10pt,landscape,letterpaper]{cheatsheet}

\usepackage{amsmath,amsthm,amssymb}
\usepackage{graphicx,lipsum}
\usepackage{enumitem}
\newcommand{\PRLsep}{\noindent\makebox[\linewidth]{\resizebox{0.3333\linewidth}{1pt}{$\bullet$}}\bigskip}
\newcommand\longdiv[2]{%
$\strut#1$\kern.25em\smash{\raise.3ex\hbox{$\big)$}}$\mkern-8mu
        \overline{\enspace\strut#2}$}
\usepackage{mdframed}
\usepackage{cancel}
\title{Calculus Cheat Sheet}
\author{Jeremy Favro}
\date{\today\\Revision 3}

\begin{document}

\maketitle

\section*{Limits}
\subsection*{Existence}
$\lim\limits_{x\to a}f(x)=L $ exists if $\forall \epsilon > 0 \exists \delta > 0 \text{ s.t. } \left\lvert x-a \right\rvert < \delta \Rightarrow \left\lvert f(x) - L \right\rvert < \epsilon $.
\subsection*{Properties}
$ \lim\limits_{x\to a}[cf(x)]=c\lim\limits_{x\to a}[f(x)] $
$ \lim\limits_{x\to a}[f(x) \pm g(x)]=\lim\limits_{x\to a}[f(x)] \pm \lim\limits_{x\to a}[g(x)] $
$ \lim\limits_{x\to a}[f(x)g(x)]=\lim\limits_{x\to a}[f(x)]\lim\limits_{x\to a}[g(x)] $
$ \lim\limits_{x\to a}\left[\frac{f(x)}{g(x)}\right]=\frac{\lim\limits_{x\to a}[f(x)]}{\lim\limits_{x\to a}[g(x)]}, \lim\limits_{x\to a}[g(x)] \neq 0 $
\subsection*{Evaluation Techniques}

\subsubsection*{Basics}
$ \lim\limits_{x\to a}\left[f(x)\right]=f(a) \text{ if f exists at a } $

\subsubsection*{L'H\^opital's Rule}
$ \text{If } \lim\limits_{x\to a}\left[\frac{f(x)}{g(x)}\right]=\frac{0}{0} \text { \textbf{or} } \frac{\pm \infty}{\pm \infty}$
$ \text{ then } $
$ \lim\limits_{x\to a}\left[\frac{f(x)}{g(x)}\right]=\lim\limits_{x\to a}\left[\frac{f^{\prime}(x)}{g^{\prime}(x)}\right] $

\subsubsection*{Factoring at Infinity}
If $p(x)$ and $q(x)$ are polynomials, to evaluate $\lim\limits_{x\to \pm \infty}\left[\frac{p(x)}{q(x)}\right]$ factor the greatest power of $x$ in $q(x)$ (the denominator) out of poth $p(x)$ and $q(x)$ then compute the limit, e.g.
$ \lim\limits_{x\to -\infty}\left[\frac{3x^2-4}{5x-2x^2}\right]=\lim\limits_{x\to -\infty}\left[\frac{\cancel{x^2}(3-\frac{4}{x^2})}{\cancel{x^2}(\frac{5}{x}-2)}\right]$
$ \frac{3-0}{0-2} = -\frac{3}{2} $
\section*{Derivatives}
\subsection*{Definition}
$ \frac{d}{dx}f(x) = f^{\prime}(x) = \lim\limits_{h\to 0}\frac{f(x+h)-f(x)}{h}$.
\subsection*{Techniques}

\subsubsection*{Sum Rule}
$ \frac{d}{dx}(f(x) \pm g(x))=f^{\prime}(x) \pm g^{\prime}(x) $

\subsubsection*{Product Rule}
$ \frac{d}{dx}(f(x)g(x))=f^{\prime}(x)g(x)+f(x)g^{\prime}(x) $

\subsubsection*{Quotient Rule}
$ \frac{d}{dx}\left(\frac{f(x)}{g(x)}\right)=\frac{f^{\prime}(x)g(x)-f(x)g^{\prime}(x)}{[g(x)]^2} $

\subsubsection*{Power Rule}
$ \frac{d}{dx}(x^n)=nx^{n-1} $

\subsubsection*{Chain Rule}
$ \frac{d}{dx}(f(g(x)))=f^{\prime}(g(x))g^{\prime}(x) $

\subsubsection*{Parametric}
$\frac{dy}{dx}=\frac{dy/dt}{dx/dt}$ recurse $y$ for $n$th derivative\\
$(x,y)+(x^\prime,y^\prime)t$ for parametric tangent

\subsection*{Common Derivatives}
$ \frac{d}{dx}(x)=1                                                                                      $
% Normal trig
$ \frac{d}{dx}(\sin {x})=\cos {x}                                                                        $
$ \frac{d}{dx}(\cos {x})=-\sin {x}                                                                       $
$ \frac{d}{dx}(\tan {x})=\sec ^{2} {x}                                                                   $
% Reciprocal
$ \frac{d}{dx}(\sec {x})=\sec {x} \tan {x}                                                               $
$ \frac{d}{dx}(\csc {x})=-\csc {x} \cot {x}                                                              $
$ \frac{d}{dx}(\cot {x})=-\csc ^2 {x}                                                                    $
% Inverse
$ \frac{d}{dx}(\sin ^{-1} {x})=\frac{1}{\sqrt{1-x^2}} \sin^{-1} {x} \neq \frac{1}{\sin {x}} $
$ \frac{d}{dx}(\cos ^{-1} {x})=-\frac{1}{\sqrt{1-x^2}}                                                   $
$ \frac{d}{dx}(\tan ^{-1} {x})=\frac{1}{1+x^2}                                                           $
% Exponentials and Logarithms
$ \frac{d}{dx}(a ^ {x})=a^{x} \ln {a}                                                                    $
$ \frac{d}{dx}(e ^ {x})=e^{x}                                                                            $
$ \frac{d}{dx}(\ln {x})=\frac{1}{x}, x > 0                                                               $
$ \frac{d}{dx}(\ln {|x|})=\frac{1}{x}, x \neq 0                                                          $
$ \frac{d}{dx}(\log {_a}{x})=\frac{1}{x \ln {a}}                                                         $

\subsection*{Common Chain Rule Derivatives}
$ \frac{d}{dx}([f(x)]^n)=n[f(x)]^{n-1}f^{\prime}(x) $
$ \frac{d}{dx}(e^{f(x)})=e^{f(x)}f^{\prime}(x)      $
$ \frac{d}{dx}(\ln [{f(x)}])=\frac{f^{\prime}(x)}{f(x)}      $
$ \frac{d}{dx}(\sin [{f(x)}])=f^{\prime}(x)\cos[{f(x)}]    $
$ \frac{d}{dx}(\cos [{f(x)}])=-f^{\prime}(x)\sin[{f(x)}]    $
$ \frac{d}{dx}(\tan [{f(x)}])=f^{\prime}(x)\sec ^{2}[{f(x)}]    $
$ \text{Trig derivatives, same old same old}$
$ \frac{d}{dx}(f(x)^{g(x)})=\frac{g(x)f^{\prime}(x)}{f(x)}+\ln{[f(x)]}g^{\prime}(x)   $
\section*{Integrals}

\subsection*{Definition}
The integral of some function $f(x)$ is a function $f^{\star}(x)$ s.t. $f^{\star\prime}(x)=f(x)$. \textbf{Don't forget your constant!}
The integral can be done using the Right Riemann Sum:
$\int_{a}^{b} f(x) \,dx = \lim\limits_{n\to \infty}\sum_{i=1}^{n}\left[\frac{b-a}{n}\cdot f(a+i\cdot\frac{b-a}{n})\right]$

\subsection*{U-Substitution}
$ \int_{a}^{b} f(g(x)) \cdot g^{\prime} \,dx = \int_{g(a)}^{g(b)} f(u) \,dx $. Using U-Substitution, $u=g(x)$ and $du=g^{\prime}(x)dx$ ($dx=\frac{du}{g^{\prime}}$).

\subsection*{Integration By Parts}
$\int u(x)v^{\prime}(x) \,dx = u(x)v(x) - \int u^{\prime}(x)v(x) \,dx $ and $ \int_{a}^{b} u(x)v^{\prime}(x) \,dx = uv\vert_{a}^{b} - \int_{a}^{b} u^{\prime}(x)v(x) \,dx$.

\subsection*{Common Integrals}
$ \int x^n \,dx = \frac{1}{n+1}x^{n+1} + C $ \\
$ \int \frac{1}{x} \,dx = \ln{\left\lvert x \right\rvert} + C $ \\
$ \int \frac{1}{ax+b} \,dx = \frac{1}{a} \ln{\left\lvert ax+b \right\rvert } + C $ \\
$ \int \ln{x} \,dx = x\ln{x} - x + C $ \\
$ \int e^x \,dx = e^x + C $ \\
$ \int a^x \,dx = \frac{a^x}{\ln{a}} + C $ \\
$ \int \sin{x} \,dx = -\cos{x} + C $ \\
$ \int \cos{x} \,dx = \sin{x} + C $ \\
$ \int \tan{x} \,dx = \ln{\left\lvert \sec{x} \right\rvert} + C $ \\
$ \int \sec^2{x} \,dx = \tan{x} + C $ \\
$ \int \sec{x}\tan{x} \,dx = \sec{x} + C $ \\
$ \int \csc{x}\cot{x} \,dx = -\csc{x} + C $ \\
$ \int \csc^2{x} \,dx = -\cot{x} + C $ \\
$ \int \sec{x} \,dx = \ln{\left\lvert \sec{x} + \tan{x} \right\rvert} + C $ \\
$ \int \frac{1}{a^2+u^2} \,dx = \frac{1}{a}\tan^{-1}\left({\frac{x}{a}}\right) + C $ \\
$ \int \frac{1}{\sqrt{a^2+u^2}} \,dx = \sin^{-1}\left({\frac{x}{a}}\right) + C $ \\

\subsection*{Trig Reduction Formulae} (For $n \geq 2$) \\
$\int \sin^n(x) \,dx = -\frac{1}{n}\sin^{n-1}(x)\cos(x)+\frac{n-1}{n}\int \sin^{n-2}(x) \,dx$\\
$\int \cos^n(x) \,dx = \frac{1}{n}\cos^{n-1}(x)\sin(x)+\frac{n-1}{n}\int \cos^{n-2}(x) \,dx$\\
$\int \tan^n(x) \,dx = \frac{1}{n-1}\tan^{n-1}(x)+\frac{n-1}{n}\int \tan^{n-2}(x) \,dx$\\
$\int \sec^n(x) \,dx = \frac{1}{n-1}\sec^{n-2}(x)\tan(x)+\frac{n-2}{n-1}\int \sec^{n-2}(x) \,dx$\\

\subsection*{Trig Identities}
$\sin^2(x)+\cos^2(x)=1$
$1+\tan^2(x)=\sec^2(x)$
$\sin(2x)=2\sin(x)\cos(x)$
$\cos(2x)=\cos^2(x)-\sin^2(x)$\\
~~~~~~~~~~~~~~$=2\cos^2(x)-1$\\
~~~~~~~~~~~~~~$=1-2\sin^2(x)$

\subsection*{Trig Substitutions}
$\sqrt{a^2-x^2} \rightsquigarrow x=a\sin(\theta)$
$\sqrt{a^2+x^2} \rightsquigarrow x=a\tan(\theta)$
$\sqrt{x^2-a^2} \rightsquigarrow x=a\sec(\theta)$

\subsection*{Partial Fractions \& Polynomial Division}
If the degree of the numerator is greater than that of a the denominator, use polynomial division to
divide the denominator into the numerator (\longdiv{denom}{numer}), then integrate the result. If the degree of the denominator is greater than
that of the numerator, use partial fractions to decompose the integral as follows:

If the denominator contains different linear terms, break it down to $\frac{A}{ax+b}$ \\
If it contains a repeated linear term ($(ax+b)^2$), break it down to $\frac{A}{ax+b}+\frac{B}{(ax+b)^2}$ \\
If it contains an irreducible quadratic ($x^2+bx+c$), break it down to $\frac{A}{dx+o}+\frac{Bx+C}{x^2+bx+c}$ \\

Then, set up equations for the coefficients of the powers of $x$ in the numerator and solve them to determine $A,\,B,\,C$ and so on.
\subsection*{Applications}
\subsubsection*{Centroids}
$C = (\hat{x}, \hat{y})$,
$\hat{x} = \frac{\int xf(x) \,dx}{\int f(x) \,dx}$ \&
$\hat{y} = \frac{\int yf(y) \,dy}{\int f(y) \,dy}$
\subsubsection*{Arc Length}
$L = \int_{a}^{b} \sqrt{1+[f^{\prime}(x)]^2} \,dx$
$L = \int_{a}^{b} \sqrt{[f^{\prime}(t)]^2+[g^{\prime}(t)]^2+[h^{\prime}(t)]^2} \,dt$ for parametric. Area is just this times $2\pi x$ or $y$ for opposite of axis.
\subsubsection*{Volume} Use the variable parallel to the axis of revolution.
$V = \int_{a}^{b} A(x) \,dx$ where $A(x)$ is a function which gives the area
of one ``slice'' of the solid. Slices are generally circular and may have holes in them, area
of a circle is $\pi r^2$, and the area of a washer is $\pi (r_{outer}^2-r_{inner}^2)$

\section*{Sequences and Series}
\subsection*{Series With Nice Formulae}
The geometric series $\sum_{n=0}^{\infty}ar^n=\frac{a}{1-r}$ provided that $|r|<1$
The telescoping series where the ``inbetween'' terms cancel. Take the limit as $k\to\infty$
\subsection{Basic Divergence Test}
If for some series $\sum_{n=0}^{\infty}a_n$ the limit $\lim_{n \to \infty} a_n \neq 0$ the series diverges.
However, if the limit is 0, no information can be determined from this test.
\subsection*{Integral Test}
If some function $f(n)=a_n$ is decreasing on $[c,\infty)$ then the
series $\sum_{n=c}^{\infty}a_n$ converges exactly as $\int_{c}^{\infty}f(x)dx$ does / does not. This works because
$\int_{c}^{\infty}f(x)dx\leq\sum_{n=c}^{\infty}a_n\leq f(c)+\int_{c}^{\infty}f(x)dx$
\subsection*{P-Test}
Generally, $\sum_{n=c}^{\infty}\frac{an^k+\dots+a_0}{bn^l+\dots+b_0}$ converges if $p=l-k<1$ and diverges if $p\geq 1$
\subsection*{Basic Comparsion Test}
If two sequences $\{a_n\}$ and $\{b_n\}$ exist, are comprised of positive terms, and satisfy $0<a_n\leq b_n$
past some point then
\begin{enumerate}[label=(\alph*)]
        \item if $\sum_{n=0}^{\infty}b_n$ converges, so does $\sum_{n=0}^{\infty}a_n$
        \item if $\sum_{n=0}^{\infty}a_n$ diverges, so does $\sum_{n=0}^{\infty}b_n$
\end{enumerate}
\subsection*{Limit Comparison Test}
If two sequences $\{a_n\}$ and $\{b_n\}$ exist and are comprised of positive terms past some point then
$\lim_{n \to \infty} \frac{a_n}{b_n} =c$
\begin{enumerate}[label=(\alph*)]
        \item if $c>0$ both series either converge or diverge (one converging means the other does and vice versa)
        \item if $c=0$ then $a_n$ diverges so does $b_n$ and if $b_n$ converges so does $a_n$
        \item if $c=\infty$ then $a_n$ converges so does $b_n$ and if $b_n$ diverges so does $a_n$
\end{enumerate}
\subsection*{Alternating Series Test}
If each $a_n$ in the series $\sum_{n=0}^{\infty}a_n$ and
\begin{enumerate}[label=(\arabic*)]
        \item $|a_{n+1}|<|a_n|$ (The series is decreasing)
        \item $a_{n+1}<0$ \& $a_n>0$ (The series is alternating)
        \item $\lim_{n \to \infty} |a_n|=0$
\end{enumerate}
then the series converges
\subsection*{Ratio Test}
If past some point $a_n\neq 0$ then for the series $\sum_{n=0}^{\infty}a_n$
if the limit $\lim_{n \to \infty} \left|\frac{a_{n+1}}{a_n}\right|=L$ then
\begin{enumerate}[label=(\alph*)]
        \item if $L<1$ the series converges absolutely
        \item if $L=1$ no information can be obtained through this test
        \item if $L>1$ the series diverges
\end{enumerate}
\subsection*{Root Test}
For the series $\sum_{n=0}^{\infty}a_n$ if $\lim_{n \to \infty}\sqrt[n]{|a_n|}=L$ then
\begin{enumerate}[label=(\alph*)]
        \item if $L<1$ the series converges absolutely
        \item if $L=1$ no information can be obtained through this test
        \item if $L>1$ the series diverges
\end{enumerate}
\subsection*{Taylor Series}
Taylor's formula states that if some function $f(x)$ can be expanded as a power series around $x=a$
its representation is given by $$\sum_{n=0}^{\infty}\frac{\frac{d^n}{dx^n}f(a)}{n!}(x-a)^n$$
\subsubsection*{Remainder Terms}
If there exists some function $f(x)$ whose Taylor Series is defined as $T_n(x)$ then the remainder term is
defined as $R_n(x)=f(x)-T_n(x)$, which is how ``far off'' the Taylor Series $T_n(x)$ is from the function $f(x)$
for some $n$.
\section*{Tidbits}
$\vec{v}$ here is the velocity vector (duh)\\
Unit tangent vector $\vec{T}=\frac{\vec{v}}{\left|\vec{v}\right|}$\\
Curvature $\kappa=\left|\frac{d\vec{T}}{d\vec{s}}\right|=\frac{1}{\left|\vec{v}\right|}\left|\frac{d\vec{T}}{dt}\right|$\\
Normal vector $\vec{N}=\frac{1}{\kappa}\frac{d\vec{T}}{ds}=\frac{d\vec{T}/dt}{\left|d\vec{T}/dt\right|}$ 
\end{document}