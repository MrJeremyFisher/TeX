\documentclass[10pt]{article}

\usepackage[margin=0.75in]{geometry}
\usepackage{amsmath,amsthm,amssymb}
\usepackage{xcolor}
\usepackage{cancel}
\usepackage{graphicx}
\usepackage{changepage}
\usepackage{circuitikz}
\usepackage{pgfplots}
%\usepackage{physics}
\usepackage{hyperref}
\usepackage{siunitx}
\usepackage[breakable]{tcolorbox}
\usepackage[inline]{enumitem}

\theoremstyle{definition}
\newtheorem{problem}{Problem}
\newtheorem{soln}{Solution}

\pgfplotsset{compat=newest}
\usetikzlibrary{lindenmayersystems}
\usetikzlibrary{arrows}
\usetikzlibrary{calc}

\definecolor{incolor}{HTML}{303F9F}
\definecolor{outcolor}{HTML}{D84315}
\definecolor{cellborder}{HTML}{CFCFCF}
\definecolor{cellbackground}{HTML}{F7F7F7}
\newcommand{\eq}{=}
\usetikzlibrary{positioning, fit, calc}
\pgfdeclarelayer{background}  
\pgfsetlayers{background,main}

\makeatletter
\newcommand{\boxspacing}{\kern\kvtcb@left@rule\kern\kvtcb@boxsep}
\makeatother
\newcommand{\prompt}[4]{
    \ttfamily\llap{{\color{#2}[#3]:\hspace{3pt}#4}}\vspace{-\baselineskip}
}

\newcommand{\thevenin}[2]{
  \begin{center}
    \begin{circuitikz} \draw
      (0,0) -- (2,0) to[battery1, l_=$V_{Th}\eq#1$] (2,2) 
      to[resistor, l_=$R_{Th}\eq#2$] (0,2)
      ;
      \draw [o-] (-.07,2.079);
      \draw [o-] (-.07,0.079);
    \end{circuitikz}
  \end{center}
}

\newcommand{\norton}[2]{
  \begin{center}
    \begin{circuitikz} \draw
      (0,0) -- (3,0) to[american current source, l_=$I_{N}\eq#1$] (3,2) -- (0,2) (2,0)
      to[resistor, l=$R_{N}\eq#2$] (2,2)
      ;
      \draw [o-] (-.07,2.079);
      \draw [o-] (-.07,0.079);
    \end{circuitikz}
  \end{center}
}

\newcommand{\highlight}[1]{\colorbox{yellow}{$\displaystyle #1$}}

\newcommand{\ti}[1]{\widetilde{#1}}

\title{Physics 2610H: Lab 1, electron charge-to-mass ratio}
\author{Jeremy Favro}
\date{\today}

\begin{document}
\maketitle

\setcounter{problem}{4}
\setcounter{soln}{4}
% PROBLEM 5
\begin{problem}
Create a graph of $r_{circ}$ versus $I_{coil}^{-1}$. Ensure the axes are labelled properly and compute a line of best fit.
Provide a caption explaining what the graph is showing. Comment whether the data appears linear or not,
and whether there are outliers that make sense to omit (might make sense to put this in the caption).
\end{problem}
\begin{soln}~\\
  \begin{center}
    \begin{tikzpicture}
      \begin{axis}[
          title=Helmholtz coil current vs. electron beam radius in an $\frac{e}{m}$ ratio experiment,
          ylabel=$r_{circ}(\unit{\centi\meter})$,
          xlabel=$I_{coil}(\unit{\ampere})$,
          legend pos=south east
        ]
        \addplot table [x=Iinv, y=rad, col sep=comma, only marks] {data.csv};
        \addplot [domain=0.36:1.1,samples=2] ({\x}, {7.08816896702972*\x-0.34285578093019});
        \addlegendentry{Data}
        \addlegendentry{$\frac{7.09}{I_{coil}}-0.34$, $R^2\approx0.98$}
      \end{axis}
    \end{tikzpicture}
  \end{center}
  As demonstrated by the calculated (via LINEST in LibreOffice Calc) coefficient of determination, $R^2$, the data is highly linear. The point at $(I_{coil},r_{circ})=(0.39, 2.95)$ may be worth removing as an
  outlier as its removal increases $R^2$ to $\approx 0.994$ and decreases slope uncertainty to $\approx 0.21$.
\end{soln}

% PROBLEM 6
\begin{problem}
Calculate the slope of the line of best fit and its associated uncertainty. This can be done with numpy.polyfit in python or LINEST in a spreadsheet.
Present these values rounded appropriately and with the correct units.
\end{problem}
\begin{soln}
  Using LINEST in LibreOffice Calc I determined that the slope is $\approx(7.09\pm0.39)\unit{\centi\meter\ampere}$ with a y-intercept of $\approx(-0.34\pm0.24)\unit{\centi\meter}$.
\end{soln}

% PROBLEM 7
\begin{problem}
Derive your experimental charge to mass ratio using the slope from the previous question. Show your values and calculation.
\end{problem}
\begin{soln}
  $$\frac{e}{m}=\frac{125V_{acc}}{32}\left(\frac{R_{coil}}{\mu_0N}\cdot\frac{1}{slope}\right)^2=\frac{125\cdot194.1\unit{\volt}}{32}\left(\frac{0.158\unit{\meter}}{130\mu_0}\cdot\frac{1}{7.09\unit{\centi\meter\ampere}}\right)^2\approx\qty{1.41d11}{\coulomb\per\kilo\gram}$$
\end{soln}
\newpage

% PROBLEM 8
\begin{problem}
Derive the uncertainty in your charge-to-mass ratio using the uncertainty in the slope. This is a brief error propagation calculation. Show your work.
\end{problem}
\begin{soln}
  $$\frac{\partial \frac{e}{m}}{\partial V}=\frac{125}{32}\left(\frac{R}{\mu_0Ns}\right)^2,\qquad \frac{\partial \frac{e}{m}}{\partial s}=-\frac{125}{16}\left(\frac{R}{\mu_0N}\right)^2s^{-3}$$
  So,
  \begin{align*}
     & =\sqrt{\left(\frac{125}{32}\left(\frac{R}{\mu_0Ns}\right)^2\cdot 0.1\unit{\volt}\right)^2+\left(-\frac{125}{16}\left(\frac{R}{\mu_0N}\right)^2s^{-3}\cdot\qty{3.9d-3}{\meter}\right)^2}                                                                                               \\
     & =\sqrt{\left(\frac{125}{32}\left(\frac{0.158\unit{\meter}}{\mu_0\cdot130\cdot0.0709\unit{\meter}}\right)^2\cdot 0.1\unit{\volt}\right)^2+\left(-\frac{125}{16}\left(\frac{0.158\unit{\meter}}{\mu_0\cdot130}\right)^2\cdot0.0709^{-3}\unit{\meter}\cdot\qty{3.9d-3}{\meter}\right)^2} \\
     & \approx\qty{3.98d10}{\coulomb\per\kilo\gram}
  \end{align*}
  Which gives a percent uncertainty of $\displaystyle\frac{3.98}{14.1}\cdot100\approx28.23\%$.
\end{soln}

% PROBLEM 9
\begin{problem}
How does your experimental charge-to-mass ratio compare with the reference value? Explain, using the uncertainty calculated in the previous question.
\end{problem}
\begin{soln}
  Using the value reccommended by NIST\footnote{Mohr, P. , Tiesinga, E. , Newell, D. and Taylor, B. (2024), Codata Internationally Recommended 2022 Values of the Fundamental Physical Constants, Codata Internationally Recommended 2022 Values of the Fundamental Physical Constants, [online], https://tsapps.nist.gov/publication/get\_pdf.cfm?pub\_id=958002, https://physics.nist.gov/constants (Accessed January 28, 2025) }
  we get a percentage error of
  $$\left\lvert \frac{\qty{1.41d11}{\coulomb\per\kilo\gram}-\qty{1.758 820 008 38d11}{\coulomb\per\kilo\gram}}{\qty{1.758 820 008 38d11}{\coulomb\per\kilo\gram}}\cdot100\right\rvert\approx 19.83\%$$
  Because the measurement uncertainty, $28.23\%$, is greater than the error our measured value agrees with the reference value within uncertainty. Woohoo!
\end{soln}

% PROBLEM 10
\begin{problem}
Say that we miss step 6 of the lab manual and forget to rotate the globe and align the electron beam to be parallel with the Helmholtz coils.
Specifically, say that there is an angle $\theta$ between the beam and the coil axis (i.e. the $x$-axis) when viewed from above.
Note that the ruler and mirror used to measure $r_{circ}$ is aligned with the coil axis. Will this have an effect on the $\frac{e}{m}$ ratio we calculate?
If so, would you expect it to be larger, smaller, or randomly distributed around the reference value? What about the uncertainty in $\frac{e}{m}$?
Explain your reasoning, showing any calculations and sketches.
\end{problem}
\begin{soln}
  Because the force on the electron is given by
  $$e\vec{v}\times \vec{B}=e\left\lVert \vec{v}\right\rVert\left\lVert \vec{B}\right\rVert\sin\theta$$
  then,
  \begin{align*}
     & F=ma=\frac{mv^2}{r_{circ}}evB\sin\theta         \\
     & r_{circ}=\frac{1}{\sin\theta}\cdot\frac{mv}{eB} \\
  \end{align*}
  so changing $\theta$ will have a constant factor effect on the radius of $\csc\theta$. Knowing that $-1\leq\csc\theta\leq-1$ for $0\leq\theta\leq2\pi$
  we should expect that any $\theta\neq\frac{\pi}{2}$ will cause an increase in the (magnitude) of the radius of the circle and therefore a (magnitude) decrease in the $\frac{e}{m}$ ratio.
\end{soln}
\end{document}