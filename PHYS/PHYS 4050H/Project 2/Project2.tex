\documentclass[10pt]{article}

\usepackage[margin=0.75in]{geometry}
\usepackage{amsmath,amsthm,amssymb}
\usepackage{xcolor}
\usepackage{cancel}
\usepackage{graphicx}
\usepackage{changepage}
\usepackage{circuitikz}
\usepackage{pgfplots}
\usepackage{physics}
\usepackage{hyperref}
\usepackage{siunitx}
\usepackage{fontspec}
\usepackage{relsize}
\usepackage{subfig}
\usepackage{minted}
\usepackage{todonotes}
\usepackage{adjustbox}
\usepackage{pdfpages}
\usepackage{multicol, multirow, booktabs}
\usepackage[breakable]{tcolorbox}
\usepackage[inline]{enumitem}
\usepackage{etoolbox}
\patchcmd{\thebibliography}{\section*{\refname}}{}{}{}

\theoremstyle{definition}
\newtheorem{problem}{Problem}
\newtheorem{soln}{Solution}

\pgfplotsset{compat=newest}
\usetikzlibrary{lindenmayersystems}
\usetikzlibrary{arrows}
\usetikzlibrary{calc}
\usetikzlibrary{positioning, fit}
\usetikzlibrary{3d, perspective}

\definecolor{incolor}{HTML}{303F9F}
\definecolor{outcolor}{HTML}{D84315}
\definecolor{cellborder}{HTML}{CFCFCF}
\definecolor{cellbackground}{HTML}{F7F7F7}
\newcommand{\ui}{\hat{i}}
\newcommand{\uj}{\hat{j}}
\newcommand{\uk}{\hat{k}}
\newcommand{\ux}{\hat{x}}
\newcommand{\uy}{\hat{y}}
\newcommand{\uz}{\hat{z}}
\newcommand{\primed}[1]{#1^\prime}
\pgfdeclarelayer{background}  
\pgfsetlayers{background,main}
\AtBeginDocument{\RenewCommandCopy\qty\SI}
\definecolor{LightGray}{gray}{0.9}

\makeatletter
\newcommand{\boxspacing}{\kern\kvtcb@left@rule\kern\kvtcb@boxsep}
\makeatother
\newcommand{\prompt}[4]{
    \ttfamily\llap{{\color{#2}[#3]:\hspace{3pt}#4}}\vspace{-\baselineskip}
}

\newcommand{\thevenin}[2]{
  \begin{center}
    \begin{circuitikz} \draw
      (0,0) -- (2,0) to[battery1, l_=$V_{Th}\eq#1$] (2,2) 
      to[resistor, l_=$R_{Th}\eq#2$] (0,2)
      ;
      \draw [o-] (-.07,2.079);
      \draw [o-] (-.07,0.079);
    \end{circuitikz}
  \end{center}
}

\newcommand{\norton}[2]{
  \begin{center}
    \begin{circuitikz} \draw
      (0,0) -- (3,0) to[american current source, l_=$I_{N}\eq#1$] (3,2) -- (0,2) (2,0)
      to[resistor, l=$R_{N}\eq#2$] (2,2)
      ;
      \draw [o-] (-.07,2.079);
      \draw [o-] (-.07,0.079);
    \end{circuitikz}
  \end{center}
}


\newcommand{\highlight}[1]{\colorbox{yellow}{$\displaystyle #1$}}

\newcommand{\ti}[1]{\widetilde{#1}}

\newfontface{\Kaufmann}{Kaufmann}
\DeclareTextFontCommand{\kf}{\Kaufmann}
\newcommand{\scriptr}{\fontsize{12pt}{12pt}\kf{r}}

\newfontface{\KaufmannB}{Kaufmann Bd BT}
\DeclareTextFontCommand{\kfb}{\KaufmannB}
\newcommand{\bscriptr}{\fontsize{12pt}{12pt}\kfb{r}}

\newcommand{\bv}[1]{\mathbf{#1}}

\title{Physics 4050H: Project 1}
\author{Jeremy Favro (0805980), \\\emph{Department of Physics \& Astronomy}\\ Trent University, Peterborough, ON, Canada}
\date{\today}

\begin{document}
\maketitle
\begin{abstract}
  Blah Blah
\end{abstract}
\todo{Reference figures when they're done}
\section{Introduction}
Normal intro stuff. Discuss why we think the datasheet plot is sketchy (i.e. real world examples of high quality mics).
Also consider units of the datasheet, dbU, dbV? Can we compare it to the freq. corr. curve from Dayton Audio??
\section{Theory??}
Idk, maybe talk about how electret mics work?? Not super relevant other than making it more evident
that the datasheet is dodgy.
\newpage
\section{Methods}
Give a general flowchart of the process
\subsection{Noise Generation}
\subsection{Constructing a Correction Profile}
In order to determine the response of the unknown (untrusted) microphone a known sound source and receiver are required.
Here a fairly trustworthy reference microphone was used in combination with a low quality speaker. The reference microphone,
a Dayton Audio iMM-6,
included a calibration file which mapped frequencies to dBV ($V=10^{dBV/20}$) correction values. Values were given with
respect to a reference value of $\qty{39.9}{\decibel\volt}$ at $\qty{1}{\kilo\hertz}$. Without a known accurate calibrated microphone
it is difficult to proceed \todo{Talk about reciprocity calibration as an option for \emph{creating} a reference mic} and so it
was assumed that audio recorded within the range of the calibration curve was, after applying the calibration curve, ideal.
\begin{figure}
  \begin{tikzpicture}
    \begin{axis}[
        xmode=log,
        log ticks with fixed point,
        xlabel={$\unit{\hertz}$},
        ylabel={$\unit{\decibel\volt}$},
        width=\linewidth,
        height=8cm]
      \addplot[mark=none] table [x={*1000Hz}, y={-39.9}, col sep=comma] {day_cal_curve.csv};
    \end{axis}
  \end{tikzpicture}
  \caption{Manufacturer provided correction curve for the reference microphone}
\end{figure}
\begin{figure}
  \centering
  \includegraphics[width=\textwidth]{demo.png}
  \caption{White noise captured on the reference microphone at various processing stages.
    The final plot is the difference between the raw and corrected signals, created by phase inversion.}
\end{figure}
With the recording microphone now assumed to be ideal any deviation from a signal with equal
contribution from all frequencies over the chosen range should be in majority caused by the speaker
with some environmental effects that remain effectively constant regardless of the recording device
used \todo{Talk about the harmonics :(}.
From here the audio recording/processing software Audacity was used to analyze the recorded audio 
and create a frequency contribution plot and dataset. The dataset contains frequencies mapped to 
decibel values representing the amount of a specific frequency present in the signal. In an
ideal system with a perfect emitter perfectly coupled to a perfect receiver the recorded signal
would contain equal amounts of every frequency emitted by the sound source. This is not the case here 
but all frequencies are present in some amount. This allows the construction of another correction curve.
By selecting a desired intensity we can add the difference between the recorded signal strength and the
desired signal strength at each frequency. This generates a correction curve which equalizes the signal across a 
desired frequency range. As, at this point, the only major influence on the signal is the quality of the speaker,
this correction curve will eliminate these effects and can therefore be used to eliminate the effects of the speaker
when testing non-ideal receivers.

% \begin{figure}
%   \begin{tikzpicture}
%     \begin{axis}[
%         xmode=log,
%         log ticks with fixed point,
%         xlabel={$\unit{\hertz}$},
%         ylabel={$\unit{\decibel}$},
%         width=\linewidth,
%         height=8cm]
%       \addplot[mark=none] table [x={Frequency (Hz)}, 
%       y={Level (dB)}, 
%       col sep=tab,
%       each nth point={10}] {spectrum.txt};
%     \end{axis}
%   \end{tikzpicture}
%   \caption{Time-averaged frequency ``amounts''.}
% \end{figure}
\section{Confounding Factors}
\subsection{}
\end{document}