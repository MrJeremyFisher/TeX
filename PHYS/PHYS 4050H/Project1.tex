\documentclass[10pt, twocolumn]{article}

\usepackage[margin=0.75in]{geometry}
\usepackage{amsmath,amsthm,amssymb}
\usepackage{xcolor}
\usepackage{cancel}
\usepackage{graphicx}
\usepackage{changepage}
\usepackage{circuitikz}
\usepackage{pgfplots}
\usepackage{physics}
\usepackage{hyperref}
\usepackage{siunitx}
\usepackage{fontspec}
\usepackage{relsize}
\usepackage{subfig}
\usepackage{todonotes}
\usepackage{multicol, multirow, booktabs}
\usepackage[breakable]{tcolorbox}
\usepackage[inline]{enumitem}

\theoremstyle{definition}
\newtheorem{problem}{Problem}
\newtheorem{soln}{Solution}

\pgfplotsset{compat=newest}
\usetikzlibrary{lindenmayersystems}
\usetikzlibrary{arrows}
\usetikzlibrary{calc}
\usetikzlibrary{positioning, fit}
\usetikzlibrary{3d, perspective}

\definecolor{incolor}{HTML}{303F9F}
\definecolor{outcolor}{HTML}{D84315}
\definecolor{cellborder}{HTML}{CFCFCF}
\definecolor{cellbackground}{HTML}{F7F7F7}
\newcommand{\ui}{\hat{i}}
\newcommand{\uj}{\hat{j}}
\newcommand{\uk}{\hat{k}}
\newcommand{\ux}{\hat{x}}
\newcommand{\uy}{\hat{y}}
\newcommand{\uz}{\hat{z}}
\newcommand{\primed}[1]{#1^\prime}
\pgfdeclarelayer{background}  
\pgfsetlayers{background,main}
\AtBeginDocument{\RenewCommandCopy\qty\SI}

\makeatletter
\newcommand{\boxspacing}{\kern\kvtcb@left@rule\kern\kvtcb@boxsep}
\makeatother
\newcommand{\prompt}[4]{
    \ttfamily\llap{{\color{#2}[#3]:\hspace{3pt}#4}}\vspace{-\baselineskip}
}

\newcommand{\thevenin}[2]{
  \begin{center}
    \begin{circuitikz} \draw
      (0,0) -- (2,0) to[battery1, l_=$V_{Th}\eq#1$] (2,2) 
      to[resistor, l_=$R_{Th}\eq#2$] (0,2)
      ;
      \draw [o-] (-.07,2.079);
      \draw [o-] (-.07,0.079);
    \end{circuitikz}
  \end{center}
}

\newcommand{\norton}[2]{
  \begin{center}
    \begin{circuitikz} \draw
      (0,0) -- (3,0) to[american current source, l_=$I_{N}\eq#1$] (3,2) -- (0,2) (2,0)
      to[resistor, l=$R_{N}\eq#2$] (2,2)
      ;
      \draw [o-] (-.07,2.079);
      \draw [o-] (-.07,0.079);
    \end{circuitikz}
  \end{center}
}


\newcommand{\highlight}[1]{\colorbox{yellow}{$\displaystyle #1$}}

\newcommand{\ti}[1]{\widetilde{#1}}

\newfontface{\Kaufmann}{Kaufmann}
\DeclareTextFontCommand{\kf}{\Kaufmann}
\newcommand{\scriptr}{\fontsize{12pt}{12pt}\kf{r}}

\newfontface{\KaufmannB}{Kaufmann Bd BT}
\DeclareTextFontCommand{\kfb}{\KaufmannB}
\newcommand{\bscriptr}{\fontsize{12pt}{12pt}\kfb{r}}

\newcommand{\bv}[1]{\mathbf{#1}}

\title{Physics 4050H: Project 1}
\author{Jeremy Favro (0805980) Melody Berhane (), \\\emph{Department of Physics \& Astronomy}\\ Trent University, Peterborough, ON, Canada}
\date{\today}

\begin{document}
\maketitle
\listoftodos

\begin{abstract}
  We successfully automated PHYS-2250H's lab 4 on operational amplifiers through microcontroller operated
  wave generation and input element switching, however due to time constraints did not automate data collection. Results are similar to those that could be obtained
  by ``manually'' completing the lab with most inaccuracy owing to additional resistance in the switching element(s) \todo{Any other areas?}.
\end{abstract}
\section{Introduction}
The goal of this project was to take a lab from PHYS-2250H (Electronics) and improve upon/automate the content of that lab. We opted to 
automate lab 4 which involved constructing different configurations of operational amplifiers and verifying that the theoretical equations 
derived in lecture corresponded to the real-world behaviour of the device. Lab 4 specifically looked at varying the input resistance of an
inverting amplifier and looked at a single configuration of differentiating amplifier. Our initial concept for automation was to go the most direct route
and use a microcontroller to control a switch which would allow selecting different components as in the case of both the inverting and differentiating amplifiers
the only component which is varied between measurements in the element lying between the signal source and inverting input of the amplifier. We additionally
sought to eliminate as many components external to those which sit directly on a breadboard and so created a function generator to replace the one used in the original lab.
\section{High Level Overview}
The automation consists of two key segments, one which generates a fixed frequency and peak-to-peak voltage input and one
which controls which element is connected to the inverting input of the amplifier. There is technically a third segment
of the device which provides a stable $\qty{5}{\unit{volts}}$ supply for the various integrated circuits but due to its simplicity
it is left off the following diagrams.
\todo[inline]{Whole circuit block diagram here}
\subsection{Wave Generator}
The input signal generator uses a single ATMmega328P \cite{microchip:ATmega328P} (through hole, 28-PDIP) microcontroller connected through SPI
to an MCP4822\cite{microchip:MCP4822} digital-to-analog converter.
\subsection{Amplifier Control \& Measurement}
\section{Detailed Methods}
\subsection{Wave Generator}
\subsection{Amplifier Control}
\subsection{Measurement}
\section{Conclusion}
\subsection{Results}
\subsection{Areas for Improvement}
\section{References}
\bibliography{sources}{}
\bibliographystyle{plain}
\section{Appendix}
\end{document}