\documentclass[10pt]{article}

\usepackage[margin=0.75in]{geometry}
\usepackage{amsmath,amsthm,amssymb,bm}
\usepackage{xcolor}
\usepackage{cancel}
\usepackage{graphicx}
\usepackage{changepage}
\usepackage{circuitikz}
\usepackage{pgfplots}
\usepackage{minted}
\usepackage{physics}
\usepackage{hyperref}
\usepackage{siunitx}
\usepackage{fontspec}
\usepackage{relsize}
\usepackage{subfig}
\usepackage{todonotes}
\usepackage{multicol, multirow, booktabs}
\usepackage[breakable]{tcolorbox}
\usepackage[inline]{enumitem}

\theoremstyle{definition}
\newtheorem{problem}{Problem}
\newtheorem{soln}{Solution}

\pgfplotsset{compat=newest}
\usetikzlibrary{lindenmayersystems}
\usetikzlibrary{arrows}
\usetikzlibrary{calc}
\usetikzlibrary{positioning, fit}
\usetikzlibrary{3d, perspective}

\definecolor{incolor}{HTML}{303F9F}
\definecolor{outcolor}{HTML}{D84315}
\definecolor{cellborder}{HTML}{CFCFCF}
\definecolor{cellbackground}{HTML}{F7F7F7}
\newcommand{\ui}{\hat{i}}
\newcommand{\uj}{\hat{j}}
\newcommand{\uk}{\hat{k}}
\newcommand{\ux}{\hat{\mathbf{x}}}
\newcommand{\uy}{\hat{\mathbf{y}}}
\newcommand{\uz}{\hat{\mathbf{z}}}
\newcommand{\uv}[1]{\hat{\bm{{#1}}}}
\newcommand{\pr}[1]{{#1^\prime}}
\newcommand{\pd}[2][]{{\frac{\partial^{#1}}{\partial {{#2}^{#1}}}}}
\newcommand{\dif}[2][]{{\frac{d^{#1}}{d{{#2}^{#1}}}}}
\pgfdeclarelayer{background}  
\pgfsetlayers{background,main}
\AtBeginDocument{\RenewCommandCopy\qty\SI}

\makeatletter
\newcommand{\boxspacing}{\kern\kvtcb@left@rule\kern\kvtcb@boxsep}
\makeatother
\newcommand{\prompt}[4]{
    \ttfamily\llap{{\color{#2}[#3]:\hspace{3pt}#4}}\vspace{-\baselineskip}
}

\newcommand{\thevenin}[2]{
  \begin{center}
    \begin{circuitikz} \draw
      (0,0) -- (2,0) to[battery1, l_=$V_{Th}\eq#1$] (2,2) 
      to[resistor, l_=$R_{Th}\eq#2$] (0,2)
      ;
      \draw [o-] (-.07,2.079);
      \draw [o-] (-.07,0.079);
    \end{circuitikz}
  \end{center}
}

\newcommand{\norton}[2]{
  \begin{center}
    \begin{circuitikz} \draw
      (0,0) -- (3,0) to[american current source, l_=$I_{N}\eq#1$] (3,2) -- (0,2) (2,0)
      to[resistor, l=$R_{N}\eq#2$] (2,2)
      ;
      \draw [o-] (-.07,2.079);
      \draw [o-] (-.07,0.079);
    \end{circuitikz}
  \end{center}
}

\DeclareMathOperator{\Div}{div}
\DeclareMathOperator{\Curl}{curl}
\DeclareMathOperator{\sgn}{sgn}

\newcommand{\highlight}[1]{\colorbox{yellow}{$\displaystyle #1$}}

\newcommand{\ti}[1]{\widetilde{#1}}

\newfontface{\Kaufmann}{Kaufmann}
\DeclareTextFontCommand{\kf}{\Kaufmann}
\newcommand{\scriptr}{\fontsize{12pt}{12pt}\kf{r}}

\newfontface{\KaufmannB}{Kaufmann Bd BT}
\DeclareTextFontCommand{\kfb}{\KaufmannB}
\newcommand{\bscriptr}{\fontsize{12pt}{12pt}\kfb{r}}
\newcommand{\justif}[2]{&{#1}&\text{#2}}
\newcommand{\bv}[1]{\mathbf{#1}}

\title{Physics 3200Y: Assignment V}
\author{Jeremy Favro (0805980) \\ Trent University, Peterborough, ON, Canada}
\date{\today}

\begin{document}
\maketitle

% PROBLEM 1
\begin{problem}
Griffiths 4.13
\end{problem}
\begin{soln}

\end{soln}

% PROBLEM 2
\begin{problem}
In lab, before Christmas, you measured the capacitance of a ferroelectric capacitor. This question asks you to
make sense of your data.
\begin{enumerate}[label=(\alph*)]
  \item Consider a parallel-plate capacitor of thickness $d$ and area $A$. The area between the capacitor plates is
        filled with a ferroelectric, with polarization
        $$P=\pm P_s +\epsilon_0\chi E,$$
        where $P_s$ is the magnitude of the polarization when the electric field is zero, $\chi$ is the dielectric susceptibility
        that tells you how the polarization changes when a field $E$ is applied, and $E$ is the electric field in the
        capacitor. The ferroelectric polarization contains a factor $\pm$ to account for the two possible states, one
        with polarization pointing up and one with polarization pointing down.
        \begin{enumerate}[label=\roman*]
          \item Find the electric field $E$ and the charge density $\sigma$ on the plates when the voltage across the capacitor
                is $V$. Note that your answer will contain $\pm P_s$.
          \item ketch a graph of $\sigma$ vs. $V$. Note that
                \begin{enumerate}[label=\textbullet]
                  \item there are two branches, corresponding to the two signs of Ps;
                  \item the direction of $P_s$ will flip when the internal electric field exceeds the so-called “coercive
                        field” $E_C$. Your graph should look something like this one: \href{https://en.wikipedia.org/wiki/
                          Ferroelectricity#/media/File:Ferroelectric_polarisation_DE.svg}{https://en.wikipedia.org/wiki/
                          Ferroelectricity\#/media/File:Ferroelectric\_polarisation\_DE.svg}.
                \end{enumerate}
                Be sure to label the coercive fields on your graph and explain (in terms of the model introduced here)
                the arrows that are shown in the figure I’ve linked to above.
        \end{enumerate}
  \item In a Sawyer-Tower circuit, the total charge that flows onto the unknown “sample” capacitor (i.e. the ferroelectric capacitor)
        is equal to the charge that flows onto the reference capacitor, $Q = C_r V_r$ . Furthermore,
        the voltage across the sample is $V_s = V - V_r \approx V$ . A graph of $V_r$ vs. $V$ should therefore look like your
        sketch, assuming the capacitor really is ferroelectric. Make such a plot and try to interpret your data in
        terms of the model in part (a). How does your data agree or disagree with the model. I know that you
        measured your data at multiple frequencies, so choose one or two frequencies that illustrate the typical
        behaviour you observed.
\end{enumerate}
\end{problem}
\begin{soln}
  \begin{enumerate}[label=(\alph*)]
    \item Consider a parallel-plate capacitor of thickness $d$ and area $A$. The area between the capacitor plates is
          filled with a ferroelectric, with polarization
          $$P=\pm P_s +\epsilon_0\chi E,$$
          where $P_s$ is the magnitude of the polarization when the electric field is zero, $\chi$ is the dielectric susceptibility
          that tells you how the polarization changes when a field $E$ is applied, and $E$ is the electric field in the
          capacitor. The ferroelectric polarization contains a factor $\pm$ to account for the two possible states, one
          with polarization pointing up and one with polarization pointing down.
          \begin{enumerate}[label=\roman*]
            \item $Q=CV=$

            \item Sketch a graph of $\sigma$ vs. $V$. Note that
                  \begin{enumerate}[label=\textbullet]
                    \item there are two branches, corresponding to the two signs of Ps;
                    \item the direction of $P_s$ will flip when the internal electric field exceeds the so-called “coercive
                          field” $E_C$. Your graph should look something like this one: \href{https://en.wikipedia.org/wiki/
                            Ferroelectricity#/media/File:Ferroelectric_polarisation_DE.svg}{https://en.wikipedia.org/wiki/
                            Ferroelectricity\#/media/File:Ferroelectric\_polarisation\_DE.svg}.
                  \end{enumerate}
                  Be sure to label the coercive fields on your graph and explain (in terms of the model introduced here)
                  the arrows that are shown in the figure I’ve linked to above.
          \end{enumerate}
  \end{enumerate}
\end{soln}
\end{document}