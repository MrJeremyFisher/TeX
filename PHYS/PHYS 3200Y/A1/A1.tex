\documentclass[10pt]{article}

\usepackage[margin=0.75in]{geometry}
\usepackage{amsmath,amsthm,amssymb}
\usepackage{xcolor}
\usepackage{cancel}
\usepackage{graphicx}
\usepackage{changepage}
\usepackage{circuitikz}
\usepackage{pgfplots}
\usepackage{physics}
\usepackage{hyperref}
\usepackage{siunitx}
\usepackage{fontspec}
\usepackage{relsize}
\usepackage{multicol, multirow, booktabs}
\usepackage[breakable]{tcolorbox}
\usepackage[inline]{enumitem}

\theoremstyle{definition}
\newtheorem{problem}{Problem}
\newtheorem{soln}{Solution}

\pgfplotsset{compat=newest}
\usetikzlibrary{lindenmayersystems}
\usetikzlibrary{arrows}
\usetikzlibrary{calc}

\definecolor{incolor}{HTML}{303F9F}
\definecolor{outcolor}{HTML}{D84315}
\definecolor{cellborder}{HTML}{CFCFCF}
\definecolor{cellbackground}{HTML}{F7F7F7}
\newcommand{\ui}{\hat{i}}
\newcommand{\uj}{\hat{j}}
\newcommand{\uk}{\hat{k}}
\newcommand{\ux}{\hat{x}}
\newcommand{\uy}{\hat{y}}
\newcommand{\uz}{\hat{z}}
\newcommand{\primed}[1]{#1^\prime}
\usetikzlibrary{positioning, fit, calc}
\pgfdeclarelayer{background}  
\pgfsetlayers{background,main}
\AtBeginDocument{\RenewCommandCopy\qty\SI}

\makeatletter
\newcommand{\boxspacing}{\kern\kvtcb@left@rule\kern\kvtcb@boxsep}
\makeatother
\newcommand{\prompt}[4]{
    \ttfamily\llap{{\color{#2}[#3]:\hspace{3pt}#4}}\vspace{-\baselineskip}
}

\newcommand{\thevenin}[2]{
  \begin{center}
    \begin{circuitikz} \draw
      (0,0) -- (2,0) to[battery1, l_=$V_{Th}\eq#1$] (2,2) 
      to[resistor, l_=$R_{Th}\eq#2$] (0,2)
      ;
      \draw [o-] (-.07,2.079);
      \draw [o-] (-.07,0.079);
    \end{circuitikz}
  \end{center}
}

\newcommand{\norton}[2]{
  \begin{center}
    \begin{circuitikz} \draw
      (0,0) -- (3,0) to[american current source, l_=$I_{N}\eq#1$] (3,2) -- (0,2) (2,0)
      to[resistor, l=$R_{N}\eq#2$] (2,2)
      ;
      \draw [o-] (-.07,2.079);
      \draw [o-] (-.07,0.079);
    \end{circuitikz}
  \end{center}
}

\newcommand{\highlight}[1]{\colorbox{yellow}{$\displaystyle #1$}}

\newcommand{\ti}[1]{\widetilde{#1}}

\newfontface{\Kaufmann}{Kaufmann}
\DeclareTextFontCommand{\kf}{\Kaufmann}
\newcommand{\scriptr}{\kf{r}}

\newfontface{\KaufmannB}{Kaufmann Bd BT}
\DeclareTextFontCommand{\kfb}{\KaufmannB}
\newcommand{\bscriptr}{\kfb{r}}

\newcommand{\bv}[1]{\mathbf{#1}}

\title{Physics 3200Y: Assignment I}
\author{Jeremy Favro (0805980) \\ Trent University, Peterborough, ON, Canada}
\date{\today}

\begin{document}
\maketitle

% PROBLEM 1
\begin{problem}
Let $\bscriptr=\bv{r}-\bv{r}^\prime$ be the separation between $\bv{r}$ and $\bv{r}^\prime$, where $\bv{r}^\prime = (x^\prime, y^\prime, z^\prime)$ is a fixed point and $\bv{r} = (x, y, z)$. Let
$\scriptr = \left|\bscriptr \right|$ be the magnitude of the separation.
\begin{enumerate}[label=(\alph*)]
  \item Show that $\nabla\left(\scriptr^2\right)=2\bscriptr$.
  \item Show that $\nabla\exp\left(\vec{k}\cdot\vec{\bscriptr}\right)=\vec{k}\exp\left(\vec{k}\cdot\vec{\bscriptr}\right)$, where $\vec{k}$ is a vector constant.
  \item Show that $\nabla\exp\left(k\scriptr\right)=k\hat{\bscriptr}\exp\left(k\scriptr\right)$.
  \item Show that $\nabla \left(\scriptr^{-1}\right)=-\hat{\bscriptr}/\scriptr^2$.
\end{enumerate}
\end{problem}
\begin{soln} ~
  \begin{enumerate}[label=(\alph*)]
    \item \begin{proof}
            \begin{align*}
               & =\nabla\left(\scriptr^2\right)                                                                                                                                                                                     \\
               & =\left[\frac{\partial}{\partial x}\ux + \frac{\partial}{\partial y}\uy + \frac{\partial}{\partial z}\uz \right]\sqrt{\left(x-\primed{x}\right)^2 + \left(y-\primed{y}\right)^2+ \left(z-\primed{z}\right)^2}^2     \\
               & =\left[\frac{\partial}{\partial x}\ux + \frac{\partial}{\partial y}\uy + \frac{\partial}{\partial z}\uz \right]\left[\left(x-\primed{x}\right)^2 + \left(y-\primed{y}\right)^2+ \left(z-\primed{z}\right)^2\right] \\
               & =\frac{\partial}{\partial x}\left(x-\primed{x}\right)^2\ux + \frac{\partial}{\partial y} \left(y-\primed{y}\right)^2\uy + \frac{\partial}{\partial z}\left(z-\primed{z}\right)^2\uz \quad \text{Note\footnotemark} \\
               & =2\left(x-\primed{x}\right)\ux + 2\left(y-\primed{y}\right)\uy+2\left(z-\primed{z}\right)\uz \quad \text{By chain rule}                                                                                            \\
               & =2(\left(x-\primed{x}\right),\left(y-\primed{y}\right),\left(z-\primed{z}\right)) = 2\bscriptr\qedhere
            \end{align*}
            \footnotetext{The partials kill the terms that don't contain their variable of differentiation, omitted for brevity}
          \end{proof}
    \item \begin{proof}
            \begin{align*}
               & =  \nabla\exp\left(\vec{k}\cdot\vec{\bscriptr}\right)                                                             \\
               & =  \nabla\exp\left(k_x(x-\primed{x})+k_y(y-\primed{y})+k_z(z-\primed{z})\right)                                   \\
               & =  \nabla\exp\left(k_x(x-\primed{x})\right) \exp\left(k_y(y-\primed{y})\right) \exp\left(k_z(z-\primed{z})\right) \\
            \end{align*}
          \end{proof}
    \item Show that $\nabla\exp\left(k\scriptr\right)=k\hat{\bscriptr}\exp\left(k\scriptr\right)$.
    \item Show that $\nabla \left(\scriptr^{-1}\right)=-\hat{\bscriptr}/\scriptr^2$.
  \end{enumerate}
\end{soln}
\end{document}