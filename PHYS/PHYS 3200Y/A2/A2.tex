\documentclass[10pt]{article}

\usepackage[margin=0.75in]{geometry}
\usepackage{amsmath,amsthm,amssymb}
\usepackage{xcolor}
\usepackage{cancel}
\usepackage{graphicx}
\usepackage{changepage}
\usepackage{circuitikz}
\usepackage{pgfplots}
\usepackage{physics}
\usepackage{hyperref}
\usepackage{siunitx}
\usepackage{fontspec}
\usepackage{relsize}
\usepackage{subfig}
\usepackage{todonotes}
\usepackage{multicol, multirow, booktabs}
\usepackage[breakable]{tcolorbox}
\usepackage[inline]{enumitem}

\theoremstyle{definition}
\newtheorem{problem}{Problem}
\newtheorem{soln}{Solution}

\pgfplotsset{compat=newest}
\usetikzlibrary{lindenmayersystems}
\usetikzlibrary{arrows}
\usetikzlibrary{calc}
\usetikzlibrary{positioning, fit}
\usetikzlibrary{3d, perspective}

\definecolor{incolor}{HTML}{303F9F}
\definecolor{outcolor}{HTML}{D84315}
\definecolor{cellborder}{HTML}{CFCFCF}
\definecolor{cellbackground}{HTML}{F7F7F7}
\newcommand{\ui}{\hat{i}}
\newcommand{\uj}{\hat{j}}
\newcommand{\uk}{\hat{k}}
\newcommand{\ux}{\hat{x}}
\newcommand{\uy}{\hat{y}}
\newcommand{\uz}{\hat{z}}
\newcommand{\primed}[1]{#1^\prime}
\pgfdeclarelayer{background}  
\pgfsetlayers{background,main}
\AtBeginDocument{\RenewCommandCopy\qty\SI}

\makeatletter
\newcommand{\boxspacing}{\kern\kvtcb@left@rule\kern\kvtcb@boxsep}
\makeatother
\newcommand{\prompt}[4]{
    \ttfamily\llap{{\color{#2}[#3]:\hspace{3pt}#4}}\vspace{-\baselineskip}
}

\newcommand{\thevenin}[2]{
  \begin{center}
    \begin{circuitikz} \draw
      (0,0) -- (2,0) to[battery1, l_=$V_{Th}\eq#1$] (2,2) 
      to[resistor, l_=$R_{Th}\eq#2$] (0,2)
      ;
      \draw [o-] (-.07,2.079);
      \draw [o-] (-.07,0.079);
    \end{circuitikz}
  \end{center}
}

\newcommand{\norton}[2]{
  \begin{center}
    \begin{circuitikz} \draw
      (0,0) -- (3,0) to[american current source, l_=$I_{N}\eq#1$] (3,2) -- (0,2) (2,0)
      to[resistor, l=$R_{N}\eq#2$] (2,2)
      ;
      \draw [o-] (-.07,2.079);
      \draw [o-] (-.07,0.079);
    \end{circuitikz}
  \end{center}
}

\DeclareMathOperator{\Div}{div}
\DeclareMathOperator{\Curl}{curl}

\newcommand{\highlight}[1]{\colorbox{yellow}{$\displaystyle #1$}}

\newcommand{\ti}[1]{\widetilde{#1}}

\newfontface{\Kaufmann}{Kaufmann}
\DeclareTextFontCommand{\kf}{\Kaufmann}
\newcommand{\scriptr}{\fontsize{12pt}{12pt}\kf{r}}

\newfontface{\KaufmannB}{Kaufmann Bd BT}
\DeclareTextFontCommand{\kfb}{\KaufmannB}
\newcommand{\bscriptr}{\fontsize{12pt}{12pt}\kfb{r}}

\newcommand{\bv}[1]{\mathbf{#1}}

\title{Physics 3200Y: Assignment I}
\author{Jeremy Favro (0805980) \\ Trent University, Peterborough, ON, Canada}
\date{\today}

\begin{document}
\maketitle

% PROBLEM 1
\begin{problem}
A hollow cone has a constant surface charge density $\sigma$. The cone's vertex is at $z = 0$ and its axis lies along the
positive $z$-axis. It has a height $h$, which is also the radius of the cone at its widest end (the top). You are asked
to find the potential difference $V (h) - V (0$) between the top (on the cone's axis) and the vertex.
\begin{enumerate}[label=(\alph*)]
  \item Draw a picture that defines $\vec{r}$, $\primed{\vec{r}}$, $\scriptr$, etc.
  \item Discuss how you will parameterize your surface and find expressions for $\primed{\vec{r}}$ and $d\primed{a}$ in terms of your parameters.
  \item Write out and solve the integrals for $V (h)$ and $V (0)$ to find the final answer. The integral for $V (0)$ is easy
        to do, and I have some hints for the integrals for $V (h)$. First, I found it helpful to make the substitution
        $\primed{z}=h/2+t$, and then with a little bit of work the integral simplifies to a constant times something of the
        form
        $$\int_{-a}^{a}\frac{dt}{\sqrt{a^2+t^2}}=2\sinh^{-1}(\frac{1}{\sqrt{2}})$$
        Show your work. Don't simply look up the integral!
\end{enumerate}
\end{problem}
\begin{soln}
\end{soln}

% PROBLEM 2
\begin{problem}
Consider the infamous triple-plate capacitor (I just made that up), consisting of three parallel plates. Let the
middle plate lie in the xy plane, let the top plate lie a distance $d_1$ above it and the bottom plate lie a distance
$d_2$ below it. Assume that the capacitor charges in such a way that the top plate has a 2D charge density $\sigma$,
while the other two plates have charge densities$-\sigma/2$; treat the plates as infinite.
\begin{enumerate}[label=(\alph*)]
  \item Find the electric field everywhere along the $z$-axis, and the voltage differences between the different plates.
  \item Find the energy per unit area required to charge up the system by incrementally moving amounts of
        charge dQ from the lower two plates to the upper plate and calculating the work done for each increment.
        Describe carefully the sequence of steps in each case, since that is what you are being graded on.
  \item Show that you get the same answer for the work by integrating the electric field over all space
\end{enumerate}
\end{problem}
\begin{soln}
\end{soln}

% PROBLEM 3
\begin{problem}
A uniformly charged solid cylinder has length $L$, radius $R$, and charge density $\rho$.
\begin{enumerate}[label=(\alph*)]
  \item Find the potential on the axis of the cylinder as a function of $z$, the distance along the central axis from
the centre of the cylinder.
  \item Does the solution to part (a) give you enough information to solve for the electric field along the $z$-axis?
Why or why not? Hint: Look at the gradient operator in cylindrical coordinates. What assumptions would
you have to make to calculate $\vec{E}$? Are they reasonable?
\end{enumerate}
\end{problem}
\begin{soln}
\end{soln}
\end{document}