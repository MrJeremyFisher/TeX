\documentclass[10pt]{article}

\usepackage[margin=0.75in]{geometry}
\usepackage{amsmath,amsthm,amssymb}
\usepackage{xcolor}
\usepackage{cancel}
\usepackage{graphicx}
\usepackage{changepage}
\usepackage{circuitikz}
\usepackage{pgfplots}
\usepackage{physics}
\usepackage{hyperref}
\usepackage{siunitx}
\usepackage{minted}
\usepackage{fontspec}
\usepackage{relsize}
\usepackage{subfig}
\usepackage{todonotes}
\usepackage{multicol, multirow, booktabs}
\usepackage[breakable]{tcolorbox}
\usepackage[inline]{enumitem}

\theoremstyle{definition}
\newtheorem{problem}{Problem}
\newtheorem{soln}{Solution}

\pgfplotsset{compat=newest}
\usetikzlibrary{lindenmayersystems}
\usetikzlibrary{arrows}
\usetikzlibrary{calc}
\usetikzlibrary{positioning, fit}
\usetikzlibrary{3d, perspective}

\definecolor{incolor}{HTML}{303F9F}
\definecolor{outcolor}{HTML}{D84315}
\definecolor{cellborder}{HTML}{CFCFCF}
\definecolor{cellbackground}{HTML}{F7F7F7}
\newcommand{\ui}{\hat{i}}
\newcommand{\uj}{\hat{j}}
\newcommand{\uk}{\hat{k}}
\newcommand{\ux}{\hat{x}}
\newcommand{\uy}{\hat{y}}
\newcommand{\uz}{\hat{z}}
\newcommand{\uv}[1]{\hat{\mathbf{{#1}}}}
\newcommand{\pr}[1]{{#1^\prime}}
\newcommand{\justif}[2]{&{#1}&\text{#2}}
\newcommand{\dul}[1]{\underline{\underline{#1}}}
\pgfdeclarelayer{background}  
\pgfsetlayers{background,main}
\AtBeginDocument{\RenewCommandCopy\qty\SI}

\makeatletter
\newcommand{\boxspacing}{\kern\kvtcb@left@rule\kern\kvtcb@boxsep}
\makeatother
\newcommand{\prompt}[4]{
    \ttfamily\llap{{\color{#2}[#3]:\hspace{3pt}#4}}\vspace{-\baselineskip}
}

\newcommand{\thevenin}[2]{
  \begin{center}
    \begin{circuitikz} \draw
      (0,0) -- (2,0) to[battery1, l_=$V_{Th}\eq#1$] (2,2) 
      to[resistor, l_=$R_{Th}\eq#2$] (0,2)
      ;
      \draw [o-] (-.07,2.079);
      \draw [o-] (-.07,0.079);
    \end{circuitikz}
  \end{center}
}

\newcommand{\norton}[2]{
  \begin{center}
    \begin{circuitikz} \draw
      (0,0) -- (3,0) to[american current source, l_=$I_{N}\eq#1$] (3,2) -- (0,2) (2,0)
      to[resistor, l=$R_{N}\eq#2$] (2,2)
      ;
      \draw [o-] (-.07,2.079);
      \draw [o-] (-.07,0.079);
    \end{circuitikz}
  \end{center}
}

\newcommand{\highlight}[1]{\colorbox{yellow}{$\displaystyle #1$}}

\newcommand{\ti}[1]{\widetilde{#1}}

\newfontface{\Kaufmann}{Kaufmann}
\DeclareTextFontCommand{\kf}{\Kaufmann}
\newcommand{\scriptr}{\fontsize{12pt}{12pt}\kf{r}}

\newfontface{\KaufmannB}{Kaufmann Bd BT}
\DeclareTextFontCommand{\kfb}{\KaufmannB}
\newcommand{\bscriptr}{\fontsize{12pt}{12pt}\kfb{r}}

\newcommand{\bv}[1]{\mathbf{#1}}

\title{Physics 3610H: Assignment XI}
\author{Jeremy Favro (0805980) \\ Trent University, Peterborough, ON, Canada}
\date{\today}

\begin{document}
\maketitle

% PROBLEM 1
\begin{problem}
Express the operator $\hat{x}$ as a matrix using as a basis the eigenstates of the harmonic
oscillator Hamiltonian. Show the upper left $5 \times 5$ elements.
\end{problem}
\begin{soln}
  First we need to obtain $\hat{x}$ in terms of operators that we can easily act on the given basis.
  These are the raising and lowering operators,
  $$\hat{a}_{\pm}=\frac{1}{\sqrt{2}}\left[\left(\frac{m\omega}{\hbar}\right)^{1/2}\hat{x}\mp i\frac{\hat{p}_x}{(m\hbar\omega)^{1/2}}\right].$$
  Rearranging these we can see that
  \begin{align*}
    \hat{a}_{+}+\hat{a}_{-} & =\frac{1}{\sqrt{2}}\left[\left(\frac{m\omega}{\hbar}\right)^{1/2}\hat{x}- i\frac{\hat{p}_x}{(m\hbar\omega)^{1/2}}
    +\left(\frac{m\omega}{\hbar}\right)^{1/2}\hat{x}+ i\frac{\hat{p}_x}{(m\hbar\omega)^{1/2}}\right]                                            \\
                            & =2\left(\frac{m\omega}{2\hbar}\right)^{1/2}\hat{x}
  \end{align*}
  so
  $$\hat{x}=\left(\frac{\hbar}{2m\omega}\right)^{1/2}\left(\hat{a}_{+}+\hat{a}_{-}\right).$$
  Now we can evaluate
  $$\bra{E_i}\left(\frac{\hbar}{2m\omega}\right)^{1/2}\left(\hat{a}_{+}+\hat{a}_{-}\right)\ket{E_j}$$
  to obtain the $ij$th element of the matrix representation.
  Calculating the value for a generic $ij$ element,
  \begin{align*}
    X_{ij} & =\bra{E_i}\left(\frac{\hbar}{2m\omega}\right)^{1/2}\left(\hat{a}_{+}+\hat{a}_{-}\right)\ket{E_j}                        \\
           & =\left(\frac{\hbar}{2m\omega}\right)^{1/2}\left(\bra{E_i}\hat{a}_{+}\ket{E_j}+\bra{E_i}\hat{a}_{-}\ket{E_j}\right)      \\
           & =\left(\frac{\hbar}{2m\omega}\right)^{1/2}\left(\sqrt{j+1}\bra{E_i}\ket{E_{j+1}}+\sqrt{j}\bra{E_i}\ket{E_{j-1}}\right).
  \end{align*}
  \newpage
  I did the actual writing out of the elements with SageMath as it's just algebra from this point,
  \begin{tcolorbox}[breakable, size=fbox, boxrule=1pt, pad at break*=1mm,colback=cellbackground, colframe=cellborder]
    \prompt{In}{incolor}{1}{\boxspacing}
    \begin{minted}[breaklines, autogobble]{sage}
        clear_vars()
        x, m = var('x m')
        w = var('w', latex_name=r'\omega')
        hbar = var('hbar', latex_name=r'\hbar') 

        coeff = sqrt(hbar/(2*m*w))

        M = matrix(SR, 5,5)

        def delta(a,b):
            if (a==b):
                return 1
            else: 
                return 0

        for i in range(0, 5): # row
            for j in range(0,5):
                M[i,j] = (coeff * (sqrt(j+1)*delta(i,j+1)+ sqrt(j)*delta(i,j-1))).full_simplify()

        show(M)
    \end{minted}
  \end{tcolorbox}
  \begin{tcolorbox}[breakable, size=fbox, boxrule=.5pt, pad at break*=1mm, opacityfill=0]
    \prompt{Out}{outcolor}{1}{\boxspacing}
    $\displaystyle \left(\begin{array}{rrrrr}
          0                                                         & \frac{1}{2} \, \sqrt{2} \sqrt{\frac{{\hbar}}{m {\omega}}} & 0                                                                  & 0                                                                  & 0                                          \\
          \frac{1}{2} \, \sqrt{2} \sqrt{\frac{{\hbar}}{m {\omega}}} & 0                                                         & \sqrt{\frac{{\hbar}}{m {\omega}}}                                  & 0                                                                  & 0                                          \\
          0                                                         & \sqrt{\frac{{\hbar}}{m {\omega}}}                         & 0                                                                  & \frac{1}{2} \, \sqrt{3} \sqrt{2} \sqrt{\frac{{\hbar}}{m {\omega}}} & 0                                          \\
          0                                                         & 0                                                         & \frac{1}{2} \, \sqrt{3} \sqrt{2} \sqrt{\frac{{\hbar}}{m {\omega}}} & 0                                                                  & \sqrt{2} \sqrt{\frac{{\hbar}}{m {\omega}}} \\
          0                                                         & 0                                                         & 0                                                                  & \sqrt{2} \sqrt{\frac{{\hbar}}{m {\omega}}}                         & 0
        \end{array}\right)$
  \end{tcolorbox}
\end{soln}

% PROBLEM 2 
\begin{problem}
Suppose the operator $\hat{A}$ acting on the state $\ket{\Psi}$ results in the state $\ket{X}$: $\hat{A}\ket{\Psi}=\ket{X}$.
What is the corresponding equation in matrix representation using the basis $\left\{\ket{\phi_i}\right\}$? What
are the elements of each matrix and vector in your equation?
\end{problem}
\begin{soln}
  I believe the corresponding equation is just
  $$\dul{A}\ket{\Psi}=\ket{X}$$
  where $\dul{A}$ is the matrix representation of $\hat{A}$,
  $$A_{ij}=\bra{\phi_i}\hat{A}\ket{\phi_j}.$$
  $\ket{\Psi}$ and $\ket{X}$ are column vectors of the coefficients in the eigenfunction expansions of their respective states.
\end{soln}
\newpage

% PROBLEM 3
\begin{problem}
~
$$\dul{M}=\begin{pmatrix}
    2  & i \\
    -i & 3
  \end{pmatrix}$$
\begin{enumerate}[label=(\alph*)]
  \item Find the eigenvalues and eigenvectors of this matrix.
  \item Construct a matrix for which the columns are the eigenvectors found in (a). Show this
        matrix is unitary.
\end{enumerate}
\end{problem}
\begin{soln}~
  \begin{enumerate}[label=(\alph*)]
    \item For eigenvalues we solve the expression
          \begin{align*}
            0 & =\left|\dul{M}-\lambda \dul{I}\right|                                                          \\
              & =\begin{vmatrix}
                   2-\lambda & i         \\
                   -i        & 3-\lambda
                 \end{vmatrix}                                                                         \\
              & =(2-\lambda)(3-\lambda)-1                                                                      \\
              & =\lambda^2-5\lambda+5\implies \lambda=\frac{5\pm \sqrt{25-4(1)(5)}}{2}=\frac{5\pm\sqrt{5}}{2}.
          \end{align*}
          Eigenvectors are vectors which satisfy
          $$\dul{M}\vec{v}=\lambda \vec{v}$$
          so for eigenvector $\vec{v}_+$ corresponding to $\lambda=\frac{5+\sqrt{5}}{2}$
          we want a $\vec{v}_+$ where
          $$\begin{pmatrix}
              2  & i \\
              -i & 3
            \end{pmatrix}\begin{pmatrix}
              v_{+1} \\
              v_{+2}
            \end{pmatrix}=\frac{5+\sqrt{5}}{2}\begin{pmatrix}
              v_{+1} \\
              v_{+2}
            \end{pmatrix}
          $$
          evaluating the left hand side,
          $$\begin{pmatrix}
              2  & i \\
              -i & 3
            \end{pmatrix}\begin{pmatrix}
              v_{+1} \\
              v_{+2}
            \end{pmatrix}=\begin{pmatrix}
              2v_{+1}+iv_{+2} \\
              3v_{+2}-iv_{+1}
            \end{pmatrix}$$
          so we have the two equation system
          \begin{equation}
            2v_{+1}+iv_{+2}=\frac{5+\sqrt{5}}{2}v_{+1}
          \end{equation}
          \begin{equation}
            3v_{+2}-iv_{+1}=\frac{5+\sqrt{5}}{2}v_{+2}.
          \end{equation}
          (1) tells us that
          $$v_{+2}=-i\left(\frac{5+\sqrt{5}}{2}-2\right)v_{+1}=-i\left(\frac{1+\sqrt{5}}{2}\right)v_{+1}.$$
          Choosing
          $v_{+1}=1$ then gives us, with $\sim$ representing normalization,
          $$\vec{v}_+=\begin{pmatrix}
              1 \\
              -\dfrac{(1+\sqrt{5})i}{2}
            \end{pmatrix}\sim \begin{pmatrix}
              \sqrt{\frac{\sqrt{5}-1}{2\sqrt{5}}} \\
              -i\frac{\sqrt{5}\sqrt{10+2\sqrt{5}}}{10}
            \end{pmatrix}.$$
          We can apply the same process to obtain
          $$\vec{v}_-=\begin{pmatrix}
              1 \\
              \dfrac{(-1+\sqrt{5})i}{2}
            \end{pmatrix}\sim
            \begin{pmatrix}
              \sqrt{\frac{\sqrt{5}+1}{2\sqrt{5}}} \\
              i\frac{\sqrt{5}\sqrt{10-2\sqrt{5}}}{10}
            \end{pmatrix}$$
    \item Our constructed matrix is
          $$\dul{U}=\begin{pmatrix}
              \sqrt{\frac{\sqrt{5}-1}{2\sqrt{5}}}      & \sqrt{\frac{\sqrt{5}+1}{2\sqrt{5}}}     \\
              -i\frac{\sqrt{5}\sqrt{10+2\sqrt{5}}}{10} & i\frac{\sqrt{5}\sqrt{10-2\sqrt{5}}}{10}
            \end{pmatrix}.$$
          To show this is unitary we evaluate
          \begin{align*}
            \dul{U}\dul{U}^\dagger & =\begin{pmatrix}
                                        \sqrt{\frac{\sqrt{5}-1}{2\sqrt{5}}}      & \sqrt{\frac{\sqrt{5}+1}{2\sqrt{5}}}     \\
                                        -i\frac{\sqrt{5}\sqrt{10+2\sqrt{5}}}{10} & i\frac{\sqrt{5}\sqrt{10-2\sqrt{5}}}{10}
                                      \end{pmatrix}
            \begin{pmatrix}
              \sqrt{\frac{\sqrt{5}-1}{2\sqrt{5}}} & i\frac{\sqrt{5}\sqrt{10+2\sqrt{5}}}{10}  \\
              \sqrt{\frac{\sqrt{5}+1}{2\sqrt{5}}} & -i\frac{\sqrt{5}\sqrt{10-2\sqrt{5}}}{10}
            \end{pmatrix}                                                                                                                                                                                                                                                                                                                                                                                                                                                                                                                   \\
                                   & =\begin{pmatrix}\sqrt{\frac{\sqrt{5}-1}{2\sqrt{5}}}\sqrt{\frac{\sqrt{5}-1}{2\sqrt{5}}}+\sqrt{\frac{\sqrt{5}+1}{2\sqrt{5}}}\sqrt{\frac{\sqrt{5}+1}{2\sqrt{5}}}                       & \sqrt{\frac{\sqrt{5}-1}{2\sqrt{5}}}i\frac{\sqrt{5}\sqrt{10+2\sqrt{5}}}{10}+\sqrt{\frac{\sqrt{5}+1}{2\sqrt{5}}}\left(-i\frac{\sqrt{5}\sqrt{10-2\sqrt{5}}}{10}\right)                       \\
               \left(-i\frac{\sqrt{5}\sqrt{10+2\sqrt{5}}}{10}\right)\sqrt{\frac{\sqrt{5}-1}{2\sqrt{5}}}+i\frac{\sqrt{5}\sqrt{10-2\sqrt{5}}}{10}\sqrt{\frac{\sqrt{5}+1}{2\sqrt{5}}} & \left(-i\frac{\sqrt{5}\sqrt{10+2\sqrt{5}}}{10}\right)i\frac{\sqrt{5}\sqrt{10+2\sqrt{5}}}{10}+i\frac{\sqrt{5}\sqrt{10-2\sqrt{5}}}{10}\left(-i\frac{\sqrt{5}\sqrt{10-2\sqrt{5}}}{10}\right)\end{pmatrix}\\
               &=\begin{pmatrix}
                1 & 0 \\
                0 & 1
               \end{pmatrix}
          \end{align*}
          where that last step is done mostly on inspection of signs in the huge matrix. I verified it with SageMath and it does come out to $\dul{I}$.
  \end{enumerate}
\end{soln}

% PROBLEM 4
\begin{problem}
Consider the Hamiltonian
$$\hat{H}=\frac{\hat{p}_x^2}{2m}+\frac{1}{2}k\hat{x}^2+\beta\hat{x}^3$$
Describe how you could, using a computer, find approximate results for the lowest energy
eigenvalue and its corresponding eigenstate.
\end{problem}
\begin{soln}
  To my understanding the idea here is to construct a truncated Hamiltonian matrix for
  the given $\hat{H}$ in the basis of the eigenstates of the ``standard'' harmonic oscillator for
  which $\beta=0$. We then use this matrix to find the approximate (due to the truncated matrix) eigenvalues and eigenvectors of the system.
  We can then use the eigenvector to write the state corresponding to the found eigenvalue as an eigenfunction expansion where the entries of the
  eigenvector are the coefficients.
\end{soln}
\end{document}