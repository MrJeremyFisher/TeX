\documentclass[10pt]{article}

\usepackage[margin=0.75in]{geometry}
\usepackage{amsmath,amsthm,amssymb}
\usepackage{xcolor}
\usepackage{cancel}
\usepackage{graphicx}
\usepackage{changepage}
\usepackage{circuitikz}
\usepackage{pgfplots}
\usepackage{physics}
\usepackage{hyperref}
\usepackage{siunitx}
\usepackage{fontspec}
\usepackage{relsize}
\usepackage{subfig}
\usepackage{todonotes}
\usepackage{minted}
\usepackage{multicol, multirow, booktabs}
\usepackage[breakable]{tcolorbox}
\usepackage[inline]{enumitem}

\theoremstyle{definition}
\newtheorem{problem}{Problem}
\newtheorem{soln}{Solution}

\pgfplotsset{compat=newest}
\usetikzlibrary{lindenmayersystems}
\usetikzlibrary{arrows}
\usetikzlibrary{calc}
\usetikzlibrary{positioning, fit}
\usetikzlibrary{3d, perspective}

\definecolor{incolor}{HTML}{303F9F}
\definecolor{outcolor}{HTML}{D84315}
\definecolor{cellborder}{HTML}{CFCFCF}
\definecolor{cellbackground}{HTML}{F7F7F7}
\newcommand{\ui}{\hat{i}}
\newcommand{\uj}{\hat{j}}
\newcommand{\uk}{\hat{k}}
\newcommand{\ux}{\hat{x}}
\newcommand{\uy}{\hat{y}}
\newcommand{\uz}{\hat{z}}
\newcommand{\primed}[1]{#1^\prime}
\pgfdeclarelayer{background}  
\pgfsetlayers{background,main}
\AtBeginDocument{\RenewCommandCopy\qty\SI}

\makeatletter
\newcommand{\boxspacing}{\kern\kvtcb@left@rule\kern\kvtcb@boxsep}
\makeatother
\newcommand{\prompt}[4]{
    \ttfamily\llap{{\color{#2}[#3]:\hspace{3pt}#4}}\vspace{-\baselineskip}
}

\newcommand{\thevenin}[2]{
  \begin{center}
    \begin{circuitikz} \draw
      (0,0) -- (2,0) to[battery1, l_=$V_{Th}\eq#1$] (2,2) 
      to[resistor, l_=$R_{Th}\eq#2$] (0,2)
      ;
      \draw [o-] (-.07,2.079);
      \draw [o-] (-.07,0.079);
    \end{circuitikz}
  \end{center}
}

\newcommand{\norton}[2]{
  \begin{center}
    \begin{circuitikz} \draw
      (0,0) -- (3,0) to[american current source, l_=$I_{N}\eq#1$] (3,2) -- (0,2) (2,0)
      to[resistor, l=$R_{N}\eq#2$] (2,2)
      ;
      \draw [o-] (-.07,2.079);
      \draw [o-] (-.07,0.079);
    \end{circuitikz}
  \end{center}
}

\newcommand{\highlight}[1]{\colorbox{yellow}{$\displaystyle #1$}}

\newcommand{\ti}[1]{\widetilde{#1}}

\newfontface{\Kaufmann}{Kaufmann}
\DeclareTextFontCommand{\kf}{\Kaufmann}
\newcommand{\scriptr}{\fontsize{12pt}{12pt}\kf{r}}

\newfontface{\KaufmannB}{Kaufmann Bd BT}
\DeclareTextFontCommand{\kfb}{\KaufmannB}
\newcommand{\bscriptr}{\fontsize{12pt}{12pt}\kfb{r}}

\newcommand{\bv}[1]{\mathbf{#1}}

\title{Physics 3610H: Assignment IV}
\author{Jeremy Favro (0805980) \\ Trent University, Peterborough, ON, Canada}
\date{\today}

\begin{document}
\maketitle

% PROBLEM 1
\begin{problem}
In class we found the general form of the wavefunction in each region of a finite well to
be
$$
  \begin{cases}
    x<-a De^{+\kappa x}
    -a<x<a A\cos kx + B \sin kx
    x> a Ce^{-\kappa x}
  \end{cases}
$$
Using the continuity of the wavefunction and its first derivative at both $x = -a$ and $x = +a$,
we arrived at the following four equations.
\begin{align}
  \psi(-a)                     & =A\cos ka - B \sin ka  = De^{-\kappa a}          \\
  \psi(+a)                     & =A\cos ka + B \sin ka  = Ce^{-\kappa a}          \\
  \eval{\frac{d\psi}{dx}}_{-a} & = kA\sin ka +kB\cos ka = D\kappa e^{-\kappa a}   \\
  \eval{\frac{d\psi}{dx}}_{+a} & = -kA\sin ka +kB\cos ka = _C\kappa e^{-\kappa a}
\end{align}
Together with normalization these determine $A$, $B$, $C$, $D$ and $E$. In particular, by considering
$(2)-(4)/\kappa$ and $(3)+(5)/\kappa$ we showed that
$$
  A\left(1-\frac{k}{\kappa}\tan ka\right) = B\left(\tan ka + \frac{k}{\kappa}\right)=0
$$
In class we considered the even solutions by setting $B = 0$. Here, consider the odd solutions
by setting $A = 0$.
\begin{enumerate}[label=(\alph*)]
  \item What two equations connect $k$ and $\kappa$ in this case?
  \item Let $x \equiv  ka$ and $y \equiv \kappa a$ and plot both functions on a single plot for $2mV_oa^2/\hbar^2$ = 25.
  \item How many allowed values of energy are there in this case?
  \item Give an approximate value of $\kappa a$ which is allowed.
  \item Use $(2)+(4)/k$ and $(3)-(5)/\kappa$ to determine the values of $C$ and $D$, and write the form of
the odd wavefunctions in each region in terms of $B$, $k$ and $\kappa$.
\end{enumerate}
\end{problem}
\begin{soln}
  
\end{soln}
\end{document}