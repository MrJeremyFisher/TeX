\documentclass[10pt]{article}

\usepackage[margin=0.75in]{geometry}
\usepackage{amsmath,amsthm,amssymb}
\usepackage{xcolor}
\usepackage{cancel}
\usepackage{graphicx}
\usepackage{changepage}
\usepackage{circuitikz}
\usepackage{pgfplots}
\usepackage{physics}
\usepackage{hyperref}
\usepackage{siunitx}
\usepackage{fontspec}
\usepackage{relsize}
\usepackage{subfig}
\usepackage{todonotes}
\usepackage{multicol, multirow, booktabs}
\usepackage[breakable]{tcolorbox}
\usepackage[inline]{enumitem}

\theoremstyle{definition}
\newtheorem{problem}{Problem}
\newtheorem{soln}{Solution}

\pgfplotsset{compat=newest}
\usetikzlibrary{lindenmayersystems}
\usetikzlibrary{arrows}
\usetikzlibrary{calc}
\usetikzlibrary{positioning, fit}
\usetikzlibrary{3d, perspective}

\definecolor{incolor}{HTML}{303F9F}
\definecolor{outcolor}{HTML}{D84315}
\definecolor{cellborder}{HTML}{CFCFCF}
\definecolor{cellbackground}{HTML}{F7F7F7}
\newcommand{\ui}{\hat{i}}
\newcommand{\uj}{\hat{j}}
\newcommand{\uk}{\hat{k}}
\newcommand{\ux}{\hat{x}}
\newcommand{\uy}{\hat{y}}
\newcommand{\uz}{\hat{z}}
\newcommand{\uv}[1]{\hat{\mathbf{{#1}}}}
\newcommand{\pr}[1]{{#1^\prime}}
\pgfdeclarelayer{background}  
\pgfsetlayers{background,main}
\AtBeginDocument{\RenewCommandCopy\qty\SI}

\makeatletter
\newcommand{\boxspacing}{\kern\kvtcb@left@rule\kern\kvtcb@boxsep}
\makeatother
\newcommand{\prompt}[4]{
    \ttfamily\llap{{\color{#2}[#3]:\hspace{3pt}#4}}\vspace{-\baselineskip}
}

\newcommand{\thevenin}[2]{
  \begin{center}
    \begin{circuitikz} \draw
      (0,0) -- (2,0) to[battery1, l_=$V_{Th}\eq#1$] (2,2) 
      to[resistor, l_=$R_{Th}\eq#2$] (0,2)
      ;
      \draw [o-] (-.07,2.079);
      \draw [o-] (-.07,0.079);
    \end{circuitikz}
  \end{center}
}

\newcommand{\norton}[2]{
  \begin{center}
    \begin{circuitikz} \draw
      (0,0) -- (3,0) to[american current source, l_=$I_{N}\eq#1$] (3,2) -- (0,2) (2,0)
      to[resistor, l=$R_{N}\eq#2$] (2,2)
      ;
      \draw [o-] (-.07,2.079);
      \draw [o-] (-.07,0.079);
    \end{circuitikz}
  \end{center}
}

\newcommand{\highlight}[1]{\colorbox{yellow}{$\displaystyle #1$}}

\newcommand{\ti}[1]{\widetilde{#1}}

\newfontface{\Kaufmann}{Kaufmann}
\DeclareTextFontCommand{\kf}{\Kaufmann}
\newcommand{\scriptr}{\fontsize{12pt}{12pt}\kf{r}}

\newfontface{\KaufmannB}{Kaufmann Bd BT}
\DeclareTextFontCommand{\kfb}{\KaufmannB}
\newcommand{\bscriptr}{\fontsize{12pt}{12pt}\kfb{r}}

\newcommand{\bv}[1]{\mathbf{#1}}

\title{Physics 3610H: Assignment VIII}
\author{Jeremy Favro (0805980) \\ Trent University, Peterborough, ON, Canada}
\date{\today}

\begin{document}
\maketitle

% PROBLEM 0
\setcounter{problem}{-1}
\setcounter{soln}{-1}
\begin{problem}
Consider a particle in one dimension subject to the potential
$V(x)=-V_0\delta(x).$
\begin{enumerate}[label=(\alph*)]
  \item Write the time-independent Schr\"odinger equation for this system
  \item What is the functional form of the solutions in the region $x<0$? Explain
        your reasoning.
  \item What is the functional form of the solutions in the region $x>0$? Again, explain.
  \item This potential has an infinite discontinuity which results in a discontinuity
        in the first derivative of the wavefunction at $x=0$. Specifically,
        $$\eval{\frac{d\psi}{dx}}_{x=0^-}-\eval{\frac{d\psi}{dx}}_{x=0^+}=\frac{2mV_0}{\hbar^2}\psi(x=0)$$
        State all additional conditions which $\psi(x)$ must satisfy.
  \item Is the energy quantized?
  \item Fully determine one solution and its corresponding energy.
\end{enumerate}
\end{problem}
\begin{soln}~
  \begin{enumerate}[label=(\alph*)]
    \item $$\hat{H}\psi(x)=E\psi(x).$$
          Here
          $$\hat{H}=\frac{\hat{p}_x^2}{2m}+V(x)=-\frac{\hbar^2}{2m}\frac{\partial^2}{\partial x^2}-V_0\delta(x).$$
    \item In order for the solutions in this region to be physical they must decay as $x\to -\infty$.
          The easy way to do this is with a positive-valued exponential, some $e^x$. I'm skipping some
          detail here because I was comfortable with these parts on the midterm.
    \item Again so that solutions in this region are physical we must be able to normalize them
          and so they must decay to zero as $x\to +\infty$. The way to have this is a negative-valued
          exponential, some $e^{-x}$.
    \item $\psi(x)$ must be normalizable as mentioned above. It must also be continuous (but not necessarily smooth)
          at $x=0$ where the two solutions meet.
    \item To determine if the energy is quantized I'll solve the T.I.S.E given above.
          There are three regions in which we need to solve. Region 1 is $x<0$.
          This was a point on the midterm where I tacked on an extra term which would not be normalizable
          that I just dropped later. Here I'm being smarter so I'll just present that the T.I.S.E
          in this region is
          $$-\frac{\hbar^2}{2m}\frac{\partial\psi(x)}{\partial x}=E\psi(x)$$
          so
          $$\frac{\partial\psi(x)}{\partial x}=\frac{2m\abs{E}}{\hbar^2}\psi(x)$$
          where we take $\abs{E}$ as $E<0$.
          The solution to this is
          $$\psi_1(x)=Ae^{kx}$$
          where
          $$k=\sqrt{\frac{2mE}{\hbar^2}}$$
          and $A$ is a constant determined by boundary and normalization conditions.
          Now in region 3 for $x>0$ we do the same process, noting our answer to part (d) which means
          $$\psi_3(x)=Be^{-kx}.$$
          Because these two have the same exponent, just differing in sign, we can note that
          $A=B$ because the exponential is symmetric about $0$ for sign changes.
          Now we must meet the condition on the discontinuity in the derivative.
          Our wavefunction at this point is
          $$
            \psi(x)=\begin{cases}
              Ae^{kx}  & x<0 \\
              Ae^{-kx} & x>0
            \end{cases}
          $$
          so the derivative is
          $$
            \pr{\psi}(x)=\begin{cases}
              Ake^{kx}   & x<0 \\
              -Ake^{-kx} & x>0
            \end{cases}
          $$
          and so the condition becomes
          $$
            Ak+Ak=2Ak=\frac{2mV_0}{\hbar^2}\psi(x=0).
          $$
          Here $\psi(x=0)$ is the point at which the two exponentials meet. This value will
          be either exponential at $x=0$ which will in both cases just be $A$, so
          we can cancel the $A$ throughout,
          \begin{align*}
            k                                  & =\frac{mV_0}{\hbar^2}     \\
            \implies \frac{2m\abs{E}}{\hbar^2} & =\frac{m^2V_0^2}{\hbar^4} \\
            \implies \abs{E}                   & =\frac{mV_0^2}{2\hbar^2}
          \end{align*}
          so
          $$E=-\frac{mV_0^2}{2\hbar^2}$$
          because we know $E<0$ and all the other quantities on the LHS are positive.
          Therefore the energy is not quantized per-se, but can only take on one specific value, dependent on the properties
          of the particle and the well.
    \item In order to fully determine the solution all that remains is to determine the value of $A$ such that
          $\psi(x)$ is normalized.
          To do this we solve
          $$1=\int_{\mathbb{R}}\abs{\psi}^2\,dx$$
          for $A$. Note that $\int_{\mathbb{R}}$ is my way of writing
          an integral over all space to make things look a little neater in handwriting. It's like
          the integral over a surface $\mathcal{S}$ for example, where you just subscript the integral with
          $\mathcal{S}$ and figure out the bounds later. Solving the integral then,
          \begin{align*}
            1 & =\int_{\mathbb{R}}\abs{\psi}^2\,dx                                                       \\
              & =\int_{-\infty}^0\abs{A}^2e^{2kx}\,dx+\int_{0}^\infty\abs{A}^2e^{-2kx}\,dx               \\
              & =\abs{A}^2\left[\frac{1}{2k}+\frac{1}{2k}\right]=\frac{\abs{A}^2}{k}\implies A=\sqrt{k}.
          \end{align*}
          So a full state/the only state is
          $$
            \psi(x)=\begin{cases}
              \sqrt{k}e^{kx}  & x<0 \\
              \sqrt{k}        & x=0 \\
              \sqrt{k}e^{-kx} & x>0
            \end{cases}
          $$
          with, as noted above, energy
          $$E=-\frac{mV_0^2}{2\hbar^2}$$
  \end{enumerate}
\end{soln}

% PROBLEM 1
\begin{problem}
Show that if $\braket{h}{\hat{Q}h}=\braket{\hat{Q}h}{h}$ for all functions $h$ in Hilbert space then $\braket{f}{\hat{Q}g}=\braket{\hat{Q}f}{g}$
where $f$ and $g$ are also functions in the Hilbert space.
\end{problem}
\begin{soln}
  As given in the hint we'll start with $h=f+g$.
  Note that
  $$
    \bra{\hat{Q}\left(f+g\right)}
    =\left(\ket{\hat{Q}(f+g)}\right)^\dagger
    =\left(\ket{\hat{Q}f}+\ket{\hat{Q}g}\right)^\dagger
    =\ket{\hat{Q}f}^\dagger+\ket{\hat{Q}g}^\dagger
    =\bra{\hat{Q}f}+\bra{\hat{Q}g}
  $$
  just to be explicit. Now, expanding $h=f+g$,
  \begin{align*}
    \braket{h}{\hat{Q}h}                                                                                  & =\braket{\hat{Q}h}{h}                                                                                  \\
    \braket{f+g}{\hat{Q}\left(f+g\right)}                                                                 & =\braket{\hat{Q}\left(f+g\right)}{f+g}                                                                 \\
    \braket{f}{\hat{Q}\left(f+g\right)}+\braket{g}{\hat{Q}\left(f+g\right)}                               & =\braket{\hat{Q}\left(f+g\right)}{f}+\braket{\hat{Q}\left(f+g\right)}{g}                               \\
    \cancel{\braket{f}{\hat{Q}f}}+\braket{f}{\hat{Q}g}+\braket{g}{\hat{Q}f}+\cancel{\braket{g}{\hat{Q}g}} & =\cancel{\braket{\hat{Q}f}{f}}+\braket{\hat{Q}g}{f}+\braket{\hat{Q}f}{g}+\cancel{\braket{\hat{Q}g}{g}} \\
    \braket{f}{\hat{Q}g}+\braket{g}{\hat{Q}f}                                                             & =\braket{\hat{Q}g}{f}+\braket{\hat{Q}f}{g}
  \end{align*}
  Now taking $h=f+ig$ and expanding,
  \begin{align*}
    \braket{h}{\hat{Q}h}                                                                                    & =\braket{\hat{Q}h}{h}                                                                                    \\
    \braket{f+ig}{\hat{Q}\left(f+ig\right)}                                                                 & =\braket{\hat{Q}\left(f+ig\right)}{f+ig}                                                                 \\
    \braket{f}{\hat{Q}\left(f+ig\right)}+\braket{ig}{\hat{Q}\left(f+ig\right)}                              & =\braket{\hat{Q}\left(f+ig\right)}{f}+\braket{\hat{Q}\left(f+ig\right)}{ig}                              \\
    \braket{f}{\hat{Q}f}+\braket{f}{\hat{Q}ig}+\braket{ig}{\hat{Q}f}+\braket{ig}{\hat{Q}ig}                 & =\braket{\hat{Q}f}{f}+\braket{\hat{Q}ig}{f}+\braket{\hat{Q}f}{ig}+\braket{\hat{Q}ig}{ig}                 \\
    \cancel{\braket{f}{\hat{Q}f}}+i\braket{f}{\hat{Q}g}-i\braket{g}{\hat{Q}f}+\cancel{\braket{g}{\hat{Q}g}} & =\cancel{\braket{\hat{Q}f}{f}}-i\braket{\hat{Q}g}{f}+i\braket{\hat{Q}f}{g}+\cancel{\braket{\hat{Q}g}{g}} \\
    i\braket{f}{\hat{Q}g}-i\braket{g}{\hat{Q}f}                                                             & =-i\braket{\hat{Q}g}{f}+i\braket{\hat{Q}f}{g}                                                            \\
    \braket{f}{\hat{Q}g}-\braket{g}{\hat{Q}f}                                                               & =-\braket{\hat{Q}g}{f}+\braket{\hat{Q}f}{g}
  \end{align*}
  Now adding these two equations,
  \begin{align*}
    \braket{f}{\hat{Q}g}-\braket{g}{\hat{Q}f}+\braket{f}{\hat{Q}g}+\braket{g}{\hat{Q}f} & =-\braket{\hat{Q}g}{f}+\braket{\hat{Q}f}{g}+\braket{\hat{Q}g}{f}+\braket{\hat{Q}f}{g} \\
    2\braket{f}{\hat{Q}g}                                                               & =2\braket{\hat{Q}f}{g}                                                                \\
    \braket{f}{\hat{Q}g}                                                                & =\braket{\hat{Q}f}{g}
  \end{align*}
\end{soln}

% PROBLEM 2
\begin{problem}~
\begin{enumerate}[label=(\alph*)]
  \item $\bv{a}_1=\frac{1}{2}\ux-\frac{\sqrt{3}}{2}\uy$ and $\bv{a}_2=\frac{1}{2}\ux+\frac{\sqrt{3}}{2}\uy$.
        Use the procedure discussed in class to construct
        a pair of vectors $\bv{b}_1$ and $\bv{b}_2$ which are orthogonal to each other and both normalized.
  \item Demonstrate that $\bv{b}_1$ and $\bv{b}_2$ form a complete set, in terms of which any vector in two
        dimensions may be expressed, by showing that
        $$\ket{b_1}\bra{b_1}+\ket{b_2}\bra{b_2}=I_2=\begin{pmatrix}
            1 & 0 \\
            0 & 1
          \end{pmatrix}$$
\end{enumerate}
\end{problem}
\begin{soln}~
  \begin{enumerate}[label=(\alph*)]
    \item The process mentioned is the Gram-Schmidt process wherein we remove the portion of each vector which
          is colinear with the other vectors. Note that here $\bv{a}_1$ and $\bv{a}_2$ are both already normalized. So,
          \begin{align*}
            \bv{b}_1 & =\bv{a}_1=\uv{b}_1                                                                                 \\
            \bv{b}_2 & =\bv{a}_2-\frac{\braket{b_2}{b_1}}{\braket{b_1}{b_1}}\bv{b}_1                                      \\
                     & =\frac{1}{2}\ux-\frac{\sqrt{3}}{2}\uy+\frac{1}{2}\left(\frac{1}{2}\ux-\frac{\sqrt{3}}{2}\uy\right) \\
                     & =\frac{3}{4}\ux-\frac{\sqrt{3}}{4}\uy
          \end{align*}
          Now to normalize $\bv{b}_2$
          $$\uv{b}_2=\frac{\bv{b}_2}{\braket{b_2}{b_2}}=\frac{\sqrt{3}}{2}\ux+\frac{1}{2}\uy$$
    \item It is convenient now to write $\uv{b}_1$ and $\uv{b}_2$ in their Dirac notation forms as column and row vectors,
          $$
            \ket{b_1}=\begin{pmatrix}
              1/2 \\
              -\sqrt{3}/2
            \end{pmatrix};\quad
            \bra{b_1}=\begin{pmatrix}
              1/2 &
              -\sqrt{3}/2
            \end{pmatrix};\quad
            \ket{b_2}=\begin{pmatrix}
              \sqrt{3}/2 \\
              1/2
            \end{pmatrix};\quad
            \bra{b_2}=\begin{pmatrix}
              \sqrt{3}/2 &
              1/2
            \end{pmatrix}.
          $$
          Now computing the given expression,
          $$\ket{b_1}\bra{b_1}=
            \begin{pmatrix}
              1/2 \\
              -\sqrt{3}/2
            \end{pmatrix}\begin{pmatrix}
              1/2 &
              -\sqrt{3}/2
            \end{pmatrix}=
            \begin{pmatrix}
              1/4         & -\sqrt{3}/4 \\
              -\sqrt{3}/4 & 3/4
            \end{pmatrix},
          $$
          and
          $$\ket{b_2}\bra{b_2}=
            \begin{pmatrix}
              \sqrt{3}/2 \\
              1/2
            \end{pmatrix}\begin{pmatrix}
              \sqrt{3}/2 &
              1/2
            \end{pmatrix}=
            \begin{pmatrix}
              3/4        & \sqrt{3}/4 \\
              \sqrt{3}/4 & 1/4
            \end{pmatrix}.
          $$
          And summing the two,
          $$
            \begin{pmatrix}
              3/4        & \sqrt{3}/4 \\
              \sqrt{3}/4 & 1/4
            \end{pmatrix}+
            \begin{pmatrix}
              1/4         & -\sqrt{3}/4 \\
              -\sqrt{3}/4 & 3/4
            \end{pmatrix}
            =\begin{pmatrix}
              1 & 0 \\
              0 & 1
            \end{pmatrix}
          $$
  \end{enumerate}
\end{soln}
\end{document}