\documentclass[10pt]{article}

\usepackage[margin=0.75in]{geometry}
\usepackage{amsmath,amsthm,amssymb}
\usepackage{xcolor}
\usepackage{cancel}
\usepackage{graphicx}
\usepackage{changepage}
\usepackage{circuitikz}
\usepackage{pgfplots}
\usepackage{physics}
\usepackage{hyperref}
\usepackage{siunitx}
\usepackage{fontspec}
\usepackage{relsize}
\usepackage{subfig}
\usepackage{todonotes}
\usepackage{multicol, multirow, booktabs}
\usepackage[breakable]{tcolorbox}
\usepackage[inline]{enumitem}

\theoremstyle{definition}
\newtheorem{problem}{Problem}
\newtheorem{soln}{Solution}

\pgfplotsset{compat=newest}
\usetikzlibrary{lindenmayersystems}
\usetikzlibrary{arrows}
\usetikzlibrary{calc}
\usetikzlibrary{positioning, fit}
\usetikzlibrary{3d, perspective}

\definecolor{incolor}{HTML}{303F9F}
\definecolor{outcolor}{HTML}{D84315}
\definecolor{cellborder}{HTML}{CFCFCF}
\definecolor{cellbackground}{HTML}{F7F7F7}
\newcommand{\ui}{\hat{i}}
\newcommand{\uj}{\hat{j}}
\newcommand{\uk}{\hat{k}}
\newcommand{\ux}{\hat{x}}
\newcommand{\uy}{\hat{y}}
\newcommand{\uz}{\hat{z}}
\newcommand{\uv}[1]{\hat{\mathbf{{#1}}}}
\newcommand{\pr}[1]{{#1^\prime}}
\newcommand{\justif}[2]{&{#1}&\text{#2}}
\newcommand{\dunderline}[1]{\underline{\underline{#1}}}
\pgfdeclarelayer{background}  
\pgfsetlayers{background,main}
\AtBeginDocument{\RenewCommandCopy\qty\SI}

\makeatletter
\newcommand{\boxspacing}{\kern\kvtcb@left@rule\kern\kvtcb@boxsep}
\makeatother
\newcommand{\prompt}[4]{
    \ttfamily\llap{{\color{#2}[#3]:\hspace{3pt}#4}}\vspace{-\baselineskip}
}

\newcommand{\thevenin}[2]{
  \begin{center}
    \begin{circuitikz} \draw
      (0,0) -- (2,0) to[battery1, l_=$V_{Th}\eq#1$] (2,2) 
      to[resistor, l_=$R_{Th}\eq#2$] (0,2)
      ;
      \draw [o-] (-.07,2.079);
      \draw [o-] (-.07,0.079);
    \end{circuitikz}
  \end{center}
}

\newcommand{\norton}[2]{
  \begin{center}
    \begin{circuitikz} \draw
      (0,0) -- (3,0) to[american current source, l_=$I_{N}\eq#1$] (3,2) -- (0,2) (2,0)
      to[resistor, l=$R_{N}\eq#2$] (2,2)
      ;
      \draw [o-] (-.07,2.079);
      \draw [o-] (-.07,0.079);
    \end{circuitikz}
  \end{center}
}

\newcommand{\highlight}[1]{\colorbox{yellow}{$\displaystyle #1$}}

\newcommand{\ti}[1]{\widetilde{#1}}

\newfontface{\Kaufmann}{Kaufmann}
\DeclareTextFontCommand{\kf}{\Kaufmann}
\newcommand{\scriptr}{\fontsize{12pt}{12pt}\kf{r}}

\newfontface{\KaufmannB}{Kaufmann Bd BT}
\DeclareTextFontCommand{\kfb}{\KaufmannB}
\newcommand{\bscriptr}{\fontsize{12pt}{12pt}\kfb{r}}

\newcommand{\bv}[1]{\mathbf{#1}}

\title{Physics 3610H: Assignment X}
\author{Jeremy Favro (0805980) \\ Trent University, Peterborough, ON, Canada}
\date{\today}

\begin{document}
\maketitle

% PROBLEM 1
\begin{problem}
In class we showed that if $\ket{E_n}$ is normalized, then $\ket{E_{n+1}}=\hat{a}_+\ket{E_n}/\sqrt{n+1}$ is also
normalized. Assume again that $\ket{E_n}$ is normalized and show that for $\ket{E_{n-1}}$ to be normalized
it must equal $\hat{a}_-\ket{E_n}/\sqrt{n}$.
\end{problem}
\begin{soln}
  Well, we know that $\ket{E_{n-1}}\propto \hat{a}_-\ket{E_n}$ by the definition of $\hat{a}_-$. So
  $$\ket{E_{n-1}}=C_{n-1}\hat{a}_-\ket{E_{n}}.$$
  And so
  \begin{align*}
    \braket{E_{n-1}}{E_{n-1}} & =\left(C_{n-1}\hat{a}_-\ket{E_{n}}\right)^\dagger C_{n-1}\hat{a}_-\ket{E_{n}} \\
                              & =\abs{C_{n-1}}^2\bra{E_{n}}\hat{a}_-^\dagger\hat{a}_-\ket{E_n}                \\
                              & =\abs{C_{n-1}}^2\bra{E_{n}}\hat{a}_+\hat{a}_-\ket{E_n}.
  \end{align*}
  We found in class that
  $$\hat{a}_+\hat{a}_-=\frac{m\omega}{2\hbar}\hat{x}^2+\frac{1}{2m\omega\hbar}\hat{p}_x^2-\frac{1}{2}$$
  and from this that
  $$\hat{H}=\hbar\omega\left(\hat{a}_+\hat{a}_-+\frac{1}{2}\right)\implies \hat{a}_+\hat{a}_-=\frac{\hat{H}}{\hbar\omega}-\frac{1}{2}$$
  and so our previous inner product becomes
  \begin{align*}
    \abs{C_{n-1}}^2\bra{E_{n}}\hat{a}_+\hat{a}_-\ket{E_n} & = \abs{C_{n-1}}^2\bra{E_{n}}\left(\frac{\hat{H}}{\hbar\omega}-\frac{1}{2}\right)\ket{E_n}                      \\
                                                          & = \abs{C_{n-1}}^2\bra{E_{n}}\left(\frac{\hat{H}}{\hbar\omega}\ket{E_n}-\frac{1}{2}\ket{E_n}\right)             \\
                                                          & = \abs{C_{n-1}}^2\left(\frac{1}{\hbar\omega}\bra{E_{n}}\hat{H}\ket{E_n}-\frac{1}{2}\bra{E_{n}}\ket{E_n}\right) \\
                                                          & = \abs{C_{n-1}}^2\left(\frac{1}{\hbar\omega}E_n-\frac{1}{2}\right)                                             \\
                                                          & = \abs{C_{n-1}}^2\left(n+\frac{1}{2}-\frac{1}{2}\right)                                                        \\
                                                          & = \abs{C_{n-1}}^2n
  \end{align*}
  we want this expression to be normalized (recall with originally began with $\braket{E_{n-1}}{E_{n-1}}$) and so we set it equal to 1,
  $$\abs{C_{n-1}}^2n=1\implies C_{n-1}=\frac{1}{\sqrt{n}}$$
  as we wanted.
\end{soln}
\newpage

% PROBLEM 2
\begin{problem}
In class, we used $\hat{a}_-\ket{E_0} = 0$ to show that $\psi_0(\xi)\propto e^{-\xi^2/2}$. In fact, when you normalize
this you find
$$\psi_0(x)=\left(\frac{m\omega}{\pi\hbar}\right)^{1/4} e^{-m\omega x^2/2\hbar}.$$
Use the raising operator to find $\psi_1(x)$.
\end{problem}
\begin{soln}
  The raising operator, $\hat{a}_+$ is, in position representation, given by
  $$
    \hat{a}_+=\frac{1}{\sqrt{2}}\left[\sqrt{\frac{m\omega}{\hbar}}\hat{x}-i\frac{\hat{p}_x}{\sqrt{m\omega\hbar}}\right]
    =\frac{1}{\sqrt{2}}\left[\sqrt{\frac{m\omega}{\hbar}}x-\sqrt{\frac{\hbar}{m\omega}}\frac{\partial}{\partial x}\right].
  $$
  Applying this to $\psi_0$,
  \begin{align*}
    \hat{a}_+\psi_0(x) & =\frac{1}{\sqrt{2}}\left[\sqrt{\frac{m\omega}{\hbar}}x-\sqrt{\frac{\hbar}{m\omega}}\frac{\partial}{\partial x}\right]\left(\frac{m\omega}{\pi\hbar}\right)^{1/4} e^{-m\omega x^2/2\hbar}      \\
                       & =\left(\frac{m\omega}{4\pi\hbar}\right)^{1/4}\left[\sqrt{\frac{m\omega}{\hbar}}xe^{-m\omega x^2/2\hbar}-\sqrt{\frac{\hbar}{m\omega}}\frac{\partial}{\partial x}e^{-m\omega x^2/2\hbar}\right] \\
                       & =\left(\frac{m\omega}{4\pi\hbar}\right)^{1/4}\left[\sqrt{\frac{m\omega}{\hbar}}xe^{-m\omega x^2/2\hbar}+\sqrt{\frac{\hbar}{m\omega}}\frac{m\omega}{\hbar}xe^{-m\omega x^2/2\hbar}\right]      \\
                       & =\left(\frac{m\omega}{4\pi\hbar}\right)^{1/4}\left[\sqrt{\frac{m\omega}{\hbar}}xe^{-m\omega x^2/2\hbar}+\sqrt{\frac{m\omega}{\hbar}}xe^{-m\omega x^2/2\hbar}\right]                           \\
                       & =\left(\frac{m\omega}{\pi\hbar}\right)^{1/4}\sqrt{\frac{2m\omega}{\hbar}}xe^{-m\omega x^2/2\hbar}.
  \end{align*}
\end{soln}

% PROBLEM 3
\begin{problem}
The state of a system is described by the vector $\left(1/3 , 1/3 , 1/\sqrt{3} , 2/3 , 0, 0, 0, \dots\right)$ in the basis of the
eigenfunctions of the infinite square well.
What is the wavefunction for this system in position representation?
\end{problem}
\begin{soln}
  Recall that the eigenfunction of the infinite square well are of the form
  $$\psi_n(x)=\sqrt{\frac{2}{a}}\sin(\frac{n\pi x}{a}).$$
  The given state vector represents the coefficients of the eigenfunction expansion of a full solution in these eigenfunctions,
  $$\Psi(x)=\sum_nc_n\psi_n(x).$$
  So the full expansion is
  $$\Psi(x)=\sqrt{\frac{2}{a}}\left[\frac{1}{3}\sin(\frac{\pi x}{a})+\frac{1}{3}\sin(\frac{2\pi x}{a})+\frac{1}{\sqrt{3}}\sin(\frac{3\pi x}{a})+\frac{2}{3}\sin(\frac{4\pi x}{a})\right].$$
\end{soln}

% PROBLEM 4
\begin{problem}
Consider the matrix $\dunderline{M}$ corresponding to the operator $\hat{x}^4$ in the basis of eigenstates
of the harmonic oscillator, i.e. $\left\{\ket{E_n}\right\}$. Using
$$\hat{x}=\sqrt{\frac{\hbar}{2m\omega}}\left[\hat{a}_++\hat{a}_-\right]$$
\begin{enumerate}[label=(\alph*)]
  \item Find $M_{54}$.
  \item Find $M_{53}$.
  \item Find $M_{n+2,n}$.
\end{enumerate}
\end{problem}
\begin{soln}
  \begin{enumerate}[label=(\alph*)]
    \item To obtain elements of the form $n+1,n$ we would need an odd power of $\hat{x}$. With an even power we can only obtain states corresponding
          to even energy shifts. Hence $M_{54}=0$.
    \item Writing $\hat{x}^4$ as a matrix we obtain
          $i,j$ elements
          \begin{align*}
            M_ij & =\bra{E_i}\hat{x}^4\ket{E_j}                                                                                                                                                                                                            \\
                 & =\left(\frac{\hbar}{2m\omega}\right)^2\bra{E_i}\left[\hat{a}_++\hat{a}_-\right]^4\ket{E_j}                                                                                                                                              \\
                 & =\left(\frac{\hbar}{2m\omega}\right)^2\bra{E_i}
            \left[\hat{a}_+^2\left[\hat{a}_+^2+\hat{a}_+\hat{a}_-+\hat{a}_-\hat{a}_++\hat{a}_-^2\right]\right.                                                                                                                                             \\
                 & \qquad+\hat{a}_+\hat{a}_-\left[\hat{a}_+^2+\hat{a}_+\hat{a}_-+\hat{a}_-\hat{a}_++\hat{a}_-^2\right]                                                                                                                                     \\
                 & \qquad+\hat{a}_-\hat{a}_+\left[\hat{a}_+^2+\hat{a}_+\hat{a}_-+\hat{a}_-\hat{a}_++\hat{a}_-^2\right]                                                                                                                                     \\
                 & \qquad+\left.\hat{a}_-^2\left[\hat{a}_+^2+\hat{a}_+\hat{a}_-+\hat{a}_-\hat{a}_++\hat{a}_-^2\right]
            \right]\ket{E_j}                                                                                                                                                                                                                               \\
                 & =\left(\frac{\hbar}{2m\omega}\right)^2\bra{E_i}
            \left[
            \left[\hat{a}_+^4+\hat{a}_+^2\hat{a}_+\hat{a}_-+\hat{a}_+^2\hat{a}_-\hat{a}_++\hat{a}_+^2\hat{a}_-^2\right]\right.                                                                                                                             \\
                 & \qquad+\left[\hat{a}_+\hat{a}_-\hat{a}_+^2+\hat{a}_+\hat{a}_-\hat{a}_+\hat{a}_-+\hat{a}_+\hat{a}_-\hat{a}_-\hat{a}_++\hat{a}_+\hat{a}_-\hat{a}_-^2\right]                                                                               \\
                 & \qquad+\left[\hat{a}_-\hat{a}_+\hat{a}_+^2+\hat{a}_-\hat{a}_+\hat{a}_+\hat{a}_-+\hat{a}_-\hat{a}_+\hat{a}_-\hat{a}_++\hat{a}_-\hat{a}_+\hat{a}_-^2\right]                                                                               \\
                 & \qquad+\left.\left[\hat{a}_-^2\hat{a}_+^2+\hat{a}_-^2\hat{a}_+\hat{a}_-+\hat{a}_-^2\hat{a}_-\hat{a}_++\hat{a}_-^2\hat{a}_-^2\right]
            \right]\ket{E_j}                                                                                                                                                                                                                               \\
                 & =\left(\frac{\hbar}{2m\omega}\right)^2
            \left[
            \left[\bra{E_i}\hat{a}_+^4\ket{E_j} + \bra{E_i}\hat{a}_+^2\hat{a}_+\hat{a}_-\ket{E_j} + \bra{E_i}\hat{a}_+^2\hat{a}_-\hat{a}_+\ket{E_j} + \bra{E_i}\hat{a}_+^2\hat{a}_-^2\ket{E_j}\right]\right.                                               \\
                 & \qquad+\left[\bra{E_i}\hat{a}_+\hat{a}_-\hat{a}_+^2\ket{E_j} + \bra{E_i}\hat{a}_+\hat{a}_-\hat{a}_+\hat{a}_-\ket{E_j} + \bra{E_i}\hat{a}_+\hat{a}_-\hat{a}_-\hat{a}_+\ket{E_j} + \bra{E_i}\hat{a}_+\hat{a}_-\hat{a}_-^2\ket{E_j}\right] \\
                 & \qquad+\left[\bra{E_i}\hat{a}_-\hat{a}_+\hat{a}_+^2\ket{E_j} + \bra{E_i}\hat{a}_-\hat{a}_+\hat{a}_+\hat{a}_-\ket{E_j} + \bra{E_i}\hat{a}_-\hat{a}_+\hat{a}_-\hat{a}_+\ket{E_j} + \bra{E_i}\hat{a}_-\hat{a}_+\hat{a}_-^2\ket{E_j}\right] \\
                 & \qquad+\left.\left[\bra{E_i}\hat{a}_-^2\hat{a}_+^2\ket{E_j} + \bra{E_i}\hat{a}_-^2\hat{a}_+\hat{a}_-\ket{E_j} + \bra{E_i}\hat{a}_-^2\hat{a}_-\hat{a}_+\ket{E_j} + \bra{E_i}\hat{a}_-^2\hat{a}_-^2\ket{E_j}\right]
              \right].
          \end{align*}
          We can then skip ahead a bit to part (c) as it asks for a general formula for the $n+2,n$ elements to which $M_{53}$ belongs.
          Here $n=3$ and so
          $$M_{53}=\left(\frac{\hbar}{2m\omega}\right)^2(4\cdot 3-2)\sqrt{3}\sqrt{3-1}=\left(\frac{\hbar}{2m\omega}\right)^210\sqrt{6}.$$
    \item $M_{n+2,n}$ will be the result of all the terms in
          \begin{align*}
             & \left(\frac{\hbar}{2m\omega}\right)^2
            \left[
            \left[\bra{E_n}\hat{a}_+^4\ket{E_n} + \bra{E_n}\hat{a}_+^2\hat{a}_+\hat{a}_-\ket{E_n} + \bra{E_n}\hat{a}_+^2\hat{a}_-\hat{a}_+\ket{E_n} + \bra{E_n}\hat{a}_+^2\hat{a}_-^2\ket{E_n}\right]\right.                                           \\
             & \qquad+\left[\bra{E_n}\hat{a}_+\hat{a}_-\hat{a}_+^2\ket{E_n} + \bra{E_n}\hat{a}_+\hat{a}_-\hat{a}_+\hat{a}_-\ket{E_n} + \bra{E_n}\hat{a}_+\hat{a}_-\hat{a}_-\hat{a}_+\ket{E_n} + \bra{E_n}\hat{a}_+\hat{a}_-\hat{a}_-^2\ket{E_n}\right] \\
             & \qquad+\left[\bra{E_n}\hat{a}_-\hat{a}_+\hat{a}_+^2\ket{E_n} + \bra{E_n}\hat{a}_-\hat{a}_+\hat{a}_+\hat{a}_-\ket{E_n} + \bra{E_n}\hat{a}_-\hat{a}_+\hat{a}_-\hat{a}_+\ket{E_n} + \bra{E_n}\hat{a}_-\hat{a}_+\hat{a}_-^2\ket{E_n}\right] \\
             & \qquad+\left.\left[\bra{E_n}\hat{a}_-^2\hat{a}_+^2\ket{E_n} + \bra{E_n}\hat{a}_-^2\hat{a}_+\hat{a}_-\ket{E_n} + \bra{E_n}\hat{a}_-^2\hat{a}_-\hat{a}_+\ket{E_n} + \bra{E_n}\hat{a}_-^2\hat{a}_-^2\ket{E_n}\right]
              \right]
          \end{align*}
          which yield a net $-2$ change in energy (because the operators apply to the left). All others will result in inner products which evaluate to zero.
          Removing all the terms we know will be zero then we obtain
          \begin{align*}
             & \left(\frac{\hbar}{2m\omega}\right)^2
            \left[\bra{E_n}\hat{a}_+\hat{a}_-\hat{a}_-^2\ket{E_n} + bra{E_n}\hat{a}_-\hat{a}_+\hat{a}_-^2\ket{E_n} + \bra{E_n}\hat{a}_-^2\hat{a}_+\hat{a}_-\ket{E_n} + \bra{E_n}\hat{a}_-^2\hat{a}_-\hat{a}_+\ket{E_n}
              \right].
          \end{align*}
          Now we simplify using the relations
          $$\hat{a}_+\ket{E_n}=\sqrt{n+1}\ket{E_{n+1}};\qquad\hat{a}_-\ket{E_n}=\sqrt{n}\ket{E_{n-1}},$$
          \begin{align*}
            M_{n+2,n} & =\left(\frac{\hbar}{2m\omega}\right)^2
            \left[\bra{E_n}\hat{a}_+\hat{a}_-\hat{a}_-^2\ket{E_n} + bra{E_n}\hat{a}_-\hat{a}_+\hat{a}_-^2\ket{E_n} + \bra{E_n}\hat{a}_-^2\hat{a}_+\hat{a}_-\ket{E_n} + \bra{E_n}\hat{a}_-^2\hat{a}_-\hat{a}_+\ket{E_n}
            \right]                                                                                \\
                      & =\left(\frac{\hbar}{2m\omega}\right)^2
            \left[\sqrt{n}\sqrt{n-1}(n-2)\bra{E_n}\ket{E_{n-2}}\right.                             \\
                      & \qquad+\sqrt{n}\sqrt{n-1}(n-1)\bra{E_n}\ket{E_{n-2}}                       \\
                      & \qquad+ n\sqrt{n}\sqrt{n-1}\bra{E_n}\ket{E_{n-2}}                          \\
                      & \qquad\left.+ \sqrt{n+1}\sqrt{n+1}\sqrt{n}\sqrt{n-1}\bra{E_n}\ket{E_{n-2}}
            \right]                                                                                \\
                      & =\left(\frac{\hbar}{2m\omega}\right)^2(4n-2)\sqrt{n}\sqrt{n-1}
          \end{align*}


          % \begin{align*}
          %   &\left(\frac{\hbar}{2m\omega}\right)^2
          %         \left[\bra{E_n}\hat{a}_+^2\hat{a}_+\hat{a}_-\ket{E_n} + \bra{E_n}\hat{a}_+^2\hat{a}_-\hat{a}_+\ket{E_n}+\bra{E_n}\hat{a}_+\hat{a}_-\hat{a}_+^2\ket{E_n}+\bra{E_n}\hat{a}_-\hat{a}_+\hat{a}_+^2\ket{E_n}\right]                                              
          % \end{align*}
          % Now we simplify using the relations
          % $$\hat{a}_+\ket{E_n}=\sqrt{n+1}\ket{E_{n+1}};\qquad\hat{a}_-\ket{E_n}=\sqrt{n}\ket{E_{n-1}},$$
          % \begin{align*}
          %   M_{n,n+2}&=\left(\frac{\hbar}{2m\omega}\right)^2
          %         \left[\bra{E_n}\hat{a}_+^2\hat{a}_+\hat{a}_-\ket{E_n} + \bra{E_n}\hat{a}_+^2\hat{a}_-\hat{a}_+\ket{E_n}+\bra{E_n}\hat{a}_+\hat{a}_-\hat{a}_+^2\ket{E_n}+\bra{E_n}\hat{a}_-\hat{a}_+\hat{a}_+^2\ket{E_n}\right]  \\
          %         &= \left(\frac{\hbar}{2m\omega}\right)^2
          %         \left[n\sqrt{n+1}\sqrt{n+2}\bra{E_n}\ket{E_{n+2}}\right.\\
          %         &\qquad+(n+1)\sqrt{n+1}\sqrt{n+2}\bra{E_n}\ket{E_{n+2}}\\
          %         &\qquad+\sqrt{n+1}(n+2)\sqrt{n+2}\bra{E_n}\ket{E_{n+2}}\\
          %         &\qquad\left.+\sqrt{n+1}\sqrt{n+2}(n+3)\bra{E_n}\ket{E_{n+2}}\right]                 \\
          %         &= \left(\frac{\hbar}{2m\omega}\right)^2
          %         \left[(4n+6)\sqrt{n+1}\sqrt{n+2}\right]                            
          % \end{align*}
  \end{enumerate}
\end{soln}
\end{document}