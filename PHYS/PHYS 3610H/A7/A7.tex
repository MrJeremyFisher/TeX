\documentclass[10pt]{article}

\usepackage[margin=0.75in]{geometry}
\usepackage{amsmath,amsthm,amssymb}
\usepackage{xcolor}
\usepackage{cancel}
\usepackage{graphicx}
\usepackage{changepage}
\usepackage{circuitikz}
\usepackage{pgfplots}
\usepackage{physics}
\usepackage{hyperref}
\usepackage{siunitx}
\usepackage{fontspec}
\usepackage{relsize}
\usepackage{subfig}
\usepackage{todonotes}
\usepackage{multicol, multirow, booktabs}
\usepackage[breakable]{tcolorbox}
\usepackage[inline]{enumitem}

\theoremstyle{definition}
\newtheorem{problem}{Problem}
\newtheorem{soln}{Solution}

\pgfplotsset{compat=newest}
\usetikzlibrary{lindenmayersystems}
\usetikzlibrary{arrows}
\usetikzlibrary{calc}
\usetikzlibrary{positioning, fit}
\usetikzlibrary{3d, perspective}

\definecolor{incolor}{HTML}{303F9F}
\definecolor{outcolor}{HTML}{D84315}
\definecolor{cellborder}{HTML}{CFCFCF}
\definecolor{cellbackground}{HTML}{F7F7F7}
\newcommand{\ui}{\hat{i}}
\newcommand{\uj}{\hat{j}}
\newcommand{\uk}{\hat{k}}
\newcommand{\ux}{\hat{x}}
\newcommand{\uy}{\hat{y}}
\newcommand{\uz}{\hat{z}}
\newcommand{\pr}[1]{#1^\prime}
\pgfdeclarelayer{background}  
\pgfsetlayers{background,main}
\AtBeginDocument{\RenewCommandCopy\qty\SI}

\makeatletter
\newcommand{\boxspacing}{\kern\kvtcb@left@rule\kern\kvtcb@boxsep}
\makeatother
\newcommand{\prompt}[4]{
    \ttfamily\llap{{\color{#2}[#3]:\hspace{3pt}#4}}\vspace{-\baselineskip}
}

\newcommand{\thevenin}[2]{
  \begin{center}
    \begin{circuitikz} \draw
      (0,0) -- (2,0) to[battery1, l_=$V_{Th}\eq#1$] (2,2) 
      to[resistor, l_=$R_{Th}\eq#2$] (0,2)
      ;
      \draw [o-] (-.07,2.079);
      \draw [o-] (-.07,0.079);
    \end{circuitikz}
  \end{center}
}

\newcommand{\norton}[2]{
  \begin{center}
    \begin{circuitikz} \draw
      (0,0) -- (3,0) to[american current source, l_=$I_{N}\eq#1$] (3,2) -- (0,2) (2,0)
      to[resistor, l=$R_{N}\eq#2$] (2,2)
      ;
      \draw [o-] (-.07,2.079);
      \draw [o-] (-.07,0.079);
    \end{circuitikz}
  \end{center}
}

\newcommand{\highlight}[1]{\colorbox{yellow}{$\displaystyle #1$}}

\newcommand{\ti}[1]{\widetilde{#1}}

\newfontface{\Kaufmann}{Kaufmann}
\DeclareTextFontCommand{\kf}{\Kaufmann}
\newcommand{\scriptr}{\fontsize{12pt}{12pt}\kf{r}}

\newfontface{\KaufmannB}{Kaufmann Bd BT}
\DeclareTextFontCommand{\kfb}{\KaufmannB}
\newcommand{\bscriptr}{\fontsize{12pt}{12pt}\kfb{r}}

\newcommand{\bv}[1]{\mathbf{#1}}

\title{Physics 3610H: Assignment VII}
\author{Jeremy Favro (0805980) \\ Trent University, Peterborough, ON, Canada}
\date{\today}

\begin{document}
\maketitle

% PROBLEM 1
\begin{problem}
In class we showed that the eigenfunctions of the harmonic oscillator Hamiltonian have
the form $\psi(\xi) = H(\xi)e^{i\xi^2/2}$ where $\xi = \alpha x$. Considering just the even solutions, we expanded
$H(\xi)$ in a power series
$$H(\xi)=\sum_{k=0}^\infty c_k\xi^{2k},$$
and we derived a recursion relation for the coefficients of this series
$$c_{k+1}=\frac{4k+1-\lambda}{2\left(k+1\right)\left(2k+1\right)}c_k.$$
Use this recursion relation to calculate $H_6(\xi)$. Be sure to scale your result, following con-
vention, such that the coefficient of $\xi^6$ is $2^6$.
\end{problem}
\begin{soln}

\end{soln}

% PROBLEM 2
\begin{problem}
Now consider the odd series
$$H(\xi)=\sum_{k=0}^\infty d_k\xi^{2k+1}.$$
Using the equation for $H(\xi)$ which we derived in class, derive the recursion relation for the
coefficients in this series. Explain your steps. You should obtain
$$d_{k+1}=\frac{4k+3-\lambda}{2\left(k+1\right)\left(2k+3\right)}d_k.$$
\end{problem}
\begin{soln}

\end{soln}

% PROBLEM 3
\begin{problem}
Using Dirac notation:
\begin{enumerate}[label=(\alph*)]
  \item If $\hat{A}$ and $\hat{B}$ are two linear operators show that $\left(\hat{A}\hat{B}\right)^\dagger=\hat{B}^\dagger\hat{A}^\dagger$.
  \item If $\hat{C}$ and $\hat{D}$ are two Hermitian operators, is the product $\hat{C}\hat{D}$ always Hermitian? (You
        may use your result from (a)).
\end{enumerate}
\end{problem}
\begin{soln}~
\begin{enumerate}[label=(\alph*)]
  \item We know that for states $\psi$ and $\chi$ $\bra{\psi}\hat{A}\ket{\chi}=\braket{\hat{A}^\dagger\psi}{\chi}$. Applying this then to the
  product we obtain
  $$\bra{\psi}\hat{A}\hat{B}\ket{\chi}=\bra{\psi}\hat{A}\ket{\pr{\chi}}$$
  where $\ket{\pr{\chi}}=\hat{B}\ket{\chi}$. Then,
  $$\bra{\psi}\hat{A}\ket{\pr{\chi}}=\braket{\hat{A}^\dagger\psi}{\pr{\chi}}=\bra{\pr{\psi}}\hat{B}\ket{\chi}=\hat{B}^\dagger\bra{\pr{\psi}}\ket{\chi}.$$
  Here we've written $\bra{\hat{A}^\dagger\psi}=\bra{\pr{\psi}}$
  undoing this we obtain
  $$\bra{\psi}\hat{A}\hat{B}\ket{\chi}=\hat{B}^\dagger\bra{\pr{\psi}}\ket{\chi}=\hat{B}^\dagger\hat{A}^\dagger\bra{\psi}\ket{\chi}.$$
  Because $\ket{\psi}$ and $\ket{\chi}$ are generic we can say then that $\left(\hat{A}\hat{B}\right)^\dagger=\hat{B}^\dagger\hat{A}^\dagger$.
  \item As we know from part (a) linear operators in general satisfy $\left(\hat{C}\hat{D}\right)^\dagger=\hat{D}^\dagger\hat{C}^\dagger$.
  If this product were Hermitian we could say that $\left(\hat{C}\hat{D}\right)^\dagger=\hat{C}\hat{D}=\hat{C}^\dagger\hat{D}^\dagger$
  which we know is not the case again from part (a). Therefore the product of two Hermitian operators is not necessarily Hermitian unless the operators commute.
\end{enumerate}
\end{soln}

\end{document}