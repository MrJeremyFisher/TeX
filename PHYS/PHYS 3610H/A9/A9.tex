\documentclass[10pt]{article}

\usepackage[margin=0.75in]{geometry}
\usepackage{amsmath,amsthm,amssymb}
\usepackage{xcolor}
\usepackage{cancel}
\usepackage{graphicx}
\usepackage{changepage}
\usepackage{circuitikz}
\usepackage{pgfplots}
\usepackage{physics}
\usepackage{hyperref}
\usepackage{siunitx}
\usepackage{fontspec}
\usepackage{relsize}
\usepackage{subfig}
\usepackage{todonotes}
\usepackage{multicol, multirow, booktabs}
\usepackage[breakable]{tcolorbox}
\usepackage[inline]{enumitem}

\theoremstyle{definition}
\newtheorem{problem}{Problem}
\newtheorem{soln}{Solution}

\pgfplotsset{compat=newest}
\usetikzlibrary{lindenmayersystems}
\usetikzlibrary{arrows}
\usetikzlibrary{calc}
\usetikzlibrary{positioning, fit}
\usetikzlibrary{3d, perspective}

\definecolor{incolor}{HTML}{303F9F}
\definecolor{outcolor}{HTML}{D84315}
\definecolor{cellborder}{HTML}{CFCFCF}
\definecolor{cellbackground}{HTML}{F7F7F7}
\newcommand{\ui}{\hat{i}}
\newcommand{\uj}{\hat{j}}
\newcommand{\uk}{\hat{k}}
\newcommand{\ux}{\hat{x}}
\newcommand{\uy}{\hat{y}}
\newcommand{\uz}{\hat{z}}
\newcommand{\uv}[1]{\hat{\mathbf{{#1}}}}
\newcommand{\pr}[1]{{#1^\prime}}
\newcommand{\justif}[2]{&{#1}&\text{#2}}
\pgfdeclarelayer{background}  
\pgfsetlayers{background,main}
\AtBeginDocument{\RenewCommandCopy\qty\SI}

\makeatletter
\newcommand{\boxspacing}{\kern\kvtcb@left@rule\kern\kvtcb@boxsep}
\makeatother
\newcommand{\prompt}[4]{
    \ttfamily\llap{{\color{#2}[#3]:\hspace{3pt}#4}}\vspace{-\baselineskip}
}

\newcommand{\thevenin}[2]{
  \begin{center}
    \begin{circuitikz} \draw
      (0,0) -- (2,0) to[battery1, l_=$V_{Th}\eq#1$] (2,2) 
      to[resistor, l_=$R_{Th}\eq#2$] (0,2)
      ;
      \draw [o-] (-.07,2.079);
      \draw [o-] (-.07,0.079);
    \end{circuitikz}
  \end{center}
}

\newcommand{\norton}[2]{
  \begin{center}
    \begin{circuitikz} \draw
      (0,0) -- (3,0) to[american current source, l_=$I_{N}\eq#1$] (3,2) -- (0,2) (2,0)
      to[resistor, l=$R_{N}\eq#2$] (2,2)
      ;
      \draw [o-] (-.07,2.079);
      \draw [o-] (-.07,0.079);
    \end{circuitikz}
  \end{center}
}

\newcommand{\highlight}[1]{\colorbox{yellow}{$\displaystyle #1$}}

\newcommand{\ti}[1]{\widetilde{#1}}

\newfontface{\Kaufmann}{Kaufmann}
\DeclareTextFontCommand{\kf}{\Kaufmann}
\newcommand{\scriptr}{\fontsize{12pt}{12pt}\kf{r}}

\newfontface{\KaufmannB}{Kaufmann Bd BT}
\DeclareTextFontCommand{\kfb}{\KaufmannB}
\newcommand{\bscriptr}{\fontsize{12pt}{12pt}\kfb{r}}

\newcommand{\bv}[1]{\mathbf{#1}}

\title{Physics 3610H: Assignment IX}
\author{Jeremy Favro (0805980) \\ Trent University, Peterborough, ON, Canada}
\date{\today}

\begin{document}
\maketitle

% PROBLEM 1
\begin{problem}
Prove that $[\hat{A},\hat{B}\hat{C}]=[\hat{A},\hat{B}]\hat{C}+\hat{B}[\hat{A},\hat{C}]$
\end{problem}
\begin{soln}
  Recall the definition of the commutator,
  $$[\hat{A},\hat{B}]=\hat{A}\hat{B}-\hat{B}\hat{A}.$$
  Applying this then to $[\hat{A},\hat{B}\hat{C}]$,
  \begin{align*}
    [\hat{A},\hat{B}\hat{C}]&=\hat{A}(\hat{B}\hat{C})-(\hat{B}\hat{C})\hat{A}\\
    &=\hat{A}\hat{B}\hat{C}-\hat{B}\hat{C}\hat{A}-\hat{B}\hat{A}\hat{C}+\hat{B}\hat{A}\hat{C}\\
    &=\hat{A}\hat{B}\hat{C}-\hat{B}\hat{A}\hat{C}+\hat{B}\hat{A}\hat{C}-\hat{B}\hat{C}\hat{A}\\
    &=(\hat{A}\hat{B}-\hat{B}\hat{A})\hat{C}+\hat{B}(\hat{A}\hat{C}-\hat{C}\hat{A})\\
    &=[\hat{A},\hat{B}]\hat{C}+\hat{B}[\hat{A},\hat{C}]\\
  \end{align*}
\end{soln}

% PROBLEM 2
\begin{problem}
For the (normalized) wavefunction
$$\psi(x)=\left(\frac{2\alpha}{\pi}\right)^{1/4}e^{-\alpha x^2},$$
what is $\Delta x \Delta p_x$?
\end{problem}
\begin{soln}
  Generally for an operator $\hat{A}$
  $$\Delta A = \sqrt{\langle \hat{A}^2\rangle - \langle\hat{A}\rangle^2}.$$
  So in order to determine the product $\Delta x \Delta p_x$ we need
  $$\langle \hat{x}^2 \rangle;\quad \langle \hat{x} \rangle^2;\quad \langle \hat{p}_x^2 \rangle;\quad \langle \hat{p}_x \rangle^2.$$
  Working these out then,
  \begin{align*}
    \langle \hat{x}^2 \rangle & =\left(\frac{2\alpha}{\pi}\right)^{1/2} \int_{\mathbb{R}}e^{-\alpha x^2}x^2e^{-\alpha x^2}\,dx \\
                              & =\left(\frac{2\alpha}{\pi}\right)^{1/2} \int_{\mathbb{R}}x^2e^{-2\alpha x^2}\,dx               \\
                              & =\frac{1}{4\alpha}
  \end{align*}
  where the big step is provided by equation 43 in \href{https://mathworld.wolfram.com/GaussianIntegral.html}{https://mathworld.wolfram.com/GaussianIntegral.html}.
  \begin{align*}
    \langle \hat{x} \rangle^2 & =\frac{2\alpha}{\pi} \left(\int_{\mathbb{R}}e^{-\alpha x^2}xe^{-\alpha x^2}\,dx\right)^2 \\
                              & =\frac{2\alpha}{\pi}\left(\int_{\mathbb{R}}xe^{-2\alpha x^2}\,dx\right)^2                \\
                              & =0
  \end{align*}
  where again the big step comes from Wolfram.
  \begin{align*}
    \langle \hat{p}_x^2 \rangle & =\left(\frac{2\alpha}{\pi}\right)^{1/2} \int_{\mathbb{R}}e^{-\alpha x^2}(-\hbar^2)\frac{\partial^2}{\partial x^2}e^{-\alpha x^2}\,dx \\
                                & =(-\hbar^2)\left(\frac{2\alpha}{\pi}\right)^{1/2} \int_{\mathbb{R}}e^{-2\alpha x^2}\left(4{\alpha}^{2} x^{2} - 2{\alpha}\right)\,dx  \\
                                & =(-\hbar^2)\left(\frac{2\alpha}{\pi}\right)^{1/2} \sqrt{\frac{\pi \alpha}{2}}-\sqrt{2\pi \alpha}                                     \\
                                & =(-\hbar^2)\left(\frac{2\alpha}{\pi}\right)^{1/2} \frac{-\sqrt{\pi \alpha}}{\sqrt{2}}                                                \\
                                & =\hbar^2\alpha
  \end{align*}
  and
  \begin{align*}
    \langle \hat{p}_x \rangle^2 & =\left(\frac{2\alpha}{\pi}\right)^{1/2} \left(\int_{\mathbb{R}}e^{-\alpha x^2}i\hbar\frac{\partial^2}{\partial x^2}e^{-\alpha x^2}\,dx\right)^2 \\
                                & =-2{\alpha}\hbar\left(\frac{2\alpha}{\pi}\right)^{1/2}i \left(\int_{\mathbb{R}}xe^{-2{\alpha}x^{2}}\,dx\right)^2                                \\
                                & =0.
  \end{align*}
  So,
  $$\Delta x = \sqrt{\left(\frac{1}{4\alpha}\right)-0}=\frac{1}{2\sqrt{\alpha}};\quad \Delta p_x = \sqrt{\left(\hbar^2\alpha\right)-0}=\hbar\sqrt{\alpha}$$
  and so
  $$\Delta x \Delta p_x =\frac{1}{2\sqrt{\alpha}}\cdot\hbar\sqrt{\alpha}=\hbar/2.$$
\end{soln}

% PROBLEM 3
\begin{problem}
Using $\left[\hat{a}_-,\hat{a}_+\right]=1$ and $\hat{H}=\hbar\omega\left(\hat{a}_+\hat{a}_-+1/2\right)$,
show that $\left[\hat{H},\hat{a}_+\right]=+\hbar\omega \hat{a}_+$ and
$\left[\hat{H},a_-\right]=-\hbar\omega \hat{a}_-$.
\end{problem}
\begin{soln}
  \begin{align*}
    \left[\hat{H},\hat{a}_+\right] & = \left[\hbar\omega\left(\hat{a}_+\hat{a}_-+1/2\right), \hat{a}_+\right]\justif{\quad}{Definition of $\hat{H}$}                         \\
                                   & = \hbar\omega\left[\hat{a}_+\hat{a}_-+1/2, \hat{a}_+\right]\justif{\quad}{$[C\hat{A},\hat{B}]=C[\hat{A},\hat{B}]$}                      \\
                                   & = -\hbar\omega\left[\hat{a}_+,\hat{a}_+\hat{a}_-+1/2\right]\justif{\quad}{$[\hat{A},\hat{B}]=-[\hat{B},\hat{A}]$}                       \\
                                   & = -\hbar\omega(\left[\hat{a}_+,\hat{a}_+\hat{a}_-\right]+\cancelto{0}{\left[\hat{a}_+,1/2\right]})                                      \\
                                   & = -\hbar\omega(\left[\hat{a}_+,\hat{a}_+\right]\hat{a}_-+\hat{a}_+\left[\hat{a}_+,\hat{a}_-\right])\justif{\quad}{Proved in question 1} \\
                                   & =\hbar\omega \hat{a}_+
  \end{align*}
  and
  \begin{align*}
    \left[\hat{H},\hat{a}_-\right] & = \left[\hbar\omega\left(\hat{a}_+\hat{a}_-+1/2\right), \hat{a}_-\right]                                       \\
                                   & = \hbar\omega\left[\hat{a}_+\hat{a}_-+1/2, \hat{a}_-\right]                                                    \\
                                   & = -\hbar\omega\left[\hat{a}_-,\hat{a}_+\hat{a}_-+1/2\right]                                                    \\
                                   & = -\hbar\omega\left(\left[\hat{a}_-,\hat{a}_+\hat{a}_-\right]+\cancelto{0}{\left[\hat{a}_-,1/2\right]}\right)  \\
                                   & = -\hbar\omega\left(\left[\hat{a}_-,\hat{a}_+\hat{a}_-\right]\right)                                           \\
                                   & = -\hbar\omega\left(\left[\hat{a}_-,\hat{a}_+\right]\hat{a}_-+\hat{a}_+\left[\hat{a}_-,\hat{a}_-\right]\right) \\
                                   & = -\hbar\omega\left(\left[\hat{a}_-,\hat{a}_+\right]\hat{a}_-\right)                                           \\
                                   & =-\hbar\omega \hat{a}_-
  \end{align*}
\end{soln}
\end{document}