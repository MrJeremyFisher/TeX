\documentclass[10pt]{article}

\usepackage[margin=0.75in]{geometry}
\usepackage{amsmath,amsthm,amssymb}
\usepackage{xcolor}
\usepackage{cancel}
\usepackage{graphicx}
\usepackage{changepage}
\usepackage{quantikz}
\usepackage{tikz}
\usepackage{circuitikz}
\usepackage{pgfplots}
\usepackage{physics}
\usepackage{hyperref}
\usepackage{siunitx}
\usepackage[breakable]{tcolorbox}
\usepackage[inline]{enumitem}

\theoremstyle{definition}
\newtheorem{problem}{Problem}
\newtheorem{soln}{Solution}

\pgfplotsset{compat=newest}
\usetikzlibrary{arrows, angles, calc, quotes}
\usetikzlibrary{quantikz2}

\definecolor{incolor}{HTML}{303F9F}
\definecolor{outcolor}{HTML}{D84315}
\definecolor{cellborder}{HTML}{CFCFCF}
\definecolor{cellbackground}{HTML}{F7F7F7}
\newcommand{\eq}{=}
\usetikzlibrary{positioning, fit, calc}
\pgfdeclarelayer{background}  
\pgfsetlayers{background,main}
\DeclareSIUnit[number-unit-product = {\,}]\calorie{cal}
\DeclareSIUnit[number-unit-product = {\,}]\atmosphere{atm}
\AtBeginDocument{\RenewCommandCopy\qty\SI}

\makeatletter
\newcommand{\boxspacing}{\kern\kvtcb@left@rule\kern\kvtcb@boxsep}
\makeatother
\newcommand{\prompt}[4]{
    \ttfamily\llap{{\color{#2}[#3]:\hspace{3pt}#4}}\vspace{-\baselineskip}
}

\newcommand{\thevenin}[2]{
  \begin{center}
    \begin{circuitikz} \draw
      (0,0) -- (2,0) to[battery1, l_=$V_{Th}\eq#1$] (2,2) 
      to[resistor, l_=$R_{Th}\eq#2$] (0,2)
      ;
      \draw [o-] (-.07,2.079);
      \draw [o-] (-.07,0.079);
    \end{circuitikz}
  \end{center}
}

\newcommand{\norton}[2]{
  \begin{center}
    \begin{circuitikz} \draw
      (0,0) -- (3,0) to[american current source, l_=$I_{N}\eq#1$] (3,2) -- (0,2) (2,0)
      to[resistor, l=$R_{N}\eq#2$] (2,2)
      ;
      \draw [o-] (-.07,2.079);
      \draw [o-] (-.07,0.079);
    \end{circuitikz}
  \end{center}
}

\newcommand{\highlight}[1]{\colorbox{yellow}{$\displaystyle #1$}}

\newcommand{\ti}[1]{\widetilde{#1}}

\NewDocumentCommand{\evalat}{sO{\big}mm}{%
  \IfBooleanTF{#1}
   {\mleft. #3 \mright|_{#4}}
   {#3#2|_{#4}}%
}

\title{Physics 2605H: Assignment III}
\author{Jeremy Favro \\ Trent University, Peterborough, ON, Canada}
\date{\today}

\begin{document}
\maketitle

% PROBLEM 1
\begin{problem}
Solve the following problems from Chapter 6 (did you mean chapter 5? That was what I assumed.):
\begin{enumerate}[label=(\alph*)]
  \item 5.1
  \item 5.3
  \item 5.4
  \item 5.5
  \item 5.9 A and 5.9 B
  \item 5.10
\end{enumerate}
\end{problem}
\begin{soln}~
  \begin{enumerate}[label=(\alph*)]
    \item \begin{enumerate}[label=(\Alph*)]
            \item Yes, $\braket{\psi}{\psi}=\left(\sqrt{\frac{5}{6}}\bra{0}+\sqrt{\frac{1}{6}}\bra{1}\right)
                    \left(\sqrt{\frac{5}{6}}\ket{0}+\sqrt{\frac{1}{6}}\ket{1}\right)=\frac{5}{6}+\frac{1}{6}=1$.
            \item $\frac{5}{6}$.
            \item $\hat{\rho}=\ket{\psi}\bra{\psi}=\left(\sqrt{\frac{5}{6}}\ket{0}+\sqrt{\frac{1}{6}}\ket{1}\right)
                    \left(\sqrt{\frac{5}{6}}\bra{0}+\sqrt{\frac{1}{6}}\bra{1}\right)$
            \item $\left[\hat{\rho}\right]_{\left\{\ket{0},\ket{1}\right\}}=\frac{1}{6}\begin{pmatrix}
                      5        & \sqrt{5} \\
                      \sqrt{5} & 1
                    \end{pmatrix}\implies \mathrm{Tr}(\hat{\rho})=\frac{5}{6}+\frac{1}{6}=1$
          \end{enumerate}
    \item \begin{enumerate}[label=(\Alph*)]
            \item $\left[\hat{\rho}\right]_{\left\{\ket{0},\ket{1}\right\}}=\ket{\psi}\bra{\psi}=
                    \frac{1}{7}\begin{pmatrix}
                      3         & 2\sqrt{3} \\
                      2\sqrt{3} & 4
                    \end{pmatrix}$
            \item $\left[\hat{\rho}\right]^2=\frac{1}{49}\begin{pmatrix}
                      3         & 2\sqrt{3} \\
                      2\sqrt{3} & 4
                    \end{pmatrix}\begin{pmatrix}
                      3         & 2\sqrt{3} \\
                      2\sqrt{3} & 4
                    \end{pmatrix}=
                    \frac{1}{49}\begin{pmatrix}
                      21        & 14\sqrt{3} \\ 1
                      4\sqrt{3} & 28
                    \end{pmatrix}\implies \mathrm{Tr}(\hat{\rho}^2)=\frac{21}{49}+\frac{28}{49}=1\therefore\mathrm{State is pure}$
            \item Change of basis matrix here is just the Hadamard gate so,\\
                  $\left[\hat{\rho}\right]_{\left\{\ket{+},\ket{-}\right\}}=\frac{1}{14}H\left[\hat{\rho}\right]_{\left\{\ket{0},\ket{1}\right\}}H
                    =\frac{1}{14}\begin{pmatrix}
                      1 & 1  \\
                      1 & -1
                    \end{pmatrix}\begin{pmatrix}
                      3         & 2\sqrt{3} \\
                      2\sqrt{3} & 4
                    \end{pmatrix}\begin{pmatrix}
                      1 & 1  \\
                      1 & -1
                    \end{pmatrix}=\frac{1}{14}\begin{pmatrix}
                      7+4\sqrt{3} & -1          \\
                      -1          & 7-4\sqrt{3}
                    \end{pmatrix}\\\implies \mathrm{Tr}(\left[\hat{\rho}\right]_{\left\{\ket{+},\ket{-}\right\}})=
                    \frac{1}{14}\left[7+4\sqrt{3}+7-4\sqrt{3}\right]=1$.
                  Then,\\
                  $\left[\hat{\rho}^2\right]_{\left\{\ket{+},\ket{-}\right\}}=
                    \frac{1}{196}\begin{pmatrix}
                      7+4\sqrt{3} & -1          \\
                      -1          & 7-4\sqrt{3}
                    \end{pmatrix}\begin{pmatrix}
                      7+4\sqrt{3} & -1          \\
                      -1          & 7-4\sqrt{3}
                    \end{pmatrix}=\frac{1}{196}\begin{pmatrix}
                      98+56\sqrt{3} & -14           \\
                      -14           & 98-56\sqrt{3}
                    \end{pmatrix}\\\implies \mathrm{Tr}(\left[\hat{\rho}^2\right]_{\left\{\ket{+},\ket{-}\right\}})=
                    \frac{1}{196}\left[98+56\sqrt{3}+98-56\sqrt{3}\right]=1$
          \end{enumerate}
    \item \begin{enumerate}[label=(\Alph*)]
            \item $\hat{\rho}=\frac{1}{3}\begin{pmatrix}
                      2        & \sqrt{2} \\
                      \sqrt{2} & 1
                    \end{pmatrix}\implies\hat{\rho}^2=\frac{1}{9}\begin{pmatrix}
                      6         & 3\sqrt{2} \\
                      3\sqrt{2} & 3
                    \end{pmatrix}\implies \mathrm{Tr}(\hat{\rho})=\mathrm{Tr}(\hat{\rho}^2)=1$
            \item $\langle X\rangle=\mathrm{Tr}(\rho X)=\mathrm{Tr}\left(\frac{1}{9}\begin{pmatrix}
                      6         & 3\sqrt{2} \\
                      3\sqrt{2} & 3
                    \end{pmatrix}\begin{pmatrix}
                      0 & 1 \\
                      1 & 0
                    \end{pmatrix}\right)=\mathrm{Tr}\left(\begin{pmatrix}\frac{\sqrt{2}}{3} & \frac{2}{3}        \\
               \frac{1}{3}        & \frac{\sqrt{2}}{3}
                      \end{pmatrix}\right)=\frac{2\sqrt{2}}{3}$
          \end{enumerate}
    \item \begin{enumerate}[label=(\Alph*)]
            \item Yes. The matrix has trace 1 and is Hermitian.
            \item Because the trace of the square of the density operator is less than one this is a mixed state.
          \end{enumerate}
    \item \begin{enumerate}[label=(\Alph*)]
            \item $\hat{\rho}=\ket{\psi}\bra{\psi}=\frac{1}{2}\left(\ket{00}+\ket{11}\right)
                    \left(\bra{00}+\bra{11}\right)$
            \item $\left[\hat{\rho}\right]=\ket{\psi}\bra{\psi}=\frac{1}{2}\begin{pmatrix}
                      1 \\
                      0 \\
                      0 \\
                      1
                    \end{pmatrix}\begin{pmatrix}
                      1 & 0 & 0 & 1
                    \end{pmatrix}=
                    \frac{1}{2}\begin{pmatrix}
                      1 & 0 & 0 & 1 \\
                      0 & 0 & 0 & 0 \\
                      0 & 0 & 0 & 0 \\
                      1 & 0 & 0 & 1
                    \end{pmatrix}\implies\mathrm{Tr}(\hat{\rho})=1$
          \end{enumerate}
    \item \begin{enumerate}[label=(\Alph*)]
            \item $\hat{\rho}^\dagger=\begin{pmatrix}
                      \frac{2}{5} & -\frac{i}{8} \\
                      \frac{i}{8} & \frac{3}{5}  \\
                    \end{pmatrix}^\dagger=\begin{pmatrix}
                      \frac{2}{5} & -\frac{i}{8} \\
                      \frac{i}{8} & \frac{3}{5}  \\
                    \end{pmatrix}=\hat{\rho}$
            \item These are not the eigenvalues. The eigenvalues for $\hat{\rho}$ are $-\frac{20+\sqrt{41}}{40}$ and $\frac{20+\sqrt{41}}{40}$.
            \item It does not as the eigenvalues are not both positive.
            \item $P(\ket{0})=\mathrm{Tr}(\bra{0}\rho\ket{0})=\mathrm{Tr}\left(\begin{pmatrix}
                        1 &
                        0
                      \end{pmatrix}\begin{pmatrix}
                        \frac{2}{5} & -\frac{i}{8} \\
                        \frac{i}{8} & \frac{3}{5}
                      \end{pmatrix}
                    \begin{pmatrix}
                        1 \\
                        0
                      \end{pmatrix}\right)=
                    \mathrm{Tr}\left(\begin{pmatrix}
                        \frac{2}{5} \\
                        \frac{i}{8}
                      \end{pmatrix}\right)=\frac{2}{5}\neq 0.66$. However this is meaningless as the matrix is not a valid density matrix.
            \item Because this matrix is not a valid density matrix it is, again, meaningless to calculate the bloch vector representation.
          \end{enumerate}
  \end{enumerate}
\end{soln}

% PROBLEM 2
\begin{problem}
Follow the procedure discussed in the class to obtain the transformation matrix for following gates:
\begin{enumerate}[label=(\roman*)]
  \item X
  \item Y
  \item Z
  \item Hadamard
\end{enumerate}
\end{problem}
\begin{soln}~
  \begin{enumerate}[label=(\roman*)]
    \item $\left[\hat{X}\right]=\begin{pmatrix}
              \bra{0}\hat{X}\ket{0} & \bra{0}\hat{X}\ket{1} \\
              \bra{1}\hat{X}\ket{0} & \bra{1}\hat{X}\ket{1} \\
            \end{pmatrix}=\begin{pmatrix}
              \bra{0}\ket{1} & \bra{0}\ket{0} \\
              \bra{1}\ket{1} & \bra{1}\ket{0} \\
            \end{pmatrix}=\begin{pmatrix}
              0 & 1 \\
              1 & 0 \\
            \end{pmatrix}$
    \item $\left[\hat{Y}\right]=\begin{pmatrix}
              \bra{0}\hat{Y}\ket{0} & \bra{0}\hat{Y}\ket{1} \\
              \bra{1}\hat{Y}\ket{0} & \bra{1}\hat{Y}\ket{1} \\
            \end{pmatrix}=\begin{pmatrix}
              -i\bra{0}\ket{1} & i\bra{0}\ket{0} \\
              -i\bra{1}\ket{1} & i\bra{1}\ket{0} \\
            \end{pmatrix}=\begin{pmatrix}
              0  & i \\
              -i & 0 \\
            \end{pmatrix}$
    \item $\left[\hat{Z}\right]=\begin{pmatrix}
              \bra{0}\hat{Z}\ket{0} & \bra{0}\hat{Z}\ket{1} \\
              \bra{1}\hat{Z}\ket{0} & \bra{1}\hat{Z}\ket{1} \\
            \end{pmatrix}=\begin{pmatrix}
              \bra{0}\ket{1} & -\bra{0}\ket{1} \\
              \bra{1}\ket{1} & -\bra{1}\ket{1} \\
            \end{pmatrix}=\begin{pmatrix}
              1 & 0  \\
              0 & -1 \\
            \end{pmatrix}$
    \item $\left[\hat{H}\right]=\begin{pmatrix}
              \bra{0}\hat{H}\ket{0} & \bra{0}\hat{H}\ket{1} \\
              \bra{1}\hat{H}\ket{0} & \bra{1}\hat{H}\ket{1} \\
            \end{pmatrix}=\begin{pmatrix}
              \bra{0}\ket{+} & \bra{0}\ket{-} \\
              \bra{1}\ket{+} & \bra{1}\ket{-} \\
            \end{pmatrix}=\frac{1}{\sqrt{2}}\begin{pmatrix}
              1 & 1  \\
              1 & -1 \\
            \end{pmatrix}$
  \end{enumerate}
\end{soln}

% PROBLEM 3
\begin{problem}~
\begin{enumerate}[label=(\roman*)]
  \item In the class we learnt about T gate. Discuss the phase shift it introduces in the quantum state
        $\ket{\alpha}$, which is a linear combination of $\ket{0}$ and $\ket{1}$ states.
  \item What is an S gate? Obtain its matrix form from its functionality.
  \item How are S and T gates related to Z gate?
\end{enumerate}
\end{problem}
\begin{soln}~
  \begin{enumerate}[label=(\roman*)]
    \item The T gate introduces a phase shift of $\qty[parse-numbers=false]{\frac{\pi}{4}}{\radian}$ to $\alpha$.
    \item An S gate is, like the T gate, a phase shift gate. It introduces a phase shift of
          $\qty[parse-numbers=false]{\frac{\pi}{2}}{\radian}$. It's matrix form is given by
          $\begin{pmatrix}
              1 & 0 \\
              0 & i \\
            \end{pmatrix}$.
    \item The S, T, and Z gates are all phase shift gates. The Z gate, as introduced in class, simply did not explicitly
          state the phase term. Because the Z gate has a bottom right entry of $-1$ the phase term must be $e^{ni\pi}$ where
          $n$ is some odd integer.
  \end{enumerate}
\end{soln}

% PROBLEM 4
\begin{problem}
In the class we discussed two qubit quantum gate C-NOT gate.
\begin{enumerate}[label=(\roman*)]
  \item Construct the input vector.
  \item Using its truth table compute the transformation matrix.
  \item Applying the transformation matrix from (ii) on the state (i), show that it follows the
        definition of C-NOT gate.
  \item Find the output state for following input states which are passed through the CNOT
        gate:
        \begin{enumerate}[label=(\alph*)]
          \item $\ket{00}$
          \item $\ket{01}$
          \item $\ket{11}$
          \item $\frac{1}{\sqrt{2}}\ket{01}+\frac{1}{\sqrt{2}}\ket{10}$
          \item $\frac{1}{\sqrt{2}}\ket{00}+\frac{1}{2}\ket{10}-\frac{1}{2}\ket{11}$
        \end{enumerate}
\end{enumerate}
\end{problem}
\begin{soln}~
  \begin{enumerate}[label=(\roman*)]
    \item \begin{align*}
            \ket{00}=\ket{0}\otimes \ket{0} & \to \ket{00}, \\
            \ket{01}                        & \to \ket{01}, \\
            \ket{10}                        & \to \ket{11}, \\
            \ket{11}                        & \to \ket{10}  \\
          \end{align*}
    \item \begin{align*}
            \left[\hat{C}\right] & =\begin{pmatrix}
                                      \bra{00}\hat{C}\ket{00} & \bra{00}\hat{C}\ket{01} & \bra{00}\hat{C}\ket{10} & \bra{00}\hat{C}\ket{11} \\
                                      \bra{01}\hat{C}\ket{00} & \bra{01}\hat{C}\ket{01} & \bra{01}\hat{C}\ket{10} & \bra{01}\hat{C}\ket{11} \\
                                      \bra{10}\hat{C}\ket{00} & \bra{10}\hat{C}\ket{01} & \bra{10}\hat{C}\ket{10} & \bra{10}\hat{C}\ket{11} \\
                                      \bra{11}\hat{C}\ket{00} & \bra{11}\hat{C}\ket{01} & \bra{11}\hat{C}\ket{10} & \bra{11}\hat{C}\ket{11} \\
                                    \end{pmatrix}=
            \begin{pmatrix}
              \bra{00}\ket{00} & \bra{00}\ket{01} & \bra{00}\ket{11} & \bra{00}\ket{10} \\
              \bra{01}\ket{00} & \bra{01}\ket{01} & \bra{01}\ket{11} & \bra{01}\ket{10} \\
              \bra{10}\ket{00} & \bra{10}\ket{01} & \bra{10}\ket{11} & \bra{10}\ket{10} \\
              \bra{11}\ket{00} & \bra{11}\ket{01} & \bra{11}\ket{11} & \bra{11}\ket{10} \\
            \end{pmatrix}                                                     \\
                                 & =\begin{pmatrix}
                                      1 & 0 & 0 & 0 \\
                                      0 & 1 & 0 & 0 \\
                                      0 & 0 & 0 & 1 \\
                                      0 & 0 & 1 & 0 \\
                                    \end{pmatrix}
          \end{align*}
    \item I'm not sure what this question is asking. Any input vector used to construct the CNOT gate will obviously follow the definition of the
          CNOT gate when it is applied as it was used to construct the CNOT gate.
    \item Find the output state for following input states which are passed through the CNOT
          gate:
          \begin{enumerate}[label=(\alph*)]
            \item $\hat{C}\ket{00}=\begin{pmatrix}
                      1 & 0 & 0 & 0 \\
                      0 & 1 & 0 & 0 \\
                      0 & 0 & 0 & 1 \\
                      0 & 0 & 1 & 0 \\
                    \end{pmatrix}\begin{pmatrix}
                      1 \\
                      0 \\
                      0 \\
                      0
                    \end{pmatrix}=\begin{pmatrix}
                      1 \\
                      0 \\
                      0 \\
                      0
                    \end{pmatrix}=\ket{00}$
            \item $\hat{C}\ket{01}=\begin{pmatrix}
                      1 & 0 & 0 & 0 \\
                      0 & 1 & 0 & 0 \\
                      0 & 0 & 0 & 1 \\
                      0 & 0 & 1 & 0 \\
                    \end{pmatrix}\begin{pmatrix}
                      0 \\
                      1 \\
                      0 \\
                      0
                    \end{pmatrix}=\begin{pmatrix}
                      0 \\
                      1 \\
                      0 \\
                      0
                    \end{pmatrix}=\ket{01}$
            \item $\hat{C}\ket{11}=\begin{pmatrix}
                      1 & 0 & 0 & 0 \\
                      0 & 1 & 0 & 0 \\
                      0 & 0 & 0 & 1 \\
                      0 & 0 & 1 & 0 \\
                    \end{pmatrix}\begin{pmatrix}
                      0 \\
                      0 \\
                      0 \\
                      1
                    \end{pmatrix}=\begin{pmatrix}
                      0 \\
                      0 \\
                      1 \\
                      0
                    \end{pmatrix}=\ket{10}$
            \item $\hat{C}\frac{1}{\sqrt{2}}\ket{01}+\hat{C}\frac{1}{\sqrt{2}}\ket{10}=
                    \frac{1}{\sqrt{2}}\ket{01}+\frac{1}{\sqrt{2}}\ket{11}$
            \item $\hat{C}\frac{1}{\sqrt{2}}\ket{00}+\hat{C}\frac{1}{2}\ket{10}-\hat{C}\frac{1}{2}\ket{11}=
                    \frac{1}{\sqrt{2}}\ket{00}+\frac{1}{2}\ket{11}-\frac{1}{2}\ket{10}$
          \end{enumerate}
  \end{enumerate}
\end{soln}

% PROBLEM 5
\begin{problem}
Identify the product and entangled states in the list given below.
\begin{enumerate}[label=(\alph*)]
  \item $\frac{1}{\sqrt{2}}\ket{01}+\frac{1}{\sqrt{2}}\ket{10}$
  \item $\frac{1}{\sqrt{2}}\ket{01}-\frac{1}{\sqrt{2}}\ket{10}$
  \item $\frac{\sqrt{3}}{2}\ket{00}+\frac{1}{2}\ket{11}$
  \item $\frac{1}{\sqrt{2}}\ket{01}+\frac{1}{\sqrt{2}}\ket{11}$
  \item $\frac{1}{2}\ket{00}+\frac{1}{2}\ket{01}+\frac{1}{2}\ket{10}-\frac{1}{2}\ket{11}$
  \item $\frac{1}{\sqrt{2}}\ket{00}+\frac{1}{2}\ket{10}-\frac{1}{2}\ket{11}$
\end{enumerate}
\end{problem}
\begin{soln}~
  \begin{enumerate}[label=(\alph*)]
    \item Because the system $\alpha_0\beta_0=0$, $\alpha_0\beta_1=\frac{1}{\sqrt{2}}$,
          $\alpha_1\beta_0=\frac{1}{\sqrt{2}}$, $\alpha_1\beta_1=0$ has no solution the state is entangled
    \item For the same general reason as above this state is entangled.
    \item For the same general reason as above this state is entangled.
    \item Because the system $\alpha_0\beta_0=0$, $\alpha_0\beta_1=\frac{1}{\sqrt{2}}$,
          $\alpha_1\beta_0=0$, $\alpha_1\beta_1=\frac{1}{\sqrt{2}}$ has a solution for $\beta_0=0$,
          $\alpha_0=\alpha_1=\frac{1}{\sqrt{2}}$, and $\beta_1=1$ this is a product state.
    \item Because the system $\alpha_0\beta_0=\frac{1}{2}$, $\alpha_0\beta_1=\frac{1}{2}$,
          $\alpha_1\beta_0=\frac{1}{2}$, $\alpha_1\beta_1=-\frac{1}{2}$ has no solution the state is entangled.
    \item Because the system $\alpha_0\beta_0=\frac{1}{\sqrt{2}}$, $\alpha_0\beta_1=0$,
          $\alpha_1\beta_0=\frac{1}{2}$, $\alpha_1\beta_1=-\frac{1}{2}$ has no solution the state is entangled.
  \end{enumerate}
\end{soln}

% PROBLEM 6
\begin{problem}
Predict the output states of the following quantum circuits
\begin{enumerate}[label=(\alph*)]
  \item \begin{quantikz}
          \lstick{$\ket{0}$}&&\ctrl{1}&&\meter{}&& \\
          \lstick{$\ket{0}$}&\gate{H}&\targ{}&&&\meter{}&
        \end{quantikz}
  \item \begin{quantikz}
          \lstick{$\ket{0}$}&\gate{H}&\ctrl{1}&&\meter{}&& \\
          \lstick{$\ket{0}$}&&\targ{}&&&\meter{}&
        \end{quantikz}
  \item \begin{quantikz}
          \lstick{$\ket{0}$}&&\ctrl{1}&&\meter{}&& \\
          \lstick{$\ket{0}$}&&\targ{}&\gate{H}&&\meter{}&
        \end{quantikz}
  \item \begin{quantikz}
          \lstick{$\ket{0}$}&&\ctrl{1}&\gate{H}&\meter{}&& \\
          \lstick{$\ket{0}$}&&\targ{}&&&\meter{}&
        \end{quantikz}
\end{enumerate}
\end{problem}
\begin{soln}~
  \begin{enumerate}[label=(\alph*)]
    \item \begin{quantikz}
            \lstick{$\ket{0}$}&&\ctrl{1}&&\meter{}&&\rstick{$\ket{0}$} \\
            \lstick{$\ket{0}$}&\gate{H}&\targ{}&&&\meter{}&\rstick{$\ket{0}$ or $\ket{1}$, equal probability (control $\ket{0}$)}
          \end{quantikz}
    \item \begin{quantikz}
            \lstick{$\ket{0}$}&\gate{H}&\ctrl{1}&&\meter{}&&\rstick{$\ket{0}$ or $\ket{1}$, equal probability} \\
            \lstick{$\ket{0}$}&&\targ{}&&&\meter{}&\rstick{$\ket{0}$ or $\ket{1}$, equal probability, dependent on control bit}
          \end{quantikz}
    \item \begin{quantikz}
            \lstick{$\ket{0}$}&&\ctrl{1}&&\meter{}&&\rstick{$\ket{0}$} \\
            \lstick{$\ket{0}$}&&\targ{}&\gate{H}&&\meter{}&\rstick{$\ket{0}$ or $\ket{1}$, equal probability}
          \end{quantikz}
    \item \begin{quantikz}
            \lstick{$\ket{0}$}&&\ctrl{1}&\gate{H}&\meter{}&&\rstick{$\ket{0}$ or $\ket{1}$, equal probability} \\
            \lstick{$\ket{0}$}&&\targ{}&&&\meter{}&\rstick{$\ket{0}$ or $\ket{1}$, equal probability, dependent on control bit}
          \end{quantikz}
  \end{enumerate}
\end{soln}
\end{document}