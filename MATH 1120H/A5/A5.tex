\documentclass[10pt]{article}

\usepackage[margin=0.75in]{geometry}
\usepackage{amsmath,amsthm,amssymb}
\usepackage{xcolor}
\usepackage{cancel}
\usepackage{xcolor}
\usepackage{tikz}
\usepackage{pgfplots}
\usepackage{physics}
\usepackage{minted}
\usepackage{pythonhighlight}
\usepackage{pdfpages}
\usepackage[inline]{enumitem}
\usepackage[breakable]{tcolorbox}

\theoremstyle{definition}
\newtheorem{problem}{Problem}
\newtheorem{soln}{Solution}

\pgfplotsset{compat=newest}
\usetikzlibrary{patterns}

\definecolor{incolor}{HTML}{303F9F}
\definecolor{outcolor}{HTML}{D84315}
\definecolor{cellborder}{HTML}{CFCFCF}
\definecolor{cellbackground}{HTML}{F7F7F7}

\tikzset
{%
  axes/.style={thick,-latex},
  cylinder/.style={right color=blue!80,left color=white,fill opacity=0.7},
  paraboloid back/.style={left color=magenta!80,fill opacity=0.4},
  paraboloid front/.style={left color=white, right color=magenta!80,fill opacity=0.4},
}

\makeatletter
\newcommand{\boxspacing}{\kern\kvtcb@left@rule\kern\kvtcb@boxsep}
\makeatother
\newcommand{\prompt}[4]{
    \ttfamily\llap{{\color{#2}[#3]:\hspace{3pt}#4}}\vspace{-\baselineskip}
}


\NewDocumentCommand{\evalat}{sO{\big}mm}{%
  \IfBooleanTF{#1}
   {\mleft. #3 \mright|_{#4}}
   {#3#2|_{#4}}%
}

\newcommand\nnfootnote[1]{%
  \begingroup
  \renewcommand\thefootnote{}\footnote{#1}%
  \addtocounter{footnote}{-1}%
  \endgroup
}


\title{Calculus II: Assignment 5}
\author{Jeremy Favro}
\date{\today}

% DOC BEGIN

\begin{document}

\maketitle

\noindent Consider the Gamma function, the function of $x$ defined by using $x$ as a constant in an integral
as follows:

$$\Gamma(x) = \int_{0}^{\infty} t^{x-1}e^{-t} \,dt = \lim_{k \to \infty} \int_{0}^{k} t^{x-1}e^{-t} \,dt$$

\noindent This definition turns out to make sense whenever $x>0$.

% PROBLEM 1
\begin{problem}
Use SageMath to compute $\Gamma\left(\frac{1}{2}\right), \: \Gamma\left(1\right), \: \Gamma\left(\frac{3}{2}\right), \: \Gamma\left(2\right), \: \Gamma\left(\frac{5}{2}\right),
    \: \Gamma\left(3\right), \: \Gamma\left(\frac{7}{2}\right), \: \Gamma\left(4\right)$
\end{problem}
\begin{soln} ~\\
    \begin{tcolorbox}[breakable, size=fbox, boxrule=1pt, pad at break*=1mm,colback=cellbackground, colframe=cellborder]
        \prompt{In}{incolor}{1}{\boxspacing}
        \begin{minted}[breaklines, autogobble]{sage}
            clear_vars()
            from IPython.core.interactiveshell import InteractiveShell
            from sage.symbolic.integration.integral import definite_integral
            
            InteractiveShell.ast_node_interactivity = "all"
            
            nums = {1/2: 0, 1: 0, 3/2: 0, 5/2: 0, 3: 0, 7/2: 0, 4: 0}
            
            x = var('x')
            t = var('t')
            
            assume(x>0)
            
            gamma(x) = definite_integral((t^(x-1)*e^(-t)), t, 0 , oo)
            
            for num in nums:
                nums[num] = gamma(num)
            nums      
        \end{minted}
    \end{tcolorbox}
    \begin{tcolorbox}[breakable, size=fbox, boxrule=.5pt, pad at break*=1mm, opacityfill=0]
        \prompt{Out}{outcolor}{1}{\boxspacing}
        \begin{minted}[breaklines, autogobble]{sage}
            {1/2: sqrt(pi),
            1: 1,
            3/2: 1/2*sqrt(pi),
            5/2: 3/4*sqrt(pi),
            3: 2,
            7/2: 15/8*sqrt(pi),
            4: 6}
        \end{minted}
    \end{tcolorbox}
\end{soln}

\newpage

% PROBLEM 2
\begin{problem}
By hand, show that $\Gamma(x + 1) = x\Gamma(x)$.
\end{problem}
\begin{soln} Assuming that $\Gamma(x + 1) = x\Gamma(x)$ is true, we should get the same integral as part of our answer, which we can use to check our work.
    \begin{align*}
         & \Gamma(x + 1) = \int_{0}^{\infty} t^{x}e^{-t} \,dt                                                                                                      \\
         & = \int_{0}^{\infty} t^{x}e^{-t} \,dt \implies \text{ Apply integration by parts, } u = t^x, \: u^{\prime}= xt^{x-1}, \: v^{\prime}=e^{-t}, \: v=-e^{-t} \\
         & = \evaluated{\left[-t^xe^{-t}\right]}_{0}^{\infty}-\int_{0}^{\infty} -xt^{x-1}e^{-t} \,dt                                                               \\
         & = \lim_{t \to \infty}  \left[-t^xe^{-t}\right]-\left[-0^xe^{0}\right]-\int_{0}^{\infty} -xt^{x-1}e^{-t} \,dt                                            \\
         & = \lim_{t \to \infty}  \left[-t^xe^{-t}\right]-0-\int_{0}^{\infty} -xt^{x-1}e^{-t} \,dt                                                                 \\
         & = \int_{0}^{\infty} xt^{x-1}e^{-t} \,dt \implies \text{ x can be taken out because it's constant }                                                      \\
         & = x\int_{0}^{\infty} t^{x-1}e^{-t} \,dt                                                                                                                 \\
    \end{align*}
    $$\therefore \Gamma(x + 1) = x\Gamma(x) $$
\end{soln}

% PROBLEM 3
\begin{problem}
Using the results of questions 1 and 2, explain why $\Gamma(n + 1) = n!$ for any integer $n \geq 0$.
\end{problem}
\begin{soln}
    Question 1 demonstrated that $\Gamma(n)=(n-1)!$ (At least for 1, 2, 3, and 4). Question 2 showed that $\Gamma(n+1)=n\Gamma(n)$. If we extend the result from question 1 to all $n\in \mathbb{Z}$ where $n\geq 0$,
    and combine it with the result from question 2:

    $$\Gamma(n+1)=n\Gamma(n)=n(n-1)!$$

    \noindent Which, because the factorial $n!$ is defined as
    \begin{align*}
         & n! = n\times (n-1) \times (n-2) \times (n-3) \times \dots \times 3 \times 2 \times 1 \\
         & n! = n(n-1)!                                                                         \\
    \end{align*}

    \noindent Simplifies to

    $$\Gamma(n+1)=n!$$
\end{soln}

\end{document}