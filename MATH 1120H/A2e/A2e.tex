\documentclass[10pt]{article}

\usepackage[margin=0.75in]{geometry}
\usepackage{amsmath,amsthm,amssymb}
\usepackage{xcolor}
\usepackage{cancel}
\usepackage{xcolor}
\usepackage{tikz}
\usepackage{pgfplots}
\usepackage{physics}
\usepackage{minted}

\usepackage{changepage}
\usepackage{hyperref}
\usepackage{pythonhighlight}
\usepackage{pdfpages}
\usepackage[inline]{enumitem}

\theoremstyle{definition}
\newtheorem{problem}{Problem}
\newtheorem{soln}{Solution}

\pgfplotsset{compat=newest}

\NewDocumentCommand{\evalat}{sO{\big}mm}{%
  \IfBooleanTF{#1}
   {\mleft. #3 \mright|_{#4}}
   {#3#2|_{#4}}%
}

\title{Calculus II: Assignment 2e}
\author{Jeremy Favro}
\date{\today}

\begin{document}

\maketitle

Bhaskara II (1114-1185), also known as Bhaskaracharya (“Bhaskara the
Teacher”), was an Indian mathematician and astronomer who continued the
Indian tradition of writing mathematics in verse. Here are two problems
from his book on arithmetic, \textit{Lilavati}, which was named after and dedicated
to his daughter. “Lilavati” apparently means “playful”, and a number of
the problems in the book fit the title. Here are two such problems from
\textit{Lilavati}, taken from the prose translation by Henry Colebrooke \cite{AAM}.
~\\
\begin{adjustwidth}{75pt}{75pt}
    \textit{
        \indent 54. Out of a swarm of bees, one-fifth part settled on a blossom
        of Columba; and one-third on a flower of Sil\`ind'hr\`i; three times
        the difference of those numbers flew to the bloom of a Cu\'taja. One
        bee, which remained, hovered and flew about in the air, allured
        at the same moment by the pleasing fragrance of a jasmin and
        pandanus. Tell me, charming woman, the number of bees.
    }
    ~\\~\\
    \textit{
        \indent 68. The square-root of half the number of a swarm of bees is
        gone to the shrub of jasmin; and so are eight-ninths of the whole
        swarm; a female is buzzing to one remaining male that is humming
        within a lotus, in which he is confined, having been allured to its
        fragrance at night. Say, lovely woman, the number of bees.
    }
\end{adjustwidth}



% PROBLEM 1
\begin{problem}
Solve problem 54, given above.
\end{problem}
\begin{soln} ~\\
    \begin{align*}
         & b = \frac{1}{5}b + \frac{1}{3}b + 3b(\frac{1}{3} - \frac{1}{5}) + 1 \\
         & b = \frac{1}{5}b + \frac{1}{3}b + 3b(\frac{1}{3} - \frac{1}{5}) + 1 \\
         & b = \frac{1}{5}b + \frac{1}{3}b + \frac{2}{5}b + 1                  \\
         & b = \frac{14}{15}b + 1                                              \\
         & 15b = 14b + 15                                                      \\
         & 15b - 14b = 15                                                      \\
         & b = 15                                                              \\
    \end{align*}

    \begin{center}
        % \begin{adjustwidth}{75pt}{75pt}
        \textit{
            Let b count the bees \\
            Give a fifth to a third and another two fifths \\
            Recall the jasmine and give one \\
            Then both by fifteen taking fourteen gives b \\
        }
        % \end{adjustwidth}
    \end{center}

    \noindent Definitely not my proudest work, but I'm doing a physics degree for a reason ;)
\end{soln}

\newpage

% PROBLEM 2
\begin{problem}
Solve problem 68, given above.
\end{problem}
\begin{soln} ~\\
    \begin{align*}
         & b = \sqrt{\frac{1}{2}b} + \frac{8}{9}b + 2                                                                                                          \\
         & b = \frac{\sqrt{b}}{\sqrt{2}} + \frac{8}{9}b + 2                                                                                                    \\
         & b - \frac{8}{9}b - 2 = \frac{\sqrt{b}}{\sqrt{2}}                                                                                                    \\
         & \frac{1}{9}b - 2 = \frac{\sqrt{b}}{\sqrt{2}}                                                                                                        \\
         & (\frac{1}{9}b - 2)^2 = \frac{1}{2}b                                                                                                                 \\
         & 0 = \frac{1}{81}b^2-\frac{4}{9}b+4 - \frac{1}{2}b                                                                                                   \\
         & 0 = \frac{1}{81}b^2-\frac{17}{18}b+4 \rightsquigarrow \frac{\frac{17}{18}\pm\sqrt{\left(-\frac{17}{18}\right)^2-4(\frac{1}{81})(4)}}{2\frac{1}{81}} \\
    \end{align*}

    \noindent So there are either $72$ or $\frac{9}{2}$ bees. Since fractional bees are unrealistic, we probably have $72$ bees.

    \begin{center}
        \textit{
            Let b count the bees \\
            Root half and add eight-ninths \\
            Think of the lovers, add two \\
            Square, expand, and set to nil \\
            Employ a formula and reason a solution \\
        }
    \end{center}
\end{soln}


\begin{thebibliography}{2}
    \bibitem{AAM}
    \textit{Algebra, with Arithmetic and mensuration, from the Sanscrit of Brahmegupta
        and Bh\`ascara}, Henry Thomas Colebrooke, 1817. Online at: \verb https://archive.org/stream/algebrawitharith00brahuoft\#mode/2up
\end{thebibliography}
\end{document}