\documentclass[10pt]{article}

\usepackage[margin=0.75in]{geometry}
\usepackage{amsmath,amsthm,amssymb}
\usepackage{xcolor}
\usepackage{cancel}
\usepackage{xcolor}
\usepackage{tikz}
\usepackage{minted}
\usepackage{pythonhighlight}
\usepackage[inline]{enumitem}

\theoremstyle{definition}
\newtheorem{problem}{Problem}
\newtheorem{soln}{Solution}

\title{Calculus II: Assignment 1}
\author{Jeremy Favro}
\date{\today}

\begin{document}

\maketitle

% PROBLEM 1
\begin{problem}
Find the area of the finite region between the parabola $y = x^2$ and the line $y = 4$ by setting
up and evaluating a suitable definite integral. The evaluation should be done both by hand
and by using SageMath.
\end{problem}
\begin{soln} The area we are looking for is the result of the subtraction of the area under $y=4$ and $y=x^2$ between their intersection points.
    This means that $\int_{P_1}^{P_2} 4 \,dx - \int_{P_1}^{P_2} x^2 \,dx$ will give us our area.
    To determine $P_1 \& P_2$, set $x^2=4$ and solve.
    \begin{align*}
         & x^2=4      \\
         & x=\sqrt{4} \\
         & x=\pm 2    \\
    \end{align*}
    So,
    \begin{align*}
         & = \int_{-2}^{2} 4 \,dx - \int_{-2}^{2} x^2 \,dx           \\
         & = (4(2) - \frac{1}{3}(2)^3) - (4(-2) - \frac{1}{3}(-2)^3) \\
         & = \frac{32}{3}                                            \\
    \end{align*}

    $$\therefore \frac{32}{3} \text{ is the area between } y = x^2 \text{ and } y = 4 \text{, on the interval } [-2,2]$$

    \noindent Using sage to evaluate:

    \begin{minted}[breaklines]{sage}
# Plain ol' integration 
clear_vars()
        
x = var('x')
        
f = x^2
g = 4
        
show(integral(g, x, -2, 2)-integral(f(x), x, -2, 2))
    \end{minted}
\end{soln}

% PROBLEM 2
\begin{problem}
Compute the definite integral you set up in solving $1$ using the Right-Hand Rule formula, as
described in the accompanying handout Right-Hand Rule Riemann Sums. The evaluation of
the formula, once you've set it up, should be done both by hand and by using SageMath.
\end{problem}
\begin{soln}
    The Right-Hand Rule formula,
    \begin{equation}
        \lim_{n \to \infty}\left[\sum_{i = 1}^{n}\left( \frac{b-a}{n} \cdot f\left(a+i\frac{b-a}{n}\right) \right) \right]
    \end{equation}
    Where $a$ is the left bound and $b$ is the right bound.
    \begin{align*}
         & = \lim_{n \to \infty}\left[\sum_{i = 1}^{n}\left( \frac{2-(-2)}{n} \cdot f\left(-2+i\frac{2-(-2)}{n}\right) \right) \right]                                                                                                       \\
         & = \lim_{n \to \infty}\left[\sum_{i = 1}^{n}\left( \frac{4}{n} \cdot f\left(-2+i\frac{4}{n}\right) \right) \right]                                                                                                                 \\
         & = \lim_{n \to \infty}\left[\frac{4}{n}\sum_{i = 1}^{n}\left(-2+i\frac{4}{n}\right)^2 \right]                                                                                                                                      \\
         & = \lim_{n \to \infty}\left[\frac{4}{n}\sum_{i = 1}^{n}\left(4-i\frac{16}{n} + i^2\frac{16}{n^2}\right) \right]                                                                                                                    \\
         & = \lim_{n \to \infty}\left[\frac{4}{n}\left(\sum_{i = 1}^{n}\left(4\right)-\sum_{i = 1}^{n}\left(i\frac{16}{n}\right) + \sum_{i = 1}^{n}\left(i^2\frac{16}{n^2}\right)\right) \right]                                             \\
         & = \lim_{n \to \infty}\left[\frac{4}{n}\left(4n-\frac{16}{n}\sum_{i = 1}^{n}i + \frac{16}{n^2}\sum_{i = 1}^{n}i^2\right) \right] \implies \text{ use summation identities for } \sum_{i = 1}^{n}i \text{ and } \sum_{i = 1}^{n}i^2 \\
         & = \lim_{n \to \infty}\left[\frac{4}{n}\left(4n-\frac{16}{\cancel{n}}\frac{\cancel{n}(n+1)}{2} + \frac{16}{n^{\cancel{2}}}\frac{\cancel{n}(2n+1)(n+1)}{6}\right) \right]                                                           \\
         & = \lim_{n \to \infty}\left[\frac{4}{n}\left(4n-\frac{16(n+1)}{2} + \frac{16(2n+1)(n+1)}{6n}\right) \right]                                                                                                                        \\
         & = \lim_{n \to \infty}\left[\frac{4}{n}\left(4n+\frac{16(n+1)}{2}\left(- 1 + \frac{2n+1}{3n}\right)\right) \right]                                                                                                                 \\
         & = \lim_{n \to \infty}\left[\frac{4}{n}\left(4n+\frac{16(n+1)}{2}\left(- 1 + \frac{2\cancel{n}}{3\cancel{n}}+\frac{1}{3n}\right)\right) \right]                                                                                    \\
         & = \lim_{n \to \infty}\left[\frac{4}{n}\left(4n+\frac{16(n+1)}{2}\left(- 1 + \frac{2}{3}+\frac{1}{3n}\right)\right) \right]                                                                                                        \\
         & = \lim_{n \to \infty}\left[\left(\frac{4}{\cancel{n}}4\cancel{n}+\frac{4}{n}\frac{16(n+1)}{2}\left(- 1 + \frac{2}{3}+\frac{1}{3n}\right)\right) \right]                                                                           \\
         & = \lim_{n \to \infty}\left[\left(16+\frac{64(n+1)}{2n}\left(- 1 + \frac{2}{3}+\frac{1}{3n}\right)\right) \right]                                                                                                                  \\
         & = \lim_{n \to \infty}\left[\left(16+\left(\frac{64\cancel{n}}{2\cancel{n}}+\frac{64}{2n}\right)\left(-\frac{1}{3}+\frac{1}{3n}\right)\right) \right]                                                                              \\
         & = \lim_{n \to \infty}\left[\left(16-\frac{32}{3}+\frac{32}{3n}-\frac{64}{6n}+\frac{64}{6n^2}\right) \right]                                                                                                                       \\
         & = \lim_{n \to \infty}\left[16-\frac{32}{3}+\cancelto{0}{\frac{32}{3n}}-\cancelto{0}{\frac{64}{6n}}+\cancelto{0}{\frac{64}{6n^2}} \right]                                                                                          \\
         & = 16-\frac{32}{3}                                                                                                                                                                                                                 \\
         & = \frac{16}{3}                                                                                                                                                                                                                    \\
    \end{align*}
    Subtracting the result of the Right-Hand Rule formula run on $x^2$ from the result when run on $y=4$ (omitted for space because it's just a souped up way to do $base \cdot height$) yields the area between the two.
    $$16-\frac{16}{3}=\frac{32}{3}$$

    \noindent Using sage to evaluate:
    \begin{minted}[breaklines]{sage}
# Right-Hand rule
clear_vars()
        
x = var('x')
i = var('i')
n = var('n')
        
g = function('g')(x)
        
g(x) = x^2
        
lower_lim = -2
upper_lim = 2
        
s = function('s')(n)
        
s(n) = sum(((upper_lim-lower_lim)/n)*g(lower_lim+i*((upper_lim-lower_lim)/n)), i, 1, n) # Sum of the expression starting at i = 1, up to i = n
        
show(4*4 - limit(s(n), n=infinity)) # area of y=4 minus limit of the above sum as n -> infinity        
    \end{minted}
\end{soln}

% PROBLEM 3
\begin{problem}
Use Archimedes' theorem and the observations above to find the area of the region described
in question 1. After you've set it up, you should do the computation both by hand and by
using SageMath.
\end{problem}
\begin{soln}
    We know that for each subsequent triangle drawn, its area is $\frac{1}{8}$ of the area of the previous triangle.
    This means that the total area added to the approximation by each step is $\frac{A_{n-1}n}{8}$, meaning that the area for any given step, $n$, can be determined by computing
    the sum of all previous areas. This can be written as follows, where $Z$ (chosen because $A$ is a point in the original triangle) is the area of our original triangle $\triangle ABC$

    \begin{equation}
        \lim_{k \to \infty}\left[\sum_{n = 1}^{k} Z \cdot \frac{2^{n-1}}{8^{n-1}}\right]
    \end{equation}

    This summation can then be computed for the given triangle, $\triangle ABC$, which has an area of $8$.

    \begin{align*}
         & = \lim_{k \to \infty}\left[\sum_{n = 1}^{k} 8 \cdot \frac{2^{n-1}}{8^{n-1}}\right]                                                  \\
         & = \lim_{k \to \infty}\left[8\sum_{n = 1}^{k} \frac{2^{n-1}}{8^{n-1}}\right]                                                         \\
         & = \lim_{k \to \infty}\left[8\sum_{n = 1}^{k} 2^{-2n+2}\right]                                                                       \\
         & = \lim_{k \to \infty}\left[8\sum_{n = 1}^{k} (2^2)^{-n+1}\right]                                                                    \\
         & = \lim_{k \to \infty}\left[8\sum_{n = 1}^{k} 4^{-n+1}\right]                                                                        \\
         & = \lim_{k \to \infty}\left[8\sum_{n = 1}^{k} 4^{-n}\cdot 4\right]                                                                   \\
         & = \lim_{k \to \infty}\left[8\sum_{n = 1}^{k} 4^{-n}\cdot 4\right] \implies \text{ use summation identity for } \sum_{n = 1}^{k}ar^n \\
         & = \lim_{k \to \infty}\left[8 \frac{4}{1-4} \right]                                                                                  \\
         & = 8 \frac{4}{1-4}                                                                                                                   \\
         & = \frac{32}{3}                                                                                                                      \\
    \end{align*}

    \noindent Using sage to evaluate:
    \begin{minted}[breaklines]{sage}
# Archimedes triangle sums
clear_vars()
        
x = var('x')
k = var('k')
n = var('n')
        
s = function('s')(n)
        
a = 8 # area of the first triangle, bh/2
        
s(n) = sum(a*((2^(n-1))/(8^(n-1))), n, 1, k) # Sum of the expression starting at n = 1, up to n = k
        
show(limit(s(n), k=infinity)) # Sum as # of triangles -> infty
    \end{minted}
\end{soln}

\end{document}