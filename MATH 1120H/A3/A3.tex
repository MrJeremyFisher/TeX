\documentclass[10pt]{article}

\usepackage[margin=0.75in]{geometry}
\usepackage{amsmath,amsthm,amssymb}
\usepackage{xcolor}
\usepackage{cancel}
\usepackage{xcolor}
\usepackage{tikz}
\usepackage{pgfplots}
\usepackage{physics}
\usepackage{minted}
\usepackage{pythonhighlight}
\usepackage{pdfpages}
\usepackage[inline]{enumitem}
\usepackage[breakable]{tcolorbox}

\theoremstyle{definition}
\newtheorem{problem}{Problem}
\newtheorem{soln}{Solution}

\pgfplotsset{compat=newest}

\definecolor{incolor}{HTML}{303F9F}
\definecolor{outcolor}{HTML}{D84315}
\definecolor{cellborder}{HTML}{CFCFCF}
\definecolor{cellbackground}{HTML}{F7F7F7}

\tikzset
{%
  axes/.style={thick,-latex},
  cylinder/.style={right color=blue!80,left color=white,fill opacity=0.7},
  paraboloid back/.style={left color=magenta!80,fill opacity=0.4},
  paraboloid front/.style={left color=white, right color=magenta!80,fill opacity=0.4},
}

\makeatletter
\newcommand{\boxspacing}{\kern\kvtcb@left@rule\kern\kvtcb@boxsep}
\makeatother
\newcommand{\prompt}[4]{
    \ttfamily\llap{{\color{#2}[#3]:\hspace{3pt}#4}}\vspace{-\baselineskip}
}


\NewDocumentCommand{\evalat}{sO{\big}mm}{%
  \IfBooleanTF{#1}
   {\mleft. #3 \mright|_{#4}}
   {#3#2|_{#4}}%
}

\newcommand\nnfootnote[1]{%
  \begingroup
  \renewcommand\thefootnote{}\footnote{#1}%
  \addtocounter{footnote}{-1}%
  \endgroup
}


\title{Calculus II: Assignment 3}
\author{Jeremy Favro}
\date{\today}

% DOC BEGIN

\begin{document}

\maketitle

\noindent Consider the region below the curve $y = \frac{1}{x}$ and above the $x$-axis for $1 \leq x < \infty$.

% PROBLEM 1
\begin{problem}
Compute the area of the given region, both by hand and using SageMath.
\end{problem}
\begin{soln}
    \begin{align*}
         & = \int_{1}^{\infty} \frac{1}{x} \,dx                               \\
         & = \lim_{n \to \infty}  \left[\int_{1}^{n} \frac{1}{x} \,dx \right] \\
         & = \lim_{n \to \infty}  \left[\evaluated{\ln(x)}_{1}^{n}\right]     \\
         & = \lim_{n \to \infty}  \left[\ln(n)-\ln(1)\right]                  \\
         & = \lim_{n \to \infty}  \ln(n)                                      \\
         & = +\infty                                                          \\
    \end{align*}

    \noindent Using sage to evaluate:
    \begin{tcolorbox}[breakable, size=fbox, boxrule=1pt, pad at break*=1mm,colback=cellbackground, colframe=cellborder]
        \prompt{In}{incolor}{1}{\boxspacing}
        \begin{minted}[breaklines, autogobble]{sage}
            clear_vars()
            from sage.symbolic.integration.integral import definite_integral # Necessary in some versions for definite_integral()
            
            x = var('x')
            t = var('n')
            
            f = 1/x
            
            assume(n-1>0)
            
            limit(definite_integral(f, x, 1, n), n=oo)
        \end{minted}
    \end{tcolorbox}
    \begin{tcolorbox}[breakable, size=fbox, boxrule=.5pt, pad at break*=1mm, opacityfill=0]
        \prompt{Out}{outcolor}{1}{\boxspacing}
        \begin{minted}[breaklines, autogobble]{sage}
            +Infinity
        \end{minted}
    \end{tcolorbox}



    $$\therefore \text{ the area of } \frac{1}{x} \text{ on } [1,\infty) \text{ is } \infty$$
\end{soln}

% PROBLEM 2
\begin{problem}
Compute the volume of the solid obtained by revolving the given region about the x-axis, both
by hand and using SageMath.
\end{problem}
\begin{soln}
    \begin{align*}
         & = \pi\int_{1}^{\infty} \frac{1}{x^2} \,dx                               \\
         & = \lim_{n \to \infty}  \left[\pi\int_{1}^{n} \frac{1}{x^2} \,dx \right] \\
         & = \lim_{n \to \infty}  \left[-\pi\evaluated{\frac{1}{x}}_{n}^{1}\right] \\
         & = -\pi\lim_{n \to \infty}  \left[\evaluated{\frac{1}{x}}_{n}^{1}\right] \\
         & = -\pi\lim_{n \to \infty}  \left[\frac{1}{n}-\frac{1}{1}\right]         \\
         & = -\pi\lim_{n \to \infty}  \frac{1}{n}-1                                \\
         & = \pi                                                                   \\
    \end{align*}

    \noindent Using sage to evaluate:
    \begin{tcolorbox}[breakable, size=fbox, boxrule=1pt, pad at break*=1mm,colback=cellbackground, colframe=cellborder]
        \prompt{In}{incolor}{2}{\boxspacing}
        \begin{minted}[breaklines, autogobble]{sage}
            clear_vars()
            from sage.symbolic.integration.integral import definite_integral # Necessary in some versions for definite_integral()
            
            x = var('x')
            n = var('n')
            
            f = 1/x
            
            assume(n-1>0)
            
            limit(definite_integral(pi*f^2, x, 1, n), n=oo)
        \end{minted}
    \end{tcolorbox}
    \begin{tcolorbox}[breakable, size=fbox, boxrule=.5pt, pad at break*=1mm, opacityfill=0]
        \prompt{Out}{outcolor}{2}{\boxspacing}
        \begin{minted}[breaklines, autogobble]{sage}
            pi
        \end{minted}
    \end{tcolorbox}

    $$\therefore \text{ the volume of } \frac{1}{x} \text{ on } [1,\infty) \text{ rotated about the } x \text{-axis} \text{ is } \pi$$
\end{soln}

% PROBLEM 3
\begin{problem}
There is something a little paradoxical about the (correct :-) answers to 1 and 2. What is the
paradox? Explain what's going on as best you can.
\end{problem}
\begin{soln}
    The area under $\frac{1}{x}$ (on $[1, \infty)$, implied) is infinite, but the volume of the solid of revolution about the $x$-axis is finite, $\pi$ in this case.
    This seems paradoxical as rotating an infinite area about the $x$-axis should give you an infinite volume. I think the reason this doesn't occur is because $\int \frac{1}{x} \,dx = \ln(x)$,
    which is strictly increasing (though very slowly, as $x\to\infty$, $\ln(x)\to\infty$), whereas $\int \frac{1}{x^2} \,dx = -\frac{1}{x}$,
    which is strictly \textit{decreasing} (again though very slowly, as $x\to\infty$, $-\frac{1}{x}\to\infty$, though $-\frac{1}{x}$ will never equal 0).
    This means that though the area is infinite, the amount added to the volume decays ("falls off") quick enough that it approaches a finite value, which is a little weird sounding but I think it makes sense.
\end{soln}

\nnfootnote{Woohoo, finally figured out how to do nice embedding of sage code in \LaTeX, now to figure out plotting solids of revolution}
\end{document}