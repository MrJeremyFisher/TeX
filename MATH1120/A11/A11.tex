\documentclass[10pt]{article}

\usepackage[margin=0.75in]{geometry}
\usepackage{amsmath,amsthm,amssymb}
\usepackage{xcolor}
\usepackage{cancel}
\usepackage{changepage}
\usepackage{tikz}
\usepackage{pgfplots}
\usepackage{physics}
\usepackage{minted}
\usepackage{hyperref}
\usepackage[breakable]{tcolorbox}
\usepackage[inline]{enumitem}

\theoremstyle{definition}
\newtheorem{problem}{Problem}
\newtheorem{soln}{Solution}

\pgfplotsset{compat=newest}
\usetikzlibrary{lindenmayersystems}

\definecolor{incolor}{HTML}{303F9F}
\definecolor{outcolor}{HTML}{D84315}
\definecolor{cellborder}{HTML}{CFCFCF}
\definecolor{cellbackground}{HTML}{F7F7F7}

\tikzset
{%
  axes/.style={thick,-latex},
  cylinder/.style={right color=blue!80,left color=white,fill opacity=0.7},
  paraboloid back/.style={left color=magenta!80,fill opacity=0.4},
  paraboloid front/.style={left color=white, right color=magenta!80,fill opacity=0.4},
}

\makeatletter
\newcommand{\boxspacing}{\kern\kvtcb@left@rule\kern\kvtcb@boxsep}
\makeatother
\newcommand{\prompt}[4]{
    \ttfamily\llap{{\color{#2}[#3]:\hspace{3pt}#4}}\vspace{-\baselineskip}
}


\hypersetup{
    colorlinks=true,
    linkcolor=blue,
    filecolor=magenta,      
    urlcolor=cyan,
    pdftitle={Overleaf Example},
    pdfpagemode=FullScreen,
    }

\NewDocumentCommand{\evalat}{sO{\big}mm}{%
  \IfBooleanTF{#1}
   {\mleft. #3 \mright|_{#4}}
   {#3#2|_{#4}}%
}

\tikzset{koch snowflake/.style={insert path={%
   l-system [l-system={rule set={F -> F-F++F-F}, axiom=F++F++F,
    step=3cm/3^#1, angle=60, order=#1,anchor=center}] -- cycle}}}

\title{Calculus II: Assignment 11}
\author{Jeremy Favro}
\date{\today}

\begin{document}

\maketitle

% PROBLEM 1
\begin{problem}
For what values of x does the series $\displaystyle \sum_{n = 0}^{\infty} (-1)^n(n+1)x^n$ converge?
\end{problem}
\begin{soln}
      \begin{align*}
             & = \lim_{n \to \infty} \abs{\frac{(-1)^{n+1}(n+2)x^{n+1}}{(-1)^n(n+1)x^n}} \\
             & = \lim_{n \to \infty} \abs{(-1)x\frac{(n+2)}{(n+1)}}                      \\
             & = \lim_{n \to \infty} \abs{x\frac{1}{1}}                                  \\
             & = \lim_{n \to \infty} \abs{x}                                             \\
      \end{align*}
      \begin{align*}
             & = \lim_{n \to \infty} \abs{(-1)^n(n+1)(\pm 1)^n} \\
             & = \lim_{n \to \infty} \abs{n+1}                  \\
             & = \infty                                         \\
      \end{align*}
      \noindent $\therefore$ By the ratio test (and divergence test for endpoints) the series $\displaystyle \sum_{n = 0}^{\infty} (-1)^n(n+1)x^n$ converges for all $\abs{x}<1$
\end{soln}

% PROBLEM 2
\begin{problem}
Use Taylor's Formula to show that $\displaystyle \frac{1}{(1+x)^2}=\sum_{n = 0}^{\infty} (-1)^n(n+1)x^n$ when the series converges.
\end{problem}
\begin{soln} ~\\
      \begin{center}
            \begin{tabular}{ c | c | c }
                  $f^{(0)}(a=0)$ & $\frac{1}{(1+x)^2}$   & 1   \\
                  $f^{(1)}(a)$   & $-\frac{2}{(1+x)^3}$  & -2  \\
                  $f^{(2)}(a)$   & $\frac{6}{(1+x)^4}$   & 6   \\
                  $f^{(3)}(a)$   & $-\frac{24}{(1+x)^5}$ & -24 \\
                  $f^{(4)}(a)$   & $\frac{120}{(1+x)^6}$ & 120
            \end{tabular}
            $\implies\displaystyle \frac{d^n}{dx^n}=(-1)^n\frac{(n+1)!}{(1+x)^{n+2}}=(-1)^n(n+1)!$ (for $x=0$)
      \end{center}
      \begin{align*}
             & = \sum_{n=0}^{\infty} \frac{(-1)^n\frac{(n+1)!}{1}}{n!}x^n \\
             & = \sum_{n=0}^{\infty} (-1)^n\frac{(n+1)!}{n!}x^n           \\
             & = \sum_{n=0}^{\infty} (-1)^n(n+1)x^n
      \end{align*}
      \noindent $\therefore$ By the ratio test (and divergence test for endpoints) the series $\displaystyle \sum_{n = 0}^{\infty} (-1)^n(n+1)x^n$ converges for all $\abs{x}<1$
\end{soln}

% PROBLEM 3
\begin{problem}
Use algebra to show that $\displaystyle \frac{1}{(1+x)^2}=\sum_{n = 0}^{\infty} (-1)^n(n+1)x^n$ when the series converges.
\end{problem}
\begin{soln} If we take $\displaystyle \sum_{n=0}^{\infty}(-1)^nx^n=\frac{1}{1+x}=\frac{1}{1-(-x)}$ then
      \begin{align*}
             & = \frac{d}{dx}\frac{1}{1+x} \\
             & = -\frac{1}{(1+x)^2}
      \end{align*}
      And
      \begin{align*}
             & = \frac{d}{dx}\sum_{n=0}^{\infty}(-1)^nx^n \\
             & = \sum_{n=0}^{\infty}(-1)^nnx^{n-1}
      \end{align*}
      Which has a leading term of 0 so we can shift it up one
      \begin{align*}
             & = \sum_{n=1}^{\infty}(-1)^{n+1} (n+1) x^{n} \\
             & = -\sum_{n=0}^{\infty}(-1)^n(n+1)x^n
      \end{align*}
      So the series for $\displaystyle \frac{1}{1+x}$ when differentiated and manipulated slightly results in the series for $\displaystyle \frac{1}{(1+x)^2}$, with the same transformations applying to
      the functions themselves. \\

      \noindent $\displaystyle \therefore \sum_{n=0}^{\infty}(-1)^n(n+1)x^n$ is indeed the power series representation of $\displaystyle \frac{1}{(1+x)^2}$ when it converges.

\end{soln}

\end{document}