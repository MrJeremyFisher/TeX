\documentclass[10pt]{article}

\usepackage[margin=0.75in]{geometry}
\usepackage{amsmath,amsthm,amssymb}
\usepackage{xcolor}
\usepackage{cancel}
\usepackage{graphicx}
\usepackage{changepage}
\usepackage{circuitikz}
\usepackage{pgfplots}
\usepackage{physics}
\usepackage{hyperref}
\usepackage{siunitx}
\usepackage[breakable]{tcolorbox}
\usepackage[inline]{enumitem}

\theoremstyle{definition}
\newtheorem{problem}{Problem}
\newtheorem{soln}{Solution}

\pgfplotsset{compat=newest}
\usetikzlibrary{lindenmayersystems}
\usetikzlibrary{arrows}
\usetikzlibrary{calc}

\definecolor{incolor}{HTML}{303F9F}
\definecolor{outcolor}{HTML}{D84315}
\definecolor{cellborder}{HTML}{CFCFCF}
\definecolor{cellbackground}{HTML}{F7F7F7}
\newcommand{\eq}{=}
\usetikzlibrary{positioning, fit, calc}
\pgfdeclarelayer{background}  
\pgfsetlayers{background,main}
\DeclareSIUnit[number-unit-product = {\,}]\calorie{cal}
\DeclareSIUnit[number-unit-product = {\,}]\atmosphere{atm}
\AtBeginDocument{\RenewCommandCopy\qty\SI}

\makeatletter
\newcommand{\boxspacing}{\kern\kvtcb@left@rule\kern\kvtcb@boxsep}
\makeatother
\newcommand{\prompt}[4]{
    \ttfamily\llap{{\color{#2}[#3]:\hspace{3pt}#4}}\vspace{-\baselineskip}
}

\newcommand{\thevenin}[2]{
  \begin{center}
    \begin{circuitikz} \draw
      (0,0) -- (2,0) to[battery1, l_=$V_{Th}\eq#1$] (2,2) 
      to[resistor, l_=$R_{Th}\eq#2$] (0,2)
      ;
      \draw [o-] (-.07,2.079);
      \draw [o-] (-.07,0.079);
    \end{circuitikz}
  \end{center}
}

\newcommand{\norton}[2]{
  \begin{center}
    \begin{circuitikz} \draw
      (0,0) -- (3,0) to[american current source, l_=$I_{N}\eq#1$] (3,2) -- (0,2) (2,0)
      to[resistor, l=$R_{N}\eq#2$] (2,2)
      ;
      \draw [o-] (-.07,2.079);
      \draw [o-] (-.07,0.079);
    \end{circuitikz}
  \end{center}
}

\newcommand{\highlight}[1]{\colorbox{yellow}{$\displaystyle #1$}}

\newcommand{\ti}[1]{\widetilde{#1}}

\title{Physics 2700H: Assignment III}
\author{Jeremy Favro (0805980) \\ Trent University, Peterborough, ON, Canada}
\date{\today}

\begin{document}
\maketitle

% PROBLEM 1
\begin{problem}
Five kg of water at $\qty{25}{\degreeCelsius}$ is added to $\qty{10.0}{\kilo\gram}$ of water at $\qty{85}{\degreeCelsius}$.
After the mixture has reached equilibrium, how much has entropy changed?
(Assume no energy is exchanged between the water and its surroundings.)
\end{problem}
\begin{soln}
  
\end{soln}

% PROBLEM 2
\begin{problem}
The July 2023 Veritasium video about entropy, https://www.youtube.com/watch?v=DxL2HoqLbyA,
introduces a Carnot engine within the first six minutes of the video. Draw on a $PV$ diagram the cycle for
this engine, with the bottom-right-most point labelled (a), and continue the cycle to points (b), (c) and
(d). Identify which timestamps in the video correspond to the four points (a)… (d) and explain using a
sentence per point why this is so.
\end{problem}
\begin{soln}

\end{soln}

% PROBLEM 3
\begin{problem}
One mole of helium gas is initially at $P_0=\qty{1.0}{\atmosphere}$ and $T_0=\qty{273}{\kelvin}$.
\begin{enumerate}[label=(\alph*)]
  \item Compute the entropy change if the gas is heated at constant pressure to temperature $\qty{400}{\kelvin}$.
  \item Starting again from the initial state $(P_0, T_0)$, what is the entropy change if
        the gas expands isothermally to twice its original volume?
\end{enumerate}
\end{problem}
\begin{soln}

\end{soln}
\end{document}