\documentclass[10pt]{article}

\usepackage[margin=0.75in]{geometry}
\usepackage{amsmath,amsthm,amssymb}
\usepackage{xcolor}
\usepackage{cancel}
\usepackage{graphicx}
\usepackage{changepage}
\usepackage{circuitikz}
\usepackage{pgfplots}
\usepackage{physics}
\usepackage{hyperref}
\usepackage{siunitx}
\usepackage[breakable]{tcolorbox}
\usepackage[inline]{enumitem}

\theoremstyle{definition}
\newtheorem{problem}{Problem}
\newtheorem{soln}{Solution}

\pgfplotsset{compat=newest}
\usetikzlibrary{lindenmayersystems}
\usetikzlibrary{arrows}
\usetikzlibrary{calc}

\definecolor{incolor}{HTML}{303F9F}
\definecolor{outcolor}{HTML}{D84315}
\definecolor{cellborder}{HTML}{CFCFCF}
\definecolor{cellbackground}{HTML}{F7F7F7}
\newcommand{\eq}{=}
\usetikzlibrary{positioning, fit, calc}
\pgfdeclarelayer{background}  
\pgfsetlayers{background,main}
\DeclareSIUnit[number-unit-product = {\,}]\calorie{cal}
\DeclareSIUnit[number-unit-product = {\,}]\atmosphere{atm}
\AtBeginDocument{\RenewCommandCopy\qty\SI}

\makeatletter
\newcommand{\boxspacing}{\kern\kvtcb@left@rule\kern\kvtcb@boxsep}
\makeatother
\newcommand{\prompt}[4]{
    \ttfamily\llap{{\color{#2}[#3]:\hspace{3pt}#4}}\vspace{-\baselineskip}
}

\newcommand{\thevenin}[2]{
  \begin{center}
    \begin{circuitikz} \draw
      (0,0) -- (2,0) to[battery1, l_=$V_{Th}\eq#1$] (2,2) 
      to[resistor, l_=$R_{Th}\eq#2$] (0,2)
      ;
      \draw [o-] (-.07,2.079);
      \draw [o-] (-.07,0.079);
    \end{circuitikz}
  \end{center}
}

\newcommand{\norton}[2]{
  \begin{center}
    \begin{circuitikz} \draw
      (0,0) -- (3,0) to[american current source, l_=$I_{N}\eq#1$] (3,2) -- (0,2) (2,0)
      to[resistor, l=$R_{N}\eq#2$] (2,2)
      ;
      \draw [o-] (-.07,2.079);
      \draw [o-] (-.07,0.079);
    \end{circuitikz}
  \end{center}
}

\newcommand{\highlight}[1]{\colorbox{yellow}{$\displaystyle #1$}}

\newcommand{\ti}[1]{\widetilde{#1}}

\title{Physics 2700H: Assignment II}
\author{Jeremy Favro}
\date{\today}

\begin{document}
\maketitle

% PROBLEM 1
\begin{problem}
Consider a mix of $N_2$ and $O_2$, which we may treat as an ideal gas, inside a car engine's cylinder that
follows the idealized Otto cycle. Assume points ${a,b,c,d}$ in Fig. 4.15(b)
correspond to $(V,P)$ values of: $\left\{(7V_1,\qty{1}{\atmosphere}), (V_1, \qty{15.25}{\atmosphere}), (V_1, \qty{30.50}{\atmosphere}), (7V_1, \qty{2}{\atmosphere})\right\}$,
respectively.
\begin{enumerate}[label=(\alph*)]
  \item Confirm that these $(V,P)$ values are consistent with the gas experiencing one adiabatic
        compression and one adiabatic expansion over each cycle.
  \item If the car's engine size (the displacement of here, \underline{four} cylinders) due to compression/expansion is $\qty{2.4}{\litre}$, what is $V_1$?
  \item Find the net work done by the gas in all four cylinders of the engine over one cycle (you may use the relevant expression for work from either p. 44 or p. 77 of the textbook).
\end{enumerate}
\end{problem}
\begin{soln} ~
  \begin{enumerate}[label=(\alph*)]
    \item Here the adiabats are along $\overrightarrow{ab}$ and $\overrightarrow{cd}$. For an adiabat $P_iV_i^\gamma=P_fV_f^\gamma$ holds.
    \begin{align*}
      & (7V_1)^\gamma\cdot\qty{1}{\atmosphere}=V_1^\gamma\cdot\qty{15.25}{\atmosphere}\\
      & V_1^\gamma\cdot\qty{1544734.579}{\pascal}=V_1^\gamma\cdot\qty{1545206.25}{\pascal}
    \end{align*}
    Which is approximately adiabatic. For $\overrightarrow{cd}$
    \begin{align*}
      & V_1^\gamma\cdot\qty{30.50}{\atmosphere}=(7V_1)^\gamma\cdot\qty{2}{\atmosphere}\\
      & V_1^\gamma\cdot\qty{3090412.5}{\pascal}=V_1^\gamma\cdot\qty{3089469.158}{\pascal}
    \end{align*}
    Which is also approximately adiabatic.
    \item The volume difference between compression and expansion \textbf{for a single cylinder} is $7V_1-V_1=6V_1$ so $4\cdot6V_1=\qty{2.4}{\liter}\implies V_1=\qty{0.1}{\liter}=\qty{1d-4}{\meter\cubed}$
    \item Find the net work done by the gas in all four cylinders of the engine over one cycle (you may use the relevant expression for work from either p. 44 or p. 77 of the textbook).
  \end{enumerate}
\end{soln}

% PROBLEM 2
\begin{problem}
An inventor claims to have developed an engine that takes in $\qty{1.1d8}{\joule}$ at $\qty{400}{\kelvin}$,
rejects $\qty{5.07d7}{\joule}$ at $\qty{200}{\kelvin}$, and delivers $\qty{16.7}{\kilo\watt}$ hours of work.
Would you advise investing money in this project?
\end{problem}
\begin{soln}
  A Carnot engine operating in this environment has $\eta=1-\frac{T_2}{T_1}=1-\frac{\qty{200}{\kelvin}}{\qty{400}{\kelvin}}=0.5$ and this engine has efficiency
  $\eta=1-\frac{\qty{5.07d7}{\joule}}{\qty{1.1d8}{\joule}}\approx0.54$. As the efficiency of a ``real'' engine cannot exceed that of a Carnot engine I would not
  reccomend investing in this project.
\end{soln}

% PROBLEM 3
\begin{problem}
Suppose a house requires 4.3 GJ of heating in a winter month. The
utility company charges \$0.14 per $\unit{\kilo\watt\hour}$.
\begin{enumerate}[label=(\alph*)]
  \item Find the cost savings of using a heat pump versus a 95\%-efficient natural gas furnace. Assume
        a Carnot heat pump with average temperatures of $\qty{20}{\degreeCelsius}$ indoors and $\qty{0}{\degreeCelsius}$ outdoors.
  \item Repeat part (a) using a more realistic coefficient of performance of 4.0 for the heat pump.
\end{enumerate}
\end{problem}
\begin{soln}
\end{soln}

% PROBLEM 4
\begin{problem}
A hypothetical engine, with an ideal gas as the working substance,
operates in the cycle shown in Figure 4.17. Show that the efficiency of
the engine is
$$\eta=1-\frac{1}{\gamma}\left(\frac{1-\frac{P_3}{P_1}}{1-\frac{V_1}{V_3}}\right)$$
\end{problem}
\begin{soln}
\end{soln}
\end{document}