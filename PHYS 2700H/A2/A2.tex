\documentclass[10pt]{article}

\usepackage[margin=0.75in]{geometry}
\usepackage{amsmath,amsthm,amssymb}
\usepackage{xcolor}
\usepackage{cancel}
\usepackage{graphicx}
\usepackage{changepage}
\usepackage{circuitikz}
\usepackage{pgfplots}
\usepackage{physics}
\usepackage{hyperref}
\usepackage{siunitx}
\usepackage[breakable]{tcolorbox}
\usepackage[inline]{enumitem}

\theoremstyle{definition}
\newtheorem{problem}{Problem}
\newtheorem{soln}{Solution}

\pgfplotsset{compat=newest}
\usetikzlibrary{lindenmayersystems}
\usetikzlibrary{arrows}
\usetikzlibrary{calc}

\definecolor{incolor}{HTML}{303F9F}
\definecolor{outcolor}{HTML}{D84315}
\definecolor{cellborder}{HTML}{CFCFCF}
\definecolor{cellbackground}{HTML}{F7F7F7}
\newcommand{\eq}{=}
\usetikzlibrary{positioning, fit, calc}
\pgfdeclarelayer{background}  
\pgfsetlayers{background,main}
\DeclareSIUnit[number-unit-product = {\,}]\calorie{cal}
\DeclareSIUnit[number-unit-product = {\,}]\atmosphere{atm}
\AtBeginDocument{\RenewCommandCopy\qty\SI}

\makeatletter
\newcommand{\boxspacing}{\kern\kvtcb@left@rule\kern\kvtcb@boxsep}
\makeatother
\newcommand{\prompt}[4]{
    \ttfamily\llap{{\color{#2}[#3]:\hspace{3pt}#4}}\vspace{-\baselineskip}
}

\newcommand{\thevenin}[2]{
  \begin{center}
    \begin{circuitikz} \draw
      (0,0) -- (2,0) to[battery1, l_=$V_{Th}\eq#1$] (2,2) 
      to[resistor, l_=$R_{Th}\eq#2$] (0,2)
      ;
      \draw [o-] (-.07,2.079);
      \draw [o-] (-.07,0.079);
    \end{circuitikz}
  \end{center}
}

\newcommand{\norton}[2]{
  \begin{center}
    \begin{circuitikz} \draw
      (0,0) -- (3,0) to[american current source, l_=$I_{N}\eq#1$] (3,2) -- (0,2) (2,0)
      to[resistor, l=$R_{N}\eq#2$] (2,2)
      ;
      \draw [o-] (-.07,2.079);
      \draw [o-] (-.07,0.079);
    \end{circuitikz}
  \end{center}
}

\newcommand{\highlight}[1]{\colorbox{yellow}{$\displaystyle #1$}}

\newcommand{\ti}[1]{\widetilde{#1}}

\title{Physics 2700H: Assignment II}
\author{Jeremy Favro \\\emph{Department of Physics \& Astronomy}\\ Trent University, Peterborough, ON, Canada}
\date{\today}

\begin{document}
\maketitle

% PROBLEM 1
\begin{problem}
Consider a mix of $N_2$ and $O_2$, which we may treat as an ideal gas, inside a car engine's cylinder that
follows the idealized Otto cycle. Assume points ${a,b,c,d}$ in Fig. 4.15(b)
correspond to $(V,P)$ values of: $\left\{(7V_1,\qty{1}{\atmosphere}), (V_1, \qty{15.25}{\atmosphere}), (V_1, \qty{30.50}{\atmosphere}), (7V_1, \qty{2}{\atmosphere})\right\}$,
respectively.
\begin{enumerate}[label=(\alph*)]
  \item Confirm that these $(V,P)$ values are consistent with the gas experiencing one adiabatic
        compression and one adiabatic expansion over each cycle.
  \item If the car's engine size (the displacement of here, \underline{four} cylinders) due to compression/expansion is $\qty{2.4}{\litre}$, what is $V_1$?
  \item Find the net work done by the gas in all four cylinders of the engine over one cycle (you may use the relevant expression for work from either p. 44 or p. 77 of the textbook).
\end{enumerate}
\end{problem}
\begin{soln} ~
  \begin{enumerate}[label=(\alph*)]
    \item Here the adiabats are along $\overrightarrow{ab}$ and $\overrightarrow{cd}$. For an adiabat $P_iV_i^\gamma=P_fV_f^\gamma$ holds so
          \begin{align*}
             & (7V_1)^\gamma\cdot\qty{1}{\atmosphere}=V_1^\gamma\cdot\qty{15.25}{\atmosphere}     \\
             & V_1^\gamma\cdot\qty{1544734.579}{\pascal}=V_1^\gamma\cdot\qty{1545206.25}{\pascal}
          \end{align*}
          Which is approximately adiabatic. For $\overrightarrow{cd}$
          \begin{align*}
             & V_1^\gamma\cdot\qty{30.50}{\atmosphere}=(7V_1)^\gamma\cdot\qty{2}{\atmosphere}    \\
             & V_1^\gamma\cdot\qty{3090412.5}{\pascal}=V_1^\gamma\cdot\qty{3089469.158}{\pascal}
          \end{align*}
          Which is also approximately adiabatic.
    \item The volume difference between compression and expansion \textbf{for a single cylinder} is $7V_1-V_1=6V_1$ so $4\cdot6V_1=\qty{2.4}{\liter}\implies V_1=\qty{0.1}{\liter}=\qty{1d-4}{\meter\cubed}$
    \item This formula gives work done on the gas, so we must flip the sign to determine work done by the gas
          \begin{align*}
            W_{\overrightarrow{ab}}= & -\frac{P_iV_i}{\gamma-1} \left[\left(\frac{V_i}{V_f}\right)^{\gamma-1}-1\right]                                                                                                 \\
            =                        & -\frac{\qty{7d-4}{\meter\cubed}\cdot\qty{101325}{\pascal}}{\frac{7}{5}-1} \left[\left(\frac{\qty{7d-4}{\meter\cubed}}{\qty{1d-4}{\meter\cubed}}\right)^{\frac{7}{5}-1}-1\right] \\
            \approx                  & \qty{-208.9}{\joule}
          \end{align*}
          \begin{align*}
            W_{\overrightarrow{cd}}= & -\frac{\qty{1d-4}{\meter\cubed}\cdot\qty{3090412.5}{\pascal}}{\frac{7}{5}-1} \left[\left(\frac{\qty{1d-4}{\meter\cubed}}{\qty{7d-4}{\meter\cubed}}\right)^{\frac{7}{5}-1}-1\right] \\
            \approx                  & \,\qty{417.9}{\joule}                                                                                                                                                              \\
          \end{align*}
          Net work for one cylinder is $W_{net}=\qty{417.9}{\joule}-\qty{208.9}{\joule}=\qty{209}{\joule}$. This means that net work in all four cylinders (in lock-step e.g. not on a camshaft, I think) is $\qty{836}{\joule}$
  \end{enumerate}
\end{soln}
\newpage

% PROBLEM 2
\begin{problem}
An inventor claims to have developed an engine that takes in $\qty{1.1d8}{\joule}$ at $\qty{400}{\kelvin}$,
rejects $\qty{5.07d7}{\joule}$ at $\qty{200}{\kelvin}$, and delivers $\qty{16.7}{\kilo\watt}$ hours of work.
Would you advise investing money in this project?
\end{problem}
\begin{soln}
  A Carnot engine operating in this environment has $\eta=1-\frac{T_2}{T_1}=1-\frac{\qty{200}{\kelvin}}{\qty{400}{\kelvin}}=0.5$ and this engine has efficiency
  $\eta=1-\frac{\qty{5.07d7}{\joule}}{\qty{1.1d8}{\joule}}\approx0.54$. As the efficiency of a ``real'' engine cannot exceed that of a Carnot engine I would not
  reccomend investing in this project.
\end{soln}

% PROBLEM 3
\begin{problem}
Suppose a house requires $\qty{4.3}{\giga\joule}$ of heating in a winter month. The
utility company charges \$0.14 per $\unit{\kilo\watt\hour}$.
\begin{enumerate}[label=(\alph*)]
  \item Find the cost savings of using a heat pump versus a 95\%-efficient natural gas furnace. Assume
        a Carnot heat pump with average temperatures of $\qty{20}{\degreeCelsius}$ indoors and $\qty{0}{\degreeCelsius}$ outdoors.
  \item Repeat part (a) using a more realistic coefficient of performance of 4.0 for the heat pump.
\end{enumerate}
\end{problem}
\begin{soln}~
  \begin{enumerate}[label=(\alph*)]
    \item $\eta_{furn}=0.95=\frac{W_{furn}}{Q_1}\implies 0.95Q_1=W_{furn}=\qty{4.085d9}{\joule}$ and $\eta_c=1-\frac{T_2}{T_1}=\frac{W_{HP}}{Q_1}=COP_{HP}^{-1}=14.6575$ which means $14.6575=\frac{Q_1}{W_{HP}}\implies W_{HP}=\frac{Q_1}{14.6575}\approx\qty{0.2933d9}{\joule}$
          so $\Delta W=W_{furn}-W_{HP}\approx\qty{3.792d9}{\joule}\approx\qty{1027.778}{\kilo\watt\hour}\approx143.89\$$. Note that this seems very high but this is an ideal heat pump working in decent conditions
          so it could be ``realistic''.
    \item The work done by the furnace stays the same so we end up with $\Delta W=W_{furn}-W_{HP}=W_{furn}-\frac{Q_1}{COP_{HP}}=\qty{3.01}{\joule}= 836.\overline{1}\approx 117.06\$$ which also seems quite high for a heat pump now operating with an apparently
    more realistic $COP$. Maybe the authors of this problem are just very out of touch.
  \end{enumerate}
\end{soln}

% PROBLEM 4
\begin{problem}
A hypothetical engine, with an ideal gas as the working substance, operates in the cycle shown in Figure 4.17. Show that the efficiency of the engine is
$$\eta=1-\frac{1}{\gamma}\left(\frac{1-\frac{P_3}{P_1}}{1-\frac{V_1}{V_3}}\right)$$
\end{problem}
\begin{soln}
  Here we know that $Q_1=nC_P\Delta T_{12}$ and $Q_2=nC_V\Delta T_{23}$ and heat across the adiabat is by definition zero. Then,
  \begin{align*}
    Q_1= & \, nC_P\left[T_2-T_1\right]=nC_P\left[\frac{P_1V_3}{nR}-\frac{P_1V_1}{nR}\right]=\frac{C_PP_1}{R}\left[V_3-V_1\right]
  \end{align*}
  And
  \begin{align*}
    Q_2= & \, nC_V\left[T_3-T_2\right]=nC_V\left[\frac{P_3V_3}{nR}-\frac{P_1V_3}{nR}\right]=\frac{C_VV_3}{R}\left[P_3-P_1\right]
  \end{align*}
  Then we know that $W=Q_1+Q_2$ because this is a cycle with $\Delta U=0$ and $\eta=\frac{W}{Q_1}=\frac{Q_1+Q_2}{Q_1}=1+\frac{Q_2}{Q_1}$ so,
  \begin{align*}
    \eta= & \,1+\frac{Q_2}{Q_1}=1+\frac{\frac{C_VV_3}{R}\left[P_3-P_1\right]}{\frac{C_PP_1}{R}\left[V_3-V_1\right]}=1+\frac{R}{C_PP_1}\frac{C_VV_3}{R}\frac{\left[P_3-P_1\right]}{\left[V_3-V_1\right]}                                                                                                      \\
    =     & \,1+\frac{1}{\gamma}\frac{V_3}{P_1}\frac{\left[P_3-P_1\right]}{\left[V_3-V_1\right]}=1+\frac{1}{\gamma}\frac{\left[\frac{P_3}{P_1}-\frac{P_1}{P_1}\right]}{\left[\frac{V_3}{V_3}-\frac{V_1}{V_3}\right]}=1+\frac{1}{\gamma}\frac{\left[\frac{P_3}{P_1}-1\right]}{\left[1-\frac{V_1}{V_3}\right]} \\
    =     & \,1-\frac{1}{\gamma}\frac{\left[1-\frac{P_3}{P_1}\right]}{\left[1-\frac{V_1}{V_3}\right]}
  \end{align*}
\end{soln}
\end{document}