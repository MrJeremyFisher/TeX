\documentclass[10pt]{article}

\usepackage[margin=0.75in]{geometry}
\usepackage{amsmath,amsthm,amssymb}
\usepackage{xcolor}
\usepackage{cancel}
\usepackage{graphicx}
\usepackage{changepage}
\usepackage{circuitikz}
\usepackage{pgfplots}
\usepackage{physics}
\usepackage{hyperref}
\usepackage{siunitx}
\usepackage[breakable]{tcolorbox}
\usepackage[inline]{enumitem}

\theoremstyle{definition}
\newtheorem{problem}{Problem}
\newtheorem{soln}{Solution}

\pgfplotsset{compat=newest}
\usetikzlibrary{lindenmayersystems}
\usetikzlibrary{arrows}
\usetikzlibrary{calc}

\definecolor{incolor}{HTML}{303F9F}
\definecolor{outcolor}{HTML}{D84315}
\definecolor{cellborder}{HTML}{CFCFCF}
\definecolor{cellbackground}{HTML}{F7F7F7}
\newcommand{\eq}{=}
\usetikzlibrary{positioning, fit, calc}
\pgfdeclarelayer{background}  
\pgfsetlayers{background,main}
\DeclareSIUnit[number-unit-product = {\,}]\calorie{cal}
\DeclareSIUnit[number-unit-product = {\,}]\atmosphere{atm}
\AtBeginDocument{\RenewCommandCopy\qty\SI}

\makeatletter
\newcommand{\boxspacing}{\kern\kvtcb@left@rule\kern\kvtcb@boxsep}
\makeatother
\newcommand{\prompt}[4]{
    \ttfamily\llap{{\color{#2}[#3]:\hspace{3pt}#4}}\vspace{-\baselineskip}
}

\newcommand{\thevenin}[2]{
  \begin{center}
    \begin{circuitikz} \draw
      (0,0) -- (2,0) to[battery1, l_=$V_{Th}\eq#1$] (2,2) 
      to[resistor, l_=$R_{Th}\eq#2$] (0,2)
      ;
      \draw [o-] (-.07,2.079);
      \draw [o-] (-.07,0.079);
    \end{circuitikz}
  \end{center}
}

\newcommand{\norton}[2]{
  \begin{center}
    \begin{circuitikz} \draw
      (0,0) -- (3,0) to[american current source, l_=$I_{N}\eq#1$] (3,2) -- (0,2) (2,0)
      to[resistor, l=$R_{N}\eq#2$] (2,2)
      ;
      \draw [o-] (-.07,2.079);
      \draw [o-] (-.07,0.079);
    \end{circuitikz}
  \end{center}
}

\newcommand{\highlight}[1]{\colorbox{yellow}{$\displaystyle #1$}}
\newcommand{\ti}[1]{\widetilde{#1}}
\newcommand{\dbar}{d\hspace*{-0.08em}\bar{}\hspace*{0.1em}}

\title{Physics 2700H: Assignment IV}
\author{Jeremy Favro (0805980) \\ Trent University, Peterborough, ON, Canada}
\date{\today}

\begin{document}
\maketitle

% PROBLEM 1
\begin{problem}
For atomic hydrogen, the allowed energy levels are given by the Bohr equation
$$E_n=-\frac{\qty{13.6}{\electronvolt}}{n^2}$$
which gives energies of $-13.6$, $-3.4$, and $\qty{1.5}{\electronvolt}$ for the first three energy levels.
Rework the example in Section 6.3.2 with atomic hydrogen at $\qty{7500}{\kelvin}$ using these three energy levels.
\begin{enumerate}[label=(\alph*)]
  \item Compute the partition function.
  \item Compute the probabilities of the first three levels.
  \item Compare your results with the example in the text.
\end{enumerate}
\end{problem}
\begin{soln}
\end{soln}

% PROBLEM 2
\begin{problem}
Suppose there is a quantized system that can be in one of three energy states, having energies $0$, $0.2$,
and $\qty{0.4}{\electronvolt}$, respectively. The system is at $\qty{5000}{\kelvin}$.
\begin{enumerate}[label=(\alph*)]
  \item Compute the partition function for this system.
  \item Find the mean energy.
  \item Compute the probability that each of the three states will be occupied.
\end{enumerate}
\end{problem}
\begin{soln}
\end{soln}

% PROBLEM 3
\begin{problem}
It is a result of statistical mechanics that the internal energy of an ideal gas is
$$U=U(S,V)=-Nk_B\left(\frac{N}{V}\right)^{2/3}\exp\left(\frac{2S}{3Nk_B}\right)$$
where $\alpha$ is a constant and the other symbols have their usual meanings.
Show that the equation of state $PV = nRT$ follows from this equation.
\end{problem}
\begin{soln}
  Where is $\alpha$ here?
\end{soln}

% PROBLEM 4
\begin{problem}
Water boils at $T = \qty{100}{\degreeCelsius}$ at one atmosphere of pressure. In the process, the entropy increase is
$\qty{109}{\joule\per\kelvin}$ for each mole of water. Find the molar enthalpy increase.
\end{problem}
\begin{soln}
\end{soln}
\end{document}