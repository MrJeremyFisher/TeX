\documentclass[10pt]{article}

\usepackage[margin=0.75in]{geometry}
\usepackage{amsmath,amsthm,amssymb}
\usepackage{xcolor}
\usepackage{cancel}
\usepackage{graphicx}
\usepackage{changepage}
\usepackage{circuitikz}
\usepackage{pgfplots}
\usepackage{physics}
\usepackage{hyperref}
\usepackage{siunitx}
\usepackage[breakable]{tcolorbox}
\usepackage[inline]{enumitem}

\theoremstyle{definition}
\newtheorem{problem}{Problem}
\newtheorem{soln}{Solution}

\pgfplotsset{compat=newest}
\usetikzlibrary{lindenmayersystems}
\usetikzlibrary{arrows}
\usetikzlibrary{calc}

\definecolor{incolor}{HTML}{303F9F}
\definecolor{outcolor}{HTML}{D84315}
\definecolor{cellborder}{HTML}{CFCFCF}
\definecolor{cellbackground}{HTML}{F7F7F7}
\newcommand{\eq}{=}
\usetikzlibrary{positioning, fit, calc}
\pgfdeclarelayer{background}  
\pgfsetlayers{background,main}
\DeclareSIUnit[number-unit-product = {\,}]\calorie{cal}
\DeclareSIUnit[number-unit-product = {\,}]\atmosphere{atm}
\AtBeginDocument{\RenewCommandCopy\qty\SI}

\makeatletter
\newcommand{\boxspacing}{\kern\kvtcb@left@rule\kern\kvtcb@boxsep}
\makeatother
\newcommand{\prompt}[4]{
    \ttfamily\llap{{\color{#2}[#3]:\hspace{3pt}#4}}\vspace{-\baselineskip}
}

\newcommand{\thevenin}[2]{
  \begin{center}
    \begin{circuitikz} \draw
      (0,0) -- (2,0) to[battery1, l_=$V_{Th}\eq#1$] (2,2) 
      to[resistor, l_=$R_{Th}\eq#2$] (0,2)
      ;
      \draw [o-] (-.07,2.079);
      \draw [o-] (-.07,0.079);
    \end{circuitikz}
  \end{center}
}

\newcommand{\norton}[2]{
  \begin{center}
    \begin{circuitikz} \draw
      (0,0) -- (3,0) to[american current source, l_=$I_{N}\eq#1$] (3,2) -- (0,2) (2,0)
      to[resistor, l=$R_{N}\eq#2$] (2,2)
      ;
      \draw [o-] (-.07,2.079);
      \draw [o-] (-.07,0.079);
    \end{circuitikz}
  \end{center}
}

\newcommand{\highlight}[1]{\colorbox{yellow}{$\displaystyle #1$}}
\newcommand{\ti}[1]{\widetilde{#1}}
\newcommand{\dbar}{d\hspace*{-0.08em}\bar{}\hspace*{0.1em}}

\title{Physics 2700H: Assignment V}
\author{Jeremy Favro (0805980) \\ Trent University, Peterborough, ON, Canada}
\date{\today}

\begin{document}
\maketitle

% PROBLEM 1
\begin{problem}
Show that the Joule-Kelvin coefficient is zero for an ideal gas.
\end{problem}
\begin{soln}
\end{soln}

% PROBLEM 2
\begin{problem}
At the critical point, $(\partial P/\partial V)T = 0$ and $(\partial^2 P/\partial V^2)T = 0$. Show that, for a
van der Waals gas (see Section 3.5.4), the critical point is at
$$P_c=\frac{a}{27b^2};V_c=3nb; T_c=\frac{8a}{27Rb}$$
\end{problem}
\begin{soln}
\end{soln}

% PROBLEM 3
\begin{problem}
Show that the chemical potential of an ideal gas at temperature $T$ varies with pressure as
$$\mu=k_B T\ln\left(\frac{P}{P_0}\right)+\mu_0$$
where $\mu_0$ is the value at reference point of pressure $P_0$ and temperature $T$. The gas consists of a single type of particle only.
This expression is of great use in chemistry.
\end{problem}
\begin{soln}
\end{soln}

% PROBLEM 4
\begin{problem}
The Sackur-Tetrode equation (similar to Equation 5.11) gives the entropy of an ideal gas as
$$S=Nk_B\left[\ln\left(\frac{V}{N}\left[\frac{4\pi mU}{3Nh^2}\right]^{3/2}\right)+\frac{5}{2}\right]$$
\begin{enumerate}[label=(\alph*)]
  \item Show that the chemical potential can be written in terms of entropy as
        $$\mu=-T\eval{\frac{\partial S}{\partial N}}_{U,V}$$
  \item Use the result of part (a) along with the fact that $U = (3/2)Nk_BT$ for
        a monatomic gas to find an expression for the chemical potential as
        a function of $V$, $N$, and $T$.
  \item Evaluate the result in (b) numerically for helium gas at $T = \qty{298}{\kelvin}$ and $P = \qty{1}{\atmosphere}$.
  \item Discuss the implications of the fact that your answer in (c) is negative.
\end{enumerate}
\end{problem}
\begin{soln}
\end{soln}
\end{document}