\documentclass[10pt]{article}

\usepackage[margin=0.75in]{geometry}
\usepackage{amsmath,amsthm,amssymb}
\usepackage{xcolor}
\usepackage{cancel}
\usepackage{graphicx}
\usepackage{changepage}
\usepackage{circuitikz}
\usepackage{pgfplots}
\usepackage{physics}
\usepackage{hyperref}
\usepackage{siunitx}
\usepackage[breakable]{tcolorbox}
\usepackage[inline]{enumitem}

\theoremstyle{definition}
\newtheorem{problem}{Problem}
\newtheorem{soln}{Solution}

\pgfplotsset{compat=newest}
\usetikzlibrary{lindenmayersystems}
\usetikzlibrary{arrows}
\usetikzlibrary{calc}

\definecolor{incolor}{HTML}{303F9F}
\definecolor{outcolor}{HTML}{D84315}
\definecolor{cellborder}{HTML}{CFCFCF}
\definecolor{cellbackground}{HTML}{F7F7F7}
\newcommand{\eq}{=}
\usetikzlibrary{positioning, fit, calc}
\pgfdeclarelayer{background}  
\pgfsetlayers{background,main}
\DeclareSIUnit[number-unit-product = {\,}]\calorie{cal}
\DeclareSIUnit[number-unit-product = {\,}]\atmosphere{atm}
\AtBeginDocument{\RenewCommandCopy\qty\SI}

\makeatletter
\newcommand{\boxspacing}{\kern\kvtcb@left@rule\kern\kvtcb@boxsep}
\makeatother
\newcommand{\prompt}[4]{
    \ttfamily\llap{{\color{#2}[#3]:\hspace{3pt}#4}}\vspace{-\baselineskip}
}

\newcommand{\thevenin}[2]{
  \begin{center}
    \begin{circuitikz} \draw
      (0,0) -- (2,0) to[battery1, l_=$V_{Th}\eq#1$] (2,2) 
      to[resistor, l_=$R_{Th}\eq#2$] (0,2)
      ;
      \draw [o-] (-.07,2.079);
      \draw [o-] (-.07,0.079);
    \end{circuitikz}
  \end{center}
}

\newcommand{\norton}[2]{
  \begin{center}
    \begin{circuitikz} \draw
      (0,0) -- (3,0) to[american current source, l_=$I_{N}\eq#1$] (3,2) -- (0,2) (2,0)
      to[resistor, l=$R_{N}\eq#2$] (2,2)
      ;
      \draw [o-] (-.07,2.079);
      \draw [o-] (-.07,0.079);
    \end{circuitikz}
  \end{center}
}

\newcommand{\highlight}[1]{\colorbox{yellow}{$\displaystyle #1$}}
\newcommand{\ti}[1]{\widetilde{#1}}
\newcommand{\dbar}{d\hspace*{-0.08em}\bar{}\hspace*{0.1em}}

\title{Physics 2700H: Assignment V}
\author{Jeremy Favro (0805980) \\ Trent University, Peterborough, ON, Canada}
\date{\today}

\begin{document}
\maketitle

% PROBLEM 1
\begin{problem}
Show that the Joule-Kelvin coefficient is zero for an ideal gas.
\end{problem}
\begin{soln}
  \begin{align*}
    \mu_{JK} & =\frac{1}{C_p}\left[T\left(\frac{\partial V}{\partial T}\right)_P-V\right] \\
             & =\frac{1}{C_p}\left[\frac{nRT}{P}-V\right]                                 \\
             & =\frac{1}{C_p}\left[\frac{nRT}{P}-\frac{nRT}{P}\right]                     \\
             & = 0
  \end{align*}
\end{soln}

% PROBLEM 2
\begin{problem}
At the critical point, $(\partial P/\partial V)_T = 0$ and $(\partial^2 P/\partial V^2)_T = 0$. Show that, for a
van der Waals gas (see Section 3.5.4), the critical point is at
$$P_c=\frac{a}{27b^2};V_c=3nb; T_c=\frac{8a}{27Rb}$$
\end{problem}
\begin{soln}
  $$(\partial P/\partial V)_T=\frac{2n^2a}{V^3}-\frac{nRT}{\left(V-nb\right)^2}$$
  and
  $$(\partial^2 P/\partial V^2)_T=\frac{2nRT}{\left(V-nb\right)^3}-\frac{6n^2a}{V^4}.$$
  Setting the first derivative to zero and solving for $T$ gives $T=\frac{2na\left(V-nb\right)^2}{RV^3}$. Doing
  the same for the second derivative gives $T=\frac{3na\left(V-nb\right)^3}{RV^4}$. Dividing these two gives
  \begin{align*}
    1        & = \frac{\frac{2na\left(V-nb\right)^2}{RV^3}}{\frac{3na\left(V-nb\right)^3}{RV^4}} \\
    1        & = \frac{2V}{3\left(V-nb\right)}                                                   \\
    \implies & 3nb  = V                                                                          \\
  \end{align*}
  which can be back substituted to get $T=\frac{2na\left(3nb-nb\right)^2}{R(3nb)^3}=\frac{8a}{27Rb}$ which can, in combination with the volume be back substituted into
  the original Van der Waals equation to get $P=\frac{nR\frac{8a}{27Rb}}{3nb-nb}-\frac{n^2a}{(3nb)^2}=\frac{a}{27b^2}$.
\end{soln}
\newpage

% PROBLEM 3
\begin{problem}
Show that the chemical potential of an ideal gas at temperature $T$ varies with pressure as
$$\mu=k_B T\ln\left(\frac{P}{P_0}\right)+\mu_0$$
where $\mu_0$ is the value at reference point of pressure $P_0$ and temperature $T$. The gas consists of a single type of particle only.
This expression is of great use in chemistry.
\end{problem}
\begin{soln}
  Using the hint from the text that $\left(\frac{\partial G}{\partial P}\right)_{T,N}=V\implies dG=VdP$ so $\int_{G_i}^{G_f}dG=\int_{P_i}^{P_f}VdP
    =\int_{P_i}^{P_f}\frac{nRT}{P}dP\implies Nk_BT\ln\left(\frac{P_i}{P_f}\right)+G_i=G_f$. From here we can differentiate with respect to $N$ and
  using $\left(\frac{\partial G}{\partial N}\right)_{T,P}=\mu$ we get $k_BT\ln\left(\frac{P_i}{P_f}\right)+\mu_i=\mu_f$ which is the same as the given expression
  if we treat the $f$ variables as the current ``position'' which is valid and just switch $i$ for $0$.
\end{soln}

% PROBLEM 4
\begin{problem}
The Sackur-Tetrode equation (similar to Equation 5.11) gives the entropy of an ideal gas as
$$S=Nk_B\left[\ln\left(\frac{V}{N}\left[\frac{4\pi mU}{3Nh^2}\right]^{3/2}\right)+\frac{5}{2}\right]$$
\begin{enumerate}[label=(\alph*)]
  \item Show that the chemical potential can be written in terms of entropy as
        $$\mu=-T\eval{\frac{\partial S}{\partial N}}_{U,V}$$
  \item Use the result of part (a) along with the fact that $U = (3/2)Nk_BT$ for
        a monatomic gas to find an expression for the chemical potential as
        a function of $V$, $N$, and $T$.
  \item Evaluate the result in (b) numerically for helium gas at $T = \qty{298}{\kelvin}$ and $P = \qty{1}{\atmosphere}$.
  \item Discuss the implications of the fact that your answer in (c) is negative.
\end{enumerate}
\end{problem}
\begin{soln}~
  \begin{enumerate}[label=(\alph*)]
    \item The derivative is $k_B\ln\left(\frac{V}{N}\left[\frac{2\pi mk_BT}{Nh^2}\right]^{3/2}\right)$ in its most simplified form (I think?)
          Where to go from there though I have no idea and I'm out of time.
    \item Substituting the given expression for $U$ we get $\mu=-k_BT\ln\left(\frac{V}{N}\left[\frac{2\pi mk_BT}{Nh^2}\right]^{3/2}\right)$.
    \item Using the ideal gas equation for one mole of helium ($m=\qty{4d-3}{\kilo\gram}$, $N=N_A\approx\num{6.02d23}$) we get $V\approx\qty{2.44d-2}{\meter\cubed}$
          which when plugged into the equation for $\mu$ we get
          $\mu=-k_B\qty{298}{\kelvin}\ln\left(\frac{\qty{2.44d-2}{\meter\cubed}}{N_A}\left[\frac{2\pi \qty{4d-3}{\kilo\gram}k_B\qty{298}{\kelvin}}{N_Ah^2}\right]^{3/2}\right)\approx
            \qty{-0.834}{\electronvolt\per\kilogram}$.
    \item The chemical potential being negative means that the internal energy of the gas will decrease if more helium is added and other state variables are held constant.
  \end{enumerate}
\end{soln}
\end{document}