\documentclass[10pt]{article}

\usepackage[margin=0.75in]{geometry}
\usepackage{amsmath,amsthm,amssymb}
\usepackage{xcolor}
\usepackage{cancel}
\usepackage{graphicx}
\usepackage{changepage}
\usepackage{circuitikz}
\usepackage{pgfplots}
%\usepackage{physics}
\usepackage{hyperref}
\usepackage{siunitx}
\usepackage[breakable]{tcolorbox}
\usepackage[inline]{enumitem}

\theoremstyle{definition}
\newtheorem{problem}{Problem}
\newtheorem{soln}{Solution}

\pgfplotsset{compat=newest}
\usetikzlibrary{lindenmayersystems}
\usetikzlibrary{arrows}
\usetikzlibrary{calc}

\definecolor{incolor}{HTML}{303F9F}
\definecolor{outcolor}{HTML}{D84315}
\definecolor{cellborder}{HTML}{CFCFCF}
\definecolor{cellbackground}{HTML}{F7F7F7}
\newcommand{\eq}{=}
\usetikzlibrary{positioning, fit, calc}
\pgfdeclarelayer{background}  
\pgfsetlayers{background,main}
\DeclareSIUnit[number-unit-product = {\,}]\calorie{cal}
\DeclareSIUnit[number-unit-product = {\,}]\atmosphere{atm}

\makeatletter
\newcommand{\boxspacing}{\kern\kvtcb@left@rule\kern\kvtcb@boxsep}
\makeatother
\newcommand{\prompt}[4]{
    \ttfamily\llap{{\color{#2}[#3]:\hspace{3pt}#4}}\vspace{-\baselineskip}
}

\newcommand{\thevenin}[2]{
  \begin{center}
    \begin{circuitikz} \draw
      (0,0) -- (2,0) to[battery1, l_=$V_{Th}\eq#1$] (2,2) 
      to[resistor, l_=$R_{Th}\eq#2$] (0,2)
      ;
      \draw [o-] (-.07,2.079);
      \draw [o-] (-.07,0.079);
    \end{circuitikz}
  \end{center}
}

\newcommand{\norton}[2]{
  \begin{center}
    \begin{circuitikz} \draw
      (0,0) -- (3,0) to[american current source, l_=$I_{N}\eq#1$] (3,2) -- (0,2) (2,0)
      to[resistor, l=$R_{N}\eq#2$] (2,2)
      ;
      \draw [o-] (-.07,2.079);
      \draw [o-] (-.07,0.079);
    \end{circuitikz}
  \end{center}
}

\newcommand{\highlight}[1]{\colorbox{yellow}{$\displaystyle #1$}}

\newcommand{\ti}[1]{\widetilde{#1}}

\hypersetup{
    colorlinks=true,
    linkcolor=blue,
    filecolor=magenta,      
    urlcolor=cyan,
    pdftitle={Overleaf Example},
    pdfpagemode=FullScreen,
    }

\NewDocumentCommand{\evalat}{sO{\big}mm}{%
  \IfBooleanTF{#1}
   {\mleft. #3 \mright|_{#4}}
   {#3#2|_{#4}}%
}

\title{Physics 2700H: Assignment I}
\author{Jeremy Favro}
\date{\today}

\begin{document}
\maketitle

% PROBLEM 1
\begin{problem}
  An ideal gas undergoes the following reversible cycle:
  % \begin{tikzpicture}
  %   \tikzstyle{block} = [rectangle, text centered, draw=black, fill=orange!30]
  %   \node (i) [block] {$(P_1,V_1)$};
  %   \node (ii) [block, below right of=i, xshift=0.75cm]{$(P_1,V_2)$};
  %   \node (iii) [block, below left of=ii, xshift=-0.75cm]{$(P_2,V_2)$};
  %   \node (iv) [block, above left of=iii]{$(P_2,V_1)$};
  % \end{tikzpicture}
  \begin{enumerate}[label=(\roman*)]
    \item an isobaric expansion from the state (P1,V1) to the state (P1,V2)
    \item an isochoric reduction in pressure to the state (P2,V2)
    \item an isobaric reduction in volume to the state (P2,V1)
    \item an isochoric increase in pressure back to the original state (P1,V1)
  \end{enumerate}
  \begin{enumerate}[label=(\alph*)]
    \item What work is done on the gas in this cycle?
    \item If P1 = 3.0 atm, P2 = 1.0 atm, V1 = 1.0 L and V2 = 2.0 L, how much work is done on the gas in traversing the cycle 100 times?
  \end{enumerate}
\end{problem}
\begin{soln}
\end{soln}

% PROBLEM 2
\begin{problem}
A hypothetical substance has an isothermal compressibility $\kappa = \frac{a}{v}$ and volume expansion coefficient $\beta=\frac{2bT}{v}$, 
where $a$ and $b$ are constants and $v$ is the molar volume. Show that the equation of state is $$v-bT^2+aP=\mathrm{constant}$$
\end{problem}
\begin{soln}
\end{soln}

% PROBLEM 3
\begin{problem}
  Researchers from the Universities of Maryland and Vermont found that adults' daily energy needs at rest (their resting metabolic rate, RMR) 
  is closely related to the mass of their bones \& muscles \& organs. By selecting any person from this data set, find:
  \begin{enumerate}[label=(\alph*)]
    \item Their total bone/muscle/organs mass and their RMR in $\unit{\kilo\calorie\per\day}$.
    \item Their minimum daily energy needs written in $\unit{\joule}$, in $\unit{\kilo\joule}$, and in $\unit{\mega\joule}$.
    \item Their heat output at rest in watts (to high accuracy all energy we use at rest ends up emitted as heat).
  \end{enumerate}
\end{problem}
\begin{soln}
\end{soln}

% PROBLEM 4
\begin{problem}
  A gas is contained in a cylinder fitted with a frictionless piston and is taken from state a to state b along the path acb shown in Figure 3.8.
  $80\unit{\joule}$ of heat flows into the system, and the system does $30\unit{\joule}$ of work.
  \begin{enumerate}[label=(\alph*)]
    \item If instead the work done by the gas system is only $10\unit{\joule}$ along adb, how much heat flows into the system?
    \item When the system is returned from b to a along the curved path, the work done on the system is $20\unit{\joule}$. What is the heat transfer?
    \item If $U_a = 0$ and $U_d = 40\unit{\joule}$, find the heat absorbed in the processes ad and db.
  \end{enumerate}
\end{problem}
\begin{soln}
\end{soln}

% PROBLEM 5
\begin{problem}
  You have a pure metal that is unknown except for the fact that it happens to be among those listed in Table 3.1. A $36.7\unit{\gram}$ piece of this metal
  at $50.0\unit{\celsius}$ is placed in a calorimeter containing $150\unit{\gram}$ of water, initially at $10\unit{\celsius}$. The final equilibrium temperature in the calorimeter is $2.0\unit{\celsius}$. 
  What is the metal?
\end{problem}
\begin{soln}
\end{soln}

% PROBLEM 6
\begin{problem}
  A gas with adiabatic exponent $\gamma$ is compressed adiabatically from an initial state $(P_i, V_i)$ to final state $(P_f, V_f)$. 
  \begin{enumerate}[label=(\alph*)]
    \item Show that the work done in this process is $W=\displaystyle\frac{P_iV_i}{\gamma-1}\left[\left(\frac{V_i}{V_f}\right)^{\gamma-1}-1\right]$
    \item Evaluate the result numerically for one mole of helium gas initially at $P=1.0\unit{\atmosphere}$ and $T=300\unit{\kelvin}$ compressed to half its initial volume. 
    \item Compute the work done in an isothermal compression from the same initial point to half the initial volume. 
    Explain the difference between the numerical results for work done in adiabatic and isothermal compression.
  \end{enumerate}
\end{problem}
\begin{soln}
\end{soln}
\end{document}