\documentclass[10pt,landscape,letterpaper]{cheatsheet}

\usepackage{amsmath,amsthm,amssymb}
\usepackage{graphicx,lipsum}
\usepackage{enumitem}
\usepackage{mdframed}
\usepackage{siunitx}
\usepackage{tikz}
\usepackage{circuitikz}
\usepackage{cancel}

% Tikz Setup
\usetikzlibrary{positioning, fit, calc}

% Quickies
\newcommand{\eq}{=}

% Circuit Macros
\newcommand{\thevenin}[2]{
  \begin{center}
    \begin{circuitikz} \draw
      (0,0) -- (2,0) to[battery1, l_=$V_{Th}\eq#1$] (2,2) 
      to[resistor, l_=$R_{Th}\eq#2$] (0,2)
      ;
      \draw [o-] (-.07,2.079);
      \draw [o-] (-.07,0.079);
    \end{circuitikz}
  \end{center}
}

\newcommand{\norton}[2]{
  \begin{center}
    \begin{circuitikz} \draw
      (0,0) -- (3,0) to[american current source, l_=$I_{N}\eq#1$] (3,2) -- (0,2) (2,0)
      to[resistor, l=$R_{N}\eq#2$] (2,2)
      ;
      \draw [o-] (-.07,2.079);
      \draw [o-] (-.07,0.079);
    \end{circuitikz}
  \end{center}
}

% Title
\title{Electronics Cheat Sheet}
\author{Jeremy Favro}
\date{\today\\Revision 1}

\begin{document}

\maketitle

\section*{Miscellaneous}
\subsection*{Time Constants}
For capacitors
$V_{C}=V_s\left[1-e^{-\frac{t}{RC}}\right]$
$V_{R}=V_se^{-\frac{t}{RC}}$ \\
And for inductors
$V_{R}=V_s\left[1-e^{-\frac{tR}{L}}\right]$
$V_{L}=V_se^{-\frac{tR}{L}}$
\subsection*{Th\'evenin}
Solve circuit for voltage across port ($V_A-V_B$), solve for equivalent resistance by replacing batteries with wires and current sources with opens and trace a current path from A to B.
Put the voltage source in \textbf{series} with the equivalent resistor.
\subsection*{Norton}
Short the port and find the current through the port short. Find resistance using the same method as Th\'evenin.
Put the current source in \textbf{parallel} (across the port) with the equivalent resistor.
\subsection*{Differentials}
$V_R=IR=R\frac{dQ}{dt}$\\
$V_L=L\frac{dI}{dt}=L\frac{d^2Q}{dt^2}$\\
$V_C=\frac{Q}{C}$\\
\subsection*{Diode}
\begin{circuitikz}[scale=0.5]
  \pgfextra{\ctikzset{bipoles/resistor/width=.4,
  bipoles/resistor/height=.15}}
  % PN Contact
  \draw[fill=black!15, draw=black, line width = 0.2mm] (-3.5,0) rectangle (-2.5,1);
  \draw[line width = 0.2mm] (-2.5,0) rectangle (-1.5,1);
  \node at (-3,0.5) {p};
  \node at (-2,0.5) {n};
  \draw (-3.5,0.5) -- (-4,0.5) node[left] {A};
  \draw (-1.5,0.5) -- (-1,0.5) node[right] {C};
  \draw [-{Triangle[round]}] (-3,-0.5) -- (-2,-0.5) node[pos=0.5,below] {$I$};
  
  % Diode
  \draw (1,0.5) node[left] {A} to[diode] (3,0.5) node[right] {C};
  \draw [-{Triangle[round]}] (1.5,-0.5) -- (2.5,-0.5) node[pos=0.5,below] {$I$};		
  
  \end{circuitikz}

\section*{Op-Amps}
\subsection*{Inverting Amplifier}
\begin{circuitikz}
  \draw 
  \pgfextra{\ctikzset{bipoles/resistor/width=.4,
      bipoles/resistor/height=.15}}
  node[op amp, scale=0.5](A){A} % (a -| b) takes the y of a and the x of b!
  (A.-) to[resistor, l_=$R_i$] ++(-1,0) node[left]{$V_i$} 
  (A.+) node[ground]{}
  (A.-) -- ++(0,0.5) coordinate(RF) to[resistor, l=$R_f$]  (RF -| A.out) -- (A.out) node[right]{$V_o=-\frac{R_f}{R_i}V_i$}
  ;
\end{circuitikz}
\subsection*{Non-Inverting Amplifier}
\begin{circuitikz}
  \draw 
  \pgfextra{\ctikzset{bipoles/resistor/width=.4,
      bipoles/resistor/height=.15}}
  node[op amp, scale=0.5](A){A} % (a -| b) takes the y of a and the x of b!
  (A.-) to[resistor, l_=$R_i$] ++(-1,0) node[ground]{}
  (A.+) node[left]{$V_i$}
  (A.-) -- ++(0,0.5) coordinate(RF) to[resistor, l=$R_f$]  (RF -| A.out) -- (A.out) node[right]{$V_o=\left(1+\frac{R_f}{R_i}\right)V_i$}
  ;
\end{circuitikz}
\subsection*{Voltage Follower}
\begin{circuitikz}
  \draw 
  \pgfextra{\ctikzset{bipoles/resistor/width=.4,
      bipoles/resistor/height=.15}}
  node[op amp, scale=0.5](A){A} % (a -| b) takes the y of a and the x of b!
  (A.+) node[left]{$V_i$}
  (A.-) -- ++(0,0.5) coordinate(RF) --  (RF -| A.out) -- (A.out) node[right]{$V_o=V_i$}
  ;
\end{circuitikz}
\subsection*{(Inverting) Adder}
\begin{circuitikz}
  \draw 
  \pgfextra{\ctikzset{bipoles/resistor/width=.4,
      bipoles/resistor/height=.15}}
  node[op amp, scale=0.5](A){A} % (a -| b) takes the y of a and the x of b!
  (A.-) -- ++(-.25,0) coordinate(J1) to[resistor, l_=$R_2$] ++(-1,0) node[left]{$V_2$}
  (J1) -- ++(0,0.6) to[resistor, l_=$R_1$] ++(-1,0) node[left]{$V_1$}
  (J1) -- ++(0,-0.5) to[resistor, l=$R_3$] ++(-1,0) node[left]{$V_3$}
  (A.+) node[ground]{}
  (A.-) -- ++(0,0.5) coordinate(RF) to[resistor, l=$R_f$]  (RF -| A.out) -- (A.out) node[right]{$V_o$}
  ;
\end{circuitikz}
$V_o=-R_f\left(\frac{V_1}{R_1}+\frac{V_2}{R_2}+\frac{V_3}{R_3}\right)$
\subsection*{Subtractor}
\begin{circuitikz}
  \draw 
  \pgfextra{\ctikzset{bipoles/resistor/width=.4,
      bipoles/resistor/height=.15}}
  node[op amp, scale=0.5](A){A} % (a -| b) takes the y of a and the x of b!
  (A.-) to[resistor, l_=$R_1$] ++(-1,0) node[left]{$V_1$} 
  (A.+) to[resistor, l=$R_2$] ++(-1,0) node[left]{$V_2$} 
  (A.-) -- ++(0,0.5) coordinate(RF) to[resistor, l=$R_f$]  (RF -| A.out) -- (A.out) node[right]{$V_o=\frac{R_f}{R_i}\left(V_2-V_1\right)$}
  (A.+) node[ground]{}
  ;
\end{circuitikz}
\subsection*{Integrator}
\begin{circuitikz}
  \draw 
  \pgfextra{\ctikzset{bipoles/resistor/width=.4,
      bipoles/resistor/height=.15, capacitors/scale=0.5}}
  node[op amp, scale=0.5](A){A} % (a -| b) takes the y of a and the x of b!
  (A.-) to[resistor, l_=$R_i$] ++(-1,0) node[left]{$V_i$} 
  (A.-) -- ++(0,0.5) coordinate(RF) to[capacitor, l=$C$]  (RF -| A.out) -- (A.out) node[right]{$V_o$}
  (A.+) node[ground]{}
  ;
\end{circuitikz}
$V_o=-\displaystyle\frac{1}{R_iC}\int_{0}^{t} V_i \, dt+V_o(t=0)$
\subsection*{Differentiator}
\begin{circuitikz}
  \draw 
  \pgfextra{\ctikzset{bipoles/resistor/width=.4,
      bipoles/resistor/height=.15, capacitors/scale=0.5}}
  node[op amp, scale=0.5](A){A} % (a -| b) takes the y of a and the x of b!
  (A.-) to[capacitor, l_=$C$] ++(-1,0) node[left]{$V_i$} 
  (A.-) -- ++(0,0.5) coordinate(RF) to[resistor, l=$R_f$]  (RF -| A.out) -- (A.out) node[right]{$V_o=-RC\frac{dV_i}{dt}$}
  (A.+) node[ground]{}
  ;
\end{circuitikz}
\end{document}