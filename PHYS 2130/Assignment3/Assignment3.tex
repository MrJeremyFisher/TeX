\documentclass[10pt]{article}

\usepackage[margin=0.75in]{geometry}
\usepackage{amsmath,amsthm,amssymb}
\usepackage{xcolor}
\usepackage{cancel}
\usepackage{graphicx}
\usepackage{changepage}
\usepackage{circuitikz}
\usepackage{pgfplots}
\usepackage{subcaption}
\usepackage{physics}
\usepackage{siunitx}
\usepackage{minted}
\usepackage[breakable]{tcolorbox}
\usepackage[inline]{enumitem}

\theoremstyle{definition}
\newtheorem{problem}{Problem}
\newtheorem{soln}{Solution}

\pgfplotsset{compat=newest}
\usetikzlibrary{lindenmayersystems}
\usetikzlibrary{arrows}

\definecolor{incolor}{HTML}{303F9F}
\definecolor{outcolor}{HTML}{D84315}
\definecolor{cellborder}{HTML}{CFCFCF}
\definecolor{cellbackground}{HTML}{F7F7F7}
\newcommand{\eq}{=}
\tikzset
{%
  axes/.style={thick,-latex},
  cylinder/.style={right color=blue!80,left color=white,fill opacity=0.7},
  paraboloid back/.style={left color=magenta!80,fill opacity=0.4},
  paraboloid front/.style={left color=white, right color=magenta!80,fill opacity=0.4},
}       

\makeatletter
\newcommand{\boxspacing}{\kern\kvtcb@left@rule\kern\kvtcb@boxsep}
\makeatother
\newcommand{\prompt}[4]{
    \ttfamily\llap{{\color{#2}[#3]:\hspace{3pt}#4}}\vspace{-\baselineskip}
}

\newcommand{\highlight}[1]{\colorbox{yellow}{$\displaystyle #1$}}

\NewDocumentCommand{\evalat}{sO{\big}mm}{%
  \IfBooleanTF{#1}
   {\mleft. #3 \mright|_{#4}}
   {#3#2|_{#4}}%
}

\title{Physics 2130: Assignment III}
\author{Jeremy Favro}
\date{\today}

\begin{document}

\maketitle

% PROBLEM 1
\begin{problem}
Consider the driven, damped harmonic oscillator. Its equation of motion is:
$$\ddot{x}+2\beta\dot{x}+\omega_0^2=A\cos\left(\omega t\right)$$
In the case of an underdamped oscillator, i.e., $\beta<\omega_0^2$ we found that the solution for the equation of motion is:
$$x(t)=x_c(t)+x_p(t)$$
Where:
$$x_c(t)e^{-\beta t}\left[c_1e^{i\omega_1 t}+c_2e^{-i\omega_1 t}\right]=Be^{-\beta t}\cos\left(\omega_1 t-\phi\right)=\Gamma(t)s(t)$$
And where:
$$\omega_1=\sqrt{w_0^2-\beta^2}$$
$$\Gamma(t)=Be^{-\beta t}$$
$$s(t)=\left(\omega_1 t-\phi\right)$$
The particular solution is instead:
$$x_p(t)=D\cos\left(\omega t-\delta\right)$$
Where:
$$D=\frac{A}{\sqrt{\left(\omega_0^2-\omega^2\right)^2+4\omega^2\beta^2}}$$
$$\delta=\arctan\left(\frac{2\omega\beta}{\omega_0^2-\omega^2}\right)$$
\begin{enumerate}[label=(\alph*)]
      \item Consider an underdamped driven oscillator that starts with the initial conditions $x(t=0)=x_0$ and $\dot{x}(t=0)=0$.
            Find the analytical expressions for the unknown coefficients in $x(t)$ using these initial conditions.

      \item Write a python program that returns (and prints) the values of $\omega_1$, $\beta$, $\phi$, $D$ and $\delta$ for a given $x_0$. This should
            be coded as a function called \verb|harm_osc_params| that accepts as inputs $\omega_0$, $\beta$, $A$, $\omega$ and $x_0$.

      \item In the same python program, now write a new function, \verb|harm_osc_x_pos|, that calculates the array of positions $x$ of the
            harmonic oscillator for a given array of times $t$. This function should receive the array $t$ as an input as well as the values of $\omega_0$,
            $\beta$, $A$, $\omega$ and $x_0$. It should call the previously written function \verb|harm_osc_params| for the calculations of all the oscillation parameters.
            It should return the array of positions $x$ of the harmonic oscillator for each value of $t$.

      \item In the same python program, now write a new function to plot the data, \verb|harm_osc_x_plot_single|. The
            function receives as inputs the arrays $t$ and $x$ generated at the previous step.

      \item Pick three values of $\beta$ (remember of the constraint $\beta<w_0$) and plot in the same graph $x(t)$ for the three
            chosen values. For reference use $w_0=1\unit{\per\second}$ and $A=1\unit{\per\second\squared}$ (but play with the values of $A$ to see the relative
            importance of the driver and the damping). Comment on what effect $\beta$ has on both the transient and steady- state solution.

      \item To help you distinguish the different effects, now write a function called \verb|harm_osc_damp_drive| that plots
            in the same graph, $s(t)$, $\Gamma(t)$, $x_c(t)$ and $x_p(t)$. Comment on the results, specifically on the contribution of each component.

      \item Finally, write a new function called \verb|harm_osc_euler_cromer| that receives has input parameters $\omega_0$, $\beta$,
            $A$, $\omega$, $x_0$, and the analytical solution $x(t)$. This function should calculate a new $x(t)$, called $x_{EC}(t)$, that uses
            the Euler-Cromer method to determine the position of the oscillator as a function of time for the same initial conditions. Additionally, the function should
            plot, on the same graph, the solutions $x(t)$ and $x_{EC}(t)$ obtained with the analytical and Euler-Cromer methods, respectively, and the residuals, i.e.,
            the difference $x(t)-x_{EC}(t)$. Discuss how you choose a value of $\Delta t$ that gives a sufficiently accurate answer (which means defining
            ``sufficiently accurate'').
\end{enumerate}
\end{problem}
\begin{soln}
      \begin{enumerate}[label=(\alph*)]
            \item \begin{align*}
                         & x(t=0)=B\cos\left(-\phi\right)+D\cos\left(-\delta\right)=x_0             \\
                         & \Rightarrow x_0-D\cos\left(\delta\right)=B\cos\left(\phi\right)          \\
                         & \Rightarrow \frac{x_0-D\cos\left(\delta\right)}{\cos\left(\phi\right)}=B \\
                  \end{align*}
                  \begin{align*}
                         & \dot{x}(t=0)=-B\omega_1\sin\left(-\phi\right)-B\beta\cos\left(-\phi\right)-D\omega\sin\left(-\delta\right)=0                                                                                                 \\
                         & =B\omega_1\sin\left(\phi\right)-B\beta\cos\left(\phi\right)+D\omega\sin\left(\delta\right)                                                                                                                   \\
                         & =\frac{x_0-D\cos\left(\delta\right)}{\cos\left(\phi\right)}\omega_1\sin\left(\phi\right)-\frac{x_0-D\cos\left(\delta\right)}{\cos\left(\phi\right)}\beta\cos\left(\phi\right)+D\omega\sin\left(\delta\right) \\
                         & =\left(x_0-D\cos\left(\delta\right)\right)\omega_1\tan\left(\phi\right)-\left(x_0-D\cos\left(\delta\right)\right)\beta+D\omega\sin\left(\delta\right)                                                        \\
                         & \Rightarrow -\frac{D\omega\sin\left(\delta\right)}{\left(x_0-D\cos\left(\delta\right)\right)}=\omega_1\tan\left(\phi\right)-\beta                                                                            \\
                         & \Rightarrow \arctan\left(\frac{\beta-\frac{D\omega\sin\left(\delta\right)}{\left(x_0-D\cos\left(\delta\right)\right)}}{\omega_1}\right)=\phi                                                                 \\
                  \end{align*}

            \item \inputminted[breaklines, autogobble]{python3}{./python/q1/q1b.py} % TODO: Figure out imports or something idk.
                  \newpage
            \item \inputminted[breaklines, autogobble]{python3}{./python/q1/q1c.py}

            \item In the same python program, now write a new function to plot the data, \verb|harm_osc_x_plot_single|. The
                  function receives as inputs the arrays $t$ and $x$ generated at the previous step.

            \item Pick three values of $\beta$ (remember of the constraint $\beta<w_0$) and plot in the same graph $x(t)$ for the three
                  chosen values. For reference use $w_0=1\unit{\per\second}$ and $A=1\unit{\per\second\squared}$ (but play with the values of $A$ to see the relative
                  importance of the driver and the damping). Comment on what effect $\beta$ has on both the transient and steady- state solution.

            \item To help you distinguish the different effects, now write a function called \verb|harm_osc_damp_drive| that plots
                  in the same graph, $s(t)$, $\Gamma(t)$, $x_c(t)$ and $x_p(t)$. Comment on the results, specifically on the contribution of each component.

            \item Finally, write a new function called \verb|harm_osc_euler_cromer| that receives has input parameters $\omega_0$, $\beta$,
                  $A$, $\omega$, $x_0$, and the analytical solution $x(t)$. This function should calculate a new $x(t)$, called $x_{EC}(t)$, that uses
                  the Euler-Cromer method to determine the position of the oscillator as a function of time for the same initial conditions. Additionally, the function should
                  plot, on the same graph, the solutions $x(t)$ and $x_{EC}(t)$ obtained with the analytical and Euler-Cromer methods, respectively, and the residuals, i.e.,
                  the difference $x(t)-x_{EC}(t)$. Discuss how you choose a value of $\Delta t$ that gives a sufficiently accurate answer (which means defining
                  ``sufficiently accurate'').
      \end{enumerate}
\end{soln}
\end{document}