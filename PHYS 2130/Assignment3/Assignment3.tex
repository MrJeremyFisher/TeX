\documentclass[10pt]{article}

\usepackage[margin=0.75in]{geometry}
\usepackage{amsmath,amsthm,amssymb}
\usepackage{xcolor}
\usepackage{cancel}
\usepackage{graphicx}
\usepackage{changepage}
\usepackage{circuitikz}
\usepackage{pgfplots}
\usepackage{subcaption}
\usepackage{physics}
\usepackage{siunitx}
\usepackage{minted}
\usepackage[breakable]{tcolorbox}
\usepackage[inline]{enumitem}

\theoremstyle{definition}
\newtheorem{problem}{Problem}
\newtheorem{soln}{Solution}

\pgfplotsset{compat=newest}
\usetikzlibrary{lindenmayersystems}
\usetikzlibrary{arrows}

\definecolor{incolor}{HTML}{303F9F}
\definecolor{outcolor}{HTML}{D84315}
\definecolor{cellborder}{HTML}{CFCFCF}
\definecolor{cellbackground}{HTML}{F7F7F7}
\newcommand{\eq}{=}
\tikzset
{%
  axes/.style={thick,-latex},
  cylinder/.style={right color=blue!80,left color=white,fill opacity=0.7},
  paraboloid back/.style={left color=magenta!80,fill opacity=0.4},
  paraboloid front/.style={left color=white, right color=magenta!80,fill opacity=0.4},
}       

\makeatletter
\newcommand{\boxspacing}{\kern\kvtcb@left@rule\kern\kvtcb@boxsep}
\makeatother
\newcommand{\prompt}[4]{
    \ttfamily\llap{{\color{#2}[#3]:\hspace{3pt}#4}}\vspace{-\baselineskip}
}

\newcommand{\highlight}[1]{\colorbox{yellow}{$\displaystyle #1$}}

\NewDocumentCommand{\evalat}{sO{\big}mm}{%
  \IfBooleanTF{#1}
   {\mleft. #3 \mright|_{#4}}
   {#3#2|_{#4}}%
}

\title{Physics 2130: Assignment III}
\author{Jeremy Favro}
\date{\today}

\begin{document}

\maketitle

% PROBLEM 1
\begin{problem}
Consider the driven, damped harmonic oscillator. Its equation of motion is:
$$\ddot{x}+2\beta\dot{x}+\omega_0^2=A\cos\left(\omega t\right)$$
In the case of an underdamped oscillator, i.e., $\beta<\omega_0^2$ we found that the solution for the equation of motion is:
$$x(t)=x_c(t)+x_p(t)$$
Where:
$$x_c(t)e^{-\beta t}\left[c_1e^{i\omega_1 t}+c_2e^{-i\omega_1 t}\right]=Be^{-\beta t}\cos\left(\omega_1 t-\phi\right)=\Gamma(t)s(t)$$
And where:
$$\omega_1=\sqrt{w_0^2-\beta^2}$$
$$\Gamma(t)=Be^{-\beta t}$$
$$s(t)=\left(\omega_1 t-\phi\right)$$
The particular solution is instead:
$$x_p(t)=D\cos\left(\omega t-\delta\right)$$
Where:
$$D=\frac{A}{\sqrt{\left(\omega_0^2-\omega^2\right)^2+4\omega^2\beta^2}}$$
$$\delta=\arctan\left(\frac{2\omega\beta}{\omega_0^2-\omega^2}\right)$$
\end{problem}
\begin{soln}
      \begin{align*}
             & x(t=0)=B\cos\left(-\phi\right)+D\cos\left(-\delta\right)=x_0             \\
             & \Rightarrow x_0-D\cos\left(\delta\right)=B\cos\left(\phi\right)          \\
             & \Rightarrow \frac{x_0-D\cos\left(\delta\right)}{\cos\left(\phi\right)}=B \\
      \end{align*}
      \begin{align*}
             & \dot{x}(t=0)=-B\omega_1\sin\left(-\phi\right)-B\beta\cos\left(-\phi\right)-D\omega\sin\left(-\delta\right)=0                                                                                                 \\
             & =B\omega_1\sin\left(\phi\right)-B\beta\cos\left(\phi\right)+D\omega\sin\left(\delta\right)                                                                                                                   \\
             & =\frac{x_0-D\cos\left(\delta\right)}{\cos\left(\phi\right)}\omega_1\sin\left(\phi\right)-\frac{x_0-D\cos\left(\delta\right)}{\cos\left(\phi\right)}\beta\cos\left(\phi\right)+D\omega\sin\left(\delta\right) \\
             & =\left(x_0-D\cos\left(\delta\right)\right)\omega_1\tan\left(\phi\right)-\left(x_0-D\cos\left(\delta\right)\right)\beta+D\omega\sin\left(\delta\right)                                                        \\
             & \Rightarrow -\frac{D\omega\sin\left(\delta\right)}{\left(x_0-D\cos\left(\delta\right)\right)}=\omega_1\tan\left(\phi\right)-\beta                                                                            \\
             & \Rightarrow \arctan\left(\frac{\beta-\frac{D\omega\sin\left(\delta\right)}{\left(x_0-D\cos\left(\delta\right)\right)}}{\omega_1}\right)=\phi                                                                 \\
      \end{align*}
\end{soln}
\end{document}