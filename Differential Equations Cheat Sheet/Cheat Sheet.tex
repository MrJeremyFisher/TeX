\documentclass[10pt,landscape,letterpaper]{cheatsheet}

\usepackage{amsmath,amsthm,amssymb}
\usepackage{graphicx,lipsum}
\usepackage{enumitem}
\usepackage{xcolor}
\newcommand{\PRLsep}{\noindent\makebox[\linewidth]{\resizebox{0.3333\linewidth}{1pt}{$\bullet$}}\bigskip}
\newcommand\longdiv[2]{%
$\strut#1$\kern.25em\smash{\raise.3ex\hbox{$\big)$}}$\mkern-8mu
        \overline{\enspace\strut#2}$}
\usepackage{mdframed}
\usepackage{cancel}

\newcommand{\laplace}[1]{\mathcal{L}\left\{#1\right\}}
\newcommand{\laplacei}[1]{\mathcal{L}^{-1}\left\{#1\right\}}

\newcommand{\highlight}[1]{\colorbox{yellow}{$\displaystyle #1$}}

\title{Diff. Eq. Cheat Sheet}
\author{Jeremy Favro}
\date{\today\\Revision 3}

\begin{document}

\maketitle

\section*{Fundamentals}
\subsection*{Classification}
$\frac{d^ny}{dx^n}=f(x,y)$ denotes an ODE of order $n$. Note that $(\frac{dy}{dx})^n\neq\frac{d^ny}{dx^n}$. ODEs of order $n$ will have $n$ constants in their general form solutions.\\
A linear ODE is one which can be written in the form $a_n(x)\frac{d^ny}{d^nx}+a_{n-1}\frac{d^{n-1}y}{d^{n-1}x}+\dots+a_1(x)\frac{dy}{dx}+a_0(x)y=g(x)$.\\
\subsection*{Solutions}
Given some IVP $\frac{dy}{dx}=f(x,y), \, y(x_0)=y_0$ if $f$ and $\frac{\partial f}{\partial y}$ are continuous in the rectangle $(x_0,y_0)\in \{(x,y):a<x<b,c<y<d\}$ then the
IVP has a unique solution $\phi(x)$ in some interval $(x_0-h,x_0+h), \, h\geq 0$


\section*{Solution Techniques $n=1$}
\subsection*{Direct Integration}
Directly integrate \dots
\subsection*{Seperable}
For some ODE $\frac{dy}{dx}=f(x,y)=g(x)p(y)$ the differential can be split $s.t.$ $\frac{1}{p(y)}dy=g(x)dx$ which can be solved by direct integration.
Note that when dividing by some function we assume that the function is nonzero. If there is a case (e.g. in an IVP) where the function is zero, the solution
is lost.
\subsection*{Linear}
For some linear ODE of the form $\frac{dy}{dx}+P(x)y=Q(x)$ we can multiply both sides of the ODE by $\mu(x)=\exp\left(\int P(x)\, dx\right)$
to obtain $\mu\frac{dy}{dx}+\mu P(x)y= \mu Q(x)$ which gives
$$y=\frac{\int\mu(x) Q(x)\,dx+C}{\mu}$$
\subsection*{Exact}
Exact equations are ODEs of the form $Mdx+Ndy=0$ where $\frac{\partial M}{\partial y}=\frac{\partial N}{\partial x}$. Then, $f(x,y)=\int M\,dx+h(y)=C$ or $f(x,y)=\int N\,dy+g(x)=C$
and $\frac{d}{dy}\left(\int M\,dx+h(y)\right)=N$ or $\frac{d}{dx}\left(\int N\,dy+g(x)\right)=M$
\subsection*{Non-Exact}
In cases where something looks exact but $\frac{\partial M}{\partial y}\neq\frac{\partial N}{\partial x}$ you can find an integrating factor
\begin{align*}
         & \mu(x)=\exp\left(\int \frac{\frac{\partial M}{\partial y}-\frac{\partial N}{\partial x}}{N}\,dx\right) \\
         & \mu(y)=\exp\left(\int \frac{\frac{\partial N}{\partial x}-\frac{\partial M}{\partial y}}{M}\,dy\right)
\end{align*}
\subsection*{Homogeneous}
If each term of the ODE is of equal order (e.g. the right hand side can be expressed as a function of only $\frac{y}{x}$) we can substitute
$y=ux\implies dy=udx+xdu$. This should result in a seperable equation.
\subsection*{Bernoulli}
If we have an equation of the form $\frac{dy}{dx}+P(x)y=Q(x)y^n$ we divide by $y^n$ and substitute $u=y^{1-n}\implies \frac{dy}{dx}=\frac{dy}{du}\frac{du}{dx}$ (you should know what $\frac{dy}{du}$ is here). This should result in a linear equation.
\subsection*{Linear Substituion}
An ODE of the form $\frac{dy}{dx}=f(Ax+By+C), \, B\neq 0$ can be solved by
\begin{align*}
         & u = Ax+By+C                                            \\
         & \Rightarrow \frac{du}{dx}=A+B\frac{dy}{dx}             \\
         & \Rightarrow \frac{dy}{dx}=(\frac{du}{dx}-A)\frac{1}{B} \\
\end{align*}

\section*{Solution Techniques $n=2$}
\subsection*{Reduction of Order}
If you solve a second order ODE and obtain a single solution $y_1$, $y_2=y_1\int \frac{e^{-\int P(x)dx}}{y_1^2}dx$ where $P(x)$ is found in $y^{\prime\prime}+P(x)y^{\prime}+Q(x)y=g(x)$
\subsection*{Constant Coefficients}
An equation of the form $ay^{\prime\prime}+by^{\prime}+cy=0$ can be solved through the characteristic equation obtained by substituting $y=e^{mt}$ and solving for $m$.
This gives a solution of the form $y_h=C_1e^{m_1t}+C_2e^{m_2t}$.
\subsection*{Undetermined Coefficients}
To obtain the particular solution of $ay^{\prime\prime}+by^{\prime}+cy=g(x)$ we try
$
        \begin{array}{|c|}
                \hline
                g(x)                                                                  \\
                \hline
                Ce^{\alpha x}                                                         \\
                C_nx^n+\dots+C_1x+C_0                                                 \\
                C\cos\left(\beta x\right), \quad C\sin\left(\beta x\right)            \\
                \left(C_nx^n+\dots+C_1x+C_0\right)e^{\alpha x}                        \\
                \hline
                y_p(x)                                                                \\
                \hline
                x^s\left(Ae^{\alpha x}\right)                                         \\
                x^s\left(A_nx^n+\dots+A_1x+A_0\right)                                 \\
                x^s\left(A\cos\left(\beta x\right)+A_1\sin\left(\beta x\right)\right) \\
                x^s\left(A_nx^n+\dots+A_1x+A_0\right)e^{\alpha x}
        \end{array}
$
\subsection*{Variation of Parameters}
For $y^{\prime\prime}+P(x)y^{\prime}+Q(x)y=g(x)$ if you have the homogeneous solutions $y_1(x)$ and $y_2(x)$, the particular solution $y_p(x)=u_1(x)y_1(x)+u_2(x)y_2(x)$ where
$$u_1(x)=-\int\frac{g(x)y_2(x)}{W\left[y_1, y_2\right]}dx$$
$$u_2(x)=\int\frac{g(x)y_1(x)}{W\left[y_1, y_2\right]}dx$$
where $W\left[y_1, y_2\right]$ is the Wronskian,
$$\begin{vmatrix}
                y_1          & y_2          \\
                y_1^{\prime} & y_2^{\prime} \\
        \end{vmatrix}$$
\subsection*{Cauchy-Euler}
An equation of the form $ax^2y^{\prime\prime}+bxy^{\prime}+cy=g(x)$ can be solved through the characteristic equation $am^2+\highlight{(b-a)}m+c=0$ obtained by substituting $y=x^{m66}$ and solving for $m$.
In the case where $m$ is complex here you end up with trig functions of logarithms.
\subsection*{Laplace Transform}
If $f(t)$ has period $T$ and is piecewise continuous on $[0,T]$ then $\laplace{f(t)}=\frac{\int_{0}^{T}e^{-st}f(t)dt}{1-e^{sT}}$
\subsection*{Properties of the Laplace Transform}
$\laplace{f_1+f_2}=\laplace{f_1}+\laplace{f_2}$
$\laplace{cf_1}=c\laplace{f_1}$
$\laplace{e^{at}f(t)}=F(s-a)$
$\laplace{f^\prime(t)}=s\laplace{f(t)}-f(0)$
$\laplace{f^{\prime\prime}(t)}=s^2\laplace{f(t)}-sf(0)-f^\prime(0)$
$\laplace{t^nf(t)}=(-1)^n\frac{d^nF(s)}{ds^n}\implies f(t)=\frac{\left(-1\right)^{n}}{t^n}\laplacei{\frac{d^nF(s)}{ds^n}}$
$\laplace{f(t-a)\mu(t-a)}=e^{-as}F(s)$
$\laplacei{e^{-as}F(s)}=f(t-a)\mu(t-a)$
\subsection*{Solving Discontinuous IVPs with Laplace Transforms}
For some ODE $ay^{\prime\prime}+by^{\prime}+cy=g(t)$ $\laplace{g(t)\mu(t-a)}=e^{-as}\laplace{g(t+a)}$
$$\mu(t-a)=
        \begin{cases}
                0 & t<a \\
                1 & t>a
        \end{cases}
$$
\section*{Applications}
\subsection*{Newton's Cooling}
$$\frac{dT}{dt}=k(T-T_m)\implies T(t)=T_m+Ce^{kt}$$ where $T$ is the temperature of an object, $T_m$ the temperature of the medium in which the object sits, and $k$ some cooling constant determined by initial/boundary conditions.
$C$ comes about as a result of solving the ODE and can also be determined using initial conditions.
\section*{Miscellaneous}
\subsection*{Partial Fractions}
$\frac{px+q}{(x-a)(x-b)}\to\frac{A}{x-a}+\frac{B}{x-b}$\\
$\frac{px+q}{(x-a)^2}\to\frac{A}{x-a}+\frac{B}{(x-a)^2}$\\
$\frac{px^2+qx+r}{(x-a)(x-b)(x-c)}\to\frac{A}{x-a}+\frac{B}{x-b}+\frac{C}{x-c}$\\
$\frac{px^2+qx+r}{(x-a)^2(x-b)}\to\frac{A}{x-a}+\frac{B}{(x-a)^2}+\frac{C}{x-c}$\\
$\frac{px^2+qx+r}{(x-a)(x^2-bx+c)}\to\frac{A}{x-a}+\frac{B}{(x^2-bx+c)}$
\subsection*{Systems}
$x^\prime=
        \begin{pmatrix}
                x^\prime \\
                y^\prime
        \end{pmatrix}=
        \begin{pmatrix}
                a & b \\
                c & d \\
        \end{pmatrix}
        \begin{pmatrix}
                x \\
                y \\
        \end{pmatrix}=Ax$
Then guess that $x=e^{\lambda t}\implies A\vec{v}=\lambda \vec{v}$.
To solve for eigenvalues find $det\left(A-\lambda I\right)=0$.
Then solve for eigenvectors $\vec{v_n}$ for each $\lambda_n$ with $\left(A-\lambda_n I\right)\vec{v_n}=0$.
For real, distinct eigenvector-value pairs write $y(t)=C_1e^{\lambda_1 t}\vec{v_1}+C_2e^{\lambda_2 t}\vec{v_2}$.
For a repeated eigenvalue write $y(t)=C_1e^{\lambda_1 t}\vec{v_1}+C_2e^{\lambda_1 t}(\vec{v_2}+t\vec{v_1})$ where $A\vec{v_2}-\lambda_1\vec{v_2}=\vec{v_1}$
\section*{Partial Differential Equations}
\subsection*{Classification}
If the coefficient of the highest derivative contains derivatives only up
to the previous order then the PDE is called Quasilinear. However, if the coefficient of the
highest derivative is not a function of the unknown or any of its derivatives then the PDE
is called Semilinear. A PDE is Fully Nonlinear if it is nonlinear in the highest order.
\subsection*{Method of Characteristics}
\dots
\subsection*{Canonical Forms}
For
$Au_{xx}+Bu_{xy}+Cu_{yy}+Du_x+Eu_y+Fu=G$\\
$\Delta=B^2-4AC$
$\begin{cases}
                \Delta < 0 \implies & \text{Elliptic}               \\
                \Delta = 0 \implies & \text{Parabolic $A_1=B_1=0$}  \\
                \Delta > 0 \implies & \text{Hyperbolic $A_1=C_1=0$} \\
        \end{cases}
$
We transform to
$A_1u_{\xi\xi}+B_1u_{\xi\eta}+C_1u_{\eta\eta}+D_1u_\xi+E_1u_\eta+F_1u=G_1$\\
with
$A_1  =A\xi_x^2+B\xi_x\xi_y+C\xi_y^2$                          \\
$B_1  =2A\xi_x\eta_x+B(\xi_x\eta_y+\xi_y\eta_x)+2C\xi_y\eta_y$ \\
$C_1  =A\eta_x^2+B\eta_x\eta_y+C\eta_y^2$                      \\
$D_1  =A\xi_{xx}+B\xi_{xy}+C\xi_{yy}+D\xi_x+E\xi_y $           \\
$E_1  =A\eta_{xx}+B\eta_{xy}+C\eta_{yy}+D\eta_x+E\eta_y$       \\
$F_1  =F(\xi,\eta)$                                          \\
$G_1  =G(\xi,\eta)$
\subsection*{Fourier Series}
For some function $f(x)$ periodic with period $2L$:\\
$a_0=\frac{1}{2L}\int_{-L}^{L}f(x)\,dx$\\
$a_n=\frac{1}{L}\int_{-L}^{L}f(x)\cos\left(\dfrac{n\pi x}{L}\right)\,dx$\\
$b_n=\frac{1}{L}\int_{-L}^{L}f(x)\sin\left(\dfrac{n\pi x}{L}\right)\,dx$\\
and
$f(x)=a_0+\sum_{n=1}^{\infty}a_n\cos\frac{n\pi x}{L}+b_n\sin\frac{n\pi x}{L}$
\subsection*{Heat Equation}
An equation of the form
$u_{xx}+g=\frac{1}{k}u_t$
where $g$ represents heat generation/loss inside the bar.
We solve first for a steady state condition where $u_t=0$
by writing $u=w(x,t)+v(x)$ and saying that $w(x,t\to\infty)=0$.
We then solve the homogenous problem for $w$ where our initial conditions become
zero and we subtract our steady-state solution $v$ from the initial temperature distribution.
We'll probably apply separation of variables to solve for the transient, $w$ but that's easy.
\section*{Laplacians}
\subsection*{Cartesian}
$$\nabla^2{V}=V_{xx}+V_{yy}+V_{zz}=0.$$
\subsection*{Spherical}
$\displaystyle\nabla^2{V}=\frac{1}{r^2}\frac{\partial}{\partial r}\left(r^2\frac{\partial V}{\partial r}\right)+
\frac{1}{r^2\sin\theta}\frac{\partial}{\partial \theta}\left(\sin\theta\frac{\partial V}{\partial \theta}\right)
+\frac{1}{r^2\sin^2\theta}\frac{\partial^2 V}{\partial \phi^2}.$
With Azimuthal symmetry this is solved by 
$V=\sum_n(A_nr^n+B_n/(r^{n+1}))P_n(\cos(\theta))$
\subsection{Cylindrical}
$$\nabla^2 V=\frac{1}{s}\frac{\partial}{\partial s}\left(s\frac{\partial V}{\partial S}\right)
+\frac{1}{s^2}\frac{\partial^2 V}{\partial\phi^2}+\frac{\partial^2 V}{\partial z^2}.
$$
\section*{Bessel Functions}
\subsection*{Normal}
An ODE of the form
$x^2\phi_{xx}+x\phi_x+(n^2-\lambda^2x^2)\phi=0$
is solved by the Bessel functions
$\phi_n=C_1J_n(\lambda x)+C_2Y_n(\lambda x)$.
Generally $C_2=0$ as $Y_n(0)\to -\infty$.
\subsection*{Modified}
An ODE of the form
$x^2\phi_{xx}+x\phi_x-(n^2+\lambda^2x^2)\phi=0$
is solved by the Bessel functions
$\phi_n=C_1I_n(\lambda x)+C_2K_n(\lambda x)$.
Generally $C_2=0$ as $K_n(0)\to \infty$. Note that 
$I_n(\infty)\to\infty$ as well.
\section*{Self-Adjoint Operators and Sturm-Liouville}
A Sturm-Liouville problem is given by 
$L\left[\phi\right]+\lambda w(x)\phi(x)=0$. 
$L$ is a self-adjoint linear differential operator. For 
$\L[\phi]=P_0\phi_{xx}+P_1\phi_x+P_2\phi$ this means that 
$P_0^\prime=P_1$. Distinct $\phi_n$ are orthogonal with respect to $w(x)$.
The coefficients of the eigenfunction expansion for $f(x)$ are 
$a_n=\int_a^bw(x)f(x)\phi_n(x)\,dx/\int_a^bw(x)\phi_n^2(x)\,dx$. We can transform a 
non-self-adjoint $L$ to a self adjoint one by multiplying by 
$P_0^{-1}\exp(\int P_1(x)/P_0(x)\,dx)$
\section*{Cartesian Laplacian Solution}
In 2D the cartesian Laplacian for a rectangular membrane
with 
$V(0,y)=g_1(y),V(L,y)=g_2(y)$
and $V(x,0)=f_1(x),V(x,H)=f_2(x)$
we get the solution 
$V=\sum_n\left[(a_n\sinh(n\pi(H-y)/L)+b_n\sinh(n\pi y/L))\sin(n\pi x/L)
+(c_n\sinh(n\pi(L-x)/H)+d_n\sinh(n\pi x/H))\sin(n\pi y/H)\right]$
with 
$a_n=\frac{2}{L\sin(n\pi H/L)}\int_0^Lf_1(x)\sin(n\pi x/L)\,dx$ and the other
coefficients following the same theme. Those for $y$ are with $g_{1,2}$ and are over $y$ with $H$ and $L$ swapped.
\section*{TDNH}
$A+x/L(B-A)$,
$\bar{Q}=Q-A^\prime-x/L(B^\prime-A^\prime)$
\end{document}