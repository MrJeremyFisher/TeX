\documentclass[10pt,landscape,letterpaper]{cheatsheet}

\usepackage{amsmath,amsthm,amssymb}
\usepackage{graphicx,lipsum}
\usepackage{enumitem}
\usepackage{xcolor}
\newcommand{\PRLsep}{\noindent\makebox[\linewidth]{\resizebox{0.3333\linewidth}{1pt}{$\bullet$}}\bigskip}
\newcommand\longdiv[2]{%
$\strut#1$\kern.25em\smash{\raise.3ex\hbox{$\big)$}}$\mkern-8mu
        \overline{\enspace\strut#2}$}
\usepackage{mdframed}
\usepackage{cancel}

\newcommand{\laplace}[1]{\mathcal{L}\left\{#1\right\}}
\newcommand{\laplacei}[1]{\mathcal{L}^{-1}\left\{#1\right\}}

\newcommand{\highlight}[1]{\colorbox{yellow}{$\displaystyle #1$}}

\title{ODE Cheat Sheet}
\author{Jeremy Favro}
\date{\today\\Revision 3}

\begin{document}

\maketitle

\section*{Fundamentals}
\subsection*{Classification}
$\frac{d^ny}{dx^n}=f(x,y)$ denotes an ODE of order $n$. Note that $(\frac{dy}{dx})^n\neq\frac{d^ny}{dx^n}$. ODEs of order $n$ will have $n$ constants in their general form solutions.\\
A linear ODE is one which can be written in the form $a_n(x)\frac{d^ny}{d^nx}+a_{n-1}\frac{d^{n-1}y}{d^{n-1}x}+\dots+a_1(x)\frac{dy}{dx}+a_0(x)y=g(x)$.\\
\subsection*{Solutions}
Given some IVP $\frac{dy}{dx}=f(x,y), \, y(x_0)=y_0$ if $f$ and $\frac{\partial f}{\partial y}$ are continuous in the rectangle $(x_0,y_0)\in \{(x,y):a<x<b,c<y<d\}$ then the
IVP has a unique solution $\phi(x)$ in some interval $(x_0-h,x_0+h), \, h\geq 0$


\section*{Solution Techniques $n=1$}
\subsection*{Direct Integration}
Directly integrate \dots
\subsection*{Seperable}
For some ODE $\frac{dy}{dx}=f(x,y)=g(x)p(y)$ the differential can be split $s.t.$ $\frac{1}{p(y)}dy=g(x)dx$ which can be solved by direct integration.
Note that when dividing by some function we assume that the function is nonzero. If there is a case (e.g. in an IVP) where the function is zero, the solution
is lost.
\subsection*{Linear}
For some linear ODE of the form $\frac{dy}{dx}+P(x)y=Q(x)$ we can multiply both sides of the ODE by $\mu(x)=\exp\left(\int P(x)\, dx\right)$
to obtain $\mu\frac{dy}{dx}+\mu P(x)y= \mu Q(x)$ which is equivalent to $\mu y=\mu Q(x)$ which can be solved by direct integration.
\subsection*{Exact}
Exact equations are ODEs of the form $Mdx+Ndy=0$ where $\frac{\partial M}{\partial y}=\frac{\partial N}{\partial x}$. Then, $f(x,y)=\int M\,dx+h(y)=C$ or $f(x,y)=\int N\,dy+g(x)=C$
and $\frac{d}{dy}\left(\int M\,dx+h(y)\right)=N$ or $\frac{d}{dx}\left(\int N\,dy+g(x)\right)=M$
\subsection*{Non-Exact}
In cases where something looks exact but $\frac{\partial M}{\partial y}\neq\frac{\partial N}{\partial x}$ you can find an integrating factor 
\begin{align*}
         & \mu(x)=\exp\left(\int \frac{\frac{\partial M}{\partial y}-\frac{\partial N}{\partial x}}{N}\,dx\right) \\
         & \mu(y)=\exp\left(\int \frac{\frac{\partial N}{\partial x}-\frac{\partial M}{\partial y}}{M}\,dy\right)
\end{align*}
\subsection*{Homogeneous}
If each term of the ODE is of equal order (e.g. the right hand side can be expressed as a function of only $\frac{y}{x}$) we can substitute 
$y=ux\implies dy=udx+xdu$. This should result in a seperable equation.
\subsection*{Bernoulli}
If we have an equation of the form $\frac{dy}{dx}+P(x)y=Q(x)y^n$ we divide by $y^n$ and substitute $u=y^{1-n}\implies \frac{dy}{dx}=\frac{dy}{du}\frac{du}{dx}$ (you should know what $\frac{dy}{du}$ is here). This should result in a linear equation.
\subsection*{Linear Substituion}
An ODE of the form $\frac{dy}{dx}=f(Ax+By+C), \, B\neq 0$ can be solved by
\begin{align*}
         & u = Ax+By+C                                            \\
         & \Rightarrow \frac{du}{dx}=A+B\frac{dy}{dx}             \\
         & \Rightarrow \frac{dy}{dx}=(\frac{du}{dx}-A)\frac{1}{B} \\
\end{align*}

\section*{Solution Techniques $n=2$}
\subsection*{Reduction of Order}
If you solve a second order ODE and obtain a single solution $y_1$, $y_2=y_1\int \frac{e^{-\int P(x)dx}}{y_1^2}dx$ where $P(x)$ is found in $y^{\prime\prime}+P(x)y^{\prime}+Q(x)y=g(x)$
\subsection*{Constant Coefficients}
An equation of the form $ay^{\prime\prime}+by^{\prime}+cy=g(x)$ can be solved through the characteristic equation obtained by substituting $y=e^{mt}$ and solving for $m$.
This gives a solution of the form $y_h=C_1e^{m_1t}+C_2e^{m_2t}$.
\subsection*{Undetermined Coefficients}
To obtain the particular solution of $ay^{\prime\prime}+by^{\prime}+cy=g(x)$ we try 
$
        \begin{array}{|c|}
                g(x)                                                                  \\ 
                \hline
                Ce^{\alpha x}                                                         \\
                C_nx^n+\dots+C_1x+C_0                                                 \\
                C\cos\left(\beta x\right), \quad C\sin\left(\beta x\right)            \\
                \left(C_nx^n+\dots+C_1x+C_0\right)e^{\alpha x}                        \\
                \hline
                y_p(x)                                                                \\
                x^s\left(Ae^{\alpha x}\right)                                         \\
                x^s\left(A_nx^n+\dots+A_1x+A_0\right)                                 \\
                x^s\left(A\cos\left(\beta x\right)+A_1\sin\left(\beta x\right)\right) \\
                x^s\left(A_nx^n+\dots+A_1x+A_0\right)e^{\alpha x}                     
        \end{array}
$
\subsection*{Variation of Parameters}
For $y^{\prime\prime}+P(x)y^{\prime}+Q(x)y=g(x)$ if you have the homogeneous solutions $y_1(x)$ and $y_2(x)$, the particular solution $y_p(x)=u_1(x)y_1(x)+u_2(x)y_2(x)$ where
$$u_1(x)=-\int\frac{g(x)y_2(x)}{W\left[y_1, y_2\right]}dx$$
$$u_2(x)=\int\frac{g(x)y_1(x)}{W\left[y_1, y_2\right]}dx$$
where $W\left[y_1, y_2\right]$ is the Wronskian, 
$$\begin{vmatrix}
                y_1          & y_2          \\ 
                y_1^{\prime} & y_2^{\prime} \\
        \end{vmatrix}$$
\subsection*{Cauchy-Euler}
An equation of the form $ax^2y^{\prime\prime}+bxy^{\prime}+cy=g(x)$ can be solved through the characteristic equation $am^2+\highlight{(b-a)}m+c=0$ obtained by substituting $y=x^{mt}$ and solving for $m$. 
In the case where $m$ is complex here you end up with trig functions of logarithms.
\subsection*{Laplace Transform}
If $f(t)$ has period $T$ and is piecewise continuous on $[0,T]$ then $\laplace{f(t)}=\frac{\int_{0}^{T}e^{-st}f(t)dt}{1-e^{sT}}$
\subsection*{Properties of the Laplace Transform}
$\laplace{f_1+f_2}=\laplace{f_1}+\laplace{f_2}$
$\laplace{cf_1}=c\laplace{f_1}$
$\laplace{e^{at}f(t)}=F(s-a)$
$\laplace{f^\prime(t)}=s\laplace{f(t)}-f(0)$
$\laplace{f^{\prime\prime}(t)}=s^2\laplace{f(t)}-sf(0)-f^\prime(0)$
$\laplace{t^nf(t)}=(-1)^n\frac{d^nF(s)}{ds^n}\implies f(t)=\frac{\left(-1\right)^{n}}{t^n}\laplacei{\frac{d^nF(s)}{ds^n}}$
$\laplace{f(t-a)\mu(t-a)}=e^{-as}F(s)$
$\laplacei{e^{-as}F(s)}=f(t-a)\mu(t-a)$
\subsection*{Solving Discontinuous IVPs with Laplace Transforms}
For some ODE $ay^{\prime\prime}+by^{\prime}+cy=g(t)$ $\laplace{g(t)\mu(t-a)}=e^{-as}\laplace{g(t+a)}$
$$\mu(t-a)=
        \begin{cases} 
                0 & t<a  \\
                1 & t>a
        \end{cases}
$$
\section*{Applications}
\subsection*{Newton's Cooling}
$$\frac{dT}{dt}=k(T-T_m)\implies T(t)=T_m+Ce^{kt}$$ where $T$ is the temperature of an object, $T_m$ the temperature of the medium in which the object sits, and $k$ some cooling constant determined by initial/boundary conditions.
$C$ comes about as a result of solving the ODE and can also be determined using initial conditions.
\subsection*{Circuit Theory}
$$V_{Resistor}=RI=R\frac{dQ}{dt}$$
$$V_{Inductor}=L\frac{dI}{dt}=L\frac{d^2Q}{dt^2}$$
$$V_{Capacitor}=\frac{Q}{C}=L\frac{d^2Q}{dt^2}$$
\end{document}