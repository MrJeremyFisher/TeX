\documentclass[10pt]{article}

\usepackage[margin=0.75in]{geometry}
\usepackage{amsmath,amsthm,amssymb}
\usepackage{xcolor}
\usepackage{cancel}
\usepackage{graphicx}
\usepackage{changepage}
\usepackage{circuitikz}
\usepackage{minted}
\usepackage{pgfplots}
\usepackage{physics}
\usepackage{siunitx}
\usepackage[breakable]{tcolorbox}
\usepackage[inline]{enumitem}

\theoremstyle{definition}
\newtheorem{problem}{Problem}
\newtheorem{soln}{Solution}

\pgfplotsset{compat=newest}
\usetikzlibrary{lindenmayersystems}
\usetikzlibrary{arrows}

\definecolor{incolor}{HTML}{303F9F}
\definecolor{outcolor}{HTML}{D84315}
\definecolor{cellborder}{HTML}{CFCFCF}
\definecolor{cellbackground}{HTML}{F7F7F7}
\newcommand{\eq}{=}
\usetikzlibrary{positioning, fit, calc}
\pgfdeclarelayer{background}  
\pgfsetlayers{background,main}

\makeatletter
\newcommand{\boxspacing}{\kern\kvtcb@left@rule\kern\kvtcb@boxsep}
\makeatother
\newcommand{\prompt}[4]{
    \ttfamily\llap{{\color{#2}[#3]:\hspace{3pt}#4}}\vspace{-\baselineskip}
}

\newcommand{\thevenin}[2]{
  \begin{center}
    \begin{circuitikz} \draw
      (0,0) -- (2,0) to[battery1, l_=$V_{Th}\eq#1$] (2,2) 
      to[resistor, l_=$R_{Th}\eq#2$] (0,2)
      ;
      \draw [o-] (-.07,2.079);
      \draw [o-] (-.07,0.079);
    \end{circuitikz}
  \end{center}
}

\newcommand{\norton}[2]{
  \begin{center}
    \begin{circuitikz} \draw
      (0,0) -- (3,0) to[american current source, l_=$I_{N}\eq#1$] (3,2) -- (0,2) (2,0)
      to[resistor, l=$R_{N}\eq#2$] (2,2)
      ;
      \draw [o-] (-.07,2.079);
      \draw [o-] (-.07,0.079);
    \end{circuitikz}
  \end{center}
}

\newcommand{\laplace}[1]{\mathcal{L}\left\{#1\right\}}
\newcommand{\laplacei}[1]{\mathcal{L}^{-1}\left\{#1\right\}}

\newcommand{\highlight}[1]{\colorbox{yellow}{$\displaystyle #1$}}

\NewDocumentCommand{\evalat}{sO{\big}mm}{%
  \IfBooleanTF{#1}
   {\mleft. #3 \mright|_{#4}}
   {#3#2|_{#4}}%
}

\title{Math 2150: Assignment III}
\author{Jeremy Favro}
\date{\today}

\begin{document}

\maketitle
\noindent Note: I've kept $p$, $q$, and $r$ throughout my solutions and only substituted the actual numbers in at the end. This is because I find it easier,
especially when dealing with things that might cancel nicely, to deal with variables rather than the numbers they represent. In my case my
student number is 0805980 so $p=9$, $q=5$, and $r=22$.
\\
% PROBLEM 1
\begin{problem}~
\begin{enumerate}[label=(\alph*)]
  \item Determine $\laplacei{\frac{\left(s-2\right)}{\left(s^2+1\right)\left(s^2-qs\right)}}$
  \item Determine $\laplacei{\frac{2s+1}{\left(s^2+4\right)\left(s^2+q^2\right)}}$
  \item Determine $\laplacei{\frac{e^{-2s}}{s^2\left(s-q\right)}}$
  \item Determine $\laplace{t\sin^2\left(qt\right)}$
  \item Determine $\laplace{e^{-qt}\sin\left(pt\right)\sin\left(qt\right)}$
  \item Determine $\laplace{e^{-qt}t\sin\left(t-\frac{\pi}{6}\right)}$
  \item Determine $\laplacei{\ln\left(\frac{7+ps}{9+qs}\right)}$
\end{enumerate}
\end{problem}
\begin{soln}~
  \begin{enumerate}[label=(\alph*)]
    \item \begin{align*}
             & =\laplacei{\frac{\left(s-2\right)}{s\left(s^2+1\right)\left(s-q\right)}}                                                                                        \\
             & =\laplacei{\frac{1}{\left(s^2+1\right)\left(s-q\right)}}-2\laplacei{\frac{1}{s\left(s^2+1\right)\left(s-q\right)}}                                              \\
             & =\laplacei{\frac{A_1s+B_1}{\left(s^2+1\right)}+\frac{C_1}{\left(s-q\right)}}-2\laplacei{\frac{A}{s}+\frac{Bs+C}{\left(s^2+1\right)}+\frac{D}{\left(s-q\right)}} \\
          \end{align*}
          Which I'll solve seperately and then recombine
          \begin{align*}
             & =\laplacei{\frac{A_1s+B_1}{\left(s^2+1\right)}+\frac{C_1}{\left(s-q\right)}}                                                                                                                                                             \\
             & \Rightarrow A_1s\left(s-q\right)+B_1\left(s-q\right)+C_1\left(s^2+1\right)=1                                                                                                                                                             \\
             & \Rightarrow s=q \implies C_1\left(q^2+1\right)=1 \implies C_1=\frac{1}{q^2+1}                                                                                                                                                            \\
             & \Rightarrow s=0 \implies B_1\left(-q\right)+\frac{1}{q^2+1}=1 \implies B_1=\frac{1-\frac{1}{q^2+1}}{-q}                                                                                                                                  \\
             & \Rightarrow s=1\implies A_1\left(1-q\right)+\frac{\left(1-q\right)\left(1-\frac{1}{q^2+1}\right)}{-q}+\frac{2}{q^2+1}=1\implies A_1=\frac{1-\frac{\left(1-q\right)\left(1-\frac{1}{q^2+1}\right)}{-q}-\frac{2}{q^2+1}}{\left(1-q\right)} \\
             & \Rightarrow f_1(t)=A_1\cos\left(t\right)+B_1\sin\left(t\right)+C_1e^{qt}                                                                                                                                                                 \\
          \end{align*}
          And
          \begin{align*}
             & =\laplacei{\frac{A}{s}+\frac{Bs+C}{\left(s^2+1\right)}+\frac{D}{\left(s-q\right)}}                                        \\
             & \Rightarrow A\left(s^2+1\right)\left(s-q\right) + Bs(s)\left(s-q\right) + C(s)\left(s-q\right) + D(s)\left(s^2+1\right)=1 \\
             & \Rightarrow s=q \implies D(q)\left(q^2+1\right)=1\implies D=\frac{1}{q\left(q^2+1\right)}                                 \\
             & \Rightarrow s=0 \implies A\left(-q\right)=1\implies A=\frac{1}{-q}                                                        \\
             & \Rightarrow A\left(s^2+1\right)\left(s-q\right) + Bs(s)\left(s-q\right) + C(s)\left(s-q\right) + D(s)\left(s^2+1\right)=1 \\
             & \implies As^3-Aqs^2+As-Aq + Bs^3-Bqs^2 + Cs^2-Cqs + Ds^3+Ds=1                                                             \\
             & \implies A+B+D=0, \, -Aq-Bq+C=0, \, A-Cq+D=0, \, -Aq=1                                                                    \\
             & \implies B=\frac{1}{q}-\frac{1}{q\left(q^2+1\right)}, \, C=-1+q\left(\frac{1}{q}-\frac{1}{q\left(q^2+1\right)}\right)     \\
             & \Rightarrow f_2(t)=A+De^{qt}+C\sin\left(t\right)+B\cos\left(t\right)                                                      \\
          \end{align*}
          So,
          \begin{align*}
             & f(t)=f_1(s)-2(f_2(s))                                                                                                                                                        \\
             & =A_1\cos\left(t\right)+B_1\sin\left(t\right)+C_1e^{qt} -2A-2De^{qt}-2C\sin\left(t\right)-2B\cos\left(t\right)                                                                \\
             & =\left(A_1-2B\right)\cos\left(t\right)+\left(B_1-2C\right)\sin\left(t\right)+\left(C_1-2D\right)e^{qt}-2A                                                                    \\
             & =
            \left(\frac{1-\frac{\left(1-q\right)\left(1-\frac{1}{q^2+1}\right)}{-q}-\frac{2}{q^2+1}}{\left(1-q\right)}-2\frac{1}{q}+2\frac{1}{q\left(q^2+1\right)}\right)\cos\left(t\right) \\
             & +\left(\frac{1-\frac{1}{q^2+1}}{-q}+2-2q\left(\frac{1}{q}-\frac{1}{q\left(q^2+1\right)}\right)\right)\sin\left(t\right)                                                      \\
             & +\left(\frac{1}{q^2+1}-2\frac{1}{q\left(q^2+1\right)}\right)e^{qt}+\frac{2}{q}                                                                                               \\
             & = -\frac{11}{26}\cos\left(t\right)-\frac{3}{26}\sin\left(t\right)+\frac{3}{130}e^{5t}+\frac{2}{5}                                                                            \\
          \end{align*}
    \item \begin{align*}
             & =\laplacei{\frac{2s+1}{\left(s^2+4\right)\left(s^2+q^2\right)}}                                                           \\
             & =\laplacei{\frac{2s}{\left(s^2+4\right)\left(s^2+q^2\right)}}+\laplacei{\frac{1}{\left(s^2+4\right)\left(s^2+q^2\right)}} \\
          \end{align*}
          Which I will again solve seperately and then add
          \begin{align*}
             & = \laplacei{\frac{2s}{\left(s^2+4\right)\left(s^2+q^2\right)}}                                     \\
             & = \laplacei{\frac{A_1s+B_1}{\left(s^2+4\right)}+\frac{C_1s+D_1}{\left(s^2+q^2\right)}}             \\
             & \Rightarrow \left(A_1s+B_1\right)\left(s^2+q^2\right) + \left(C_1s+D_1\right)\left(s^2+4\right)=2s \\
             & \implies A_1s^3+A_1q^2s+B_1s^2+B_1q^2+C_1s^3+4C_1s+D_1s^2+4D_1=2s                                  \\
             & \implies A_1+C_1=0 \implies A_1=-C_1                                                               \\
             & \implies B_1+D_1=0 \implies B_1=-D_1                                                               \\
             & \implies A_1q^2+4C_1=2\implies A_1=\frac{2}{q^2-4}\implies C_1=-\frac{2}{q^2-4}                    \\
             & \implies B_1q^2+4D_1=0\implies B_1,D_1=0                                                           \\
             & \Rightarrow f_1(t)=A_1\cos\left(2t\right)+C_1\cos\left(qt\right)
          \end{align*}
          And
          \begin{align*}
             & = \laplacei{\frac{1}{\left(s^2+4\right)\left(s^2+q^2\right)}}                                                       \\
             & = \laplacei{\frac{As+B}{\left(s^2+4\right)}+\frac{Cs+D}{\left(s^2+q^2\right)}}                                      \\
             & \Rightarrow \left(As+B\right)\left(s^2+q^2\right) + \left(Cs+D\right)\left(s^2+4\right)=1                           \\
             & \implies As^3+Aq^2s+Bs^2+Bq^2+Cs^3+4Cs+Ds^2+4D=1                                                                    \\
             & \implies A+C=0 \implies A=-C                                                                                        \\
             & \implies B+D=0 \implies B=-D                                                                                        \\
             & \implies Aq^2+4C=0\implies A,C=0                                                                                    \\
             & \implies Bq^2+4D=1\implies B=\frac{1}{q^2-4}\implies D=-\frac{1}{q^2-4}                                             \\
             & \Rightarrow \laplacei{\frac{B}{\left(s^2+4\right)}+\frac{D}{\left(s^2+q^2\right)}}                                  \\
             & \Rightarrow \frac{B}{2}\laplacei{\frac{2}{\left(s^2+4\right)}}+\frac{D}{q}\laplacei{\frac{q}{\left(s^2+q^2\right)}} \\
             & \Rightarrow f_2(t)=\frac{B}{2}\sin\left(2t\right)+\frac{D}{q}\sin\left(qt\right)
          \end{align*}
          So,
          \begin{align*}
             & f(t)=f_1(t)+f_2(t)                                                                                                                                                \\
             & =A_1\cos\left(2t\right)+C_1\cos\left(qt\right)+\frac{B}{2}\sin\left(2t\right)+\frac{D}{q}\sin\left(qt\right)                                                      \\
             & =\frac{2}{q^2-4}\cos\left(2t\right)-\frac{2}{q^2-4}\cos\left(qt\right)+\frac{\frac{1}{q^2-4}}{2}\sin\left(2t\right)+\frac{-\frac{1}{q^2-4}}{q}\sin\left(qt\right) \\
             & =\frac{2}{21}\cos\left(2t\right)-\frac{2}{21}\cos\left(5t\right)+\frac{1}{42}\sin\left(2t\right)-\frac{1}{105}\sin\left(5t\right)
          \end{align*}
    \item Here $e^{-2s}$ can be ``eliminated'' using the theorem $$\laplacei{e^{-\alpha s}F(s)}=f(t-\alpha)\mu(t-\alpha)$$ where $\mu(t)$ is the Heaviside function.
          \begin{align*}
             & =\laplacei{\frac{e^{-2s}}{s^2\left(s-q\right)}}    \\
             & =\laplacei{\frac{1}{s^2\left(s-q\right)}}\mu(t-2)  \\
             & =\laplacei{\frac{As+B}{s^2}+\frac{C}{s-q}}\mu(t-2)
          \end{align*}
          Then
          \begin{align*}
             & Cs^2+As\left(s-q\right)+B\left(s-q\right)=1                                            \\
             & \Rightarrow s = -q \implies Cq^2=1\implies C=\frac{1}{q^2}                             \\
             & \Rightarrow s=0 \implies B\left(-q\right)=1 \implies B=\frac{1}{-q}                    \\
             & \Rightarrow s=1 \implies \frac{1}{q^2}+A\left(1-q\right)-\frac{1}{q}\left(1-q\right)=1 \\
             & \implies A=\frac{1-\frac{1}{q^2}+\frac{1}{q}\left(1-q\right)}{1-q}
          \end{align*}
          Then
          \begin{align*}
             & f(t)=A+Bt+Ce^{qt}                                                   \\
             & =\frac{1-\frac{1}{q^2}+\frac{1}{q}\left(1-q\right)}{1-q}+Bt+Ce^{qt} \\
             & =-\frac{1}{25}-\frac{1}{5}t+\frac{1}{25}Ce^{5t}
          \end{align*}
          So the full solution is $$\left(-\frac{1}{25}-\frac{1}{5}\left(t-2\right)+\frac{1}{25}Ce^{5\left(t-2\right)}\right)\mu(t-2)$$
    \item Here the $t$ can be ``eliminated'' through the theorem $\displaystyle\laplace{t^nf(t)}=\left(-1\right)^{n}\frac{d^nF(s)}{ds^n}$ so,
          \begin{align*}
             & = \laplace{\sin^2\left(qt\right)}                                           \\
             & F(s)= \frac{2q^2}{s\left(s^2+4q^2\right)} \rightsquigarrow \text{ by table}
          \end{align*}
          Now because $n=1$ here we must differentiate $F(s)$ once and switch its sign,
          \begin{align*}
             & =-\frac{d}{ds}\left[\frac{2q^2}{s\left(s^2+4q^2\right)}\right] \\
             & =\frac{2q^2\left(3s^2+4q^2\right)}{s^2\left(s^2+4q^2\right)^2} \\
             & =\frac{50\left(3s^2+100\right)}{s^2\left(s^2+100\right)^2}
          \end{align*}
    \item Here the $e^{-qt}$ can be ``eliminated'' through the theorem $\laplace{f(t)e^{at}}=F(s-a)$ so,
          \begin{align*}
             & =\laplace{\sin\left(pt\right)\sin\left(qt\right)}                                           \\
             & =\frac{1}{2}\laplace{\cos\left(\left(p-q\right)t\right)-\cos\left(\left(p+q\right)t\right)} \\
             & =\frac{1}{2}\laplace{\cos\left(\left(p-q\right)t\right)-\cos\left(\left(p+q\right)t\right)} \\
             & =\frac{1}{2}\left[\frac{s}{s^2+\left(p-q\right)^2}-\frac{s}{s^2+\left(p+q\right)^2}\right]
          \end{align*}
          Then, applying the shift,
          \begin{align*}
             & F(s)=\frac{1}{2}\left[\frac{\left(s+q\right)}{\left(s+q\right)^2+\left(p-q\right)^2}-\frac{\left(s+q\right)}{\left(s+q\right)^2+\left(p+q\right)^2}\right] \\
             & =\frac{1}{2}\left[\frac{\left(s+5\right)}{\left(s+5\right)^2+16}-\frac{\left(s+5\right)}{\left(s+5\right)^2+196}\right]
          \end{align*}
    \item The expression here can be simplified using the two theorems from the previous questions, $\laplace{f(t)e^{at}}=F(s-a)$ and $\displaystyle\laplace{t^nf(t)}=\left(-1\right)^{n}\frac{d^nF(s)}{ds^n}$
          \begin{align*}
             & =\laplace{e^{-qt}t\sin\left(t-\frac{\pi}{6}\right)}                                                          \\
             & =\laplace{\sin\left(t-\frac{\pi}{6}\right)}                                                                  \\
             & =\laplace{\sin\left(t\right)\cos\left(\frac{\pi}{6}\right)-\sin\left(\frac{\pi}{6}\right)\cos\left(t\right)} \\
             & =\laplace{\frac{\sqrt{3}}{2}\sin\left(t\right)-\frac{1}{2}\cos\left(t\right)}                                \\
             & =\frac{\sqrt{3}}{2}\laplace{\sin\left(t\right)}-\frac{1}{2}\laplace{\cos\left(t\right)}                      \\
             & =\frac{\sqrt{3}}{2\left(s^2+1\right)}-\frac{s}{2\left(s^2+1\right)}                                          \\
          \end{align*}
          Then we need to apply the derivative and shift (I don't think order should matter here),
          \begin{align*}
             & =-\frac{d}{ds}\left[\frac{\sqrt{3}}{2\left(s^2+1\right)}-\frac{s}{2\left(s^2+1\right)}\right]                \\
             & =\frac{s^2-2\sqrt{3}s-1}{2\left(s^2+1\right)^2}                                                              \\
             & \Rightarrow F(s)=\frac{\left(s+q\right)^2-2\sqrt{3}\left(s+q\right)-1}{2\left(\left(s+q\right)^2+1\right)^2} \\
             & =\frac{\left(s+5\right)^2-2\sqrt{3}\left(s+5\right)-1}{2\left(\left(s+5\right)^2+1\right)^2}
          \end{align*}
    \item ~
          \begin{align*}
             & =\laplacei{\ln\left(\frac{7+ps}{9+qs}\right)}                    \\
             & =\laplacei{\ln\left(7+ps\right)-\ln\left(9+qs\right)}            \\
             & =\laplacei{\ln\left(7+ps\right)}-\laplacei{\ln\left(9+qs\right)}
          \end{align*}
          Here we can use the theorem $\displaystyle\laplace{t^nf(t)}=\left(-1\right)^{n}\frac{d^nF(s)}{ds^n}\implies f(t)=\frac{\left(-1\right)^{n}}{t^n}\laplacei{\frac{d^nF(s)}{ds^n}}$
          \begin{align*}
             & =\laplacei{\frac{1}{7+ps}}-\laplacei{\frac{1}{9+qs}}                   \\
             & =\laplacei{\frac{1}{\frac{7}{p}+s}}-\laplacei{\frac{1}{\frac{9}{q}+s}} \\
             & =e^{-\frac{7}{p}t}-e^{-\frac{9}{q}t}                                   \\
          \end{align*}
          Then, using the theorem
          \begin{align*}
             & =\frac{1}{t}\left[-e^{-\frac{9}{q}t}-e^{-\frac{7}{p}t}\right]     \\
             & f(t)=\frac{1}{t}\left[-e^{-\frac{9}{5}t}-e^{-\frac{7}{9}t}\right] \\
          \end{align*}
  \end{enumerate}
\end{soln}
% PROBLEM 2
\begin{problem}
Use the \textbf{\underline{Laplace Transform}} method to solve the following initial value problems:
\begin{enumerate}[label=(\alph*)]
  \item $\displaystyle y^{\prime\prime}+2qy^\prime+q^2y=t^pe^{-qt}\qquad y(0)=y^\prime(0)=0$
  \item $\displaystyle y^{\prime\prime}-3qy^\prime+2q^2y=e^{qt}\cos\left(pt\right)\qquad y(0)=y^\prime(0)=0$
\end{enumerate}
\end{problem}
\begin{soln}~
  \begin{enumerate}[label=(\alph*)]
    \item
          \begin{align*}
             & \laplace{y^{\prime\prime}+2qy^\prime+q^2y}=\laplace{t^pe^{-qt}}                     \\
             & \laplace{y^{\prime\prime}}+\laplace{2qy^\prime}+\laplace{q^2y}=\laplace{t^pe^{-qt}} \\
             & s^2Y(s)+2qsY(s)+q^2Y(s)=\frac{p!}{\left(s+q\right)^{p+1}}                           \\
             & Y(s)=\frac{p!}{\left(s^2+2qs+q^2\right)\left(s+q\right)^{p+1}}
          \end{align*}
          Then
          \begin{align*}
             & y(t)=\laplacei{\frac{p!}{\left(s^2+2qs+q^2\right)\left(s+q\right)^{p+1}}}        \\
             & =p!\laplacei{\frac{1}{\left(s+q\right)^2\left(s+q\right)^{p+1}}}                 \\
             & =p!\laplacei{\frac{1}{\left(s+q\right)^{p+3}}}                                   \\
             & =p!e^{-qt}\laplacei{\frac{1}{s^{p+3}}}                                           \\
             & =\frac{p!e^{-qt}}{\left(p+2\right)!}\laplacei{\frac{\left(p+2\right)!}{s^{p+3}}} \\
             & =\frac{p!e^{-qt}}{\left(p+2\right)!}t^{p+2}                                      \\
             & =\frac{e^{-qt}}{\left(p+1\right)\left(p+2\right)}t^{p+2}                         \\
             & =\frac{e^{-5t}}{110}t^{11}
          \end{align*}
    \item
          \begin{align*}
             & \laplace{y^{\prime\prime}-3qy^\prime+2q^2y}=\laplace{e^{qt}\cos\left(pt\right)}                     \\
             & \laplace{y^{\prime\prime}}-\laplace{3qy^\prime}+\laplace{2q^2y}=\laplace{e^{qt}\cos\left(pt\right)} \\
             & Y(s)\left(s^2-3qs+2q^2\right)=\frac{\left(s-q\right)}{\left(s-q\right)^2+k^2}                       \\
             & Y(s)=\frac{\left(s-q\right)}{\left(s^2-3qs+2q^2\right)\left(\left(s-q\right)^2+p^2\right)}
          \end{align*}
          Then,
          \begin{align*}
             & = \laplacei{\frac{1}{\left(s-2q\right)\left(\left(s-q\right)^2+p^2\right)}}     \\
             & = \laplacei{\frac{A}{s-2q}+\frac{Bs+C}{s^2-2qs+q^2+p^2}} \\
             & \Rightarrow A\left(s^2-2qs+q^2+p^2\right)+Bs\left(s-2q\right)+C\left(s-2q\right)=1                  \\
             & \Rightarrow s=2q \implies A\left(q^2+p^2\right)=1 \implies A=\frac{1}{q^2+p^2}  \\
             & \Rightarrow s=0 \implies 1+C\left(-2q\right)=1 \implies C=0  \\
             & \Rightarrow s=1 \implies \frac{1}{q^2+p^2}\left(1-2q+q^2+p^2\right)+B\left(1-2q\right)=1 \implies B=\frac{1-\frac{1}{q^2+p^2}\left(1-2q+q^2+p^2\right)}{\left(1-2q\right)}  \\
             & = \laplacei{\frac{A}{s-2q}+\frac{Bs}{\left(s-q\right)^2+p^2}}\\
             & = \laplacei{\frac{A}{s-2q}+\frac{B\left(s-q\right)}{\left(s-q\right)^2+p^2}+\frac{Bq}{\left(s-q\right)^2+p^2}}\\
             & = Ae^{2qt}+Be^{qt}\cos\left(pt\right)+\frac{Bq}{p}e^{qt}\sin\left(pt\right)\\
             & = \frac{1}{106}e^{10t}-\frac{1}{106}e^{5t}\cos\left(9t\right)-\frac{5}{954}e^{5t}\sin\left(9t\right)
          \end{align*}

  \end{enumerate}
\end{soln}
\end{document}