\documentclass[10pt]{article}

\usepackage[margin=0.75in]{geometry}
\usepackage{amsmath,amsthm,amssymb}
\usepackage{xcolor}
\usepackage{cancel}
\usepackage{graphicx}
\usepackage{changepage}
\usepackage{circuitikz}
\usepackage{pgfplots}
\usepackage{physics}
\usepackage{hyperref}
\usepackage{siunitx}
\usepackage[breakable]{tcolorbox}
\usepackage[inline]{enumitem}

\theoremstyle{definition}
\newtheorem{problem}{Problem}
\newtheorem{soln}{Solution}

\pgfplotsset{compat=newest}
\usetikzlibrary{lindenmayersystems}
\usetikzlibrary{arrows}
\usetikzlibrary{calc}

\definecolor{incolor}{HTML}{303F9F}
\definecolor{outcolor}{HTML}{D84315}
\definecolor{cellborder}{HTML}{CFCFCF}
\definecolor{cellbackground}{HTML}{F7F7F7}
\newcommand{\eq}{=}
\usetikzlibrary{positioning, fit, calc}
\pgfdeclarelayer{background}  
\pgfsetlayers{background,main}
\DeclareSIUnit[number-unit-product = {\,}]\calorie{cal}
\DeclareSIUnit[number-unit-product = {\,}]\atmosphere{atm}
\AtBeginDocument{\RenewCommandCopy\qty\SI}

\makeatletter
\newcommand{\boxspacing}{\kern\kvtcb@left@rule\kern\kvtcb@boxsep}
\makeatother
\newcommand{\prompt}[4]{
    \ttfamily\llap{{\color{#2}[#3]:\hspace{3pt}#4}}\vspace{-\baselineskip}
}

\newcommand{\thevenin}[2]{
  \begin{center}
    \begin{circuitikz} \draw
      (0,0) -- (2,0) to[battery1, l_=$V_{Th}\eq#1$] (2,2) 
      to[resistor, l_=$R_{Th}\eq#2$] (0,2)
      ;
      \draw [o-] (-.07,2.079);
      \draw [o-] (-.07,0.079);
    \end{circuitikz}
  \end{center}
}
\newcommand{\norton}[2]{
  \begin{center}
    \begin{circuitikz} \draw
      (0,0) -- (3,0) to[american current source, l_=$I_{N}\eq#1$] (3,2) -- (0,2) (2,0)
      to[resistor, l=$R_{N}\eq#2$] (2,2)
      ;
      \draw [o-] (-.07,2.079);
      \draw [o-] (-.07,0.079);
    \end{circuitikz}
  \end{center}
}
\newcommand{\highlight}[1]{\colorbox{yellow}{$\displaystyle #1$}}
\newcommand{\ti}[1]{\widetilde{#1}}
\newcommand{\vlen}[1]{\| {#1}\Vert }

\NewDocumentCommand{\evalat}{sO{\big}mm}{%
  \IfBooleanTF{#1}
   {\mleft. #3 \mright|_{#4}}
   {#3#2|_{#4}}%
}

\title{Math 2350H: Assignment IV}
\author{Jeremy Favro (0805980) \\ Trent University, Peterborough, ON, Canada}
\date{\today}

\begin{document}
\maketitle

% PROBLEM 1 
\begin{problem}
Consider the subspace
$$U=\mathrm{span}\left(
  \left(2,-1,-2,4\right),
  \left(-2,1,-5,5\right),
  \left(-1,3,7,11\right)
  \right)$$
\begin{enumerate}[label=(\alph*)]
  \item Apply the Gram-Schmidt process (with normalization) to find an orthonormal basis of $U$.
  \item Find a basis for $U^\perp$
  \item Express the vector $v = (-11, 8, -4, 18)$ as $v = x + y$ where
        $x\in U$ and $y \in U^\perp$.
\end{enumerate}
\end{problem}
\begin{soln}~
  \begin{enumerate}[label=(\alph*)]
    \item \begin{align*}
            u_1 & =\left(2,-1,-2,4\right)\implies \hat{u}_1=\left(\frac{2}{5},-\frac{1}{5},-\frac{2}{5},\frac{4}{5}\right)                                                                                      \\
            u_2 & =v_2-\frac{\braket{v_2}{u_1}}{\vlen{u_1}^2}u_1                                                                                                                                                \\
                & =\left(-2,1,-5,5\right)-\frac{\left(-2,1,-5,5\right)\cdot\left(2,-1,-2,4\right)}{25}\left(2,-1,-2,4\right)                                                                                    \\
                & =\left(-4,2,-3,1\right)\implies \hat{u}_2=\left(-\frac{4}{\sqrt{30}},\frac{2}{\sqrt{30}},-\frac{3}{\sqrt{30}},\frac{1}{\sqrt{30}}\right)                                                      \\
            u_3 & =v_3-\frac{\braket{v_3}{u_1}}{\vlen{u_1}^2}u_1-\frac{\braket{v_3}{u_2}}{\vlen{u_2}^2}u_2                                                                                                      \\
                & =\left(-1,3,7,11\right)-\frac{\left(-1,3,7,11\right)\cdot\left(2,-1,-2,4\right)}{25}\left(2,-1,-2,4\right)-\frac{\left(-1,3,7,11\right)\cdot\left(-4,2,-3,1\right)}{30}\left(-4,2,-3,1\right) \\
                & = \left(-3,4,9,7\right)\implies \hat{u}_3=\left(-\frac{3}{\sqrt{155}},\frac{4}{\sqrt{155}},\frac{9}{\sqrt{155}},\frac{7}{\sqrt{155}}\right)
          \end{align*}
    \item We are looking for the vector $v\in U^\perp$ such that $\braket{v}{u_n}=0$ which can be found by solving the system represented by the matrix of
          the basis vectors ($u_n$) we found previously (un-normalized because it yields the same result and is easier),
          $$\begin{pmatrix}
              2  & -1 & -2 & 4 \\
              -4 & 2  & -3 & 1 \\
              -3 & 4  & 9  & 7
            \end{pmatrix}\sim
            \begin{pmatrix}
              1 & 0 & 0 & 34/7 \\
              0 & 1 & 0 & 58/7 \\
              0 & 0 & 1 & -9/7
            \end{pmatrix}$$
          which means that $U^\perp=\mathrm{span}(-34/7,-58/7,9/7,1)$.
    \item Using the fact that the coefficients of a linear combination are given by the inner products of the result of that linear combination and the vector paired with the coefficient
          and calling $u_4=(-34/7,-58/7,9/7,1)$
          \begin{align*}
            \braket{v}{u_1} & =\braket{x+y}{u_1}=\braket{x}{u_1}+\braket{y}{u_1}=\braket{x}{u_1}=c_1, \\
            \braket{v}{u_2} & =\braket{x+y}{u_2}=\braket{x}{u_2}+\braket{y}{u_2}=\braket{x}{u_2}=c_2, \\
            \braket{v}{u_3} & =\braket{x+y}{u_3}=\braket{x}{u_3}+\braket{y}{u_3}=\braket{x}{u_3}=c_3, \\
            \braket{v}{u_4} & =\braket{x+y}{u_4}=\braket{x}{u_4}+\braket{y}{u_4}=\braket{y}{u_4}=c_4
          \end{align*} so
          \begin{align*}
            v & =\braket{v}{u_1}u_1+\braket{v}{u_2}u_2+\braket{v}{u_3}u_3+\braket{v}{u_4}u_4
          \end{align*}
          which gives $x=2u_1+3u_2+u_3$ and $y=0u_4$.
  \end{enumerate}
\end{soln}

% PROBLEM 2
\begin{problem}
The dot product is defined on $\mathcal{M}_{n\cross 1}(\mathbb{R})$ and $\mathcal{M}_{n\cross 1}(\mathbb{C})$ just as it is
for $\mathbb{R}^n$ and $\mathbb{C}^n$. For $u,v\in\mathcal{M}_{n\cross 1}(\mathbb{C})$, with
$$u=\begin{pmatrix}
    u_1    \\
    u_2    \\
    \vdots \\
    u_n
  \end{pmatrix},\quad v=\begin{pmatrix}
    v_1    \\
    v_2    \\
    \vdots \\
    v_n
  \end{pmatrix}$$
the dot product (or standard inner product) is
$$\braket{u}{v}=\sum_{k=1}^{n}u_kv_k^*$$
This can be written more compactly with matrix multiplication as
$$\braket{u}{v}=v^\dagger u$$
where $v^\dagger=\left(v^T\right)^*$. We use this inner product below.
\begin{enumerate}[label=(\alph*)]
  \item Let $P\in\mathcal{M}_{n\cross n}(\mathbb{R})$. Relative to the dot product on column vectors, show that the following are equivalent:
        \begin{enumerate}[label=(\roman*)]
          \item The columns $u_1,\dots,u_n$ of $P$ form an orthonormal basis $\mathcal{M}_{n\cross 1}(\mathbb{R})$.
          \item $P^T=P^{-1}$.
          \item The rows of $P$ form an orthonormal basis for $\mathbb{R}^n$.
        \end{enumerate}
  \item A matrix $P\in\mathcal{M}_{n\cross n}(\mathbb{R})$ is called an \emph{orthogonal matrix} if $P^T=P^{-1}$. Determine which
        of the following matrices are orthogonal.
        \begin{enumerate}[label=(\roman*)]
          \item $\begin{pmatrix}
                    3/5  & 4/5  \\
                    -4/5 & -3/5
                  \end{pmatrix}$
          \item $\begin{pmatrix}
                    5/13  & 12/13 \\
                    12/13 & -5/13
                  \end{pmatrix}$
          \item $\begin{pmatrix}
                    1 & 0 \\
                    0 & 1
                  \end{pmatrix}$
          \item $\begin{pmatrix}
                    1/\sqrt{2} & -1/\sqrt{2} \\
                    1/\sqrt{2} & 1/\sqrt{2}
                  \end{pmatrix}$
          \item $\begin{pmatrix}
                    1/3 & 2/3  & -2/3 \\
                    1/3 & -2/3 & 1/3  \\
                    2/3 & 1/3  & 2/3
                  \end{pmatrix}$
          \item $\begin{pmatrix}
                    1/3 & 2/3  & 2/3  \\
                    2/3 & -2/3 & 1/3  \\
                    2/3 & 1/3  & -2/3
                  \end{pmatrix}$
        \end{enumerate}
  \item Let $P,Q\in\mathcal{M}_{n\cross n}(\mathbb{R})$ be orthogonal matrices and $u,v\in\mathcal{M}_{n\cross 1}(\mathbb{R})$.
        Prove that
        \begin{enumerate}[label=(\roman*)]
          \item $\braket{Pu}{Pv}=\braket{u}{v}$
          \item $\|Pu \Vert =\|u \Vert $
          \item $PQ$ is an orthogonal matrix.
        \end{enumerate}
\end{enumerate}
\end{problem}
\begin{soln}~
  \begin{enumerate}[label=(\alph*)]
    \item To prove the equivalency of these statements we need to prove that $(\mathrm{i})\implies(\mathrm{ii})\implies(\mathrm{iii})\implies(\mathrm{i})$.
          For $(\mathrm{i})\implies(\mathrm{ii})$ we know that
          $$\braket{u_i}{u_j}=\begin{cases}
              1\quad i=j \\
              0\quad i\neq j
            \end{cases}$$
          because these columns are linearly independent. By the definition of this inner product this is equal to $u_j^Tu_i$ which is the $ij$-th entry of
          $P^TP$ which we know from above must be $I$ so $P^TP=I\implies P^T=P^{-1}$.
          \\
          For $(\mathrm{ii})\implies(\mathrm{iii})$ we are trying to show that if $P^T=P^{-1}$ then the rows of $P$ form a basis for $\mathbb{R}^n$
          where $P\in\mathcal{M}_{n\cross n}(\mathbb{R}$). We start by noting that $P^T=P^{-1}\implies P^TP=I$ so
            $$P_{ji}P_{ij}=\begin{cases}
                1\quad i=j \\
                0\quad i\neq j
              \end{cases}.$$
            The entries of $P^TP$ are given by the inner product $\braket{P_k}{P_h^T}$ where $P_k$ is the $k$-th row of $P$ and $P_h$ the $h$-th column of $P^T$ which is equivalent
            to the $h$-th \emph{row} of $P$ which means we have that
            $$\braket{P_k}{P_h}=\begin{cases}
                1\quad h=k \\
                0\quad h\neq k
              \end{cases}$$
            which means that the rows of $P$ are by definition linearly independent and normalized and because we have $n$ of them form an orthonormal basis for $\mathbb{R}^n$.\\
            For $\implies(\mathrm{iii})\implies(\mathrm{i})$ we can see that a matrix with linearly independent rows will reduce to the appropriate identity matrix which will also
          have independent columns because Gaussian elimination preserves linear independence between rows/columns.
    \item A matrix $P\in\mathcal{M}_{n\cross n}(\mathbb{R})$ is called an \emph{orthogonal matrix} if $P^T=P^{-1}$. Determine which
          of the following matrices are orthogonal.
          \begin{enumerate}[label=(\roman*)]
            \item From part (a) it's enough to note that $\braket{u_1}{u_2}=\begin{pmatrix}
                      3/5 & -4/5
                    \end{pmatrix}\begin{pmatrix}
                      4/5 \\
                      -3/5
                    \end{pmatrix}=24/25\neq 0$ so the columns of this matrix do not form an orthonormal basis and therefore $P^T\neq P^{-1}$
            \item Here again it is sufficient to note that, by inspection (as the columns are flipped and the 22 entry is negative), the columns
                  of this matrix form an orthonormal basis for $\mathcal{M}_{n\cross 1}(\mathbb{R})$ and therefore $P^T=P^{-1}$
            \item Here again it is sufficient to note that, by inspection, the columns
                  of this matrix form an orthonormal basis for $\mathcal{M}_{n\cross 1}(\mathbb{R})$ and therefore $P^T=P^{-1}$
            \item Here again it is sufficient to note that, by inspection (as the columns are flipped and the 21 entry is negative), the columns
                  of this matrix form an orthonormal basis for $\mathcal{M}_{n\cross 1}(\mathbb{R})$ and therefore $P^T=P^{-1}$
            \item Here it is sufficient to note that the first column is not normalized ($u_1=\sqrt{2\cdot(1/3)^2+(2/3)^2}=\sqrt{6}/3=\neq 1$) and therefore the columns
                  cannot form an orthonormal basis so $P^T\neq P^{-1}$
            \item Here $\braket{u_1}{u_2}=\begin{pmatrix}
                      2/3 & -2/3 & 1/3
                    \end{pmatrix}\begin{pmatrix}
                      1/3 \\
                      2/3 \\
                      2/3
                    \end{pmatrix}=0$,
                  $\braket{u_1}{u_3}=\begin{pmatrix}
                      2/3 & 1/3 & -2/3
                    \end{pmatrix}\begin{pmatrix}
                      1/3 \\
                      2/3 \\
                      2/3
                    \end{pmatrix}=0$, and
                  $\braket{u_2}{u_3}=\begin{pmatrix}
                      2/3 & 1/3 & -2/3
                    \end{pmatrix}\begin{pmatrix}
                      2/3  \\
                      -2/3 \\
                      1/3
                    \end{pmatrix}=0$ and $u_1$, $u_2$, and $u_3$ are normalized so $P^T=P^{-1}$
          \end{enumerate}
    \item \begin{enumerate}[label=(\roman*)]
            \item $\braket{Pu}{Pv}=v^TP^TPu=v^TIu=v^Tu=\braket{u}{v}$
            \item $\vlen{Pu} =\braket{Pu}{Pu}=u^TP^TPu=u^TIu=\braket{u}{u}=\vlen{u} $
            \item Because $P,Q$ are of the same size we can multiply them (and their transposes),
                  $(PQ)^T PQ=Q^TP^T PQ=Q^TIQ=Q^TQ=I$ which satisfies the property of an orthogonal matrix
                  $A$ that $A^TA=I$
          \end{enumerate}
  \end{enumerate}
\end{soln}

% PROBLEM 3
\begin{problem}
Let $V$ be the vector space of continuous, real-valued functions defined
on the interval $[0, 1]$. Then $V$ is an inner product space with inner
product
$$\braket{f}{g}=\int_{0}^{1}f(x)g(x)dx,$$
for $f,g\in V$. Consider the subspace $U$ of $V$ spanned by the functions $f(x)=\sqrt{x}$, $g(x)=x$, $h(x)=x^2$.
\begin{enumerate}[label=(\alph*)]
  \item Show that $f, g, h$ is linearly independent.
  \item Find an orthonormal basis for $U$
  \item Let $p(x)=x^3$. Find the closest approximation of $p$ in $U$.
\end{enumerate}
\end{problem}
\begin{soln}~
  \begin{enumerate}[label=(\alph*)]
    \item By the definition of linear independence
          $$c_1f(x)+c_2g(x)+c_3h(x)=0.$$
          Using the Wronskian here for fun we get $\frac{3\sqrt{x}}{4}$ (calculated using SageMath) which gives $0$ only at $x=0$ which, by the definition of the Wronskian, implies
          linear independence.
    \item Here we apply the Gram-Schmidt process,
          $$w_1=\sqrt{x}$$
          $$w_2=x-\frac{\braket{x}{\sqrt{x}}}{\vlen{\sqrt{x}}^2}\sqrt{x}=x-\frac{\int_{0}^{1}x\sqrt{x}\,dx}{\int_{0}^{1}x\,dx}\sqrt{x}=x-\frac{4}{5}\sqrt{x}$$
          \begin{align*}
            w_3 & =x^2-\frac{\braket{x^2}{\sqrt{x}}}{\vlen{\sqrt{x}}^2}\sqrt{x}-\frac{\braket{x^2}{x-\frac{4}{5}\sqrt{x}}}{\vlen{x-\frac{4}{5}\sqrt{x}}^2}\left(x-\frac{4}{5}\sqrt{x}\right) \\
                & =x^2-\frac{\int_{0}^{1}x^2\sqrt{x}\,dx}{\frac{1}{2}}\sqrt{x}-\frac{\int_{0}^{1}x^2\left(x-\frac{4}{5}\sqrt{x}\right)\,dx}{\frac{1}{75}}\left(x-\frac{4}{5}\sqrt{x}\right)  \\
                & =x^2-\frac{4}{7}\sqrt{x}-\frac{45}{28}\left(x-\frac{4}{5}\sqrt{x}\right)                                                                                                   \\
                & =x^2-\frac{45}{28}x+\frac{5}{7}\sqrt{x}
          \end{align*}
          then we normalize,
          $$\vlen{w_1}=\frac{1}{\sqrt{2}}\implies\hat{u}_1=\sqrt{2x}$$
          $$\vlen{w_2}=\frac{1}{\sqrt{75}}\implies\hat{u}_2=\sqrt{75}\left(x-\frac{4}{5}\sqrt{x}\right)$$
          $$\vlen{w_3}=\int_{0}^{1}\left(x^2-\frac{45}{28}x+\frac{5}{7}\sqrt{x}\right)^2\,dx=\sqrt{\frac{9}{3920}}
            \implies\hat{u}_3=\sqrt{\frac{3920}{9}}\left(x^2-\frac{45}{28}x+\frac{5}{7}\sqrt{x}\right)$$
    \item Here we are looking for the projection map
          \begin{align*}
            P_U(x^3) & =\braket{x^3}{\sqrt{2x}}\left(\sqrt{2x}\right)+                                                                                                                                       \\
                     & \quad\braket{x^3}{\sqrt{75}\left(x-\frac{4}{5}\sqrt{x}\right)}\sqrt{75}\left(x-\frac{4}{5}\sqrt{x}\right)+                                                                            \\
                     & \quad\braket{x^3}{\sqrt{\frac{3920}{9}}\left(x^2-\frac{45}{28}x+\frac{5}{7}\sqrt{x}\right)}\sqrt{\frac{3920}{9}}\left(x^2-\frac{45}{28}x+\frac{5}{7}\sqrt{x}\right)                   \\
                     & = \left[\int_{0}^{1}x^3\sqrt{2x}\,dx\right]\sqrt{2x} +                                                                                                                                \\
                     & \quad\left[\int_{0}^{1}x^3\sqrt{75}\left(x-\frac{4}{5}\sqrt{x}\right)\,dx\right]\sqrt{75}\left(x-\frac{4}{5}\sqrt{x}\right)+                                                          \\
                     & \quad\left[\int_{0}^{1}x^3\sqrt{\frac{3920}{9}}\left(x^2-\frac{45}{28}x+\frac{5}{7}\sqrt{x}\right)\,dx\right]\sqrt{\frac{3920}{9}}\left(x^2-\frac{45}{28}x+\frac{5}{7}\sqrt{x}\right) \\
                     & =\frac{4}{9}\sqrt{x}+\frac{5}{3}\left(x-\frac{4}{5}\sqrt{x}\right)+\frac{140}{81}\left(x^2-\frac{45}{28}x+\frac{5}{7}\sqrt{x}\right)
          \end{align*}
  \end{enumerate}
\end{soln}
\end{document}