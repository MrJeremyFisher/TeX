\documentclass[10pt]{article}

\usepackage[margin=0.75in]{geometry}
\usepackage{amsmath,amsthm,amssymb}
\usepackage{xcolor}
\usepackage{cancel}
\usepackage{graphicx}
\usepackage{changepage}
\usepackage{circuitikz}
\usepackage{pgfplots}
\usepackage{physics}
\usepackage{hyperref}
\usepackage{siunitx}
\usepackage{fontspec}
\usepackage{relsize}
\usepackage{subfig}
\usepackage{todonotes}
\usepackage{emoji}
\usepackage{minted}
\usepackage{multicol, multirow, booktabs}
\usepackage[breakable]{tcolorbox}
\usepackage[inline]{enumitem}

\theoremstyle{definition}
\newtheorem{problem}{Problem}
\newtheorem{soln}{Solution}

\pgfplotsset{compat=newest}
\usetikzlibrary{lindenmayersystems}
\usetikzlibrary{arrows}
\usetikzlibrary{calc}
\usetikzlibrary{positioning, fit}
\usetikzlibrary{3d, perspective}

\definecolor{incolor}{HTML}{303F9F}
\definecolor{outcolor}{HTML}{D84315}
\definecolor{cellborder}{HTML}{CFCFCF}
\definecolor{cellbackground}{HTML}{F7F7F7}
\newcommand{\ui}{\hat{i}}
\newcommand{\uj}{\hat{j}}
\newcommand{\uk}{\hat{k}}
\newcommand{\ux}{\hat{x}}
\newcommand{\uy}{\hat{y}}
\newcommand{\uz}{\hat{z}}
\newcommand{\primed}[1]{#1^\prime}
\pgfdeclarelayer{background}  
\pgfsetlayers{background,main}
\AtBeginDocument{\RenewCommandCopy\qty\SI}
\newcommand{\justif}[2]{&{#1}&\text{#2}}

\makeatletter
\newcommand{\boxspacing}{\kern\kvtcb@left@rule\kern\kvtcb@boxsep}
\makeatother
\newcommand{\prompt}[4]{
    \ttfamily\llap{{\color{#2}[#3]:\hspace{3pt}#4}}\vspace{-\baselineskip}
}

\newcommand{\thevenin}[2]{
  \begin{center}
    \begin{circuitikz} \draw
      (0,0) -- (2,0) to[battery1, l_=$V_{Th}\eq#1$] (2,2) 
      to[resistor, l_=$R_{Th}\eq#2$] (0,2)
      ;
      \draw [o-] (-.07,2.079);
      \draw [o-] (-.07,0.079);
    \end{circuitikz}
  \end{center}
}

\newcommand{\norton}[2]{
  \begin{center}
    \begin{circuitikz} \draw
      (0,0) -- (3,0) to[american current source, l_=$I_{N}\eq#1$] (3,2) -- (0,2) (2,0)
      to[resistor, l=$R_{N}\eq#2$] (2,2)
      ;
      \draw [o-] (-.07,2.079);
      \draw [o-] (-.07,0.079);
    \end{circuitikz}
  \end{center}
}

\newcommand{\highlight}[1]{\colorbox{yellow}{$\displaystyle #1$}}

\newcommand{\ti}[1]{\widetilde{#1}}

\newfontface{\Kaufmann}{Kaufmann}
\DeclareTextFontCommand{\kf}{\Kaufmann}
\newcommand{\scriptr}{\fontsize{12pt}{12pt}\kf{r}}

\newfontface{\KaufmannB}{Kaufmann Bd BT}
\DeclareTextFontCommand{\kfb}{\KaufmannB}
\newcommand{\bscriptr}{\fontsize{12pt}{12pt}\kfb{r}}

\newcommand{\bv}[1]{\mathbf{#1}}

\title{Math 3310H: Assignment IV}
\author{Jeremy Favro (0805980) \\ Trent University, Peterborough, ON, Canada}
\date{\today}

\begin{document}
\maketitle

% PROBLEM 1
\begin{problem}~
\begin{enumerate}[label=(\alph*)]
  \item Show that $S_n$ for $n \geq 2$ is generated by the $n$ transpositions
        $(1, 2), (1, 3),\dots, (1, n)$.
  \item Let $\alpha,\beta\in S_n$ for $n\geq 2$ and suppose that $\beta$, in cycle notation,
        is given by
        $$\beta=(\beta_1,\dots,\beta_k)$$
        where $\beta_1,\dots,\beta_k \in \left\{1,\dots, n\right\}$.
        It can be shown that
        $$\alpha\beta\alpha^{-1}=(\alpha(\beta_1),\dots,\alpha(\beta_k)).$$
        Practice using this property by computing $\alpha\beta\alpha^{-1}$ for
        $\alpha=(1,2,5,4,3),\,\beta=(1,3,7,4,8,5)$
  \item Show that $S_n$ for $n \geq 2$ is generated by the $n - 1$ transpositions
        $(1, 2), (2, 3), \dots , (n - 1, n)$.
  \item Let $\sigma=(1,2,\dots,n)$ and $\tau=(1,2)$. Show that
        $$\sigma^k\tau\sigma^{-k}=(k+1,k+2)$$
        for $k\in\left\{0,\dots,n-2\right\}$.
  \item Show that $S_n$ for $n\geq 2$ is generated by $\sigma$ and $\tau$.
\end{enumerate}
\end{problem}
\begin{soln}~
  \begin{enumerate}[label=(\alph*)]
    \item Because any cycle can be factored as a product of $(1,k)$ transpositions
          and all such transpositions for $S_n$ are in the given set any cycle
          in $S_n$ can be written as a product of these transpositions, even the two-cycles (bicycles?)
          as $(i,j)=(1,i)(1,j)(1,i)$.
    \item $\alpha^{-1}=(3,4,5,2,1)$.
          Then
          $$\alpha\beta\alpha^{-1}=(1,2,5,4,3)(1,3,7,4,8,5)(3,4,5,2,1)=(1,7,3,8,4,2)$$
          which is indeed the element-wise application of
          $\alpha$ onto $\beta$ as
          $$(\alpha(1),\alpha(3),\alpha(7),\alpha(4),\alpha(8),\alpha(5))
            =(2,1,7,3,8,4)=(1,7,3,8,4,2).
          $$
    \item Because any cycle can be decomposed as a product
          of the given cycles we can write any cycle of length greater than 2
          as a product of transpositions. So as in (a) we need to show that
          we can construct all the transpositions from the given set that are not in said set.
          So, we need to be able to write the transposition $(i,j)$ as a product of $(a,a+1)$
          $a<n$ transpositions. Since $(i,j)=(j,i)$ we can assume
          without loss of generality that $i<j-1$ (i.e. it is not already in the set)
          it is evident then that the product
          $$(i,i+1)(i+1,i+2)\dots(j-1,j)=(i,j)$$
          as we are just ``shuffling'' $i$ and $j$ up and down the chain. All of these permutations
          are in the given set and therefore we are able to construct any transposition and therefore
          the entirety of $S_n$.
    \item $\sigma^{-k}$ will downshift anything it is applied to $k$ times unless it ``underflows''
          in which case $\sigma^{-k}(a)=a-k+n$. $\sigma^{k}$ will do the same except it will upshift
          until it ``overflows'' in which case it will $\sigma^{k}(a)=a+k-n$. Then what the full
          composition will do is first shift everything up (really down and around)
          by $k$ so that $k+1$ sits in the first spot and $k+2$ the second spot,
          then swap these two, then shift back by $k$ so that we return to the original
          setup with the $k+1$ and $k+2$ elements switched around. Returning means we'll end up
          only affecting the elements we hit with the $\tau$ transposition and moved out of place so
          $$\sigma^k\tau\sigma^{-k}=(\sigma^k(1),\sigma^k(2))=(k+1,k+2).$$
          I'm not super happy with the clarity of that explanation but it makes a fair bit of sense.
          I don't know how to really ``prove'' it though beyond my kind of sketch of a proof.
    \item We already have the $(1,2)$ transposition and
          know that as long as we can generate all the transpositions then we can generate any
          cycle. We also have that we can construct any $(k+1,k+2)=(c,c+1)$ transposition from above.
          We also have, from (c), that the $(n,n-1)$ set of transpositions will generate $S_n$, which we can
          create using $\sigma$ and $\tau$ again as above.
  \end{enumerate}
\end{soln}

% PROBLEM 2 
\begin{problem}
Compute all left and right cosets of $\langle3\rangle$ in $\mathbb{Z}_{15}$ under $+_{15}$.
\end{problem}
\begin{soln}
  First note that
  $$\langle 3 \rangle=\left\{0,3,6,9,12\right\}.$$
  Our left cosets then are
  \begin{align*}
    0+\langle 3 \rangle & = \left\{0,3,6,9,12\right\},  \\
    1+\langle 3 \rangle & = \left\{1,4,7,10,13\right\}, \\
    2+\langle 3 \rangle & = \left\{2,5,8,11,14\right\}.
  \end{align*}
  As cosets are disjoint and these contain all of $\mathbb{Z}_15$ other left cosets will just be repetitions.
  The right cosets will the the same as addition is commutative.
\end{soln}

% PROBLEM 3
\begin{problem}
Compute all left and right cosets of $\langle(1,2,3,4)\rangle$ in $S_4$.
\end{problem}
\begin{soln}
  Note that
  $$\langle(1,2,3,4)\rangle=\left\{\iota,(1,2,3,4),(1,3)(2,4), (1,4,3,2)\right\}.$$
  Our left cosets then are
  \begin{align*}
    \iota\langle(1,2,3,4)\rangle & =\left\{\iota,(1,2,3,4),(1,3)(2,4),(1,4,3,2)\right\}, \\
    (1,2)\langle(1,2,3,4)\rangle & =\left\{(1,2),(2,3,4),(1,3,2,4),(1,4,3)\right\},      \\
    (1,3)\langle(1,2,3,4)\rangle & =\left\{(1,3),(1,2)(3,4),(2,4),(1,4)(2,3)\right\},    \\
    (1,4)\langle(1,2,3,4)\rangle & =\left\{(1,4),(1,2,3),(1,3,4,2),(2,4,3)\right\},      \\
    (2,3)\langle(1,2,3,4)\rangle & =\left\{(2,3),(1,3,4),(1,2,4,3),(1,4,2)\right\},      \\
    (3,4)\langle(1,2,3,4)\rangle & =\left\{(3,4),(1,2,4),(1,4,2,3),(1,3,2)\right\}.
  \end{align*}
  Again because cosets are disjoint and we've covered all of $S_4$ these are the only unique cosets.
  Our right cosets are
  \begin{align*}
    \langle(1,2,3,4)\rangle\iota & =\left\{\iota,(1,2,3,4),(1,3)(2,4),(1,4,3,2)\right\}, \\
    \langle(1,2,3,4)\rangle(1,2) & =\left\{(1,2),(1,3,4),(1,4,2,3),(2,4,3)\right\},      \\
    \langle(1,2,3,4)\rangle(1,3) & =\left\{(1,3),(1,4)(2,3),(2,4),(1,2)(3,4)\right\},    \\
    \langle(1,2,3,4)\rangle(1,4) & =\left\{(1,4),(2,3,4),(1,2,4,3),(1,3,2)\right\},      \\
    \langle(1,2,3,4)\rangle(2,3) & =\left\{(2,3),(1,2,4),(1,3,4,2),(1,4,3)\right\},      \\
    \langle(1,2,3,4)\rangle(3,4) & =\left\{(3,4),(1,2,3),(1,3,2,4),(1,4,2)\right\}.
  \end{align*}
\end{soln}

% PROBLEM 4
\begin{problem}
Compute all left and right cosets of $\left\{(1,1),(3,3),(5,5),(7,7)\right\}$ in
$U(8)\times U(8)$.
\end{problem}
\begin{soln}
  Our only left or right coset will be the original set
  $$\left\{(1,1),(3,3),(5,5),(7,7)\right\}$$
  by the property that
  $$aH=H\iff a\in H$$
  because here $H=U(8)\times U(8)$.
\end{soln}

% PROBLEM 5
\begin{problem}
Let $H$ be a subgroup of a group $G$. For $a\in G$, define
$$a^{-1}Ha=\left\{a^{-1}ha|h\in H\right\}$$
\begin{enumerate}[label=(\alph*)]
  \item Prove or disprove: $a^{-1}Ha$ is a subgroup of $G$.
  \item Prove or disprove: $a^{-1}Ha=H$ for each $a\in G$.
\end{enumerate}
\end{problem}
\begin{soln}~
  \begin{enumerate}[label=(\alph*)]
    \item By the subgroup criterion,
          \begin{proof}~
            \begin{enumerate}[label=(\roman*)]
              \item Identity: Let $e\in G$ be the identity in $G$. As $H$ is a subgroup $e\in H$.
                    $e\in a^{-1}Ha$ because $a^{-1}ea=a^{-1}a=e$.
              \item Inverses: Let $a^{-1}ha\in a^{-1}Ha$. Then $(a^{-1}ha)^{-1}=a^{-1}h^{-1}(a^{-1})^{-1}=a^{-1}ha\in a^{-1}Ha$.
              \item Closure: Let $a^{-1}ha,a^{-1}ka\in a^{-1}Ha$. Then their product is
                    $a^{-1}haa^{-1}ka=a^{-1}hka\in a^{-1}Ha$.
            \end{enumerate}
          \end{proof}
    \item Consider $G=S_3=\left\{\iota, (1,2), (1,3), (2,3), (1,2,3), (1,3,2)\right\}$
          and $H=\left\{\iota, (1,2)\right\}$. $(1,3)H(1,3)=\left\{\iota, (2,3)\right\}\neq H$.
  \end{enumerate}
\end{soln}
\end{document}