\documentclass[10pt]{article}

\usepackage[margin=0.75in]{geometry}
\usepackage{amsmath,amsthm,amssymb}
\usepackage{xcolor}
\usepackage{cancel}
\usepackage{graphicx}
\usepackage{changepage}
\usepackage{circuitikz}
\usepackage{pgfplots}
\usepackage{physics}
\usepackage{hyperref}
\usepackage{siunitx}
\usepackage{fontspec}
\usepackage{relsize}
\usepackage{subfig}
\usepackage{todonotes}
\usepackage{emoji}
\usepackage{multicol, multirow, booktabs}
\usepackage[breakable]{tcolorbox}
\usepackage[inline]{enumitem}

\theoremstyle{definition}
\newtheorem{problem}{Problem}
\newtheorem{soln}{Solution}

\pgfplotsset{compat=newest}
\usetikzlibrary{lindenmayersystems}
\usetikzlibrary{arrows}
\usetikzlibrary{calc}
\usetikzlibrary{positioning, fit}
\usetikzlibrary{3d, perspective}

\definecolor{incolor}{HTML}{303F9F}
\definecolor{outcolor}{HTML}{D84315}
\definecolor{cellborder}{HTML}{CFCFCF}
\definecolor{cellbackground}{HTML}{F7F7F7}
\newcommand{\ui}{\hat{i}}
\newcommand{\uj}{\hat{j}}
\newcommand{\uk}{\hat{k}}
\newcommand{\ux}{\hat{x}}
\newcommand{\uy}{\hat{y}}
\newcommand{\uz}{\hat{z}}
\newcommand{\primed}[1]{#1^\prime}
\pgfdeclarelayer{background}  
\pgfsetlayers{background,main}
\AtBeginDocument{\RenewCommandCopy\qty\SI}
\newcommand{\justif}[2]{&{#1}&\text{#2}}

\makeatletter
\newcommand{\boxspacing}{\kern\kvtcb@left@rule\kern\kvtcb@boxsep}
\makeatother
\newcommand{\prompt}[4]{
    \ttfamily\llap{{\color{#2}[#3]:\hspace{3pt}#4}}\vspace{-\baselineskip}
}

\newcommand{\thevenin}[2]{
  \begin{center}
    \begin{circuitikz} \draw
      (0,0) -- (2,0) to[battery1, l_=$V_{Th}\eq#1$] (2,2) 
      to[resistor, l_=$R_{Th}\eq#2$] (0,2)
      ;
      \draw [o-] (-.07,2.079);
      \draw [o-] (-.07,0.079);
    \end{circuitikz}
  \end{center}
}

\newcommand{\norton}[2]{
  \begin{center}
    \begin{circuitikz} \draw
      (0,0) -- (3,0) to[american current source, l_=$I_{N}\eq#1$] (3,2) -- (0,2) (2,0)
      to[resistor, l=$R_{N}\eq#2$] (2,2)
      ;
      \draw [o-] (-.07,2.079);
      \draw [o-] (-.07,0.079);
    \end{circuitikz}
  \end{center}
}

\newcommand{\highlight}[1]{\colorbox{yellow}{$\displaystyle #1$}}

\newcommand{\ti}[1]{\widetilde{#1}}

\newfontface{\Kaufmann}{Kaufmann}
\DeclareTextFontCommand{\kf}{\Kaufmann}
\newcommand{\scriptr}{\fontsize{12pt}{12pt}\kf{r}}

\newfontface{\KaufmannB}{Kaufmann Bd BT}
\DeclareTextFontCommand{\kfb}{\KaufmannB}
\newcommand{\bscriptr}{\fontsize{12pt}{12pt}\kfb{r}}

\newcommand{\bv}[1]{\mathbf{#1}}

\title{Math 3310H: Assignment II}
\author{Jeremy Favro (0805980) \\ Trent University, Peterborough, ON, Canada}
\date{\today}

\begin{document}
\maketitle

% PROBLEM 1
\begin{problem}
Complete the following Cayley table (for a group). (Justify your
results.)
\begin{center}
  \begin{tabular}{c | c c c c c c c c}
    ~ & 1 & 2 & 3 & 4 & 5 & 6 & 7 & 8 \\
    \cline{1-9}
    1 & 1 & 2 & 3 & 4 & 5 & 6 & 7 & 8 \\
    2 & 2 & 1 & 4 & 3 & 6 & 5 & 8 & 7 \\
    3 & 3 & 4 & 2 & 1 & 7 & 8 & 6 & 5 \\
    4 & 4 & 3 & 1 & 2 & 8 & 7 & 5 & 6 \\
    5 & 5 & 6 & 8 & 7 & 1 &   &   &   \\
    6 & 6 & 5 & 7 & 8 &   & 1 &   &   \\
    7 & 7 & 8 & 5 & 6 &   &   & 1 &   \\
    8 & 8 & 7 & 6 & 5 &   &   &   & 1
  \end{tabular}
\end{center}
\end{problem}
\begin{soln}~
  The strategy I settled on for this is to use the existing information in the table (obviously) and to break
  down the unknown entries using the known entries. I found it easiest to look for
  combinations that gave the identity, like in the case of $8* 7$ we'll look for
  some way to make 8 which is the $*$ of something and $7$ as $7*7=1$ which is the identity in this case.
  We see that $8=2*7\implies 8*7=2*7*7=2*1=\colorbox{red!35!white}{2}$.
  Repeating this process for $8*5$ we get $8*5=3*5*5=3*1=\colorbox{blue!20!white}{3}$. Using the Sudoku theorem
  we can automatically fill in the $8*6$ spot with a \colorbox{green!50!white}{4} as that is the only element we haven't used.
  \begin{center}
    \begin{tabular}{c | c c c c c c c c}
      ~ & 1 & 2 & 3 & 4 & 5 & 6 & 7 & 8                            \\
      \cline{1-9}
      1 & 1 & 2 & 3 & 4 & 5 & 6 & 7 & 8                            \\
      2 & 2 & 1 & 4 & 3 & 6 & 5 & 8 & 7                            \\
      3 & 3 & 4 & 2 & 1 & 7 & 8 & 6 & 5                            \\
      4 & 4 & 3 & 1 & 2 & 8 & 7 & 5 & 6                            \\
      5 & 5 & 6 & 8 & 7 & 1 &   &   & \colorbox{blue!20!white}{3}  \\
      6 & 6 & 5 & 7 & 8 &   & 1 &   & \colorbox{green!50!white}{4} \\
      7 & 7 & 8 & 5 & 6 &   &   & 1 & \colorbox{red!35!white}{2}   \\
      8 & 8 & 7 & 6 & 5 &   &   &   & 1
    \end{tabular}
  \end{center}
  Now we can start on the $8$ row. Beginning with $5*8=4*8*8=4*1=\colorbox{blue!20!white}{4}$.
  Now $6*8=3*8*8=3*1=\colorbox{red!35!white}{3}$. Then using the Sudoku theorem $7*8$ must be
  \colorbox{green!50!white}{2}
  \begin{center}
    \begin{tabular}{c | c c c c c c c c}
      ~ & 1 & 2 & 3 & 4 & 5                           & 6                          & 7                            & 8 \\
      \cline{1-9}
      1 & 1 & 2 & 3 & 4 & 5                           & 6                          & 7                            & 8 \\
      2 & 2 & 1 & 4 & 3 & 6                           & 5                          & 8                            & 7 \\
      3 & 3 & 4 & 2 & 1 & 7                           & 8                          & 6                            & 5 \\
      4 & 4 & 3 & 1 & 2 & 8                           & 7                          & 5                            & 6 \\
      5 & 5 & 6 & 8 & 7 & 1                           &                            &                              & 3 \\
      6 & 6 & 5 & 7 & 8 &                             & 1                          &                              & 4 \\
      7 & 7 & 8 & 5 & 6 &                             &                            & 1                            & 2 \\
      8 & 8 & 7 & 6 & 5 & \colorbox{blue!20!white}{4} & \colorbox{red!35!white}{3} & \colorbox{green!50!white}{2} & 1
    \end{tabular}
  \end{center}
  Now if we just fill in the $5*6$ position with $5*6=2*6*6=2*1=\colorbox{red!35!white}{2}$ we obtain
  that $7*6$ must be 3 by Sudoku, then that $7*5$ must be 4, then that $6*5$ must be 2, then that
  $6*7$ must 4, and $5*7$ must be 3.
  \begin{center}
    \begin{tabular}{c | c c c c c c c c}
      ~ & 1 & 2 & 3 & 4 & 5                          & 6 & 7 & 8 \\
      \cline{1-9}
      1 & 1 & 2 & 3 & 4 & 5                          & 6 & 7 & 8 \\
      2 & 2 & 1 & 4 & 3 & 6                          & 5 & 8 & 7 \\
      3 & 3 & 4 & 2 & 1 & 7                          & 8 & 6 & 5 \\
      4 & 4 & 3 & 1 & 2 & 8                          & 7 & 5 & 6 \\
      5 & 5 & 6 & 8 & 7 & 1                          & 2 & 4 & 3 \\
      6 & 6 & 5 & 7 & 8 & \colorbox{red!35!white}{2} & 1 & 3 & 4 \\
      7 & 7 & 8 & 5 & 6 & 3                          & 4 & 1 & 2 \\
      8 & 8 & 7 & 6 & 5 & 4                          & 3 & 2 & 1
    \end{tabular}
  \end{center}
  I think this is sufficient because we've only used the already known-to-be-a-group elements and Sudoku theorem so the new elements
  should preserve associativity. Not entirely sure but pretty confident.
\end{soln}

% PROBLEM 2
\begin{problem}
Determine whether the given subset $H$ is a subgroup of the group
$G$.
\begin{enumerate}[label=(\alph*)]
  \item Let
        $$H=\left\{\left.\begin{pmatrix}
            a & b \\
            c & d
          \end{pmatrix}\in\mathcal{M}_{2\cross 2}(\mathbb{Z})\right| a+b+c+d=0\right\}$$
        and $G=\left(\mathcal{M}_{2\cross 2}, +\right)$.
  \item Let $H=\left\{\left.\frac{1+2m}{1+2n}\right|m,n\in \mathbb{Z} \right\}$ and $G=\left(\mathbb{Q}\backslash \left\{0\right\},\cdot\right)$
  \item Let $H=\left\{(0,0),(1,9),(2,6),(3,3)\right\}$ and $G=\left(\mathbb{Z}_4\cross\mathbb{Z}_{12},+\right)$
\end{enumerate}
\end{problem}
\begin{soln}~
  \begin{enumerate}[label=(\alph*)]
    \item Here we need: \begin{enumerate}[label=(\roman*)]
            \item Closure: Let
                  $$A,B\in S;\quad A=\begin{pmatrix}
                      a & b \\
                      c & d
                    \end{pmatrix};\quad B=\begin{pmatrix}
                      e & f \\
                      g & h
                    \end{pmatrix}.$$
                  Then
                  $$A+B=\begin{pmatrix}
                      a & b \\
                      c & d
                    \end{pmatrix}+\begin{pmatrix}
                      e & f \\
                      g & h
                    \end{pmatrix}=\begin{pmatrix}
                      a+e & b+f \\
                      c+g & d+h
                    \end{pmatrix}$$
                  and so
                  $$(a+e)+(b+f)+(c+g)+(d+h)=(a+b+c+d)+(e+f+g+h)=0+0=0.$$
                  Therefore $H$ is closed under $+$.
            \item $H$ must contain the identity: The identity for matrix addition is just the zero matrix which is obviously
                  contained in $H$ as its elements will sum to zero.
            \item $H$ must contain inverses: For some
                  $$A=\begin{pmatrix}
                      a & b \\
                      c & d
                    \end{pmatrix}$$
                  the inverse of $A$ is $-A$ whose elements sum to zero as their sum will just be the negative of the sum of the elements of $A$,
                  which is zero by definition.
          \end{enumerate}
          As all of the subgroup criterion are satisfied, $H$ is a subgroup of $G$.
    \item Again we need the following:
          \begin{enumerate}[label=(\roman*)]
            \item Closure: Let $a=\frac{1+2m}{1+2n}$ and $b=\frac{1+2k}{1+2q}$. Then
                  $$a\cdot b =\frac{1+2m}{1+2n}\cdot \frac{1+2k}{1+2q}=\frac{1+2m+2k+4mk}{1+2n+2q+4nq}=\frac{1+2(m+k+2mk)}{1+2(n+q+2nq)}.$$
                  Because both $m+k+2mk$ and $n+q+2nq$ will still be integers, $a\cdot b\in H$ and $H$ is therefore closed under $\cdot$.
            \item $H$ must contain the identity: The identity for multiplication is
                  $$1=\frac{1}{1}=\frac{1+2\cdot 0}{1+2\cdot 0}\in H.$$
            \item $H$ must contain inverses: For any element $a=\frac{1+2m}{1+2n}$, $a^{-1}=\frac{1+2n}{1+2m}\in H.$
          \end{enumerate}
          As all subgroup criterion are satisfied, $H$ is a subgroup of $G$.
    \item Again we need the following:
          \begin{enumerate}[label=(\roman*)]
            \item $H$ must contain inverses:
                  \begin{align*}
                    (1,9)+(3,3)=(0,0) \\
                    (2,6)+(2,6)=(0,0) \\
                    (3,3)+(1,9)=(0,0).
                  \end{align*}
            \item The identity here is obviously $(0,0)$ as the operation is addition.
            \item Closure: Here we need to check that repeated addition of each element is closed
                  and that adding each element to another element is closed (recursively).
                  First  for $n\in \mathbb{Z}$,
                  $$\begin{array}{ccccc}
                              & n=2    & n=3    & n=4   &    \\
                      n(1,9)= & (2,6), & (3,3), & (0,0) & ~  \\
                      n(3,3)= & (2,6), & (1,9), & (0,0) & ~  \\
                      n(2,6)= & (0,0)  & ~      & ~     & ~.
                    \end{array}$$
                  Then for addition of individual elements there are several we don't need to check because they are inverses
                  and the operation here is commutative, all we need is:
                  $$(1,9)+(2,6)=(3,3)$$
                  and
                  $$(2,6)+(3,3)=(1,9).$$
                  These actually come up in checking the repeated addition but I find this a little more clear.
          \end{enumerate}
  \end{enumerate}
\end{soln}

% PROBLEM 3
\begin{problem}
Prove that any nonabelian group contains nontrivial subgroups.
\end{problem}
\begin{soln}
  Any nonabelian group $G\neq \left\{e\right\}$ as the trivial group is abelian.
  This means that there exists some $a \in G$ with $a\neq e$ which means that there
  exists a subgroup $H=\langle a\rangle=\left\{a^n|n\in \mathbb{Z}\right\}$.
\end{soln}

% PROBLEM 4
\begin{problem}
Let $G$ be an abelian group. Show that the elements of finite order
in $G$ form a subgroup.
\end{problem}
\begin{soln}
  Let $H$ be the potential subgroup we are considering here. Following the subgroup criterion:
  \begin{enumerate}
    \item Identity: the identity, $e$, (whatever the operation may be) will have order 1 and therefore will be in $H$.
    \item Closure: Let $a,b\in H$. Then $\abs{a}=n$ and $\abs{b}=m$ for some positive $n,m\in \mathbb{Z}$.
          Then, by definition $a^n=e$ and $b^m=e$. We also have that, because $G$ is abelian, $(ab)^k=a^kb^k$ for all $k$.
          So, $(ab)^{nm}=a^{nm}b^{nm}=(a^n)^m(b^m)^n=e^me^n=e\implies \abs{ab}\leq nm$ and so $ab\in H$.
    \item Inverses: Let $a\in H$. Then $\abs{a}=n$ and $a^n=e$ for some positive $n\in \mathbb{Z}$.
          $a^{-1}\in H$ because
          $$(a^{-1})^n=(a^n)^{-1}=e^{-1}=e$$
          which means that $a^{-1}$ has finite order $\abs{a^{-1}}\leq n$ and therefore belongs to $H$.
  \end{enumerate}
\end{soln}

% PROBLEM 5
\begin{problem}
For $n\in \mathbb{N}$ define
$$\mathcal{U}(n)=\left\{\left.x\in\mathbb{Z}_n\right|\gcd(x,n)=1\right\}.$$
Find the order of the groups
\begin{enumerate}[label=(\alph*)]
  \item $\mathcal{U}(10)$
  \item $\mathcal{U}(19)$
  \item $\mathcal{U}(20)$
  \item $\mathcal{U}(36)$
\end{enumerate}
\end{problem}
\begin{soln}~
  \begin{enumerate}[label=(\alph*)]
    \item This is the group of all integers coprime with 10. These are 9, 8, 7, 6, 4, 3, and 1. Therefore $\abs{\mathcal{U}(10)}=7$
    \item 19 is prime so $\mathcal{U}(19)$ will be the group of all $n\in\mathbb{Z}$, $0< n< 19$. These
          are $$\left\{1,2,3,4,5,6,7,8,9,10,11,12,13,14,15,16,17,18\right\}$$ so $\abs{\mathcal{U}(19)}=18$.
    \item The set of $n\in\mathbb{Z}$ which satisfy $\gcd(x,20)=1$ is $\left\{1,3,6,7,8,9,11,12,13,14,15,16,17,18,19\right\}$
          so $\abs{\mathcal{U}(20)}=15$
    \item While looking up coprime numbers to see if there's anything interesting about them (there is), I found out you can
          calculate the number of numbers coprime to another number using Euler's (of course) ``totient function'':
          $$n\prod_{p|n}(1-\frac{1}{p})$$
          where $n$ is the number whose count of coprimes we want to know and the product runs over the  prime numbers $p$ which divide $n$.
          In this case the prime factorization of 36 is $2^23^3$ so the product is
          $$36\left[\left(1-\frac{1}{2}\right)\left(1-\frac{1}{3}\right)\right]=12\implies \abs{\mathcal{U}(36)}=12.$$
  \end{enumerate}
\end{soln}

% PROBLEM 6
\begin{problem}
Let $G$ be a group and $a,b\in G$ such that $ab\neq ba$. Prove that $aba\neq e$.
\end{problem}
\begin{soln}
  Assume, by way of contradiction, that $aba=e$. Then if we multiply $aba=e$ by $a^{-1}$ on the right we get $ab=a^{-1}$.
  But if we multiply by $a^{-1}$ on the left we get $ba=a^{-1}\implies ba=ab$ which is in contradiction with our original statement.
  Therefore $aba\neq e$ if $ab\neq ba$.
\end{soln}
\end{document}