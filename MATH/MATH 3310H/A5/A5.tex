\documentclass[10pt]{article}

\usepackage[margin=0.75in]{geometry}
\usepackage{amsmath,amsthm,amssymb}
\usepackage{xcolor}
\usepackage{cancel}
\usepackage{graphicx}
\usepackage{changepage}
\usepackage{circuitikz}
\usepackage{pgfplots}
\usepackage{physics}
\usepackage{hyperref}
\usepackage{siunitx}
\usepackage{fontspec}
\usepackage{relsize}
\usepackage{subfig}
\usepackage{todonotes}
\usepackage{emoji}
\usepackage{minted}
\usepackage{multicol, multirow, booktabs}
\usepackage[breakable]{tcolorbox}
\usepackage[inline]{enumitem}

\theoremstyle{definition}
\newtheorem{problem}{Problem}
\newtheorem{soln}{Solution}

\pgfplotsset{compat=newest}
\usetikzlibrary{lindenmayersystems}
\usetikzlibrary{arrows}
\usetikzlibrary{calc}
\usetikzlibrary{positioning, fit}
\usetikzlibrary{3d, perspective}

\definecolor{incolor}{HTML}{303F9F}
\definecolor{outcolor}{HTML}{D84315}
\definecolor{cellborder}{HTML}{CFCFCF}
\definecolor{cellbackground}{HTML}{F7F7F7}
\newcommand{\ui}{\hat{i}}
\newcommand{\uj}{\hat{j}}
\newcommand{\uk}{\hat{k}}
\newcommand{\ux}{\hat{x}}
\newcommand{\uy}{\hat{y}}
\newcommand{\uz}{\hat{z}}
\newcommand{\primed}[1]{#1^\prime}
\pgfdeclarelayer{background}  
\pgfsetlayers{background,main}
\AtBeginDocument{\RenewCommandCopy\qty\SI}
\newcommand{\justif}[2]{&{#1}&\text{#2}}

\makeatletter
\newcommand{\boxspacing}{\kern\kvtcb@left@rule\kern\kvtcb@boxsep}
\makeatother
\newcommand{\prompt}[4]{
    \ttfamily\llap{{\color{#2}[#3]:\hspace{3pt}#4}}\vspace{-\baselineskip}
}

\newcommand{\thevenin}[2]{
  \begin{center}
    \begin{circuitikz} \draw
      (0,0) -- (2,0) to[battery1, l_=$V_{Th}\eq#1$] (2,2) 
      to[resistor, l_=$R_{Th}\eq#2$] (0,2)
      ;
      \draw [o-] (-.07,2.079);
      \draw [o-] (-.07,0.079);
    \end{circuitikz}
  \end{center}
}

\newcommand{\norton}[2]{
  \begin{center}
    \begin{circuitikz} \draw
      (0,0) -- (3,0) to[american current source, l_=$I_{N}\eq#1$] (3,2) -- (0,2) (2,0)
      to[resistor, l=$R_{N}\eq#2$] (2,2)
      ;
      \draw [o-] (-.07,2.079);
      \draw [o-] (-.07,0.079);
    \end{circuitikz}
  \end{center}
}

\newcommand{\highlight}[1]{\colorbox{yellow}{$\displaystyle #1$}}

\newcommand{\ti}[1]{\widetilde{#1}}

\newfontface{\Kaufmann}{Kaufmann}
\DeclareTextFontCommand{\kf}{\Kaufmann}
\newcommand{\scriptr}{\fontsize{12pt}{12pt}\kf{r}}

\newfontface{\KaufmannB}{Kaufmann Bd BT}
\DeclareTextFontCommand{\kfb}{\KaufmannB}
\newcommand{\bscriptr}{\fontsize{12pt}{12pt}\kfb{r}}

\newcommand{\bv}[1]{\mathbf{#1}}

\title{Math 3310H: Assignment V}
\author{Jeremy Favro (0805980) \\ Trent University, Peterborough, ON, Canada}
\date{\today}

\begin{document}
\maketitle

% PROBLEM 1
\begin{problem}
Find all abelian groups (up to isomorphism) of the given order.
\begin{enumerate}[label=(\alph*)]
  \item 2025
  \item 1234
\end{enumerate}
\end{problem}
\begin{soln}
  \begin{enumerate}[label=(\alph*)]
    \item Note that $2025=3^4\cdot5^2$.
          So we get
          $$\mathbb{Z}_{2025}$$
          $$ \mathbb{Z}_3\times \mathbb{Z}_3 \times \mathbb{Z}_3 \times \mathbb{Z}_3 \times \mathbb{Z}_5 \times \mathbb{Z}_5$$
          $$ \mathbb{Z}_9 \times \mathbb{Z}_3 \times \mathbb{Z}_3 \times \mathbb{Z}_5 \times \mathbb{Z}_5$$
          $$ \mathbb{Z}_{27} \times \mathbb{Z}_3 \times \mathbb{Z}_5 \times \mathbb{Z}_5$$
          $$ \mathbb{Z}_{81} \times \mathbb{Z}_5 \times \mathbb{Z}_5$$
          $$ \mathbb{Z}_3\times \mathbb{Z}_3 \times \mathbb{Z}_3 \times \mathbb{Z}_3 \times \mathbb{Z}_{25}$$
          $$ \mathbb{Z}_9 \times \mathbb{Z}_3 \times \mathbb{Z}_3 \times \mathbb{Z}_{25}$$
          $$ \mathbb{Z}_{27} \times \mathbb{Z}_3 \times \mathbb{Z}_{25}$$
          $$ \mathbb{Z}_9 \times \mathbb{Z}_9 \times \mathbb{Z}_{25}$$
          $$ \mathbb{Z}_9 \times \mathbb{Z}_9 \times \mathbb{Z}_5 \times \mathbb{Z}_5$$
    \item Note that $1234=2\cdot617$.
          So we get just
          $$\mathbb{Z}_{1234}.$$
  \end{enumerate}
\end{soln}

% PROBLEM 2
\begin{problem}
Determine whether or not $U (8)$ is isomorphic to $U (5)$.
\end{problem}
\begin{soln}
  Well,
  $$U(8)=\left\{1,3,5,7\right\}$$
  and
  $$U(5)=\left\{1,2,3,4\right\}$$
  so they are both of order 4. They are also both abelian and so must both be isomorphic to
  $\mathbb{Z}_4$ or $\mathbb{Z}_2\times\mathbb{Z}_2$ respectively. If one is cyclic then it will be
  $\mathbb{Z}_4$, otherwise it will be $\mathbb{Z}_2\times\mathbb{Z}_2$.
  $U(8)$ is not cyclic as none of its elements are of order $8$. $U(5)$ is cyclic with generator $2$.
  Since these two are not isomorphic to the same groups (and those groups are not isomorphic) the groups themselves are not isomorphic.
\end{soln}

% PROBLEM 3 
\begin{problem}
Find all homomorphisms $\phi$ between the given groups.
\begin{enumerate}[label=(\alph*)]
  \item $\phi:\mathbb{Z}_6\to \mathbb{Z}_9$
  \item $\phi:\mathbb{Z}_{18}\to \mathbb{Z}_{12}$
  \item $\phi:\mathbb{Z}_{30}\to \mathbb{Z}_{13}$
  \item $\phi:\mathbb{Z}_4\to \mathbb{Z}_2\times \mathbb{Z}_2$
\end{enumerate}
\end{problem}
\begin{soln}~
  \begin{enumerate}[label=(\alph*)]
    \item We know that a map $\phi:\mathbb{Z}_n\to\mathbb{Z}_m$ where $\phi(x)=kx$ will only be a homomorphism iff
          $\abs{k}$ in $\mathbb{Z}_n$ is a divisor of both $n$ and $m$. Here the divisors of the first group are
          $1,2,3,6$ and the divisors of the second group are $1,3,9$. The common divisors are $1,3$ so we need to find elements of
          $\mathbb{Z}_{6}$ with orders $1$ or $3$. We know that elements $k$ of order $3$ will satisfy
          $$3=\frac{6}{\gcd(k,6)}\implies 2=\gcd(k,6).$$
          The elements in $\mathbb{Z}_6$ which satisfy this are $2,4$. Now for $1$
          we need elements which satisfy $6=\gcd(k,6)$ which is just $0$. So all the homomorphisms are
          $\phi_{0}(x)=0$, $\phi_2(x)=2x$, $\phi_4(x)=4x$.
    \item Following same steps as previously we have common divisors of the order of both groups
          $1,2,3,6$ and hence we need all elements of these orders in $\mathbb{Z}_{18}$
          so
          $$6=\frac{18}{\gcd(k,18)}\implies k=3,15$$
          $$3=\frac{18}{\gcd(k,18)}\implies k=6,12$$
          $$2=\frac{18}{\gcd(k,18)}\implies k=9$$
          $$1=\frac{18}{\gcd(k,18)}\implies k=0$$
          so we have all homomorphisms being all
          $\phi_k(x)=kx$ where $k\in\left\{0,3,6,9,12,15\right\}$.
    \item As $13$ is prime, the only common divisor here is $1$ and the only element of order $1$ is zero, hence the only homomorphism
          here is the trivial one.
    \item Since all the elements of $\mathbb{Z}_2\times \mathbb{Z}_2$ are of order 1 (the identity) or 2, both of which divide the order
          of $\mathbb{Z}_4$, all the elements of $\mathbb{Z}_2\times \mathbb{Z}_2$ form homomorphisms. So we have
          $\phi_k(x)=kx$ where $k\in \mathbb{Z}_2\times \mathbb{Z}_2=\left\{(0,0),(1,0),(0,1),(1,1)\right\}$.
  \end{enumerate}
\end{soln}

% PROBLEM 4 
\begin{problem}
Suppose $\phi:\mathbb{Z}_{50}\to\mathbb{Z}_{15}$ is a group homomorphism with $\phi(7)=6$.
\begin{enumerate}[label=(\alph*)]
  \item Determine $\phi(x)$ where $x\in\mathbb{Z}_{50}$.
  \item Find $\phi\left[\mathbb{Z}_{50}\right]$.
  \item Find $\ker \phi$.
  \item Find $\phi^{-1}(3)$.
\end{enumerate}
\end{problem}
\begin{soln}~
  \begin{enumerate}[label=(\alph*)]
    \item Because $\gcd(7,50)=1$, $\langle 7\rangle=\mathbb{Z}_{50}$ so
          $\phi(7+7+\dots)=\phi(7)+\phi(7)+\dots=6+6+\dots\implies \phi(7n)=6n$, mod their respective
          group orders of course. Hence for $k\in\mathbb{Z}_{50}$ where $k=7n$, $\phi(k)=6n$. Note that
          $n$ for $1$ is 43 and so $\phi(1)=6\cdot_{15}43=3\implies \phi(x)=3\cdot_{15}x$.
    \item Under the definition above this is just
          $$\phi\left[\mathbb{Z}_{50}\right]=\langle 3\rangle=\left\{0,3,6,9,12\right\}.$$
    \item By definition,
          \begin{align*}
            \ker \phi & =\left\{x\in \mathbb{Z}_{50}|3x\equiv_{15}0 \right\} \\
                      & =\left\{0,5,10,15,20,25,30,35,40,45\right\}
          \end{align*}
    \item Recalling that if $\phi(g)=h$ then $\phi^{-1}(h)=g\ker\phi$ we have that because
          $\phi(1)=3$, $\phi^{-1}(3)=\ker\phi$.
  \end{enumerate}
\end{soln}

% PROBLEM 5
\begin{problem}
Let $G$ be a group of order 24, and suppose $\phi:\mathbb{Z}_{36}\to G$ is a group homomorphism.
\begin{enumerate}[label=(\alph*)]
  \item Find all possible images $\phi\left[\mathbb{Z}_{36}\right]$ for such a map $\phi$.
  \item What is $\ker\phi$ for each of the maps in part (a)?
\end{enumerate}
\end{problem}
\begin{soln}~
  \begin{enumerate}[label=(\alph*)]
    \item Since $\phi(1)=k$ defines $\phi(x)=kx$ and we know that $\abs{k}$ must divide the order of
          the input and output groups, $k$ must take on one of the values $1,2,3,4,6,12$ as these are the common divisors
          of $36$ and $24$. Additionally, since homomorphisms preserve the cyclic and abelian nature of a group in its image
          we know that the image must be both cyclic and abelian, as $\mathbb{Z}_{36}$ is both of these. As all cyclic groups
          of order $n$ are isomorphic to $\mathbb{Z}_n$, the image $\mathbb{Z}_{36}$ under $\phi$ could be
          $\mathbb{Z}_k$ for $k\in \left\{1,2,3,4,6,12\right\}$.
    \item For $\phi_1:\mathbb{Z}_{36}\to \mathbb{Z}_1$ we know that $\ker\phi_1 < \mathbb{Z}_{36}$
          and hence that $\mathbb{Z}_{36}/\ker\phi_1\cong \mathbb{Z}_1$. The order of $\mathbb{Z}_1$ is 1 as it
          contains only the identity and hence $\abs{\mathbb{Z}_{36}/\ker\phi}=1\implies \abs{\ker\phi_1 }=36$. We also know that this
          kernel group should be cyclic and generated by an element of order $36$. We know that
          for $k\in\mathbb{Z}_{36}$ $k$ must satisfy $\abs{\langle k\rangle}=36=\frac{36}{\gcd(k,36)}\implies \gcd(k,36)=1$
          so we can pick anything relatively prime to $36$. $1$ will work and so $\ker\phi_1=\mathbb{Z}_{36}$.

          Now for
          $\phi_2:\mathbb{Z}_{36}\to \mathbb{Z}_2$ we repeat the same process: $\abs{\mathbb{Z}_2}=2\implies
            \abs{\mathbb{Z}_{36}/\ker\phi}=2\implies \abs{\ker\phi_2 }=18$, so we seek a $k$ of order $18$ in $\mathbb{Z}_{36}$. $k$ will
          satisfy $18=\frac{36}{\gcd(k,36)}\implies 2=\gcd(k,36)$ so we can pick $2$ as a generator which gives
          $$\ker\phi_2=\left\{0,2,4,6,8,10,12,14,16,18,20,22,24,26,28,30,32,34\right\}.$$


          Again we go by the same process for $\phi_3:\mathbb{Z}_{36}\to \mathbb{Z}_3$:
          $\abs{\mathbb{Z}_3}=3\implies
            \abs{\mathbb{Z}_{36}/\ker\phi}=3\implies \abs{\ker\phi_3 }=12$, so want a $k$ of order $12$ which will satisfy
          $12=\frac{36}{\gcd(k,36)}\implies 3=\gcd(k,36)$ so we can pick $3$ as a generator which gives
          $$\ker\phi_3=\left\{0,3,6,9,12,15,18,21,24,27,30,33\right\}.$$


          Again for $\phi_4:\mathbb{Z}_{36}\to \mathbb{Z}_4$: $\abs{\mathbb{Z}_4}=4\implies
            \abs{\mathbb{Z}_{36}/\ker\phi}=4\implies \abs{\ker\phi_4 }=9$ so we want $k$ such that
          $9=\frac{36}{\gcd(k,36)}\implies 4=\gcd(k,36)\implies k=4$ so
          $$\ker\phi_4=\left\{0,4,8,12,16,20,24,28,32\right\}.$$

          Again for $\phi_6:\mathbb{Z}_{36}\to \mathbb{Z}_6$: $\abs{\mathbb{Z}_6}=6\implies
            \abs{\mathbb{Z}_{36}/\ker\phi}=6\implies \abs{\ker\phi_6 }=6$ so we want $k$ such that
          $6=\frac{36}{\gcd(k,36)}\implies 6=\gcd(k,36)\implies k=6$ so 
          $$\ker\phi_6=\left\{0,6,12,18,24,30\right\}.$$

          And one last time for $\phi_6:\mathbb{Z}_{36}\to \mathbb{Z}_{12}$: $\abs{\mathbb{Z}_{12}}=12\implies
            \abs{\mathbb{Z}_{36}/\ker\phi}=12\implies \abs{\ker\phi_{12} }=3$ so we want $k$ such that
          $3=\frac{36}{\gcd(k,36)}\implies 12=\gcd(k,36)\implies k=12$ so 
          $$\ker\phi_{12}=\left\{0,12,24\right\}.$$


  \end{enumerate}
\end{soln}

% PROBLEM 6
\begin{problem}
Find the isomorphism class of the following factor groups:
\begin{enumerate}[label=(\alph*)]
  \item $U(32)/H$ where $H=\left\{1,15\right\}$.
  \item $\mathbb{Z}_{18}\times \mathbb{Z}_{24}/\langle\left(2,12\right)\rangle$.
\end{enumerate}
\end{problem}
\begin{soln}~
  \begin{enumerate}[label=(\alph*)]
    \item We know that $\abs{U(32)/\left\{1,15\right\}}=16/2=8$. Since this group is abelian
          it is isomorphic to $\mathbb{Z}_2\times\mathbb{Z}_2\times\mathbb{Z}_2$, or
          $\mathbb{Z}_2\times \mathbb{Z}_4$, or $\mathbb{Z}_8$. We can look at the orders of the elements of the factor
          group to determine which these it is. Consider
          the $3\cdot\left\{1,15\right\}$ coset where we'll check the orders that would determine which of the previous groups
          this is isomorphic to
          \begin{align*}
            3\cdot \left\{1,15\right\}  & =\left\{3,13\right\}  \\
            3^2\cdot\left\{1,15\right\} & =\left\{9,4\right\}   \\
            3^4\cdot\left\{1,15\right\} & =\left\{17,31\right\} \\
            3^8\cdot\left\{1,15\right\} & =\left\{1,15\right\}.
          \end{align*}
          So $U(32)/\left\{1,15\right\}$ contains an element of order $8$ and can therefore only be isomorphic to $\mathbb{Z}_8$
    \item We know that $\abs{\mathbb{Z}_{18}\times \mathbb{Z}_{24}/\langle\left(2,12\right)\rangle}=432/9=48$
          since
          $$\langle\left(2,12\right)\rangle=\left\{
            (0,0), (2,12), (4,0), (6,12),
            (8,0), (10, 12), (12, 0), (14, 12),
            (16, 12)
            \right\}.$$
          This could be isomorphic to quite a few things however we can see that
          the element $(1,0) + \langle\left(2,12\right)\rangle$ has order 18 and so we can drop all potential groups
          that could not produce an element of such an order. This cuts us down to just
          $\mathbb{Z}_{48}$ as $48=2^4\cdot3$ so there is no other way we can construct a group containing elements of
          order $18$ that is abelian.
  \end{enumerate}
\end{soln}
\end{document}