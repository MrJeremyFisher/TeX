\documentclass[10pt]{article}

\usepackage[margin=0.75in]{geometry}
\usepackage{amsmath,amsthm,amssymb}
\usepackage{xcolor}
\usepackage{cancel}
\usepackage{graphicx}
\usepackage{changepage}
\usepackage{circuitikz}
\usepackage{pgfplots}
\usepackage{physics}
\usepackage{hyperref}
\usepackage{siunitx}
\usepackage{fontspec}
\usepackage{relsize}
\usepackage{subfig}
\usepackage{todonotes}
\usepackage{multicol, multirow, booktabs}
\usepackage[breakable]{tcolorbox}
\usepackage[inline]{enumitem}

\theoremstyle{definition}
\newtheorem{problem}{Problem}
\newtheorem{soln}{Solution}

\pgfplotsset{compat=newest}
\usetikzlibrary{lindenmayersystems}
\usetikzlibrary{arrows}
\usetikzlibrary{calc}
\usetikzlibrary{positioning, fit}
\usetikzlibrary{3d, perspective}

\definecolor{incolor}{HTML}{303F9F}
\definecolor{outcolor}{HTML}{D84315}
\definecolor{cellborder}{HTML}{CFCFCF}
\definecolor{cellbackground}{HTML}{F7F7F7}
\newcommand{\ui}{\hat{i}}
\newcommand{\uj}{\hat{j}}
\newcommand{\uk}{\hat{k}}
\newcommand{\ux}{\hat{x}}
\newcommand{\uy}{\hat{y}}
\newcommand{\uz}{\hat{z}}
\newcommand{\primed}[1]{#1^\prime}
\pgfdeclarelayer{background}  
\pgfsetlayers{background,main}
\AtBeginDocument{\RenewCommandCopy\qty\SI}

\makeatletter
\newcommand{\boxspacing}{\kern\kvtcb@left@rule\kern\kvtcb@boxsep}
\makeatother
\newcommand{\prompt}[4]{
    \ttfamily\llap{{\color{#2}[#3]:\hspace{3pt}#4}}\vspace{-\baselineskip}
}

\newcommand{\thevenin}[2]{
  \begin{center}
    \begin{circuitikz} \draw
      (0,0) -- (2,0) to[battery1, l_=$V_{Th}\eq#1$] (2,2) 
      to[resistor, l_=$R_{Th}\eq#2$] (0,2)
      ;
      \draw [o-] (-.07,2.079);
      \draw [o-] (-.07,0.079);
    \end{circuitikz}
  \end{center}
}

\newcommand{\norton}[2]{
  \begin{center}
    \begin{circuitikz} \draw
      (0,0) -- (3,0) to[american current source, l_=$I_{N}\eq#1$] (3,2) -- (0,2) (2,0)
      to[resistor, l=$R_{N}\eq#2$] (2,2)
      ;
      \draw [o-] (-.07,2.079);
      \draw [o-] (-.07,0.079);
    \end{circuitikz}
  \end{center}
}

\newcommand{\highlight}[1]{\colorbox{yellow}{$\displaystyle #1$}}

\newcommand{\ti}[1]{\widetilde{#1}}

\newfontface{\Kaufmann}{Kaufmann}
\DeclareTextFontCommand{\kf}{\Kaufmann}
\newcommand{\scriptr}{\fontsize{12pt}{12pt}\kf{r}}

\newfontface{\KaufmannB}{Kaufmann Bd BT}
\DeclareTextFontCommand{\kfb}{\KaufmannB}
\newcommand{\bscriptr}{\fontsize{12pt}{12pt}\kfb{r}}

\newcommand{\bv}[1]{\mathbf{#1}}

\title{Math 3310H: Assignment I}
\author{Jeremy Favro (0805980) \\ Trent University, Peterborough, ON, Canada}
\date{\today}

\begin{document}
\maketitle

% PROBLEM 1
\begin{problem}
Define a relation $\mathbb{R}\cross\mathbb{R}$ by $(a,b)~(c,d)$ if $2(a-c)-3(b-d)=0$
\begin{enumerate}[label=(\alph*)]
  \item Show that $~$ is an equivalence relation on $\mathbb{R}$.
  \item Give an example of two pairs $(a,b),(c,d)\in\mathbb{R}\cross\mathbb{R}$, which lie in the same
        equivalence class, and two pairs that don't.
  \item This equivalence relation partitions the 2D plane $\mathbb{R}\cross\mathbb{R}$ into subregions.
        What does the equivalence class $(a,b)$ look like as a region of the plane?
\end{enumerate}
\end{problem}
\begin{soln}

\end{soln}

% PROBLEM 2
\begin{problem}
For each of the following sets $S$, determine whether $S$ is closed under
addition modulo $n$, or multiplication modulo $n$, or both or neither.
(Addition and multiplication modulo $n$ are defined in Exercise Set 2).
\begin{enumerate}[label=(\alph*)]
  \item $S=\left\{0,4,8,12\right\},n=16$.
  \item $S=\left\{0,3,6,9,12\right\},n=15$.
  \item $S=\left\{1,2,3,4\right\},n=5$.
  \item $S=\left\{0,2,3,4,6,8,9,10\right\},n=12$.
  \item $S=\left\{1,5,7,11\right\},n=12$.
\end{enumerate}
\end{problem}
\begin{soln}
\end{soln}

% PROBLEM 3
\begin{problem}
Determine whether the given binary operation $*$ is commutative,
associative, both or neither. Justify your answers with proof.
\begin{enumerate}[label=(\alph*)]
  \item The operation $*$ on $\mathbb{Z}$ given by $a*b=a+b+ab$
  \item The operation $*$ on $\mathbb{R}$ given by $a*b=a+b-ab$
  \item The operation $*$ on $\mathbb{R}$ given by $a*b=a+2ab$
  \item The operation $*$ on $\mathbb{Z}\cross\mathbb{Z}$ given by $(a,b)*(c,d)=(ad+bc, bd)$
  \item The operation $*$ on $\mathbb{Z}\cross\mathbb{Z}$ given by $(a,b)*(c,d)=(ad,bc)$
\end{enumerate}
\end{problem}
\begin{soln}
\end{soln}

% PROBLEM 4
\begin{problem}
Let $S$ be a nonempty set. A binary algebraic structure $(S, *)$ is called
a semigroup if $*$ is associative.
\begin{enumerate}[label=(\alph*)]
  \item Let $S$ be the set of positive rational numbers. Show that $(S, *)$
        is a commutative semigroup if
        $$a*b=\frac{ab}{a+b}$$
        (the usual operations on the right) for all $a,b\in S$
  \item Let $S$ be a set containing more than one element. Define
        $$a*b=b$$
        for all $a,b\in S$. Show that $(S,*)$ is a noncommutative semigroup with no identity element.
\end{enumerate}
\end{problem}
\begin{soln}
\end{soln}
\end{document}