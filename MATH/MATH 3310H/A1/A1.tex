\documentclass[10pt]{article}

\usepackage[margin=0.75in]{geometry}
\usepackage{amsmath,amsthm,amssymb}
\usepackage{xcolor}
\usepackage{cancel}
\usepackage{graphicx}
\usepackage{changepage}
\usepackage{circuitikz}
\usepackage{pgfplots}
\usepackage{physics}
\usepackage{hyperref}
\usepackage{siunitx}
\usepackage{fontspec}
\usepackage{relsize}
\usepackage{subfig}
\usepackage{todonotes}
\usepackage{multicol, multirow, booktabs}
\usepackage[breakable]{tcolorbox}
\usepackage[inline]{enumitem}

\theoremstyle{definition}
\newtheorem{problem}{Problem}
\newtheorem{soln}{Solution}

\pgfplotsset{compat=newest}
\usetikzlibrary{lindenmayersystems}
\usetikzlibrary{arrows}
\usetikzlibrary{calc}
\usetikzlibrary{positioning, fit}
\usetikzlibrary{3d, perspective}

\definecolor{incolor}{HTML}{303F9F}
\definecolor{outcolor}{HTML}{D84315}
\definecolor{cellborder}{HTML}{CFCFCF}
\definecolor{cellbackground}{HTML}{F7F7F7}
\newcommand{\ui}{\hat{i}}
\newcommand{\uj}{\hat{j}}
\newcommand{\uk}{\hat{k}}
\newcommand{\ux}{\hat{x}}
\newcommand{\uy}{\hat{y}}
\newcommand{\uz}{\hat{z}}
\newcommand{\primed}[1]{#1^\prime}
\pgfdeclarelayer{background}  
\pgfsetlayers{background,main}
\AtBeginDocument{\RenewCommandCopy\qty\SI}
\newcommand{\justif}[2]{&{#1}&\text{#2}}

\makeatletter
\newcommand{\boxspacing}{\kern\kvtcb@left@rule\kern\kvtcb@boxsep}
\makeatother
\newcommand{\prompt}[4]{
    \ttfamily\llap{{\color{#2}[#3]:\hspace{3pt}#4}}\vspace{-\baselineskip}
}

\newcommand{\thevenin}[2]{
  \begin{center}
    \begin{circuitikz} \draw
      (0,0) -- (2,0) to[battery1, l_=$V_{Th}\eq#1$] (2,2) 
      to[resistor, l_=$R_{Th}\eq#2$] (0,2)
      ;
      \draw [o-] (-.07,2.079);
      \draw [o-] (-.07,0.079);
    \end{circuitikz}
  \end{center}
}

\newcommand{\norton}[2]{
  \begin{center}
    \begin{circuitikz} \draw
      (0,0) -- (3,0) to[american current source, l_=$I_{N}\eq#1$] (3,2) -- (0,2) (2,0)
      to[resistor, l=$R_{N}\eq#2$] (2,2)
      ;
      \draw [o-] (-.07,2.079);
      \draw [o-] (-.07,0.079);
    \end{circuitikz}
  \end{center}
}

\newcommand{\highlight}[1]{\colorbox{yellow}{$\displaystyle #1$}}

\newcommand{\ti}[1]{\widetilde{#1}}

\newfontface{\Kaufmann}{Kaufmann}
\DeclareTextFontCommand{\kf}{\Kaufmann}
\newcommand{\scriptr}{\fontsize{12pt}{12pt}\kf{r}}

\newfontface{\KaufmannB}{Kaufmann Bd BT}
\DeclareTextFontCommand{\kfb}{\KaufmannB}
\newcommand{\bscriptr}{\fontsize{12pt}{12pt}\kfb{r}}

\newcommand{\bv}[1]{\mathbf{#1}}

\title{Math 3310H: Assignment I}
\author{Jeremy Favro (0805980) \\ Trent University, Peterborough, ON, Canada}
\date{\today}

\begin{document}
\maketitle

% PROBLEM 1
\begin{problem}
Define a relation $\mathbb{R}\cross\mathbb{R}$ by $(a,b)\sim(c,d)$ if $2(a-c)-3(b-d)=0$
\begin{enumerate}[label=(\alph*)]
  \item Show that $\sim$ is an equivalence relation on $\mathbb{R}$.
  \item Give an example of two pairs $(a,b),(c,d)\in\mathbb{R}\cross\mathbb{R}$, which lie in the same
        equivalence class, and two pairs that don't.
  \item This equivalence relation partitions the 2D plane $\mathbb{R}\cross\mathbb{R}$ into subregions.
        What does the equivalence class $(a,b)$ look like as a region of the plane?
\end{enumerate}
\end{problem}
\begin{soln}~
  \begin{enumerate}[label=(\alph*)]
    \item For $\sim$ to be an equivalence relation it must satisfy the following properties for a set $S$ (proofs included)
          \begin{enumerate}[label=(\roman*)]
            \item Reflexivity: $x\sim x\,\forall x\in S$.
                  \begin{proof}
                    Let $(a,b)\in\mathbb{R}\cross\mathbb{R}$, then
                    \begin{align*}
                               & (a,b)\stackrel{?}{\sim}(a,b) \\
                      \implies & 2(a-a)-3(b-b) = 0
                    \end{align*}
                    Which satisfies our relation as defined. Therefore the relation is reflexive. \qedhere
                  \end{proof}
            \item Symmetry: $x\sim y\implies  y\sim x\,\forall x,y\in S$
                  \begin{proof}
                    Let $(a,b), (c,d)\in\mathbb{R}\cross\mathbb{R}$, then
                    \begin{align*}
                               & (a,b)\sim(c,d)          \\
                      \implies & 2(a-c)-3(b-d) = 0       \\
                      \implies & 2(a-c) = 3(b-d)         \\
                      \implies & -2(a-c) = -3(b-d)       \\
                      \implies & 2(c-a) = 3(d-b)         \\
                      \implies & 2(c-a)-3(d-b) =0        \\
                      \implies & (c,d)\sim(a,b) \qedhere
                    \end{align*}
                  \end{proof}
                  \newpage
            \item Transitivity: $x\sim y\sim z\implies  x\sim z\,\forall x,y,z\in S$
                  \begin{proof}
                    Let $(a,b), (c,d), (e,f)\in\mathbb{R}\cross\mathbb{R}$, then
                    \begin{align*}
                               & (a,b)\sim(c,d)    \\
                      \implies & 2(a-c)-3(b-d) = 0 \\
                    \end{align*}
                    and
                    \begin{align*}
                               & (c,d)\sim(e,f)    \\
                      \implies & 2(c-e)-3(d-f) = 0 \\
                    \end{align*}
                    so
                    \begin{align*}
                               & 2(a-c)-3(c-d) + 2(c-e)-3(d-f)  = 0 \\
                      \implies & 2(a-c + c-e)-3(b-d + d-f)  = 0     \\
                      \implies & 2(a-e)-3(b-f)  = 0                 \\
                      \implies & (a,b)\sim(e,f) \qedhere            \\
                    \end{align*}
                  \end{proof}
          \end{enumerate}
          Therefore $\sim$ is an equivalence relation on $\mathbb{R}\cross\mathbb{R}$.
    \item For representative element $(1,1)$ we get that for an element $(a,b)\in \mathbb{R}\cross\mathbb{R}$
          to belong to the associated equivalence class we must have
          $$2(1-a)-3(1-b)=0$$
          which can be rearranged to obtain
          $$a=-\frac{1-3b}{2}$$
          so for $b=\pm 1$ we get two members of the equivalence class represented by $(1,1)$ under $\sim$, $(1,1)$ and $(-2,1)$.
          The elements $(\pi, e)$ and $(\phi, i^i)$ where $\pi,e$ take on their usual definitions, $\phi$ is the golden ratio and $i^i$ is, interestingly,
          both transcendental \emph{and} real!
    \item The equivalence class with representative $(a,b)$ is the set $E=\left\{(x,y)\in \mathbb{R}\cross\mathbb{R} | x\sim (a,b)\right\}$. This gives the equation
          $$2(a-x)-3(b-y)=0\implies y=\frac{2(a-x)-3b}{-3}$$
          so the class looks like a line with slope $2/3$ and y-intercept $b-2a/3$
  \end{enumerate}
\end{soln}
\newpage

% PROBLEM 2
\begin{problem}
For each of the following sets $S$, determine whether $S$ is closed under
addition modulo $n$, or multiplication modulo $n$, or both or neither.
(Addition and multiplication modulo $n$ are defined in Exercise Set 2).
\begin{enumerate}[label=(\alph*)]
  \item $S=\left\{0,4,8,12\right\},n=16$.
  \item $S=\left\{0,3,6,9,12\right\},n=15$.
  \item $S=\left\{1,2,3,4\right\},n=5$.
  \item $S=\left\{0,2,3,4,6,8,9,10\right\},n=12$.
  \item $S=\left\{1,5,7,11\right\},n=12$.
\end{enumerate}
\end{problem}
\begin{soln}~\\
  \begin{enumerate}[label=(\alph*)]
    \item ~\begin{center}
            \setlength\extrarowheight{3pt}
            \noindent\begin{tabular}{c | c c c c c}
              $+_{16}$ & 0  & 4  & 8  & 12 \\
              \cline{1-5}
              0        & 0  & 4  & 8  & 12 \\
              4        & 4  & 8  & 12 & 0  \\
              8        & 8  & 12 & 0  & 4  \\
              12       & 12 & 0  & 4  & 8  \\
            \end{tabular}
            \qquad
            \setlength\extrarowheight{3pt}
            \noindent\begin{tabular}{c | c c c c c}
              $\cdot_{16}$ & 0 & 4 & 8 & 12 \\
              \cline{1-5}
              0            & 0 & 0 & 0 & 0  \\
              4            & 0 & 0 & 0 & 0  \\
              8            & 0 & 0 & 0 & 0  \\
              12           & 0 & 0 & 0 & 0  \\
            \end{tabular}
          \end{center}
          That these tables, being every possible combination of elements on each set with their respective operations
          contain no elements not members of $S$ means that both are closed under $+_{16}$ and $\cdot_{16}$.
    \item ~\begin{center}
            \setlength\extrarowheight{3pt}
            \noindent\begin{tabular}{c | c c c c c}
              $+_{15}$ & 0  & 3  & 6  & 9  & 12 \\
              \cline{1-6}
              0        & 0  & 3  & 6  & 9  & 12 \\
              3        & 3  & 6  & 9  & 12 & 0  \\
              6        & 6  & 9  & 12 & 0  & 3  \\
              9        & 9  & 12 & 0  & 3  & 6  \\
              12       & 12 & 0  & 3  & 6  & 9  \\
            \end{tabular}
            \qquad
            \noindent\begin{tabular}{c | c c c c c}
              $\cdot_{15}$ & 0 & 3  & 6  & 9  & 12 \\
              \cline{1-6}
              0            & 0 & 0  & 0  & 0  & 0  \\
              3            & 0 & 9  & 3  & 12 & 6  \\
              6            & 0 & 3  & 6  & 9  & 12 \\
              9            & 0 & 12 & 9  & 6  & 3  \\
              12           & 0 & 6  & 12 & 3  & 9  \\
            \end{tabular}
          \end{center}
          Again because these tables contain only elements of $S$ $S$ is closed under both of their respective operations.
    \item ~\begin{center}
            \setlength\extrarowheight{3pt}
            \noindent\begin{tabular}{c | c c c c}
              $+_{5}$ & 1 & 2 & 3 & 4 \\
              \cline{1-5}
              1       & 2 & 3 & 4 & 0 \\
              2       & 3 & 4 & 0 & 1 \\
              3       & 4 & 0 & 1 & 2 \\
              4       & 0 & 1 & 2 & 3 \\
            \end{tabular}
            \qquad
            \noindent\begin{tabular}{c | c c c c}
              $\cdot_{5}$ & 1 & 2 & 3 & 4 \\
              \cline{1-5}
              1           & 1 & 2 & 3 & 4 \\
              2           & 2 & 4 & 1 & 3 \\
              3           & 3 & 1 & 4 & 2 \\
              4           & 4 & 3 & 2 & 1 \\
            \end{tabular}
          \end{center}
          Here because $0\notin S\implies S$ is not closed under $+_5$ but is closed under $\cdot_5$ for the same reasons as previously.
          \newpage
    \item ~\begin{center}
            \setlength\extrarowheight{3pt}
            \noindent\begin{tabular}{c | c c c c c c c c}
              $+_{12}$ & 0  & 2  & 3  & 4  & 6  & 8  & 9  & 10 \\
              \cline{1-9}
              0        & 0  & 2  & 3  & 4  & 6  & 8  & 9  & 10 \\
              2        & 2  & 4  & 5  & 6  & 8  & 10 & 11 & 0  \\
              3        & 3  & 5  & 6  & 7  & 9  & 11 & 0  & 1  \\
              4        & 4  & 6  & 7  & 8  & 10 & 0  & 1  & 2  \\
              6        & 6  & 8  & 9  & 10 & 0  & 2  & 3  & 4  \\
              8        & 8  & 10 & 11 & 0  & 2  & 4  & 5  & 6  \\
              9        & 9  & 11 & 0  & 1  & 3  & 5  & 6  & 7  \\
              10       & 10 & 0  & 1  & 2  & 4  & 6  & 7  & 8  \\
            \end{tabular}
            \qquad
            \noindent\begin{tabular}{c | c c c c c c c c}
              $\cdot_{12}$ & 0 & 2 & 3 & 4 & 6 & 8 & 9 & 10 \\
              \cline{1-9}
              0            & 0 & 0 & 0 & 0 & 0 & 0 & 0 & 0  \\
              2            & 0 & 4 & 6 & 8 & 0 & 4 & 6 & 8  \\
              3            & 0 & 6 & 9 & 0 & 6 & 0 & 3 & 6  \\
              4            & 0 & 8 & 0 & 4 & 0 & 8 & 0 & 4  \\
              6            & 0 & 0 & 6 & 0 & 0 & 0 & 6 & 0  \\
              8            & 0 & 4 & 0 & 8 & 0 & 4 & 0 & 8  \\
              9            & 0 & 6 & 3 & 0 & 6 & 0 & 9 & 6  \\
              10           & 0 & 8 & 6 & 4 & 0 & 8 & 6 & 4  \\
            \end{tabular}
          \end{center}
          Here because $1,5,7,11\notin S$ $S$ is not closed under $+_{12}$ but is closed under $\cdot_{12}$ for the same reasons as previously.
    \item ~\begin{center}
            \setlength\extrarowheight{3pt}
            \noindent\begin{tabular}{c | c c c c}
              $+_{12}$ & 1 & 5  & 7 & 11 \\
              \cline{1-5}
              1        & 2 & 6  & 8 & 0  \\
              5        & 6 & 10 & 0 & 4  \\
              7        & 8 & 0  & 2 & 6  \\
              11       & 0 & 4  & 6 & 10 \\
            \end{tabular}
            \qquad
            \noindent\begin{tabular}{c | c c c c}
              $\cdot_{12}$ & 1  & 5  & 7  & 11 \\
              \cline{1-5}
              1            & 1  & 5  & 7  & 11 \\
              5            & 5  & 1  & 11 & 7  \\
              7            & 7  & 11 & 1  & 5  \\
              11           & 11 & 7  & 5  & 1  \\
            \end{tabular}
          \end{center}
          Here because $0,2,4,6,8,10\notin S$ $S$ is not closed under $+_{12}$ but is closed under $\cdot_{12}$ for the same reasons as previously.
  \end{enumerate}
\end{soln}
\begin{figure}[ht]
  \centering
  \includegraphics[scale=0.5]{space filling cat.jpg}
  \caption{Space Filling Cat Picture}
\end{figure}
\newpage

% PROBLEM 3
\begin{problem}
Determine whether the given binary operation $*$ is commutative,
associative, both or neither. Justify your answers with proof.
\begin{enumerate}[label=(\alph*)]
  \item The operation $*$ on $\mathbb{Z}$ given by $a*b=a+b+ab$
  \item The operation $*$ on $\mathbb{R}$ given by $a*b=a+b-ab$
  \item The operation $*$ on $\mathbb{R}$ given by $a*b=a+2ab$
  \item The operation $*$ on $\mathbb{Z}\cross\mathbb{Z}$ given by $(a,b)*(c,d)=(ad+bc, bd)$
  \item The operation $*$ on $\mathbb{Z}\cross\mathbb{Z}$ given by $(a,b)*(c,d)=(ad,bc)$
\end{enumerate}
\end{problem}
\begin{soln}~
  \begin{enumerate}[label=(\alph*)]
    \item For commutativity,
          \begin{proof}
            Let $a,b\in\mathbb{Z}$, then
            \begin{align*}
              a*b & = a+b+ab                                                            \\
                  & =b+a+ba\justif{\quad}{Commutativity of $+$ and $-$ on $\mathbb{Z}$} \\
                  & =b*a\justif{\quad}{Definition of $*$}
            \end{align*}
            Therefore $*$ is commutative. \qedhere
          \end{proof}
          For associativity,
          \begin{proof}
            Let $a,b,c\in\mathbb{Z}$, then
            \begin{align*}
              a*(b*c) & = a*(b+c+bc)                                                                 \\
                      & = a + b+c+bc + a(b+c+bc)                                                     \\
                      & = a + b+c+bc + ab+ac+abc\justif{\quad}{$\cdot$ distributive on $\mathbb{Z}$} \\
            \end{align*}
            and
            \begin{align*}
              (a*b)*c & = (a+b+ab)*c                                                                  \\
                      & = a+b+ab + c + (a+b+ab)c                                                      \\
                      & = a+b+ab + c + ac+bc+abc \justif{\quad}{$\cdot$ distributive on $\mathbb{Z}$} \\
                      & = a+b+c+ bc + ab+ac+abc \justif{\quad}{$+$ commutative on $\mathbb{Z}$}       \\
            \end{align*}
            Because the two are equal we have associativity. \qedhere
          \end{proof}
          \newpage
    \item For commutativity,
          \begin{proof}
            Let $a,b\in\mathbb{Z}$, then
            \begin{align*}
              a*b & = a+b-ab                                                             \\
                  & =b+a-ba \justif{\quad}{Commutativity of $+$ and $-$ on $\mathbb{Z}$} \\
                  & =b*a\justif{\quad}{Definition of $*$}
            \end{align*}
            Therefore $*$ is commutative. \qedhere
          \end{proof}
          For associativity,
          \begin{proof}
            Let $a,b,c\in\mathbb{Z}$, then
            \begin{align*}
              a*(b*c) & = a*(b+c-bc)                                                                 \\
                      & = a + b+c-bc + a(b+c-bc)                                                     \\
                      & = a + b+c-bc + ab+ac-abc\justif{\quad}{$\cdot$ distributive on $\mathbb{Z}$} \\
            \end{align*}
            and
            \begin{align*}
              (a*b)*c & = (a+b-ab)*c                                                                  \\
                      & = a+b-ab + c + (a+b-ab)c                                                      \\
                      & = a+b-ab + c + ac+bc-abc \justif{\quad}{$\cdot$ distributive on $\mathbb{Z}$} \\
            \end{align*}
            Because of the difference in sign on the $ab$ terms these two cannot be made to be equal, therefore $*$ is not associative. \qedhere
          \end{proof}
    \item For commutativity,
          \begin{proof}
            Let $a,b\in\mathbb{Z}$, then
            $$
              a*b = a+2ab\neq
              b*a = b+2ba
            $$
            Therefore $*$ is not commutative here. \qedhere
          \end{proof}
          For associativity,
          \begin{proof}
            Let $a,b,c\in\mathbb{Z}$, then
            \begin{align*}
              a*(b*c) & = a*(b+2bc)                                                       \\
                      & = a+ 2a(b+2bc)                                                    \\
                      & = a+ 2ab+4abc\justif{\quad}{$\cdot$ distributive on $\mathbb{Z}$}
            \end{align*}
            and
            \begin{align*}
              (a*b)*c & = (a+2ab)*c                                                        \\
                      & = c+ 2c(a+2ab)                                                     \\
                      & = c+ 2ca+4cab \justif{\quad}{$\cdot$ distributive on $\mathbb{Z}$} \\
            \end{align*}
            Which cannot be manipulated to be equal, therefore $*$ is not associative here. \qedhere
          \end{proof}
          \newpage
    \item For commutativity,
          \begin{proof}
            Let $(a,b),(c,d)\in\mathbb{Z}\cross\mathbb{Z}$, then
            \begin{align*}
              (a,b)*(c,d) & =(ad+bc, bd)                                                                 \\
                          & =(cb+da, db)\justif{\quad}{Associativity of $+$ and $\cdot$ on $\mathbb{Z}$} \\
                          & =(c,d)*(a,b)\justif{\quad}{Definition of $*$ in reverse} 
            \end{align*}
            Therefore $*$ is commutative. \qedhere
          \end{proof}
          For associativity,
          \begin{proof}
            Let $(a,b),(c,d),(e,f)\in\mathbb{Z}\cross\mathbb{Z}$, then
            \begin{align*}
              (a,b)*((c,d)*(e,f)) & = (a,b)*(cf+de,df)                                                    \\
                                  & = (adf+b(cf+de),bdf)                                                  \\
                                  & = (adf+bcf+bde,bdf)\justif{\quad}{\cdot distributive on $\mathbb{Z}$}
            \end{align*}
            and
            \begin{align*}
              ((a,b)*(c,d))*(e,f) & = (ad+bc, bd)*(e,f)                                                            \\
                                  & =  ((ad+bc)f+bde, bdf)                                                         \\
                                  & =  (fad+fbc+bde, bdf)\justif{\quad}{\cdot distributive on $\mathbb{Z}$}        \\
                                  & =  (adf+bcf+bde, bdf)\justif{\quad}{\cdot commutative on $\mathbb{Z}$}
            \end{align*}
            As these two are equal, $*$ is associative. \qedhere
          \end{proof}
    \item For commutativity,
          \begin{proof}
            Let $(a,b),(c,d)\in\mathbb{Z}\cross\mathbb{Z}$, then
            $$
              (a,b)*(c,d)=(ad,bc)\neq (c,d)*(a,b) =(cb,da).
            $$
          \end{proof}
          Therefore $*$ is not commutative. \qedhere
          For associativity,
          \begin{proof}
            Let $(a,b),(c,d),(e,f)\in\mathbb{Z}\cross\mathbb{Z}$, then

            \begin{align*}
              (a,b)*((c,d)*(e,f)) & = (a,b)*(cf,de) \\
                                  & = (ade,bcf)
            \end{align*}
            and
            \begin{align*}
              ((a,b)*(c,d))*(e,f) & = (ad,bc)*(e,f) \\
                                  & = (adf,bce)
            \end{align*}
            As these two cannot be made equal $*$ is not associative. \qedhere
          \end{proof}
  \end{enumerate}
\end{soln}
\newpage

% PROBLEM 4
\begin{problem}
Let $S$ be a nonempty set. A binary algebraic structure $(S, *)$ is called
a semigroup if $*$ is associative.
\begin{enumerate}[label=(\alph*)]
  \item Let $S$ be the set of positive rational numbers. Show that $(S, *)$
        is a commutative semigroup if
        $$a*b=\frac{ab}{a+b}$$
        (the usual operations on the right) for all $a,b\in S$
  \item Let $S$ be a set containing more than one element. Define
        $$a*b=b$$
        for all $a,b\in S$. Show that $(S,*)$ is a noncommutative semigroup with no identity element.
\end{enumerate}
\end{problem}
\begin{soln}~
  \begin{enumerate}[label=(\alph*)]
    \item For associativity (semigroupness)
          \begin{proof}
            Let $a,b,c\in S$, then
            \begin{align*}
              a*(b*c) & = a*\left(\frac{bc}{b+c}\right)                                                                \\
                      & = \frac{a(bc)}{a+(b+c)}                                                                        \\
                      & = \frac{abc}{a+(b+c)} \justif{\quad}{Associativity of $+$ and $\cdot$ on $\mathbb{Q}^{\geq0}$}
            \end{align*}
            and
            \begin{align*}
              (a*b)*c & = \left(\frac{ab}{a+b}\right)*c                                                               \\
                      & = \frac{(ab)c}{(a+b)+c}                                                                       \\
                      & = \frac{abc}{a+b+c}  \justif{\quad}{Associativity of $+$ and $\cdot$ on $\mathbb{Q}^{\geq0}$}
            \end{align*}
            $\therefore$ $(S, *)$ is a semigroup\qedhere
          \end{proof}
          For commutativity
          \begin{proof}
            Let $a,b\in S$, then
            \begin{align*}
              a*b & = \frac{ab}{a+b}                                                                         \\
                  & = \frac{ba}{b+a}\justif{\quad}{Commutativity of $+$ and $\cdot$ on $\mathbb{Q}^{\geq0}$} \\                                                                     \\
                  & = b*a \justif{\quad}{Definition of $*$ in reverse}
            \end{align*}
            $\therefore$ $(S, *)$ is commutative\qedhere
          \end{proof}
          \newpage
    \item For associativity (semigroupness)
          \begin{proof}
            Let $a,b,c\in S$, then
            \begin{align*}
              a*(b*c) & = a*c \\
                      & = c
            \end{align*}
            and
            \begin{align*}
              (a*b)*c & = b*c \\
                      & = c
            \end{align*}
            $\therefore$ $(S, *)$ is a semigroup\qedhere
          \end{proof}
          For commutativity
          \begin{proof}
            Let $a,b\in S$, then
            $$a*b=b\neq b*a=a$$
            $\therefore$ $(S, *)$ is not commutative\qedhere
          \end{proof}
          For the identity element if we assume, by way of contradiction, that such an
          element, $e$, exists then it must satisfy
          $$a*e=e*a=a\forall a\in S.$$
          We can note however that by our definition of $*$ we have
          $$a*e=e\neq e*a=a$$
          which is only satisfied by $e=a$ which would not satisfy any other such equation
          where we swap out $a$ for some other element of $S$ unless we also swap out $e$
          which would violate the uniqueness requirement imposed by the definition of $e$ as
          an identity element meaning that it must be the same for all expressions and therefore
          cannot exist in $S$.
  \end{enumerate}
\end{soln}
\end{document}