\documentclass[10pt]{article}

\usepackage[margin=0.75in]{geometry}
\usepackage{amsmath,amsthm,amssymb}
\usepackage{xcolor}
\usepackage{cancel}
\usepackage{graphicx}
\usepackage{changepage}
\usepackage{circuitikz}
\usepackage{pgfplots}
\usepackage{physics}
\usepackage{hyperref}
\usepackage{siunitx}
\usepackage{fontspec}
\usepackage{relsize}
\usepackage{subfig}
\usepackage{todonotes}
\usepackage{emoji}
\usepackage{minted}
\usepackage{multicol, multirow, booktabs}
\usepackage[breakable]{tcolorbox}
\usepackage[inline]{enumitem}

\theoremstyle{definition}
\newtheorem{problem}{Problem}
\newtheorem{soln}{Solution}

\pgfplotsset{compat=newest}
\usetikzlibrary{lindenmayersystems}
\usetikzlibrary{arrows}
\usetikzlibrary{calc}
\usetikzlibrary{positioning, fit}
\usetikzlibrary{3d, perspective}

\definecolor{incolor}{HTML}{303F9F}
\definecolor{outcolor}{HTML}{D84315}
\definecolor{cellborder}{HTML}{CFCFCF}
\definecolor{cellbackground}{HTML}{F7F7F7}
\newcommand{\ui}{\hat{i}}
\newcommand{\uj}{\hat{j}}
\newcommand{\uk}{\hat{k}}
\newcommand{\ux}{\hat{x}}
\newcommand{\uy}{\hat{y}}
\newcommand{\uz}{\hat{z}}
\newcommand{\primed}[1]{#1^\prime}
\pgfdeclarelayer{background}  
\pgfsetlayers{background,main}
\AtBeginDocument{\RenewCommandCopy\qty\SI}
\newcommand{\justif}[2]{&{#1}&\text{#2}}

\makeatletter
\newcommand{\boxspacing}{\kern\kvtcb@left@rule\kern\kvtcb@boxsep}
\makeatother
\newcommand{\prompt}[4]{
    \ttfamily\llap{{\color{#2}[#3]:\hspace{3pt}#4}}\vspace{-\baselineskip}
}

\newcommand{\thevenin}[2]{
  \begin{center}
    \begin{circuitikz} \draw
      (0,0) -- (2,0) to[battery1, l_=$V_{Th}\eq#1$] (2,2) 
      to[resistor, l_=$R_{Th}\eq#2$] (0,2)
      ;
      \draw [o-] (-.07,2.079);
      \draw [o-] (-.07,0.079);
    \end{circuitikz}
  \end{center}
}

\newcommand{\norton}[2]{
  \begin{center}
    \begin{circuitikz} \draw
      (0,0) -- (3,0) to[american current source, l_=$I_{N}\eq#1$] (3,2) -- (0,2) (2,0)
      to[resistor, l=$R_{N}\eq#2$] (2,2)
      ;
      \draw [o-] (-.07,2.079);
      \draw [o-] (-.07,0.079);
    \end{circuitikz}
  \end{center}
}

\newcommand{\highlight}[1]{\colorbox{yellow}{$\displaystyle #1$}}

\newcommand{\ti}[1]{\widetilde{#1}}

\newfontface{\Kaufmann}{Kaufmann}
\DeclareTextFontCommand{\kf}{\Kaufmann}
\newcommand{\scriptr}{\fontsize{12pt}{12pt}\kf{r}}

\newfontface{\KaufmannB}{Kaufmann Bd BT}
\DeclareTextFontCommand{\kfb}{\KaufmannB}
\newcommand{\bscriptr}{\fontsize{12pt}{12pt}\kfb{r}}

\newcommand{\bv}[1]{\mathbf{#1}}

\title{Math 3310H: Assignment III}
\author{Jeremy Favro (0805980) \\ Trent University, Peterborough, ON, Canada}
\date{\today}

\begin{document}
\maketitle

% PROBLEM 1
\begin{problem}
Show that a group $G$ cannot be the union of two proper subgroups,
in other words, if $G = H \cup K$ where $H$ and $K$ are subgroups of $G$,
then $H = G$ or $K = G$.
\end{problem}
\begin{soln}
  Suppose, by way of contradiction, that $G=H\cup K$ and $H\neq G \neq K$.
  Then there are elements $a\in H$ and $b\in K$ but $a\notin K$ and $b\notin H$.
  Because $G=H\cup K$ and $H$, $K$, and $G$ are closed by definition, $ab\in H$ or $ab\in K$.
  First then suppose that $ab\in H\implies a^{-1}ab\in H\implies eb\in H\implies b\in H$, but we began
  with the assumption that $b \notin H$, so unless $H=K=G$, $K$ cannot be a subgroup. The same argument works
  in the other direction: Suppose $ab \in K\implies abb^{-1}\in K\implies ae \in K\implies a\in K$, but
  $a$ was created to be something only in $H$, not $K$, meaning $H$ is not closed unless $H=K=G$.
\end{soln}

% PROBLEM 2
\begin{problem}
Let $G$ be a group with identity $e$ and $e\in G$. Show that if $a^n = e$
then the order of $a$ divides $n$.
\end{problem}
\begin{soln}
  Let $\abs{a}=k$ be the order of $a$. By the division algorithm we can write
  $n=qk+r$ for some $q,r\in \mathbb{Z}$ with $0\leq r < k$. So
  \begin{align*}
    e & =a^n                                        \\
      & =a^{qk+r}                                   \\
      & =a^{qk}a^r                                  \\
      & =(a^k)^qa^r                                 \\
      & =e^q a^r\justif{\,}{$a^k=e$ by definition.} \\
      & =a^r.
  \end{align*}
  For the expression $e=a^r$ to hold true $r$ must be some multiple of the order of $a$, $k$. This means
  that our expression using the division algorithm becomes $n=qk+sk$ for $sk=r$ which means that $n/k=q+s$ which
  is an integer meaning that the order of $a$, $k$, divides $n$.
\end{soln}

% PROBLEM 3
\begin{problem}
Let $G$ be a cyclic group of order $n$ with identity $e$. Suppose 15
divides $n$. How many solutions to $x^{15}=e$ are there in $G$?
\end{problem}
\begin{soln}
Let $\langle g\rangle = G$. Then, as $G$ is cyclic there is one subgroup of $G$
of order $15$ generated by $g^{n/15}$ where every element will satisfy 
$x^{15}=e$ as $x=g^{kn/15}\implies x^{15}=g^{kn}=e$. This means (I think, this was
a really tough one) that there are 15 solutions in $G$.
\end{soln}

% PROBLEM 4
\begin{problem}
Show that $H = \left\{\sigma \in S_n|\sigma(1) = 1\right\}$ is a subgroup of $S_n$.
\end{problem}
\begin{soln}
  For $H$ to be a subgroup of $S_n$ it must satisfy the following:
  \begin{enumerate}[label=(\roman*)]
    \item Closure: This is fairly obvious, constructing any $\sigma^{\prime\prime}=\sigma\circ \primed{\sigma}$
          will always satisfy $\sigma^{\prime\prime}(1)=1$ as both $\sigma$ and $\primed{\sigma}$ must map $1\to 1$ to
          belong to $H$ in the first place.
    \item Contains the identity: The identity map looks like
          $$\iota = \begin{pmatrix}
              1 & 2 & 3 & \dots & n \\
              1 & 2 & 3 & \dots & n
            \end{pmatrix}$$
          which satisfies $\sigma(1)=1$
    \item Contains inverses: All inverses for a $\sigma\in H$ will map $1\to 1$ by the definition of $\sigma$ and
          so will belong to $H$.
  \end{enumerate}
\end{soln}

% PROBLEM 5
\begin{problem}
Let
$$
  \sigma=\begin{pmatrix}
    1 & 2 & 3 & 4 & 5 & 6 & 7 & 8 & 9 \\
    3 & 2 & 4 & 1 & 7 & 5 & 8 & 9 & 6
  \end{pmatrix}
$$
and
$$
  \tau=\begin{pmatrix}
    1 & 2 & 3 & 4 & 5 & 6 & 7 & 8 & 9 \\
    6 & 3 & 7 & 9 & 1 & 8 & 2 & 4 & 5
  \end{pmatrix}.
$$
\begin{enumerate}[label=(\alph*)]
  \item Compute $\sigma^2$, $\sigma\tau$, $\tau\sigma$, $\sigma^{-1}$, $\sigma\tau\sigma^{-1}$, and $\tau\sigma\tau^{-1}$.
  \item Find the order of $\tau$
\end{enumerate}
\end{problem}
\begin{soln}~
  \begin{enumerate}[label=(\alph*)]
    \item
          $$\sigma^2=\begin{pmatrix}
              1 & 2 & 3 & 4 & 5 & 6 & 7 & 8 & 9 \\
              3 & 2 & 4 & 1 & 7 & 5 & 8 & 9 & 6
            \end{pmatrix}\begin{pmatrix}
              1 & 2 & 3 & 4 & 5 & 6 & 7 & 8 & 9 \\
              3 & 2 & 4 & 1 & 7 & 5 & 8 & 9 & 6
            \end{pmatrix}
            =\begin{pmatrix}
              1 & 2 & 3 & 4 & 5 & 6 & 7 & 8 & 9 \\
              4 & 2 & 1 & 3 & 8 & 7 & 9 & 6 & 5
            \end{pmatrix}$$

          $$\tau\sigma =
            \begin{pmatrix}
              1 & 2 & 3 & 4 & 5 & 6 & 7 & 8 & 9 \\
              6 & 3 & 7 & 9 & 1 & 8 & 2 & 4 & 5
            \end{pmatrix}
            \begin{pmatrix}
              1 & 2 & 3 & 4 & 5 & 6 & 7 & 8 & 9 \\
              3 & 2 & 4 & 1 & 7 & 5 & 8 & 9 & 6
            \end{pmatrix}
            =\begin{pmatrix}
              1 & 2 & 3 & 4 & 5 & 6 & 7 & 8 & 9 \\
              7 & 3 & 9 & 6 & 2 & 1 & 4 & 5 & 8
            \end{pmatrix}
          $$
          $$
            \sigma^{-1}=
            \begin{pmatrix}
              3 & 2 & 4 & 1 & 7 & 5 & 8 & 9 & 6 \\
              1 & 2 & 3 & 4 & 5 & 6 & 7 & 8 & 9
            \end{pmatrix}
            =
            \begin{pmatrix}
              1 & 2 & 3 & 4 & 5 & 6 & 7 & 8 & 9 \\
              4 & 2 & 1 & 3 & 6 & 9 & 5 & 7 & 8
            \end{pmatrix}
          $$
          \begin{align*}
            \sigma\tau\sigma^{-1}
             & = \begin{pmatrix}
                   1 & 2 & 3 & 4 & 5 & 6 & 7 & 8 & 9 \\
                   3 & 2 & 4 & 1 & 7 & 5 & 8 & 9 & 6
                 \end{pmatrix}\begin{pmatrix}
                                1 & 2 & 3 & 4 & 5 & 6 & 7 & 8 & 9 \\
                                6 & 3 & 7 & 9 & 1 & 8 & 2 & 4 & 5
                              \end{pmatrix}\begin{pmatrix}
                                             1 & 2 & 3 & 4 & 5 & 6 & 7 & 8 & 9 \\
                                             4 & 2 & 1 & 3 & 6 & 9 & 5 & 7 & 8
                                           \end{pmatrix} \\
             & =\begin{pmatrix}
                  1 & 2 & 3 & 4 & 5 & 6 & 7 & 8 & 9 \\
                  6 & 4 & 5 & 8 & 9 & 7 & 3 & 2 & 1 \\
                \end{pmatrix}
          \end{align*}
          \begin{align*}
            \tau\sigma\tau^{-1} & =
            \begin{pmatrix}
              1 & 2 & 3 & 4 & 5 & 6 & 7 & 8 & 9 \\
              6 & 3 & 7 & 9 & 1 & 8 & 2 & 4 & 5
            \end{pmatrix}
            \begin{pmatrix}
              1 & 2 & 3 & 4 & 5 & 6 & 7 & 8 & 9 \\
              3 & 2 & 4 & 1 & 7 & 5 & 8 & 9 & 6
            \end{pmatrix}
            \begin{pmatrix}
              1 & 2 & 3 & 4 & 5 & 6 & 7 & 8 & 9 \\
              5 & 7 & 2 & 8 & 9 & 1 & 3 & 6 & 4
            \end{pmatrix} \\
                                & =
            \begin{pmatrix}
              1 & 2 & 3 & 4 & 5 & 6 & 7 & 8 & 9 \\
              2 & 4 & 3 & 5 & 8 & 7 & 9 & 1 & 6 \\
            \end{pmatrix}
          \end{align*}
    \item I wrote some Python code to do this to check if I could get $\iota$ in a reasonable number of steps
          \begin{minted}{python3}
  tau = {
    1: 6,
    2: 3,
    3: 7,
    4: 9,
    5: 1,
    6: 8,
    7: 2,
    8: 4,
    9: 5
}
iota = [1, 2, 3, 4, 5, 6, 7, 8, 9]

t = [6, 3, 7, 9, 1, 8, 2, 4, 5]

cnt = 1
while t != iota:
    cnt = cnt + 1
    for i in range(0, 9):
        t[i] = tau[t[i]]
    print(t)

print(f"Got iota on {cnt}")
\end{minted}
          Which gives the order $\tau$ as 6.
  \end{enumerate}
\end{soln}

% PROBLEM 6
\begin{problem}
Below are four recommended car tire rotation patterns.
\begin{center}
  \begin{tikzpicture}
    \node (BL) at (0,0) {$\bigcirc$};
    \node (BR) at (1.5,0) {$\bigcirc$};
    \node (TL) at (0,2) {$\bigcirc$};
    \node (TR) at (1.5,2) {$\bigcirc$};

    \draw[->] (BL) -- (TL);
    \draw[->] (BR) -- (TR);
    \draw[->] (TL) -- (BR);
    \draw[->] (TR) -- (BL);

    \node (BL) at (0+3,0) {$\bigcirc$};
    \node (BR) at (1.5+3,0) {$\bigcirc$};
    \node (TL) at (0+3,2) {$\bigcirc$};
    \node (TR) at (1.5+3,2) {$\bigcirc$};

    \draw[->] (TL) -- (BL);
    \draw[->] (TR) -- (BR);
    \draw[->] (BL) -- (TR);
    \draw[->] (BR) -- (TL);

    \node (BL) at (0+3+3,0) {$\bigcirc$};
    \node (BR) at (1.5+3+3,0) {$\bigcirc$};
    \node (TL) at (0+3+3,2) {$\bigcirc$};
    \node (TR) at (1.5+3+3,2) {$\bigcirc$};

    \draw[<->] (BL) -- (TR);
    \draw[<->] (BR) -- (TL);

    \node (BL) at (0+3+3+3,0) {$\bigcirc$};
    \node (BR) at (1.5+3+3+3,0) {$\bigcirc$};
    \node (TL) at (0+3+3+3,2) {$\bigcirc$};
    \node (TR) at (1.5+3+3+3,2) {$\bigcirc$};

    \draw[<->] (BL) -- (TL);
    \draw[<->] (BR) -- (TR);
  \end{tikzpicture}
\end{center}
\begin{enumerate}[label=(\alph*)]
  \item Explain how these patterns can be represented as elements of
        $S_4$.
  \item Find the smallest subgroup $H$ of $S_4$ that contains these four
        patterns.
  \item Is $H$ abelian?
\end{enumerate}
\end{problem}
\begin{soln} ~\\
  \begin{enumerate}[label=(\alph*)]
    \item If we represent the ``default'' state of the tires as
          \begin{center}
            \begin{tikzpicture}
              \node (BL) at (0,0) {1};
              \node (BR) at (2,0) {2};
              \node (TL) at (0,2) {4};
              \node (TR) at (2,2) {3};

              \draw[] (BL) -- (TL);
              \draw[] (TL) -- (TR);
              \draw[] (TR) -- (BR);
              \draw[] (BR) -- (BL);
            \end{tikzpicture}
          \end{center}
          then each rotation of the tires is a permutation of this default state. By definition
          $S_4$ is the group containing all permutations of 4 elements and so these will belong to $S_4$. Going from
          left to right in the previous figure:
          $$\sigma_1=\begin{pmatrix}
              1 & 2 & 3 & 4 \\
              3 & 4 & 2 & 1
            \end{pmatrix};\quad
            \sigma_2=\begin{pmatrix}
              1 & 2 & 3 & 4 \\
              4 & 3 & 1 & 2
            \end{pmatrix};\quad
            \sigma_3=\begin{pmatrix}
              1 & 2 & 3 & 4 \\
              3 & 4 & 1 & 2
            \end{pmatrix};\quad
            \sigma_4=\begin{pmatrix}
              1 & 2 & 3 & 4 \\
              4 & 3 & 2 & 1
            \end{pmatrix}.
          $$
    \item Let's assume that the smallest subgroup is just the $\sigma$s, their inverses, and the identity $\iota$.
          The inverses are
          $$\sigma_1^{-1}=
            \begin{pmatrix}
              3 & 4 & 2 & 1 \\
              1 & 2 & 3 & 4
            \end{pmatrix}=
            \begin{pmatrix}
              1 & 2 & 3 & 4 \\
              4 & 3 & 1 & 2
            \end{pmatrix}=\sigma_2
          $$

          $$\sigma_2^{-1}=
            \begin{pmatrix}
              4 & 3 & 1 & 2 \\
              1 & 2 & 3 & 4
            \end{pmatrix}
            =
            \begin{pmatrix}
              1 & 2 & 3 & 4 \\
              3 & 4 & 2 & 1
            \end{pmatrix}=\sigma_1
          $$

          $$\sigma_3^{-1}=
            \begin{pmatrix}
              3 & 4 & 1 & 2 \\
              1 & 2 & 3 & 4
            \end{pmatrix}
            =
            \begin{pmatrix}
              1 & 2 & 3 & 4 \\
              3 & 4 & 1 & 2
            \end{pmatrix}=\sigma_3
          $$
          $$\sigma_4^{-1}=
            \begin{pmatrix}
              4 & 3 & 2 & 1 \\
              1 & 2 & 3 & 4
            \end{pmatrix}
            =
            \begin{pmatrix}
              1 & 2 & 3 & 4 \\
              4 & 3 & 2 & 1
            \end{pmatrix}=\sigma_4
          $$
          So
          $$H=\left\{\iota,\sigma_1,\sigma_2,\sigma_3,\sigma_4\right\}$$
          But we need to check that this is closed. This involves checking
          all the combinations that we don't already know give an inverse. I'll set this up as a Cayley table,
          \begin{center}
            \begin{tabular}{c | c c c c}
                         & $\sigma_1$ & $\sigma_2$ & $\sigma_3$ & $\sigma_4$ \\
              \cline{1-5}
              $\sigma_1$ &            & $\iota$    &            &            \\
              $\sigma_2$ & $\iota$    &            &            &            \\
              $\sigma_3$ &            &            & $\iota$    &            \\
              $\sigma_4$ &            &            &            & $\iota$    \\
            \end{tabular}
          \end{center}
          $$
            \sigma_1\sigma_1=
            \begin{pmatrix}
              1 & 2 & 3 & 4 \\
              3 & 4 & 2 & 1
            \end{pmatrix}\begin{pmatrix}
              1 & 2 & 3 & 4 \\
              3 & 4 & 2 & 1
            \end{pmatrix}
            =\begin{pmatrix}
              1 & 2 & 3 & 4 \\
              2 & 1 & 4 & 3
            \end{pmatrix}
          $$
          Which is not in $H$ so we'll call it $\sigma_5$ and continue
          \begin{center}
            \begin{tabular}{c | c c c c}
                         & $\sigma_1$ & $\sigma_2$ & $\sigma_3$ & $\sigma_4$ \\
              \cline{1-5}
              $\sigma_1$ & $\sigma_5$ & $\iota$    &            &            \\
              $\sigma_2$ & $\iota$    &            &            &            \\
              $\sigma_3$ &            &            & $\iota$    &            \\
              $\sigma_4$ &            &            &            & $\iota$    \\
            \end{tabular}
          \end{center}
          I again wrote some Python code to quickly fill these in,
          \begin{minted}{python3}
 class Permutation:
    def __init__(self, map: dict):
        self.map = map

    def __mul__(self, other):
        result = [1,2,3,4]
        for i in range(len(self.map)):
            result[i] = self.map[other.map[i+1]]
        return result
    
    def inverse(self):
        inv = {v: k for k, v in self.map.items()}
        return list(dict(sorted(inv.items())).values())


sig_1 = Permutation({
    1: 3,
    2: 4,
    3: 2,
    4: 1,
})

sig_2 = Permutation({
    1: 4,
    2: 3,
    3: 1,
    4: 2,
})

sig_3 = Permutation({
    1: 3,
    2: 4,
    3: 1,
    4: 2,
})

sig_4 = Permutation({
    1: 4,
    2: 3,
    3: 2,
    4: 1,
})
\end{minted}
          Which gives
          \begin{center}
            \begin{tabular}{c | c c c c}
                         & $\sigma_1$ & $\sigma_2$ & $\sigma_3$ & $\sigma_4$ \\
              \cline{1-5}
              $\sigma_1$ & $\sigma_5$ & $\iota$    & $\sigma_6$ & $\sigma_7$ \\
              $\sigma_2$ & $\iota$    & $\sigma_6$ & $\sigma_7$ & $\sigma_6$ \\
              $\sigma_3$ & $\sigma_7$ & $\sigma_6$ & $\iota$    & $\sigma_5$ \\
              $\sigma_4$ & $\sigma_6$ & $\sigma_7$ & $\sigma_5$ & $\iota$    \\
            \end{tabular}
          \end{center}
          where
          $$
            \sigma_6=\begin{pmatrix}
              1 & 2 & 3 & 4 \\
              1 & 2 & 4 & 3
            \end{pmatrix};\quad
            \sigma_7=\begin{pmatrix}
              1 & 2 & 3 & 4 \\
              2 & 1 & 3 & 4
            \end{pmatrix}.
          $$
          Now,
          $$
            \sigma_5^{-1}=\begin{pmatrix}
              1 & 2 & 3 & 4 \\
              2 & 1 & 4 & 3
            \end{pmatrix}=\sigma_5;\quad
            \sigma_6^{-1}=\begin{pmatrix}
              1 & 2 & 3 & 4 \\
              1 & 2 & 4 & 3
            \end{pmatrix}=\sigma_6;\quad
            \sigma_7^{-1}=\begin{pmatrix}
              1 & 2 & 3 & 4 \\
              2 & 1 & 3 & 4
            \end{pmatrix}=\sigma_7.
          $$
          So our new smallest subgroup becomes
          $$
            H=\left\{\iota, \sigma_1,\sigma_2,\sigma_3,
            \sigma_4,\sigma_5,\sigma_6,\sigma_7
            \right\}.
          $$
          The Cayley table for this new group is
          \begin{center}
            \begin{tabular}{c | c c c c c c c }
                         & $\sigma_1$ & $\sigma_2$ & $\sigma_3$ & $\sigma_4$ & $\sigma_5$ & $\sigma_6$ & $\sigma_7$ \\
              \cline{1-8}
              $\sigma_1$ & $\sigma_5$ & $\iota$    & $\sigma_6$ & $\sigma_7$ & $\sigma_2$ & $\sigma_4$ & $\sigma_3$ \\
              $\sigma_2$ & $\iota$    & $\sigma_6$ & $\sigma_7$ & $\sigma_6$ & $\sigma_1$ & $\sigma_3$ & $\sigma_4$ \\
              $\sigma_3$ & $\sigma_7$ & $\sigma_6$ & $\iota$    & $\sigma_5$ & $\sigma_4$ & $\sigma_2$ & $\sigma_1$ \\
              $\sigma_4$ & $\sigma_6$ & $\sigma_7$ & $\sigma_5$ & $\iota$    & $\sigma_3$ & $\sigma_1$ & $\sigma_2$ \\
              $\sigma_5$ & $\sigma_2$ & $\sigma_1$ & $\sigma_4$ & $\sigma_3$ & $\iota$    & $\sigma_7$ & $\sigma_6$ \\
              $\sigma_6$ & $\sigma_3$ & $\sigma_4$ & $\sigma_1$ & $\sigma_2$ & $\sigma_7$ & $\iota$    & $\sigma_5$ \\
              $\sigma_7$ & $\sigma_4$ & $\sigma_3$ & $\sigma_2$ & $\sigma_1$ & $\sigma_6$ & $\sigma_5$ & $\iota$
            \end{tabular}
          \end{center}
          So the group is closed and is therefore actually a subgroup. Yay!
    \item The Cayley table above is not diagonally symmetric and therefore $H$ is not abelian.
  \end{enumerate}
\end{soln}
\end{document}