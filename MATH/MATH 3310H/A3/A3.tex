\documentclass[10pt]{article}

\usepackage[margin=0.75in]{geometry}
\usepackage{amsmath,amsthm,amssymb}
\usepackage{xcolor}
\usepackage{cancel}
\usepackage{graphicx}
\usepackage{changepage}
\usepackage{circuitikz}
\usepackage{pgfplots}
\usepackage{physics}
\usepackage{hyperref}
\usepackage{siunitx}
\usepackage{fontspec}
\usepackage{relsize}
\usepackage{subfig}
\usepackage{todonotes}
\usepackage{emoji}
\usepackage{multicol, multirow, booktabs}
\usepackage[breakable]{tcolorbox}
\usepackage[inline]{enumitem}

\theoremstyle{definition}
\newtheorem{problem}{Problem}
\newtheorem{soln}{Solution}

\pgfplotsset{compat=newest}
\usetikzlibrary{lindenmayersystems}
\usetikzlibrary{arrows}
\usetikzlibrary{calc}
\usetikzlibrary{positioning, fit}
\usetikzlibrary{3d, perspective}

\definecolor{incolor}{HTML}{303F9F}
\definecolor{outcolor}{HTML}{D84315}
\definecolor{cellborder}{HTML}{CFCFCF}
\definecolor{cellbackground}{HTML}{F7F7F7}
\newcommand{\ui}{\hat{i}}
\newcommand{\uj}{\hat{j}}
\newcommand{\uk}{\hat{k}}
\newcommand{\ux}{\hat{x}}
\newcommand{\uy}{\hat{y}}
\newcommand{\uz}{\hat{z}}
\newcommand{\primed}[1]{#1^\prime}
\pgfdeclarelayer{background}  
\pgfsetlayers{background,main}
\AtBeginDocument{\RenewCommandCopy\qty\SI}
\newcommand{\justif}[2]{&{#1}&\text{#2}}

\makeatletter
\newcommand{\boxspacing}{\kern\kvtcb@left@rule\kern\kvtcb@boxsep}
\makeatother
\newcommand{\prompt}[4]{
    \ttfamily\llap{{\color{#2}[#3]:\hspace{3pt}#4}}\vspace{-\baselineskip}
}

\newcommand{\thevenin}[2]{
  \begin{center}
    \begin{circuitikz} \draw
      (0,0) -- (2,0) to[battery1, l_=$V_{Th}\eq#1$] (2,2) 
      to[resistor, l_=$R_{Th}\eq#2$] (0,2)
      ;
      \draw [o-] (-.07,2.079);
      \draw [o-] (-.07,0.079);
    \end{circuitikz}
  \end{center}
}

\newcommand{\norton}[2]{
  \begin{center}
    \begin{circuitikz} \draw
      (0,0) -- (3,0) to[american current source, l_=$I_{N}\eq#1$] (3,2) -- (0,2) (2,0)
      to[resistor, l=$R_{N}\eq#2$] (2,2)
      ;
      \draw [o-] (-.07,2.079);
      \draw [o-] (-.07,0.079);
    \end{circuitikz}
  \end{center}
}

\newcommand{\highlight}[1]{\colorbox{yellow}{$\displaystyle #1$}}

\newcommand{\ti}[1]{\widetilde{#1}}

\newfontface{\Kaufmann}{Kaufmann}
\DeclareTextFontCommand{\kf}{\Kaufmann}
\newcommand{\scriptr}{\fontsize{12pt}{12pt}\kf{r}}

\newfontface{\KaufmannB}{Kaufmann Bd BT}
\DeclareTextFontCommand{\kfb}{\KaufmannB}
\newcommand{\bscriptr}{\fontsize{12pt}{12pt}\kfb{r}}

\newcommand{\bv}[1]{\mathbf{#1}}

\title{Math 3310H: Assignment III}
\author{Jeremy Favro (0805980) \\ Trent University, Peterborough, ON, Canada}
\date{\today}

\begin{document}
\maketitle

% PROBLEM 1
\begin{problem}
Show that a group $G$ cannot be the union of two proper subgroups,
in other words, if $G = H \cup K$ where $H$ and $K$ are subgroups of $G$,
then $H = G$ or $K = G$.
\end{problem}
\begin{soln}
  Suppose, by way of contradiction, that $G=H\cup K$ and $H\neq G \neq K$.
  Then there are elements $a\in H$ and $b\in K$ but $a\notin K$ and $b\notin H$.
  Because $G=H\cup K$ and $H$, $K$, and $G$ are closed by definition, $ab\in H$ or $ab\in K$.
  First then suppose that $ab\in H\implies a^{-1}ab\in H\implies eb\in H\implies b\in H$, but we began
  with the assumption that $b \notin H$, so unless $H=K=G$, $K$ cannot be a subgroup. The same argument works
  in the other direction: Suppose $ab \in K\implies abb^{-1}\in K\implies ae \in K\implies a\in K$, but
  $a$ was created to be something only in $H$, not $K$, meaning $H$ is not closed unless $H=K=G$.
\end{soln}

% PROBLEM 2
\begin{problem}
Let $G$ be a group with identity $e$ and $e\in G$. Show that if $a^n = e$
then the order of $a$ divides $n$.
\end{problem}
\begin{soln}
  Let $\abs{a}=k$ be the order of $a$. By the division algorithm we can write
  $n=qk+r$ for some $q,r\in \mathbb{Z}$ with $0\leq r < k$. So
  \begin{align*}
    e & =a^n                                        \\
      & =a^{qk+r}                                   \\
      & =a^{qk}a^r                                  \\
      & =(a^k)^qa^r                                 \\
      & =e^q a^r\justif{\,}{$a^k=e$ by definition.} \\
      & =a^r.
  \end{align*}
  For the expression $e=a^r$ to hold true $r$ must be some multiple of the order of $a$, $k$. This means
  that our expression using the division algorithm becomes $n=qk+sk$ for $sk=r$ which means that $n/k=q+s$ which
  is an integer meaning that the order of $a$, $k$, divides $n$.
\end{soln}

% PROBLEM 3
\begin{problem}
Let $G$ be a cyclic group of order $n$ with identity $e$. Suppose 15
divides $n$. How many solutions to $x^15=e$ are there in $G$?
\end{problem}
\begin{soln}

\end{soln}

% PROBLEM 4
\begin{problem}
Show that $H = {\sigma \in S_n|\sigma(1) = 1}$ is a subgroup of $S_n$.
\end{problem}
\begin{soln}
  For $H$ to be a subgroup of $S_n$ it must satisfy the following:
  \begin{enumerate}[label=(\roman*)]
    \item Closure: This is fairly obvious, constructing any $\sigma^{\prime\prime}=\sigma\circ \primed{\sigma}$
          will always satisfy $\sigma^{\prime\prime}(1)=1$ as both $\sigma$ and $\primed{\sigma}$ must map $1\to 1$ to
          belong to $H$ in the first place.
    \item Contains the identity: The identity map looks like
          $$\iota = \begin{pmatrix}
              1 & 2 & 3 & \dots & n \\
              1 & 2 & 3 & \dots & n
            \end{pmatrix}$$
          which satisfies $\sigma(1)=1$
    \item Contains inverses: All inverses for a $\sigma\in H$ will map $1\to 1$ by the definition of $\sigma$ and
          so will belong to $H$.
  \end{enumerate}
\end{soln}

% PROBLEM 5
\begin{problem}
Let
$$
  \sigma=\begin{pmatrix}
    1 & 2 & 3 & 4 & 5 & 6 & 7 & 8 & 9 \\
    3 & 2 & 4 & 1 & 7 & 5 & 8 & 9 & 6
  \end{pmatrix}
$$
and
$$
  \tau=\begin{pmatrix}
    1 & 2 & 3 & 4 & 5 & 6 & 7 & 8 & 9 \\
    6 & 3 & 7 & 9 & 1 & 8 & 2 & 4 & 5
  \end{pmatrix}.
$$
\begin{enumerate}[label=(\alph*)]
  \item Compute $\sigma^2$, $\sigma\tau$, $\tau\sigma$, $\sigma^{-1}$, $\sigma\tau\sigma^{-1}$, and $\tau\sigma\tau^{-1}$.
  \item Find the order of $\tau$
\end{enumerate}
\end{problem}
\begin{soln}

\end{soln}

% PROBLEM 6
\begin{problem}
Below are four recommended car tire rotation patterns.
\begin{center}
  \begin{tikzpicture}
    \node (BL) at (0,0) {$\bigcirc$};
    \node (BR) at (1.5,0) {$\bigcirc$};
    \node (TL) at (0,2) {$\bigcirc$};
    \node (TR) at (1.5,2) {$\bigcirc$};

    \draw[->] (BL) -- (TL);
    \draw[->] (BR) -- (TR);
    \draw[->] (TL) -- (BR);
    \draw[->] (TR) -- (BL);

    \node (BL) at (0+3,0) {$\bigcirc$};
    \node (BR) at (1.5+3,0) {$\bigcirc$};
    \node (TL) at (0+3,2) {$\bigcirc$};
    \node (TR) at (1.5+3,2) {$\bigcirc$};

    \draw[->] (BL) -- (TL);
    \draw[->] (BR) -- (TR);
    \draw[->] (BL) -- (TR);
    \draw[->] (BR) -- (TL);

    \node (BL) at (0+3+3,0) {$\bigcirc$};
    \node (BR) at (1.5+3+3,0) {$\bigcirc$};
    \node (TL) at (0+3+3,2) {$\bigcirc$};
    \node (TR) at (1.5+3+3,2) {$\bigcirc$};

    \draw[<->] (BL) -- (TR);
    \draw[<->] (BR) -- (TL);

    \node (BL) at (0+3+3+3,0) {$\bigcirc$};
    \node (BR) at (1.5+3+3+3,0) {$\bigcirc$};
    \node (TL) at (0+3+3+3,2) {$\bigcirc$};
    \node (TR) at (1.5+3+3+3,2) {$\bigcirc$};

    \draw[<->] (BL) -- (TL);
    \draw[<->] (BR) -- (TR);
  \end{tikzpicture}
\end{center}
\begin{enumerate}[label=(\alph*)]
  \item Explain how these patterns can be represented as elements of
        $S_4$.
  \item Find the smallest subgroup $H$ of $S_4$ that contains these four
        patterns.
  \item Is $H$ abelian?
\end{enumerate}
\end{problem}
\begin{soln} ~\\
  \begin{enumerate}[label=(\alph*)]
    \item If we represent the ``default'' state of the tires as
          \begin{center}
            \begin{tikzpicture}
              \node (BL) at (0,0) {1};
              \node (BR) at (2,0) {2};
              \node (TL) at (0,2) {3};
              \node (TR) at (2,2) {4};

              \draw[->] (BL) -- (TL);
              \draw[->] (BR) -- (TR);
              \draw[->] (TL) -- (BR);
              \draw[->] (TR) -- (BL);
            \end{tikzpicture}
          \end{center}
          then each rotation of the tires is a permutation of this default state. By definition
          $S_4$ is the group containing all permutations of 4 elements and so these will belong to $S_4$.
          We can express them as compositions of known permutations. The first 
          $$a=\begin{pmatrix}
            1&2&3&4\\
            3&4&1&2
          \end{pmatrix};\quad b=\begin{pmatrix}
            1&2&3&4\\
            
          \end{pmatrix}$$
    \item Find the smallest subgroup $H$ of $S_4$ that contains these four
          patterns.
    \item Is $H$ abelian?
  \end{enumerate}
\end{soln}
\end{document}