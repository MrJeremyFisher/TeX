\documentclass[10pt]{article}

\usepackage[margin=0.75in]{geometry}
\usepackage{amsmath,amsthm,amssymb}
\usepackage{xcolor}
\usepackage{cancel}
\usepackage{graphicx}
\usepackage{changepage}
\usepackage{circuitikz}
\usepackage{pgfplots}
\usepackage{physics}
\usepackage{hyperref}
\usepackage{siunitx}
\usepackage{fontspec}
\usepackage{relsize}
\usepackage{subfig}
\usepackage{todonotes}
\usepackage{sagetex}
\usepackage{multicol, multirow, booktabs}
\usepackage[breakable]{tcolorbox}
\usepackage[inline]{enumitem}

\theoremstyle{definition}
\newtheorem{problem}{Problem}
\newtheorem{soln}{Solution}
\makeatletter
\newenvironment{subtheorem}[1]{%
  \def\subtheoremcounter{#1}%
  \refstepcounter{#1}%
  \protected@edef\theparentnumber{\csname the#1\endcsname}%
  \setcounter{parentnumber}{\value{#1}}%
  \setcounter{#1}{0}%
  \expandafter\def\csname the#1\endcsname{\theparentnumber.\Alph{#1}}%
  \ignorespaces
}{%
  \setcounter{\subtheoremcounter}{\value{parentnumber}}%
  \ignorespacesafterend
}
\makeatother
\newcounter{parentnumber}

\pgfplotsset{compat=newest}
\usetikzlibrary{lindenmayersystems}
\usetikzlibrary{arrows}
\usetikzlibrary{calc}
\usetikzlibrary{positioning, fit}
\usetikzlibrary{3d, perspective}
\usetikzlibrary{patterns.meta}

\definecolor{incolor}{HTML}{303F9F}
\definecolor{outcolor}{HTML}{D84315}
\definecolor{cellborder}{HTML}{CFCFCF}
\definecolor{cellbackground}{HTML}{F7F7F7}
\newcommand{\ui}{\hat{i}}
\newcommand{\uj}{\hat{j}}
\newcommand{\uk}{\hat{k}}
\newcommand{\ux}{\hat{x}}
\newcommand{\uy}{\hat{y}}
\newcommand{\uz}{\hat{z}}
\newcommand{\primed}[1]{#1^\prime}
\pgfdeclarelayer{background}  
\pgfsetlayers{background,main}
\AtBeginDocument{\RenewCommandCopy\qty\SI}
\newcommand{\justif}[2]{&{#1}&\text{#2}}
\DeclareMathOperator\Arg{Arg}
\DeclareMathOperator\Log{Log}

\makeatletter
\newcommand{\boxspacing}{\kern\kvtcb@left@rule\kern\kvtcb@boxsep}
\makeatother
\newcommand{\prompt}[4]{
    \ttfamily\llap{{\color{#2}[#3]:\hspace{3pt}#4}}\vspace{-\baselineskip}
}

\newcommand{\thevenin}[2]{
  \begin{center}
    \begin{circuitikz} \draw
      (0,0) -- (2,0) to[battery1, l_=$V_{Th}\eq#1$] (2,2) 
      to[resistor, l_=$R_{Th}\eq#2$] (0,2)
      ;
      \draw [o-] (-.07,2.079);
      \draw [o-] (-.07,0.079);
    \end{circuitikz}
  \end{center}
}

\newcommand{\norton}[2]{
  \begin{center}
    \begin{circuitikz} \draw
      (0,0) -- (3,0) to[american current source, l_=$I_{N}\eq#1$] (3,2) -- (0,2) (2,0)
      to[resistor, l=$R_{N}\eq#2$] (2,2)
      ;
      \draw [o-] (-.07,2.079);
      \draw [o-] (-.07,0.079);
    \end{circuitikz}
  \end{center}
}

\newcommand{\highlight}[1]{\colorbox{yellow}{$\displaystyle #1$}}

\newcommand{\ti}[1]{\widetilde{#1}}

\newfontface{\Kaufmann}{Kaufmann}
\DeclareTextFontCommand{\kf}{\Kaufmann}
\newcommand{\scriptr}{\fontsize{12pt}{12pt}\kf{r}}

\newfontface{\KaufmannB}{Kaufmann Bd BT}
\DeclareTextFontCommand{\kfb}{\KaufmannB}
\newcommand{\bscriptr}{\fontsize{12pt}{12pt}\kfb{r}}

\newcommand{\bv}[1]{\mathbf{#1}}

\title{Math 3770H: Assignment III}
\author{Jeremy Favro (0805980) \\ Trent University, Peterborough, ON, Canada}
\date{\today}

\begin{document}
\maketitle

% PROBLEM 1
\begin{problem}
Show that $\abs{\exp(z^2)} \leq \exp(|z|^2)$.
\end{problem}
\begin{soln}
\end{soln}

% PROBLEM 2
\begin{problem}
Find the principal value of
\begin{center}
  \begin{enumerate*}[label=(\alph*)]
    \item $(-i)^i$;\qquad~
    \item $\left[\frac{e}{2}\left(-1-\sqrt{3}i\right)\right]^{3\pi i}$;\qquad~
    \item $\left(1-i\right)^{4i}$.
  \end{enumerate*}
\end{center}
\end{problem}
\begin{soln}
\end{soln}

% PROBLEM 3
\begin{problem}
Derive expression (9), Sec. 40, for $\cosh^{-1}(z)$.
\end{problem}
\begin{soln}
  Expression (9) says 
  $$\cosh^{-1}(z)=\log\left[z+(z^2-1)^{1/2}\right].$$
\end{soln}

% PROBLEM 4
\begin{problem}
Let $C_R$ denote the upper half of the circle $|z| = R\,(R > 2)$, taken in the counterclockwise
direction. Show that
$$\abs{\int_{C_R}\frac{2z^2-1}{z^4+5z^2+4}\,dz}\leq \frac{\pi R(2R^2+1)}{(R^2-1)(R^2-4)}.$$
Then, by dividing the numerator and denominator on the right here by $R^4$, show that the
value of the integral tends to zero as $R$ tends to infinity. (Compare with Example 2 in
Sec. 47.)
\end{problem}
\begin{soln}
\end{soln}

% PROBLEM 5
\begin{problem}
Apply the Cauchy-Goursat theorem to show that
$$\int_C f(z)\,dz=0$$
when the contour $C$ is the unit circle $|z| = 1$, in either direction, and when\\
\begin{center}
    \begin{enumerate*}[label=(\alph*)]
      \item $\displaystyle f(z)=\frac{z^2}{z+3}$;\qquad~
      \item $\displaystyle f(z)=ze^{-z}$;\qquad~
      \item $\displaystyle f(z)=\frac{1}{z^2+2z+2}$;\newline
      \item $\displaystyle f(z)=\sech(z)$;\qquad~
      \item $\displaystyle f(z)=\tan(z)$;\qquad~
      \item $\displaystyle f(z)=\Log(z+2)$ 
    \end{enumerate*}
\end{center}
\end{problem}
\begin{soln}
\end{soln}
\end{document}