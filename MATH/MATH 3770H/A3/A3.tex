\documentclass[10pt]{article}

\usepackage[margin=0.75in]{geometry}
\usepackage{amsmath,amsthm,amssymb}
\usepackage{xcolor}
\usepackage{cancel}
\usepackage{graphicx}
\usepackage{changepage}
\usepackage{circuitikz}
\usepackage{pgfplots}
\usepackage{physics}
\usepackage{hyperref}
\usepackage{siunitx}
\usepackage{fontspec}
\usepackage{relsize}
\usepackage{subfig}
\usepackage{todonotes}
% \usepackage{sagetex}
\usepackage{multicol, multirow, booktabs}
\usepackage[breakable]{tcolorbox}
\usepackage[inline]{enumitem}

\theoremstyle{definition}
\newtheorem{problem}{Problem}
\newtheorem{soln}{Solution}
\makeatletter
\newenvironment{subtheorem}[1]{%
  \def\subtheoremcounter{#1}%
  \refstepcounter{#1}%
  \protected@edef\theparentnumber{\csname the#1\endcsname}%
  \setcounter{parentnumber}{\value{#1}}%
  \setcounter{#1}{0}%
  \expandafter\def\csname the#1\endcsname{\theparentnumber.\Alph{#1}}%
  \ignorespaces
}{%
  \setcounter{\subtheoremcounter}{\value{parentnumber}}%
  \ignorespacesafterend
}
\makeatother
\newcounter{parentnumber}

\pgfplotsset{compat=newest}
\usetikzlibrary{lindenmayersystems}
\usetikzlibrary{arrows}
\usetikzlibrary{calc}
\usetikzlibrary{positioning, fit}
\usetikzlibrary{3d, perspective}
\usetikzlibrary{patterns.meta}

\definecolor{incolor}{HTML}{303F9F}
\definecolor{outcolor}{HTML}{D84315}
\definecolor{cellborder}{HTML}{CFCFCF}
\definecolor{cellbackground}{HTML}{F7F7F7}
\newcommand{\ui}{\hat{i}}
\newcommand{\uj}{\hat{j}}
\newcommand{\uk}{\hat{k}}
\newcommand{\ux}{\hat{x}}
\newcommand{\uy}{\hat{y}}
\newcommand{\uz}{\hat{z}}
\newcommand{\primed}[1]{#1^\prime}
\pgfdeclarelayer{background}  
\pgfsetlayers{background,main}
\AtBeginDocument{\RenewCommandCopy\qty\SI}
\newcommand{\justif}[2]{&{#1}&\text{#2}}
\DeclareMathOperator\Arg{Arg}
\DeclareMathOperator\Log{Log}

\makeatletter
\newcommand{\boxspacing}{\kern\kvtcb@left@rule\kern\kvtcb@boxsep}
\makeatother
\newcommand{\prompt}[4]{
    \ttfamily\llap{{\color{#2}[#3]:\hspace{3pt}#4}}\vspace{-\baselineskip}
}

\newcommand{\thevenin}[2]{
  \begin{center}
    \begin{circuitikz} \draw
      (0,0) -- (2,0) to[battery1, l_=$V_{Th}\eq#1$] (2,2) 
      to[resistor, l_=$R_{Th}\eq#2$] (0,2)
      ;
      \draw [o-] (-.07,2.079);
      \draw [o-] (-.07,0.079);
    \end{circuitikz}
  \end{center}
}

\newcommand{\norton}[2]{
  \begin{center}
    \begin{circuitikz} \draw
      (0,0) -- (3,0) to[american current source, l_=$I_{N}\eq#1$] (3,2) -- (0,2) (2,0)
      to[resistor, l=$R_{N}\eq#2$] (2,2)
      ;
      \draw [o-] (-.07,2.079);
      \draw [o-] (-.07,0.079);
    \end{circuitikz}
  \end{center}
}

\newcommand{\highlight}[1]{\colorbox{yellow}{$\displaystyle #1$}}

\newcommand{\ti}[1]{\widetilde{#1}}

\newfontface{\Kaufmann}{Kaufmann}
\DeclareTextFontCommand{\kf}{\Kaufmann}
\newcommand{\scriptr}{\fontsize{12pt}{12pt}\kf{r}}

\newfontface{\KaufmannB}{Kaufmann Bd BT}
\DeclareTextFontCommand{\kfb}{\KaufmannB}
\newcommand{\bscriptr}{\fontsize{12pt}{12pt}\kfb{r}}

\newcommand{\bv}[1]{\mathbf{#1}}

\title{Math 3770H: Assignment III}
\author{Jeremy Favro (0805980) \\ Trent University, Peterborough, ON, Canada}
\date{\today}

\begin{document}
\maketitle

% PROBLEM 1
\begin{problem}
Show that $\abs{\exp(z^2)} \leq \exp(|z|^2)$.
\end{problem}
\begin{soln}
  With $z=x+iy$ we have that
  $$\abs{\exp(z^2)}=\abs{\exp(x^2+2ixy-y^2)}=\abs{\exp(x^2-y^2)}\abs{\exp(2ixy)}=\abs{\exp(x^2-y^2)}$$
  because
  $$\abs{\exp(2ixy)}=\abs{\cos(2xy)+i\sin(2xy)}=\sqrt{\cos^2(2xy)+\sin^2(2xy)}=1.$$
  We also have that
  $$\exp(|z|^2)=\exp(x^2+y^2).$$
  Since these are both real-valued exponentials they are increasing functions
  and since their arguments satisfy
  $$x^2-y^2\leq x^2+y^2$$
  then
  $$\abs{\exp(z^2)} \leq \exp(|z|^2).$$
\end{soln}

% PROBLEM 2
\begin{problem}
Find the principal value of
\begin{center}
  \begin{enumerate*}[label=(\alph*)]
    \item $(-i)^i$;\qquad~
    \item $\left[\frac{e}{2}\left(-1-\sqrt{3}i\right)\right]^{3\pi i}$;\qquad~
    \item $\left(1-i\right)^{4i}$.
  \end{enumerate*}
\end{center}
\end{problem}
\begin{soln}~
  \begin{enumerate}[label=(\alph*)]
    \item \begin{align*}
            (-i)^i & =(e^{i3\pi/2})^i \\
                   & =e^{-3\pi/2}
          \end{align*}
    \item \begin{align*}
            \left[\frac{e}{2}\left(-1-\sqrt{3}i\right)\right]^{3\pi i} & =e^{3\pi i}\left[\left(-\frac{1}{2}-\frac{\sqrt{3}}{2}i\right)\right]^{3\pi i} \\
                                                                       & =e^{3\pi i}\left[e^{i2\pi/3}\right]^{3\pi i}                                   \\
                                                                       & =-e^{-2\pi^2}
          \end{align*}
    \item \begin{align*}
            \left(1-i\right)^{4i} & =\left(\sqrt{2}e^{-i\pi/4}\right)^{4i} \\
                                  & =\sqrt{2}e^{\pi}
          \end{align*}
  \end{enumerate}
\end{soln}
\newpage 

% PROBLEM 3
\begin{problem}
Derive expression (9), Sec. 40, for $\cosh^{-1}(z)$.
\end{problem}
\begin{soln}
  If $\cosh^{-1}(z)\implies z=\cosh(\theta)$.
  Then
  \begin{align*}
    z                                & =\cosh(\theta)=\cos(i\theta)    \\
    \implies z                       & =\frac{e^\theta+e^{-\theta}}{2} \\
    \implies 2z                      & =e^\theta+e^{-\theta}           \\
    \implies 2ze^\theta              & =e^2\theta+1                    \\
    \implies -e^2\theta+2ze^\theta-1 & =0
  \end{align*}
  which is a quadratic in $e^\theta$ and can be solved using the quadratic formula,
  $$e^\theta=\frac{-2z\pm\sqrt{4z^2-4(-1)(-1)}}{2(-1)}=\frac{2z\mp2\sqrt{z^2-1}}{2}=z\mp\sqrt{z^2-1}.$$
  so, taking the logarithm,
  $$\theta=\cosh^{-1}(z)=\ln\left(z\mp\sqrt{z^2-1}\right)$$
  which isn't quite right as we've still got the $\pm$.
  To get rid of it we can (or rather, I did) look at the graph of $\cosh^{-1}(x)$ for real values
  and note that it is only defined for positive $x$. We can also note that
  $$\ln\left(z\mp\sqrt{z^2-1}\right)=\ln\left(z+\sqrt{z^2-1}\right)^\mp=\mp\ln\left(z+\sqrt{z^2-1}\right).$$
  So we must remove the negative case to obey the definition of $\cosh^{-1}$.
\end{soln}
\newpage

% PROBLEM 4
\begin{problem}
Let $C_R$ denote the upper half of the circle $|z| = R\,(R > 2)$, taken in the counterclockwise
direction. Show that
$$\abs{\int_{C_R}\frac{2z^2-1}{z^4+5z^2+4}\,dz}\leq \frac{\pi R(2R^2+1)}{(R^2-1)(R^2-4)}.$$
Then, by dividing the numerator and denominator on the right here by $R^4$, show that the
value of the integral tends to zero as $R$ tends to infinity. (Compare with Example 2 in
Sec. 47.)
\end{problem}
\begin{soln}
  Note that
  $$
    \frac{2z^2-1}{z^4+5z^2+4}=\frac{2z^2-1}{(z^2+4)(z^2+1)}
  $$
  and that
  $$\abs{2z^2-1}\leq \abs{2z^2}-1=2R^2-1$$
  and that
  $$\abs{z^2+4}\geq \abs{\abs{z}^2-4}=R^2-4$$
  and that
  $$\abs{z^2+1}\geq \abs{\abs{z}^2-1}=R^2-1$$
  by the triangle inequality.
  And so the absolute value of the integrand is bounded as
  $$\abs{\frac{2z^2-1}{z^4+5z^2+4}}=\frac{\abs{2z^2-1}}{\abs{z^2+4}\abs{z^2+1}}\leq\frac{2R^2-1}{\left(R^2-4\right)\left(R^2-1\right)}.$$
  So by the theorem presented in section 47 regarding the relationship between the bounds on integrands and their integrals we can say that
  $$\abs{\int_{C_R}\frac{2z^2-1}{(z^2+4)(z^2+1)}\,dz}\leq\frac{2R^2-1}{\left(R^2+4\right)\left(R^2+1\right)}L=\frac{\pi R\left(2R^2-1\right)}{\left(R^2-4\right)\left(R^2-1\right)}$$
  as the length of $C_R$ here is half the length of a circle.
  Now note that
  \begin{align*}
    \lim_{R\to\infty}\frac{\pi R\left(2R^2-1\right)}{\left(R^2-4\right)\left(R^2-1\right)} & =\lim_{R\to\infty}\frac{\pi R\left(2R^2-1\right)}{\left(R^2-4\right)\left(R^2-1\right)}\frac{1/R^4}{1/R^4}                                            \\
                                                                                           & =\lim_{R\to\infty}\frac{\pi \left(\frac{2}{R}-\frac{1}{R^3}\right)}{\left(\frac{1}{R^2}-\frac{4}{R^4}\right)\left(\frac{1}{R^2}-\frac{1}{R^4}\right)} \\
                                                                                           & =\lim_{R\to{\infty}}{\frac{\pi\left(\frac{2}{R}-\frac{1}{R^{3}}\right)}{-\frac{5}{R^{2}}+\frac{4}{R^{4}}+1}}                                          \\
                                                                                           & =0/1=0.
  \end{align*}
\end{soln}
\newpage

% PROBLEM 5
\begin{problem}
Apply the Cauchy-Goursat theorem to show that
$$\int_C f(z)\,dz=0$$
when the contour $C$ is the unit circle $|z| = 1$, in either direction, and when\\

\begin{center}
  \begin{multicols}{3}
    \begin{enumerate}[label=(\alph*)]
      \item $\displaystyle f(z)=\frac{z^2}{z+3}$;
      \item $\displaystyle f(z)=ze^{-z}$;
      \item $\displaystyle f(z)=\frac{1}{z^2+2z+2}$;
      \item $\displaystyle f(z)=\sech(z)$;
      \item $\displaystyle f(z)=\tan(z)$;
      \item $\displaystyle f(z)=\Log(z+2)$
    \end{enumerate}
  \end{multicols}
\end{center}
\end{problem}
\begin{soln}~
  \begin{enumerate}[label=(\alph*)]
    \item Since this function is analytic everywhere except at $z=3$ its integral is zero over the given contour.
    \item Since this function is the product of two entire functions it is itself entire and its integral is therefore zero.
    \item Using the quadratic formula the denominator here factors as $z^2+2z+2=(z+1+i)(z+1-i)$. Since both
          of these points lie at distance $\sqrt{2}$ from the origin they are not contained in or on the border of $C$ and
          the integral of the function is therefore zero.
    \item Here note that $\sech(z)=\frac{1}{\cosh(z)}=\frac{1}{\cos(iz)}=\frac{2}{e^{z}+e^{-z}}$ which is undefined for $z=\frac{\pi}{2}\left(2k+1\right)$ for
          $k\in \mathbb{Z}$ as $\cos$ is zero. These points are however at a distance of $\pi/2\approx 1.57>1$ and so do not lie in or on $C$ meaning the integral
          is zero.
    \item Note that $\tan(z)=\frac{\sin(z)}{\cos(z)}$ which is analytic in $C$, only diverging for $z=\left(2k+1\right)\frac{\pi}{2}$, which is, as above, not in or
          on the edge of $C$ and therefore integral is zero.
    \item Note that $\Log(z+2)=\ln\abs{z+2}+i\Arg(z)$. $\Log(z)$ is analytic everywhere except on the real axis where $z\leq 0$ and
          so $\Log(z+2)$ is analytic except for $z+2\leq 0$. The points on or inside $C$ do not satisfy this inequality (the closest we can get is
          $z=-1$) and so the function is analytic in the region and the integral is zero.
  \end{enumerate}
\end{soln}
\newpage

% PROBLEM 6
\begin{problem}
Let $C$ denote the positively oriented boundary of the square whose sides lie along the
lines $x =\pm 2$ and $y = \pm 2$. Evaluate each of these integrals:
\begin{center}
  \begin{multicols}{3}
    \begin{enumerate}[label=(\alph*)]
      \item $\displaystyle \int_C\frac{e^{-z}}{z-i\pi/2}\,dz$;
      \item $\displaystyle \int_C\frac{\cos(z)}{z\left(z^2+8\right)}\,dz$;
      \item $\displaystyle \int_C\frac{z}{2z+1}\,dz$;
      \item $\displaystyle \int_C\frac{\cosh(z)}{z^4}\,dz$;
      \item $\displaystyle \int_C\frac{\tan(z/2)dz}{\left(z-x_0\right)^2}\quad (-2<x_0<2)$
    \end{enumerate}
  \end{multicols}
\end{center}
\end{problem}
\begin{soln}
  \begin{enumerate}[label=(\alph*)]
    \item Using Cauchy's integral formula,
          $$f(z_0)=\frac{1}{2\pi i}\int_C\frac{f(z)}{z-z_0}\,dz,$$
          here we have $f(z)=e^{-z}$ and $z_0=i\pi/2$. So
          $$\int_C\frac{e^{-z}}{z-i\pi/2}\,dz=2\pi i e^{-i\pi/2}=2\pi i (-i)=2\pi.$$
    \item Using the extended form of Cauchy's integral formula,
          $$\int_C\frac{f(z)}{(z-z_0)^{n+1}}\,dz=\frac{n!}{2\pi i}f^(n)(z_0),$$
          and rewriting our integral as
          $$\int_C\frac{(\cos(z)/\left(z^2+8\right))}{(z-0)^{0+1}}\,dz$$
          we can clearly see that
          $n=1$, $z_0=0$, and $f(z)=\cos(z)/\left(z^2+8\right)$. Evaluating then we obtain that the integral is equal to
          $$\frac{\cos(0)}{8}\cdot\frac{2\pi i}{0!}=\frac{\pi i}{4}.$$
    \item The integral can be rewritten as
          $$\int_C\frac{z/2}{z+1/2}\,dz=\int_C\frac{z/2}{(z-(-1/2))^{0+1}}\,dz$$
          and so by Cauchy's extended integral formula we obtain
          $$\int_C\frac{z}{2z+1}\,dz=\frac{(-1/4)\cdot2\pi i}{0!}=-\frac{\pi i}{2}.$$
    \item Here we can directly see that $f(z)=\cosh(z)$, $z_0=0$, and $n=3$. The third derivative
          of $\cosh(z)$ is $\sinh(z)$ and so the integral evaluates to zero as $\sinh(0)=0$ in the numerator.
    \item Here we have that $f(z)=\tan(z/2)$, $z_0=x_0$, and $n=1$. The first derivative of $\tan(z/2)$
          is $\sec^2(z/2)/2$ and so by Cauchy's extended integral formula we have
          $$\int_C\frac{\tan(z/2)}{\left(z-x_0\right)^2}\,dz=\frac{2\pi i \sec^2(x_0/2)}{2\cdot0!}=\pi i \sec^2(x_0/2).$$
  \end{enumerate}
\end{soln}
\end{document}