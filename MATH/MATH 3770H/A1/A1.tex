\documentclass[10pt]{article}

\usepackage[margin=0.75in]{geometry}
\usepackage{amsmath,amsthm,amssymb}
\usepackage{xcolor}
\usepackage{cancel}
\usepackage{graphicx}
\usepackage{changepage}
\usepackage{circuitikz}
\usepackage{pgfplots}
\usepackage{physics}
\usepackage{hyperref}
\usepackage{siunitx}
\usepackage{fontspec}
\usepackage{relsize}
\usepackage{subfig}
\usepackage{todonotes}
\usepackage{multicol, multirow, booktabs}
\usepackage[breakable]{tcolorbox}
\usepackage[inline]{enumitem}

\theoremstyle{definition}
\newtheorem{problem}{Problem}
\newtheorem{soln}{Solution}

\pgfplotsset{compat=newest}
\usetikzlibrary{lindenmayersystems}
\usetikzlibrary{arrows}
\usetikzlibrary{calc}
\usetikzlibrary{positioning, fit}
\usetikzlibrary{3d, perspective}

\definecolor{incolor}{HTML}{303F9F}
\definecolor{outcolor}{HTML}{D84315}
\definecolor{cellborder}{HTML}{CFCFCF}
\definecolor{cellbackground}{HTML}{F7F7F7}
\newcommand{\ui}{\hat{i}}
\newcommand{\uj}{\hat{j}}
\newcommand{\uk}{\hat{k}}
\newcommand{\ux}{\hat{x}}
\newcommand{\uy}{\hat{y}}
\newcommand{\uz}{\hat{z}}
\newcommand{\primed}[1]{#1^\prime}
\pgfdeclarelayer{background}  
\pgfsetlayers{background,main}
\AtBeginDocument{\RenewCommandCopy\qty\SI}

\makeatletter
\newcommand{\boxspacing}{\kern\kvtcb@left@rule\kern\kvtcb@boxsep}
\makeatother
\newcommand{\prompt}[4]{
    \ttfamily\llap{{\color{#2}[#3]:\hspace{3pt}#4}}\vspace{-\baselineskip}
}

\newcommand{\thevenin}[2]{
  \begin{center}
    \begin{circuitikz} \draw
      (0,0) -- (2,0) to[battery1, l_=$V_{Th}\eq#1$] (2,2) 
      to[resistor, l_=$R_{Th}\eq#2$] (0,2)
      ;
      \draw [o-] (-.07,2.079);
      \draw [o-] (-.07,0.079);
    \end{circuitikz}
  \end{center}
}

\newcommand{\norton}[2]{
  \begin{center}
    \begin{circuitikz} \draw
      (0,0) -- (3,0) to[american current source, l_=$I_{N}\eq#1$] (3,2) -- (0,2) (2,0)
      to[resistor, l=$R_{N}\eq#2$] (2,2)
      ;
      \draw [o-] (-.07,2.079);
      \draw [o-] (-.07,0.079);
    \end{circuitikz}
  \end{center}
}

\newcommand{\highlight}[1]{\colorbox{yellow}{$\displaystyle #1$}}

\newcommand{\ti}[1]{\widetilde{#1}}

\newfontface{\Kaufmann}{Kaufmann}
\DeclareTextFontCommand{\kf}{\Kaufmann}
\newcommand{\scriptr}{\fontsize{12pt}{12pt}\kf{r}}

\newfontface{\KaufmannB}{Kaufmann Bd BT}
\DeclareTextFontCommand{\kfb}{\KaufmannB}
\newcommand{\bscriptr}{\fontsize{12pt}{12pt}\kfb{r}}

\newcommand{\bv}[1]{\mathbf{#1}}

\title{Math 3770H: Assignment I}
\author{Jeremy Favro (0805980) \\ Trent University, Peterborough, ON, Canada}
\date{\today}

\begin{document}
\maketitle

% PROBLEM 1
\begin{problem}
Verify that each of the two numbers $z = 1 \pm i$ satisfies the equation $z^2 - 2z + 2 = 0$.
\end{problem}
\begin{soln}
\end{soln}

% PROBLEM 2
\begin{problem}
Prove that multiplication of complex numbers is commutative, as stated at the beginning
of Sec. 2.
\end{problem}
\begin{soln}
\end{soln}

% PROBLEM 3
\begin{problem}
Reduce each of these quantities to a real number:\\
\begin{center}
  \begin{enumerate*}[label=(\alph*)]
    \item $\displaystyle\frac{1+2i}{3-4i}+\frac{2-i}{5i}$;\qquad~
    \item $\displaystyle\frac{5i}{(1-i)(2-i)(3-i)}$;\qquad~
    \item $\displaystyle(1-i)^4$.
  \end{enumerate*}
\end{center}
\end{problem}
\begin{soln}
\end{soln}

% PROBLEM 4
\begin{problem}
Verify that $\displaystyle\sqrt{2}\abs{z}\geq \abs{\Re{z}} +\abs{\Im{z}}$
\end{problem}
\begin{soln}
\end{soln}

% PROBLEM 5
\begin{problem}
Using the fact that $\abs{z_1-z_2}$ is the distance between two points $z_1$ and $z_2$, give a geometric
argument that $\abs{z-1}=\abs{z+1}$ represents the line through the origin whose slope is $-1$.
\end{problem}
\begin{soln}
\end{soln}

% PROBLEM 6
\begin{problem}
By factoring $z^4 - 4z^2 + 3$ into two quadratic factors and using inequality (2), Sec. 5,
show that if $z$ lies on the circle $\abs{z} = 2$, then
$$\abs{\frac{1}{z^4-4z^2+3}}\leq \frac{1}{3}$$
\end{problem}
\begin{soln}
\end{soln}

% PROBLEM 7
\begin{problem}
Find the principal argument $\arg{z}$ when
  \begin{enumerate*}[label=(\alph*)]
    \item $\displaystyle z=\frac{-2}{1+\sqrt{3}i}$;\qquad~
    \item $\displaystyle z=\left(\sqrt{3}-i\right)^6$.
  \end{enumerate*}
\end{problem}
\begin{soln}
\end{soln}

% PROBLEM 8
\begin{problem}
Find $\left(-8-8\sqrt{3}i\right)^{1/4}$, express the roots in rectangular coordinates, 
exhibit them as the vertices of a certain square, and point out which is the principal root.
\end{problem}
\begin{soln}
\end{soln}
\end{document}