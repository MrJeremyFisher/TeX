\documentclass[10pt]{article}

\usepackage[margin=0.75in]{geometry}
\usepackage{amsmath,amsthm,amssymb}
\usepackage{xcolor}
\usepackage{cancel}
\usepackage{changepage}
\usepackage{tikz}
\usepackage{pgfplots}
\usepackage{physics}
\usepackage{minted}
\usepackage{hyperref}
\usepackage[breakable]{tcolorbox}
\usepackage[inline]{enumitem}

\theoremstyle{definition}
\newtheorem{problem}{Problem}
\newtheorem{soln}{Solution}

\pgfplotsset{compat=newest}
\usetikzlibrary{lindenmayersystems}

\definecolor{incolor}{HTML}{303F9F}
\definecolor{outcolor}{HTML}{D84315}
\definecolor{cellborder}{HTML}{CFCFCF}
\definecolor{cellbackground}{HTML}{F7F7F7}

\tikzset
{%
  axes/.style={thick,-latex},
  cylinder/.style={right color=blue!80,left color=white,fill opacity=0.7},
  paraboloid back/.style={left color=magenta!80,fill opacity=0.4},
  paraboloid front/.style={left color=white, right color=magenta!80,fill opacity=0.4},
}

\makeatletter
\newcommand{\boxspacing}{\kern\kvtcb@left@rule\kern\kvtcb@boxsep}
\makeatother
\newcommand{\prompt}[4]{
    \ttfamily\llap{{\color{#2}[#3]:\hspace{3pt}#4}}\vspace{-\baselineskip}
}


\hypersetup{
    colorlinks=true,
    linkcolor=blue,
    filecolor=magenta,      
    urlcolor=cyan,
    pdftitle={Overleaf Example},
    pdfpagemode=FullScreen,
    }

\NewDocumentCommand{\evalat}{sO{\big}mm}{%
  \IfBooleanTF{#1}
   {\mleft. #3 \mright|_{#4}}
   {#3#2|_{#4}}%
}

\tikzset{koch snowflake/.style={insert path={%
   l-system [l-system={rule set={F -> F-F++F-F}, axiom=F++F++F,
    step=3cm/3^#1, angle=60, order=#1,anchor=center}] -- cycle}}}

\title{Calculus II: Assignment 10}
\author{Jeremy Favro}
\date{\today}

\begin{document}

\maketitle

% PROBLEM 1
\begin{problem}
For what values of x does the series $\displaystyle \sum_{n = 0}^{\infty} \frac{(-1)^nx^{2n+1}}{(2n+1)!}$ converge?
\end{problem}
\begin{soln}
      \begin{align*}
             & = \lim_{n \to \infty} \abs{\frac{(-1)^{n+1}x^{2(n+1)+1}}{(2(n+1)+1)!}\frac{(2n+1)!}{(-1)^nx^{2n+1}}}     \\
             & = \lim_{n \to \infty} \abs{\frac{(-1)^{n+1}}{(-1)^n}\frac{x^{2n+3}}{x^{2n+1}}\frac{(2n+1)!}{(2n+3)!}}    \\
             & = \lim_{n \to \infty} \abs{(-1)(x^2)\frac{(2n+1)!}{(2n+3)!}}                                             \\
             & = \lim_{n \to \infty} \abs{\frac{(-1)(x^2)}{(2n+3)(2n+2)}}                                               \\
             & = \lim_{n \to \infty} \abs{\frac{(-1)(x^2)}{4n^2+10n+6}}                                                 \\
             & = \lim_{n \to \infty} \abs{\frac{\frac{(-1)(x^2)}{n^2}}{\frac{4n^2}{n^2}+\frac{10n}{n^2}+\frac{6}{n^2}}} \\
             & = 0
      \end{align*}
      \noindent $\therefore$ By the ratio test the series $\displaystyle \sum_{n = 0}^{\infty} \frac{(-1)^nx^{2n+1}}{(2n+1)!}$ converges for all $x$
\end{soln}

% PROBLEM 2
\begin{problem}
What function does the series $\displaystyle \sum_{n = 0}^{\infty} \frac{(-1)^nx^{2n+1}}{(2n+1)!}$ equal when it converges?
\end{problem}
\begin{soln} The series should converge to some alternating function, so it's probably a trig function of some kind. To determine which I used SageMath so I didn't have to go through all of them,
      but I'll also work through determining properly. Knowing which function the series equals, I look at the $n^{th}$ derivative pattern for the Taylor Series of $\sin(x)$ about 0:
      \begin{center}
            \begin{tabular}{ c | c | c }
                  $f^{(0)}(a=0)$ & $\sin(a)$  & 0  \\
                  $f^{(1)}(a)$   & $\cos(a)$  & 1  \\
                  $f^{(2)}(a)$   & $-\sin(a)$ & 0  \\
                  $f^{(3)}(a)$   & $-\cos(a)$ & -1 \\
                  $f^{(4)}(a)$   & $\sin(a)$  & 0
            \end{tabular}
      \end{center}
      \noindent So the series expansion should look like
      \begin{align*}
             & = \frac{0}{0!}x^0+\frac{1}{1!}x^1+\frac{0}{2!}x^2+\frac{-1}{3!}x^3+\frac{0}{4!}x^4+\frac{1}{5!}x^5+\ldots \\
             & = \frac{1}{1!}x^1+\frac{-1}{3!}x^3+\frac{1}{5!}x^5+\ldots                                                 \\
             & = \sum_{n = 0}^{\infty} \frac{(-1)^nx^{2n+1}}{(2n+1)!}
      \end{align*}
      $\therefore$ $\displaystyle \sum_{n = 0}^{\infty} \frac{(-1)^nx^{2n+1}}{(2n+1)!}=\sin(x)$ when it converges
\end{soln}

% PROBLEM 3
\begin{problem}
For what values of x does the series $\displaystyle \sum_{n = 0}^{\infty} (n+1)x^n$ converge?
\end{problem}
\begin{soln}
      \begin{align*}
             & = \lim_{n \to \infty} \abs{\frac{(n+2)x^{n+1}}{(n+1)x^n}}          \\
             & = \lim_{n \to \infty} \abs{x\frac{(n+2)}{(n+1)}}                   \\
             & = \lim_{n \to \infty} \abs{x\frac{1+\frac{2}{n}}{(1+\frac{1}{n})}} \\
             & = \lim_{n \to \infty} \abs{x}
      \end{align*}
      \noindent Testing endpoints as well
      \begin{align*}
             & = \lim_{n \to \infty} \abs{(n+1)(\pm 1)^n} \\
             & = \lim_{n \to \infty} (n+1)                \\
             & = \infty
      \end{align*}
      \noindent $\therefore$ By the ratio test the series $\displaystyle \sum_{n = 0}^{\infty} (n+1)x^n$ converges for all $\abs{x}<1$
\end{soln}

% PROBLEM 4
\begin{problem}
What function does the series $\displaystyle \sum_{n = 0}^{\infty} (n+1)x^n$ equal when it converges?
\end{problem}
\begin{soln} The easiest way I can think of to approach this is to try and get $(n+1)x^n$ into the form $ax^n$ which can then be simplified using the formula for geometric series.
      \begin{align*}
             & = \int \sum_{n = 0}^{\infty} (n+1)x^n \,dx \\
             & = \sum_{n = 0}^{\infty} \int (n+1)x^n \,dx \\
             & = \sum_{n = 0}^{\infty} x^{n+1}            \\
             & = \sum_{n = 0}^{\infty} xx^n
      \end{align*}
      Which is a geometric series with common ratio $x$ and first term $x$, which can be rewritten as $\frac{x}{1-x}$ (for all $x<1$) which we can then differentiate to cancel out the previous integration and arrive at a
      function for the original series.
      \begin{align*}
             & = \frac{d}{dx} \frac{x}{1-x}                                                          \\
             & = \frac{\left[\frac{d}{dx}(x)\right](1-x)-\left[\frac{d}{dx}(1-x)\right](x)}{(1-x)^2} \\
             & = \frac{1-x+x}{(1-x)^2}                                                               \\
             & = \frac{1}{(1-x)^2}
      \end{align*}
      \noindent $\therefore$ $\displaystyle \sum_{n = 0}^{\infty} (n+1)x^n = \frac{1}{(1-x)^2}$ when it converges
\end{soln}

$$\{\}=0, \, n+1=n\cup \{n\} \implies \{\{\}\}=1 \therefore \{\{\}\}\cup\{\{\{\}\}\}=\{\{\},\{\{\}\}\}=2$$
\end{document}