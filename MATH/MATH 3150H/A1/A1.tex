\documentclass[10pt]{article}

\usepackage[margin=0.75in]{geometry}
\usepackage{amsmath,amsthm,amssymb}
\usepackage{xcolor}
\usepackage{cancel}
\usepackage{graphicx}
\usepackage{changepage}
\usepackage{circuitikz}
\usepackage{pgfplots}
\usepackage{physics}
\usepackage{hyperref}
\usepackage{siunitx}
\usepackage{fontspec}
\usepackage{relsize}
\usepackage{subfig}
\usepackage{todonotes}
\usepackage{multicol, multirow, booktabs}
\usepackage[breakable]{tcolorbox}
\usepackage[inline]{enumitem}

\theoremstyle{definition}
\newtheorem{problem}{Problem}
\newtheorem{soln}{Solution}

\pgfplotsset{compat=newest}
\usetikzlibrary{lindenmayersystems}
\usetikzlibrary{arrows}
\usetikzlibrary{calc}
\usetikzlibrary{positioning, fit}
\usetikzlibrary{3d, perspective}

\definecolor{incolor}{HTML}{303F9F}
\definecolor{outcolor}{HTML}{D84315}
\definecolor{cellborder}{HTML}{CFCFCF}
\definecolor{cellbackground}{HTML}{F7F7F7}
\newcommand{\ui}{\hat{i}}
\newcommand{\uj}{\hat{j}}
\newcommand{\uk}{\hat{k}}
\newcommand{\ux}{\hat{x}}
\newcommand{\uy}{\hat{y}}
\newcommand{\uz}{\hat{z}}
\newcommand{\primed}[1]{#1^\prime}
\pgfdeclarelayer{background}  
\pgfsetlayers{background,main}
\AtBeginDocument{\RenewCommandCopy\qty\SI}
\newcommand{\justif}[2]{&{#1}&\text{#2}}
\DeclareMathOperator\Arg{Arg}

\makeatletter
\newcommand{\boxspacing}{\kern\kvtcb@left@rule\kern\kvtcb@boxsep}
\makeatother
\newcommand{\prompt}[4]{
    \ttfamily\llap{{\color{#2}[#3]:\hspace{3pt}#4}}\vspace{-\baselineskip}
}

\newcommand{\thevenin}[2]{
  \begin{center}
    \begin{circuitikz} \draw
      (0,0) -- (2,0) to[battery1, l_=$V_{Th}\eq#1$] (2,2) 
      to[resistor, l_=$R_{Th}\eq#2$] (0,2)
      ;
      \draw [o-] (-.07,2.079);
      \draw [o-] (-.07,0.079);
    \end{circuitikz}
  \end{center}
}

\newcommand{\norton}[2]{
  \begin{center}
    \begin{circuitikz} \draw
      (0,0) -- (3,0) to[american current source, l_=$I_{N}\eq#1$] (3,2) -- (0,2) (2,0)
      to[resistor, l=$R_{N}\eq#2$] (2,2)
      ;
      \draw [o-] (-.07,2.079);
      \draw [o-] (-.07,0.079);
    \end{circuitikz}
  \end{center}
}

\newcommand{\highlight}[1]{\colorbox{yellow}{$\displaystyle #1$}}

\newcommand{\ti}[1]{\widetilde{#1}}

\newfontface{\Kaufmann}{Kaufmann}
\DeclareTextFontCommand{\kf}{\Kaufmann}
\newcommand{\scriptr}{\fontsize{12pt}{12pt}\kf{r}}

\newfontface{\KaufmannB}{Kaufmann Bd BT}
\DeclareTextFontCommand{\kfb}{\KaufmannB}
\newcommand{\bscriptr}{\fontsize{12pt}{12pt}\kfb{r}}

\newcommand{\bv}[1]{\mathbf{#1}}

\title{Math 3150H: Assignment I}
\author{Jeremy Favro (0805980) \\ Trent University, Peterborough, ON, Canada}
\date{\today}

\begin{document}
\maketitle
My student number is 0805980 so $p=9$, $q=5$, and $r=22$.
% PROBLEM 1
\begin{problem}
Consider the second order linear PDE given by
$$pu_{xx}+10pu_{xy}+9pu_{yy}+qu_x+qu_y=8pqx+e^{8ry}$$
\begin{enumerate}[label=(\alph*)]
  \item Find a canonical form of the PDE.
  \item Determine the general solution of the PDE.
  \item Show that the general solution you obtained satisfies the original equation.
\end{enumerate}
\end{problem}
\begin{soln}~
  \begin{enumerate}[label=(\alph*)]
    \item Here we have
          $$\Delta = B^2-4AC=100p^2-4(p)(9p)=64p^2>0$$
          So the PDE is hyperbolic. Now we solve
          $$\frac{dy}{dx}=\frac{B\pm\sqrt{64p^2}}{2A}=\frac{10p\pm8p}{2p}=5\pm4.$$
          Which gives in the plus case
          $$\frac{dy}{dx}=9\implies y=9x+\xi\implies \xi=y-9x$$
          and in the minus case
          $$\frac{dy}{dx}=1\implies y=x+\eta\implies \eta=y-x.$$
          Now we do our partials
          \begin{align*}
             & \xi_x = -9 \qquad \xi_{xx}=0 \qquad \xi_y=1 \qquad \xi_{yy}=0 \qquad \xi_{xy}=0       \\
             & \eta_x = -1 \qquad \eta_{xx}=0 \qquad \eta_y=1 \qquad \eta_{yy}=0 \qquad \eta_{xy}=0.
          \end{align*}
          Now we find our new coefficients. We expect $A_1=C_1=0$ but we'll check just to be sure,
          \begin{align*}
            A_1 & =A\xi_x^2+B\xi_x\xi_y+C\xi_y^2                                     \\
                & =p\cdot(-9)^2+10p\cdot(-9)\cdot(1)+9p\cdot(1)^2                    \\
                & =0                                                                 \\
            B_1 & =2A\xi_x\eta_x+B\left(\xi_x\eta_y+\xi_y\eta_x\right)+2C\xi_y\eta_y \\
                & =2p\cdot(-9)\cdot(-1)+10p\cdot
            \left((-9)\cdot(1)+(1)\cdot(-1)\right)+2\cdot(9p)\cdot(1)\cdot(1)        \\
                & = -64p                                                             \\
            C_1 & =A\eta_x^2+B\eta_x\eta_y+C\eta_y^2                                 \\
                & =p\cdot(-1)^2+10p\cdot(-1)\cdot(1)+9p\cdot(1)^2                    \\
                & =0                                                                 \\
            D_1 & =A\xi_{xx}+B\xi_{xy}+C\xi_{yy}+D\xi_{x}+E\xi_{y}                   \\
                & =q\cdot(-9)+q\cdot(1)                                              \\
                & =-8q                                                               \\
            E_1 & =A\eta_{xx}+B\eta_{xy}+C\eta_{yy}+D\eta_{x}+E\eta_{y}              \\
                & =q\cdot(-1)+q\cdot(1)                                              \\
                & =0                                                                 \\
            F_1 & =0                                                                 \\
            G_1 & =pq\left(\eta-\xi\right)+e^{r\left(9\eta-\xi\right)}
          \end{align*}
          Where for $G_1$ we've made the substitution
          \begin{align*}
            x & =\frac{1}{8}\left(\eta-\xi\right)   \\
            y & =\frac{1}{8}\left(9\eta-\xi\right).
          \end{align*}
          This gives our new canonical form PDE as (with some manipulation):
          $$64pu_{\xi\eta}+8qu_\xi=pq\left(\eta-\xi\right)+e^{r\left(9\eta-\xi\right)}$$
    \item First we integrate with respect to $\xi$,
          \begin{align*}
            \int64pu_{\xi\eta}+8qu_\xi\,d\xi & =\int pq\left(\eta-\xi\right)+e^{r\left(9\eta-\xi\right)}\,d\xi                                                  \\
            64pu_{\eta}+8qu                  & =pq \left({\eta}{\xi} - \frac{{\xi}^{2}}{2}\right) - \frac{e^{r \left(9{\eta} - {\xi}\right)}}{r}                \\
            u_{\eta}+\frac{8q}{64p}u         & =\frac{pq}{64p} \left({\eta}{\xi} - \frac{{\xi}^{2}}{2}\right) - \frac{e^{r \left(9{\eta} - {\xi}\right)}}{64pr}
          \end{align*}
          Which is a linear first order ODE so we find an integrating factor $\mu$,
          $$\mu=\exp\left(\int\frac{8q}{64p}\,d\eta\right)=\exp\left(\frac{8q}{64p}\eta\right).$$
          This gives us
          \begin{align*}
            u\exp\left(\frac{8q}{64p}\eta\right) & =\int\exp\left(\frac{8q}{64p}\eta\right)
            \left[\frac{q}{64} \left({\eta}{\xi} - \frac{{\xi}^{2}}{2}\right) - \frac{e^{r \left(9{\eta} - {\xi}\right)}}{64pr}\right]\,d\eta                                                         \\
                                                 & =-\frac{e^{9r{\eta} + \frac{q{\eta}}{8p} - r{\xi}}}{64pr \left(9r + \frac{q}{8p}\right)} + \frac{{\xi} \left(8pq{\eta} - 64p^{2}\right) e^{\frac{q{\eta}}{8p}}}{64q} - \frac{p{\xi}^{2} e^{\frac{q{\eta}}{8p}}}{16}
          \end{align*}
          Transforming this back to something in terms of $x$ and $y$ we get
          \begin{align*}
            u & =\exp\left(-\frac{8q}{64p}(y-x)\right)\left(-\frac{e^{9r{(y-x)} + \frac{q{(y-x)}}{8p} - r{(y-9x)}}}{64pr \left(9r + \frac{q}{8p}\right)} + \frac{(y-9x) \left(8pq{(y-x)} - 64p^{2}\right) e^{\frac{q{(y-x)}}{8p}}}{64q} - \frac{p(y-9x)^{2} e^{\frac{q{(y-x)}}{8p}}}{16}\right)            \\
            % & =\left(\frac{q \left({(y-9x)}{(y-x)} - \frac{{(y-9x)}^{2}}{2}\right)}{64} - \frac{e^{r \left(9{(y-x)} - {(y-9x)}\right)}}{64pr}\right) \\
            %   & = \frac{q\left(y^2-2yx-63x^2\right)}{128}-\frac{e^{8yr}}{64pr}
          \end{align*}
    \item For this we first calculate the partials,
          \begin{align*}
             & u_x=-\frac{q \left(63x + y\right)}{64} \qquad u_{xx}=-\frac{63q}{64} \qquad u_{xy}=-\frac{q}{64}                     \\
             & u_y=\frac{q \left(2y - 2x\right)}{128} - \frac{e^{8ry}}{8p} \qquad u_{yy}=\frac{q}{64} - \frac{re^{8ry}}{p}
          \end{align*}
          Then evaluate the original equation with these values,
          \begin{align*}
             & =pu_{xx}+10pu_{xy}+9pu_{yy}+qu_x+qu_y                            \\
             & =p\left[-\frac{63q}{64}\right]+
            10p\left[-\frac{q}{64}\right]+
            9p\left[\frac{q}{64} - \frac{re^{8ry}}{p}\right]+
            q\left[-\frac{q \left(63x + y\right)}{64}\right]+
            q\left[\frac{q \left(2y - 2x\right)}{128}\right]                    \\
             & =-q\left(p+qx\right)-9e^{8ry}r \\
             8pqx+e^{8ry}
          \end{align*}
  \end{enumerate}
\end{soln}
\end{document}