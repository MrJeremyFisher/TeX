\documentclass[10pt]{article}

\usepackage[margin=0.75in]{geometry}
\usepackage{amsmath,amsthm,amssymb}
\usepackage{xcolor}
\usepackage{cancel}
\usepackage{graphicx}
\usepackage{changepage}
\usepackage{circuitikz}
\usepackage{pgfplots}
\usepackage{physics}
\usepackage{hyperref}
\usepackage{siunitx}
\usepackage{fontspec}
\usepackage{relsize}
\usepackage{subfig}
\usepackage{sagetex}
\usepackage{minted}
\usepackage{pdfpages}
\usepackage{todonotes}
\usepackage{multicol, multirow, booktabs}
\usepackage[breakable]{tcolorbox}
\usepackage[inline]{enumitem}

\theoremstyle{definition}
\newtheorem{problem}{Problem}
\newtheorem{soln}{Solution}

\pgfplotsset{compat=newest}
\usetikzlibrary{lindenmayersystems}
\usetikzlibrary{arrows}
\usetikzlibrary{calc}
\usetikzlibrary{positioning, fit}
\usetikzlibrary{3d, perspective}

\definecolor{incolor}{HTML}{303F9F}
\definecolor{outcolor}{HTML}{D84315}
\definecolor{cellborder}{HTML}{CFCFCF}
\definecolor{cellbackground}{HTML}{F7F7F7}
\newcommand{\ui}{\hat{i}}
\newcommand{\uj}{\hat{j}}
\newcommand{\uk}{\hat{k}}
\newcommand{\ux}{\hat{\mathbf{x}}}
\newcommand{\uy}{\hat{\mathbf{y}}}
\newcommand{\uz}{\hat{\mathbf{z}}}
\newcommand{\uv}[1]{\hat{\bm{{#1}}}}
\newcommand{\pr}[1]{{#1^\prime}}
\newcommand{\pd}[2][]{{\frac{\partial^{#1}}{\partial {{#2}^{#1}}}}}
\newcommand{\dif}[2][]{{\frac{d^{#1}}{d{{#2}^{#1}}}}}
\pgfdeclarelayer{background}  
\pgfsetlayers{background,main}
\AtBeginDocument{\RenewCommandCopy\qty\SI}
\newcommand{\justif}[2]{&{#1}&\text{#2}}
\DeclareMathOperator\Arg{Arg}
\DeclareMathOperator{\sgn}{sgn}

\makeatletter
\newcommand{\boxspacing}{\kern\kvtcb@left@rule\kern\kvtcb@boxsep}
\makeatother
\newcommand{\prompt}[4]{
    \ttfamily\llap{{\color{#2}[#3]:\hspace{3pt}#4}}\vspace{-\baselineskip}
}

\newcommand{\thevenin}[2]{
  \begin{center}
    \begin{circuitikz} \draw
      (0,0) -- (2,0) to[battery1, l_=$V_{Th}\eq#1$] (2,2) 
      to[resistor, l_=$R_{Th}\eq#2$] (0,2)
      ;
      \draw [o-] (-.07,2.079);
      \draw [o-] (-.07,0.079);
    \end{circuitikz}
  \end{center}
}

\newcommand{\norton}[2]{
  \begin{center}
    \begin{circuitikz} \draw
      (0,0) -- (3,0) to[american current source, l_=$I_{N}\eq#1$] (3,2) -- (0,2) (2,0)
      to[resistor, l=$R_{N}\eq#2$] (2,2)
      ;
      \draw [o-] (-.07,2.079);
      \draw [o-] (-.07,0.079);
    \end{circuitikz}
  \end{center}
}

\newcommand{\highlight}[1]{\colorbox{yellow}{$\displaystyle #1$}}

\newcommand{\ti}[1]{\widetilde{#1}}

\newfontface{\Kaufmann}{Kaufmann}
\DeclareTextFontCommand{\kf}{\Kaufmann}
\newcommand{\scriptr}{\fontsize{12pt}{12pt}\kf{r}}

\newfontface{\KaufmannB}{Kaufmann Bd BT}
\DeclareTextFontCommand{\kfb}{\KaufmannB}
\newcommand{\bscriptr}{\fontsize{12pt}{12pt}\kfb{r}}

\newcommand{\bv}[1]{\mathbf{#1}}

\title{Math 3150H: Assignment II}
\author{Jeremy Favro (0805980) \\ Trent University, Peterborough, ON, Canada}
\date{\today}

\begin{document}
\maketitle
My student number is 0805980 so $p=9$, $q=5$, and $r=22$.
% PROBLEM 1
\begin{problem}
Solve for $u(x,t)$.

$$\frac{\partial u}{\partial t}=2\frac{\partial^2 u}{\partial x^2}-8\frac{\partial u}{\partial x},\qquad 0<x<2,\quad t>0,$$
$$u(0,t)=0,\quad u(2,t)=0,\quad t>0,\quad \text{and}\quad u(x,0)=(x^2-2x)e^{2x}$$

\end{problem}
\begin{soln}
  Since this problem is already homogenous we can go directly to separation of variables. Let
  $$u(x,t)=\phi(x)T(t).$$
  Then our PDE becomes
  $$\frac{\partial \phi(x)T(t)}{\partial t}=2\frac{\partial^2 \phi(x)T(t)}{\partial x^2}-8\frac{\partial \phi(x)T(t)}{\partial x}$$
  which we divide through by
  $\phi(x)T(t)$ to obtain
  $$\frac{1}{T(t)}\frac{\partial T(t)}{\partial t}=\frac{2}{\phi(x)}\frac{\partial^2 \phi(x)}{\partial x^2}-\frac{8}{\phi(x)}\frac{\partial \phi(x)}{\partial x}=P.$$
  Which gives us two ODEs to solve,
  $$2\frac{\partial^2 \phi(x)}{\partial x^2}-8\frac{\partial \phi(x)}{\partial x}=P\phi(x),\qquad\phi(0)=\phi(2)=0$$
  and
  $$\frac{\partial T(t)}{\partial t}=PT(t).$$
  Let's start with the $\phi$ equation. We have three cases to check:
  \begin{enumerate}[label=(\arabic*)]
    \item $P=0$: Here our ODE becomes
          $$\frac{\partial^2 \phi(x)}{\partial x^2}-4\frac{\partial \phi(x)}{\partial x}=0$$
          which is constant coefficient with characteristic polynomial
          $$m^2-4m=0\implies m=0,4$$
          and so our solution is
          $$\phi(x)=C_1e^{4x}+C_2.$$
          Applying initial conditions,
          $$\phi(0)=C_1+C_2=0\implies C_1=-C_2$$
          and
          $$\phi(2)=C_1e^8-C_1=0\implies C_1=C_2=0$$
          and so this is a trivial case and $P=0$ is not an eigenvalue.
    \item $P>0\implies P=\lambda^2$: Here our ODE becomes
          $$2\frac{\partial^2 \phi(x)}{\partial x^2}-8\frac{\partial \phi(x)}{\partial x}-\phi\lambda^2=0$$
          which is again constant coefficient now with characteristic polynomial
          $$2m^2-8m-\lambda^2=0$$
          which has roots
          $$m=\frac{8\pm\sqrt{64-4(2)(-\lambda^2)}}{4}=\frac{4\pm\sqrt{16+2\lambda^2}}{2}$$
          and so
          $$\phi(x)=C_1e^{(\frac{4+\sqrt{16+2\lambda^2}}{2})x}+C_2e^{(\frac{4-\sqrt{16+2\lambda^2}}{2})x}.$$
          Applying initial conditions then,
          $$\phi(0)=C_1+C_2=0\implies C_1=-C_2$$
          and
          $$\phi(2)=C_1e^{4+\sqrt{16+2\lambda^2}}-C_1e^{4-\sqrt{16+2\lambda^2}}=0\implies
            C_1e^{4+\sqrt{16+2\lambda^2}}=C_1e^{4-\sqrt{16+2\lambda^2}}\implies C_1=C_2=0
          $$
          and so this is a trivial case and $P=\lambda^2$ is not an eigenvalue.
    \item $P<0\implies P=-\lambda^2$: Here our ODE becomes
          $$2\frac{\partial^2 \phi(x)}{\partial x^2}-8\frac{\partial \phi(x)}{\partial x}+\phi\lambda^2=0$$
          which is again constant coefficient with characteristic polynomial
          $$2m^2-8m+\lambda^2=0$$
          which has roots
          $$m=\frac{8\pm\sqrt{64-8\lambda^2}}{4}=\frac{4\pm\sqrt{16-2\lambda^2}}{2}$$
          which only produces something unique from the previous two cases when
          $$16-2\lambda^2\leq 0\implies \lambda \geq\sqrt{8}.$$
          We must again check cases for $\lambda$ in this range then,
          \begin{enumerate}[label=(\roman*)]
            \item $\lambda=\sqrt{8}$ gives us one value for $m$, $m=2$ and hence
                  $$\phi(x)=C_1e^{2x}+C_2xe^{2x}.$$
                  Applying initial conditions then,
                  $$\phi(0)=C_1=0$$
                  and
                  $$\phi(2)=2C_2e^4=0\implies C_2=0$$
                  and so this is a trivial case.
            \item $\lambda>\sqrt{8}$ means we obtain
                  $$m=\frac{4\pm i\sqrt{2\lambda^2-16}}{2}=2\pm\frac{\sqrt{2\lambda^2-16}}{2}i$$
                  because the discriminant becomes negative.
                  This gives us
                  $$\phi(x)  =e^{2x}\left(C_1\cos(\frac{\sqrt{2\lambda^2-16}}{2}x)+C_2\sin(\frac{\sqrt{2\lambda^2-16}}{2}x)\right).$$
                  Applying initial conditions,
                  $$\phi(0)=C_1=0$$
                  and
                  $$\phi(2)=C_2e^{4}\sin(\sqrt{2\lambda^2-16})=0$$
                  which means that either $C_2$ is zero and we obtain another trivial case, or
                  $$\sqrt{2\lambda^2-16}=n\pi\implies \lambda=\sqrt{\frac{(n\pi)^2+16}{2}}$$
                  a non trivial solution!
          \end{enumerate}
          So we obtain, finally, a nontrivial solution:
          $$\phi_n(x)=Ce^{2x}\sin(\frac{n\pi}{2}x)$$
          we've also obtained that
          $$P=-\lambda^2=-\frac{(n\pi)^2+16}{2}.$$
          We can also now write our full $\phi$ as a sum of the $\phi_n$s:
          $$\phi(x)=\sum_{n=1}^\infty C_ne^{2x}\sin(\frac{n\pi}{2}x)$$
  \end{enumerate}
  Now we have to solve
  $$\frac{\partial T(t)}{\partial t}=PT(t)$$
  which is easy,
  $$T(t)=Ce^{Pt}=C\exp(-\frac{(n\pi)^2+16}{2}t).$$
  We can now recombine our solutions to obtain
  $$u=\phi(x)T(t)=\sum_{n=1}^\infty C_ne^{2x}\sin(\frac{n\pi}{2}x)\exp(-\frac{(n\pi)^2+16}{2}t)$$
  and use our last initial condition to determine $C$:
  $$u(x,0)=(x^2-2x)\cancel{e^{2x}}=\sum_{n=1}^\infty C_n\cancel{e^{2x}}\sin(\frac{n\pi}{2}x)$$
  which gives
  $$C_n=\int_0^2(x^2-2x)\sin(\frac{n\pi}{2}x)\,dx$$
  which I threw into Sage:
  \begin{tcolorbox}[breakable, size=fbox, boxrule=1pt, pad at break*=1mm,colback=cellbackground, colframe=cellborder]
    \prompt{In}{incolor}{1}{\boxspacing}
    \begin{minted}[breaklines, autogobble]{sage}
x, n = var('x n')
assume(n, "integer")

fn = (x^2-2*x)*sin((n*pi*x)/2)

show(integral(fn, x, 0, 2).full_simplify())
              \end{minted}
  \end{tcolorbox}
  \begin{tcolorbox}[breakable, size=fbox, boxrule=.5pt, pad at break*=1mm, opacityfill=0]
    \prompt{Out}{outcolor}{1}{\boxspacing}
    $\displaystyle \frac{16 \, {\left(\left(-1\right)^{n} - 1\right)}}{\pi^{3} n^{3}}$
  \end{tcolorbox}
  So the problem is solved,
  $$u(x,t)=\sum_{n=1}^\infty \frac{16{\left(\left(-1\right)^{n} - 1\right)}}{\pi^{3} n^{3}}e^{2x}\sin(\frac{n\pi}{2}x)\exp(-\frac{(n\pi)^2+16}{2}t)$$
\end{soln}

% PROBLEM 2
\begin{problem}
Solve for $u(x,t)$.
$$\frac{\partial u}{\partial t}=2q^2t\frac{\partial^2 u}{\partial x^2},\qquad 0<x<1,\quad t>0,$$
$$u(0,t)=t^2,\quad u(1,t)=1+t^2,\quad t>0, \quad\text{and}\quad u(x,0)=1$$
\end{problem}
\begin{soln}
  Here we start with
  $$u(x,t)=v(x,t)+w(x,t)$$
  with $$w(x,t)=t^2+x$$ so that
  $$w(0,t)=u(0,t)=t^2\implies v(0,t)=0$$
  and
  $$w(1,t)=u(1,t)=t^2+1\implies v(1,t)=0.$$
  Now we are solving
  $$\frac{\partial}{\partial t}\left(v(x,t)+w(x,t)\right)=2q^2t\frac{\partial^2}{\partial x^2}\left(v(x,t)+w(x,t)\right)$$
  with boundary conditions
  $$v(0,t)=v(1,t)=0,\quad v(x,0)=u(x,0)-w(x,0)=1-x.$$
  We begin by simplifying the PDE to
  $$\frac{\partial v}{\partial t}=2q^2t\frac{\partial^2 v}{\partial x^2}-2t$$
  and we write
  $$v(x,t)=\sum_{n=1}^\infty b_n(t)\sin(n\pi x)$$
  which when plugged into our PDE gives
  $$\sum_{n=1}^\infty \frac{db_n}{dt}\sin(n\pi x)=\left(-\frac{2q^2t}{n^2\pi^2}\sum_{n=1}^\infty \frac{db_n}{dt}\sin(n\pi x)\right)-2t$$
  so
  $$\sum_{n=1}^\infty \left[\frac{db_n}{dt}+2(qn\pi)^2tb_n\right]\sin(n\pi x)=-2t$$
  which is a Fourier sine series for $-2t$
  and hence the coefficient in the series, a differential equation here, is given by the usual integral,
  $$\frac{db_n}{dt}+2(qn\pi)^2tb_n=2\int_0^1(-2t)\sin(n\pi x)\,dx=-4t\left[\frac{(-1)^n-1}{n\pi}\right]$$
  which is linear with integrating factor
  $$\mu=\exp(\int 2(qn\pi)^2t\,dt)=\exp((qn\pi)^2t^2)$$
  and therefore has solution
  $$b_n=\exp(-(qn\pi)^2t^2)\int -4t\exp((qn\pi)^2t^2)\left[\frac{(-1)^n-1}{n\pi}\right]\,dt+C\exp(-(qn\pi)^2t^2)$$
  which I solved with Sage to obtain
  $$b_n= -\frac{2{((-1)^{n} - 1)} }{\pi^{3} n^{3} q^{2}}+C\exp(-(qn\pi)^2t^2)$$
  so we have now
  $$v(x,t)=\sum_{n=1}^\infty \left(-\frac{2{((-1)^{n} - 1)} }{\pi^{3} n^{3} q^{2}}+C\exp(-(qn\pi)^2t^2)\right)\sin(n\pi x).$$
  We can now apply our initial conditions:
  $$v(0,t)=\sum_{n=1}^\infty \left(-\frac{2{((-1)^{n} - 1)} }{\pi^{3} n^{3} q^{2}}+C\exp(-(qn\pi)^2t^2)\right)\cdot \sin(0)=0$$
  and
  $$v(1,t)=\sum_{n=1}^\infty \left(-\frac{2{((-1)^{n} - 1)} }{\pi^{3} n^{3} q^{2}}+C\exp(-(qn\pi)^2t^2)\right)\cdot \sin(n\pi)=0$$
  and our boundary condition:
  $$v(x,0)=\sum_{n=1}^\infty \left(-\frac{2{((-1)^{n} - 1)} }{\pi^{3} n^{3} q^{2}}+C\right)\sin(n\pi x)=1-x$$
  which is another Fourier sine series and must satisfy
  $$-\frac{2{((-1)^{n} - 1)} }{\pi^{3} n^{3} q^{2}}+C=2\int_0^1(1-x)\sin(n\pi x)\,dx=\frac{2}{n\pi}\implies C=\frac{2}{n\pi}+\frac{2{((-1)^{n} - 1)} }{\pi^{3} n^{3} q^{2}}=\frac{2n^2\pi^2q^2+2((-1)^{n} - 1)}{n^{3}\pi^{3}q^{2}}$$
  so
  $$v(x,t)=\sum_{n=1}^\infty\frac{2}{n\pi}\exp(-(qn\pi)^2t^2)\sin(n\pi x).$$
  So we have our full solution (with my $q=5$),
  $$u(x,t)=v(x,t)+w(x,t)=t^2+x+\sum_{n=1}^\infty\frac{2}{n\pi}\exp(-(5n\pi)^2t^2)\sin(n\pi x)$$
\end{soln}
\newpage

% PROBLEM 3
\begin{problem}
Solve for $u(x,t)$.
$$\frac{\partial^2u}{\partial t^2}=q^2\frac{\partial^2 u }{\partial x^2}+6q^2(x-x^2),\qquad 0<x<1,\quad t>0,$$
$$u(0,t)=2,\quad \frac{\partial u}{\partial x}(1,t)=1,\quad t>0,\quad \text{and}\quad u(x,0)=\frac{1}{2}x^4-x^3+x^2+(2+p)x,\quad \frac{\partial u}{\partial t}(x,0)=0$$
\end{problem}
\begin{soln}
  Let
  $$u(x,t)=w(x,t)+v(x)$$
  so we get that
  $$w_{tt}=q^2(w_{xx}+v_{xx})+6q^2(x-x^2).$$
  We want $$q^2v_{xx}+6q^2(x-x^2)=0$$
  so $$v(x)=-x^3+\frac{1}{2}x^4-Cx-C_1.$$
  Our boundary conditions on $u(x,t)$ are that
  $$u(0,t)=w(0,t)+v(0)=2$$
  and we want $w(0,t)=0$ so we say that
  $v(0)=2$. We do the same for
  $u_x(1,t)=w_x(1,t)+v_x(1)=1$ so
  $v_x(1)=1$. Applying these,
  $v(0)=-C_1=2$ and $v_x(1)=-3+2-C=1\implies C=2$
  so $v(x)=\frac{1}{2}x^4-x^3+2x-2$. Now our pde becomes
  $$w_{tt}=q^2w_{xx}+6q^2(x^2-x)+6q^2(x-x^2)=q^2w_{xx}$$
  with conditions
  $$w(0,t)=0,\quad \frac{\partial w}{\partial x}(1,t)=0,\quad t>0,\quad \text{and}\quad w(x,0)=x^2+px+2,\quad \frac{\partial w}{\partial t}(x,0)=0.$$
  Now we can do separation of variables with
  $$w(x,t)=\phi(x)T(t)$$
  which gives us
  $$\frac{1}{T}\frac{\partial^2T}{\partial t^2}=\frac{q^2}{\phi}\frac{\partial^2 \phi}{\partial x^2}=P$$
  which gives us two equations,
  $$\phi_{xx}-\frac{P}{q^2}\phi(x)=0$$
  and
  $$\frac{\partial^2T}{\partial t^2}-PT=0.$$
  Solving the position equation first we look at the usual cases:
  \begin{enumerate}[label=(\arabic*)]
    \item $P=0$: this gives us $\phi_{xx}=0\implies\phi=Cx$
          and $\phi(1)=0\implies C=0$.
    \item $P>0\implies P=\lambda^2$: $\phi_{xx}-\frac{\lambda^2}{q^2}\phi(x)=0$
          which gives $\phi(x)=C_1e^{\lambda x/q}+C_2e^{-\lambda x/q}$
          and
          $\phi(0)=C_1+C_2\implies C_1=-C_2$
          and
          $\phi(1)=C_1e^{\lambda/q}-C_1e^{-\lambda/q}\implies C_1=C_2=0$.
    \item $P<0\implies P=-\lambda^2$: $\phi_{xx}+\frac{\lambda^2}{q^2}\phi(x)=0$
          which gives $\phi(x)=C_1\cos\left(\frac{\lambda}{q}x\right)+C_2\sin\left(\frac{\lambda}{q}x\right)$
          and
          $\phi(0)=C_2=0$
          and
          $\phi(1)=C_1\sin\left(\frac{\lambda}{q}\right)=0\implies \lambda=qn\pi$
  \end{enumerate}
  Now solving 
  $$\frac{\partial^2T}{\partial t^2}+q^2n^2\pi^2 T=0$$
  which gives
  $$T=C_3\cos(qn\pi t)+C_4\sin(qn\pi t)$$
  so our full solution is 
  $$w(x,t)=\sum_{n=1}^\infty (C_1\cos(qn\pi t)+C_2\sin(qn\pi t))\sin\left(n\pi x\right).$$
  Now we apply our initial conditions:
  $$w(x,0)=\sum_{n=1}^\infty C_1\sin\left(n\pi x\right)=x^2+px+2$$
  so
  $$C_1=2\int_{0}^{1}(x^2+px+2)\sin(n\pi x)\,dx$$
  which Sage evaluates as 
  $$ -\frac{2(\pi^2(-1)^n n^2 p - 2 \pi^2 n^2 + (3 \pi^2 n^2 - 2) (-1)^n + 2)}{\pi^3 n^3}$$
  and the other condition on 
  $$\frac{\partial}{\partial t}\sum_{n=1}^\infty qn\pi(-C_1\sin(qn\pi t)+C_2\cos(qn\pi t))\sin\left(n\pi x\right)$$
  gives 
  $$w_t(x,0)=\sum_{n=1}^\infty C_2qn\pi\cos(qn\pi t)\sin\left(n\pi x\right)\implies C_2=0$$
  so 
  $$w(x,t)=-2\sum_{n=1}^\infty \frac{\pi^2(-1)^n n^2 p - 2 \pi^2 n^2 + (3 \pi^2 n^2 - 2) (-1)^n + 2}{\pi^3 n^3}\cos(qn\pi t)\sin\left(n\pi x\right)$$
  so
  $$u(x,t)=w(x,t)+v(x)=
  \frac{1}{2}x^4-x^3+2x-2-2\sum_{n=1}^\infty \frac{\pi^2(-1)^n n^2 p - 2 \pi^2 n^2 + (3 \pi^2 n^2 - 2) (-1)^n + 2}{\pi^3 n^3}\cos(qn\pi t)\sin\left(n\pi x\right)
  $$








































  % Let
  % $$u(x,t)=v(x,t)+w(x,t)$$
  % with
  % $$w(x,t)=2+x\implies w(0,t)=2,\quad \frac{\partial w}{\partial x}(1,t)=1,\quad w(x,0)=2+x,\quad\frac{\partial w}{\partial t}(x,0)=0$$
  % and so
  % $$v(0,t)=\frac{\partial v}{\partial x}(1,t)=0,\quad v(x,0)=u(x,0)-w(x,0)=\frac{1}{2}x^4-x^3+x^2+(1+p)x-2,\quad \frac{\partial v}{\partial t}(x,0)=\frac{\partial u}{\partial t}(x,0)-\frac{\partial w}{\partial t}(x,0)=0$$
  % and so our PDE simplifies to
  % $$\frac{\partial^2v}{\partial t^2}=q^2\frac{\partial^2 v }{\partial x^2}+6q^2(x-x^2)$$
  % with the above (homogenous) initial/boundary conditions.
  % Now we say that
  % $$v(x,t)=\sum_{n=1}^{\infty}b_n(t)\sin(n\pi x)$$
  % will be the solution as $v(x,t)$ can be expanded by Fourier series. Plugging this into the PDE we obtain that
  % $$\sum_{n=1}^{\infty}\frac{\partial^2 b_n}{\partial t^2}\sin(n\pi x)=q^2\sum_{n=1}^{\infty}-n^2\pi^2b_n(t)\sin(n\pi x)+6q^2(x-x^2)$$
  % rearranging,
  % $$\sum_{n=1}^{\infty}\left[\frac{\partial^2 b_n}{\partial t^2}+q^2n^2\pi^2b_n(t)\right]\sin(n\pi x)=6q^2(x-x^2)$$
  % which is a Fourier series for the left hand side.
  % We know that the coefficient for this series, an ODE in $t$ here, is given by
  % $$\frac{\partial^2 b_n}{\partial t^2}+q^2n^2\pi^2b_n(t)=2\int_{0}^{1}6q^2(x-x^2)\sin(n\pi x)\,dx=-\frac{24q^2\left((-1)^n - 1\right) }{n^3\pi^3}.$$
  % which has solution
  % $$b_n(t)=C_{1}\sin\left(\pi nqt\right)+C_{2}\cos\left(\pi nqt\right)-\frac{24\left((-1)^{n}-1\right)}{{\pi}^{5}{n}^{5}}.$$
  % So our solution is
  % $$v(x,t)=\sum_{n=1}^{\infty}\left[C_{1}\sin\left(\pi nqt\right)+C_{2}\cos\left(\pi nqt\right)-\frac{24\left((-1)^{n}-1\right)}{{\pi}^{5}{n}^{5}}\right]\sin(n\pi x).$$
  % Applying initial conditions we obtain that
  % $$v(x,0)=\frac{1}{2}x^4-x^3+x^2+(1+p)x-2=\sum_{n=1}^{\infty}\left[C_{2}-\frac{24\left((-1)^{n}-1\right)}{{\pi}^{5}{n}^{5}}\right]\sin(n\pi x)$$
  % Which is yet again a Fourier series whose coefficient is given by
  % $$C_{2}-\frac{24\left((-1)^{n}-1\right)}{{\pi}^{5}{n}^{5}}=2\int_{0}^{1}\left(\frac{1}{2}x^4-x^3+x^2+(1+p)x-2\right)\sin(n\pi x)\,dx.$$
  % Giving this integral to Sage an having it do all the work for us we get
  % $$C_2=-\frac{2  \pi^{4} \left(-1\right)^{n} n^{4} p + 4  \pi^{4} n^{4} + 4\pi^{2} n^{2} - {\left(\pi^{4} n^{4} + 4\pi^{2} n^{2} + 24\right)} \left(-1\right)^{n} + 24}{2\pi^{5} n^{5}}$$
  % and applying the other condition
  % $$\frac{\partial v}{\partial t}=v(x,t)=\sum_{n=1}^{\infty}\left[C_{1}\pi nq\cos\left(\pi nqt\right)-C_{2}\pi nq\sin\left(\pi nqt\right)\right]\sin(n\pi x)$$
  % and
  % $$\frac{\partial v}{\partial t}(0)=\sum_{n=1}^{\infty}C_{1}\pi nq\sin(n\pi x)=0\implies C_1=0.$$
  % So we have that
  % $$v(x,t)=\sum_{n=1}^{\infty}\left[C_n\cos\left(\pi nqt\right)-\frac{24\left((-1)^{n}-1\right)}{{\pi}^{5}{n}^{5}}\right]\sin(n\pi x).$$
  % where
  % $$C_n=-\frac{2  \pi^{4} \left(-1\right)^{n} n^{4} p + 4  \pi^{4} n^{4} + 4\pi^{2} n^{2} - {\left(\pi^{4} n^{4} + 4\pi^{2} n^{2} + 24\right)} \left(-1\right)^{n} + 24}{2\pi^{5} n^{5}}$$
  % and so our final solution
  % $$u(x,t)=w(x,t)+v(x,t)$$
  % is, with my values for $p$ and $q$,
  % $$u(x,t)=2+x+\sum_{n=1}^{\infty}\left[C_n\cos\left(5\pi nt\right)-\frac{24\left((-1)^{n}-1\right)}{{\pi}^{5}{n}^{5}}\right]\sin(n\pi x)$$
  % with
  % $$C_n=-\frac{18  \pi^{4} \left(-1\right)^{n} n^{4}  + 4  \pi^{4} n^{4} + 4\pi^{2} n^{2} - {\left(\pi^{4} n^{4} + 4\pi^{2} n^{2} + 24\right)} \left(-1\right)^{n} + 24}{2\pi^{5} n^{5}}$$
\end{soln}

% PROBLEM 4
\begin{problem}
Consider your own homogeneous convection problem with:
$$u(x,0)=10x(L-x)\sin\left(\frac{\pi x}{L}\right)$$
$$h=q,\quad \kappa=p,\quad L=qp,\quad k=\frac{p}{q+p}$$
where $p$ is the largest digit and $q$ is the smallest nonzero digit of the last four digits of your student
I.D. number. Using SAGE to carry out an investigation of the temperature distribution in the bar:
\begin{enumerate}[label=(\alph*)]
  \item Find the first 20 eigenvalues of the problem
  \item Obtain the temperature distribution as a function of $x$ and $t$.
  \item Plot the temperature of the right end of the bar for the first 120 seconds.
  \item Plot the temperature of the right end of the bar for the first 120 seconds and determine the
        maximum temperature that the right of the bar.
  \item Obtain the time(s) at which the middle of the bar temperature is 25 degrees.
  \item Plot the temperature distribution in the bar at time(s) found in part (e).
\end{enumerate}
\end{problem}
\begin{soln}
  \includepdf[
    %% Include all pages of the PDF
    pages=-,
    %% make this page have the usual page style
    %% (you can change it to plain etc). By default pdfpages
    %% sets the pagecommand to \pagestyle{empty}
    pagecommand={\pagestyle{headings}},
    %% Add a "section" entry to the ToC with the heading
    %% "Quilling Shapes" and the label "sec:shapes"
    addtotoc={1,section,1,Quilling Shapes,sec:shapes}]
  %% The pdf file itself
  {q4.pdf}
\end{soln}
\end{document}